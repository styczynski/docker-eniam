% \RequirePackage[hyphens]{url}
\documentclass[a4paper, 12pt]{article}
% \documentclass[preprint, a4paper, 11pt]{article}

\usepackage[T1]{fontenc}
\usepackage[english, polish]{babel}
\usepackage[utf8]{inputenc}

\usepackage{fullpage}
\usepackage{amsmath}
\usepackage{amsthm}
\theoremstyle{remark}
\newtheorem{test}{Test}

\usepackage{multicol}

\usepackage[hidelinks]{hyperref}
% \urlstyle{sf}
\expandafter\def\expandafter\UrlBreaks\expandafter{\UrlBreaks%  save the current one
      \do\\~} % this stops urls from sticking out of the damn screen and the paper. I hate this arrggh.


      \usepackage{tocloft}
      \cftsetindents{section}{0pt}{2.25em}
      \cftsetindents{subsection}{0pt}{2.25em}
      \cftsetindents{subsubsection}{0pt}{2.25em}
% \addtocontents{toc}{\cftpagenumbersoff{section}}

\usepackage{xcolor}

\newcommand{\eniam}{\textsc{Eniam}}
\newcommand{\experiencer}{\textsc{Experiencer}}
\newcommand{\agent}{\textsc{Agent}}
\newcommand{\theme}{\textsc{Theme}}

\frenchspacing

\title{
\normalsize
\setlength{\topsep}{0pt} 
\noindent
Zleceniodawca: IPI PAN \hfill Dzieło do umowy z dnia:  28.11.2016 \\
Zleceniobiorca: Jakub Kozakoszczak \hfill Termin zakończenia pracy: 15.12.2016 \\
\rule{\linewidth}{2pt} \\[6pt] 
\huge Reprezentacja semantyczna \\ przyimków i ich modyfikatorów \\ dla parsera \eniam  \\[-2pt]}
\author{\Large Wojciech Jaworski, Jakub Kozakoszczak\\[6pt]
\hspace{-5pt}\rule{\linewidth}{2pt}  }
\date{}
% \date{\normalsize Date Submitted}

\begin{document}

\maketitle

\vspace{-26pt}
\begin{description}
	\setlength{\itemsep}{4pt} 
	\setlength{\parskip}{0pt} 
	\item[Zrealizowane w ramach projektu:] „CLARIN -- Polskie wspólne zasoby językowe i infrastruktura technologiczna”
	\item[Tytuł pracy zamówionej:] ,,Koncepcja reprezentacji semantycznej przyimków uwzględniająca ich modyfikowalność''
	\item[Adres dzieła:] \url{http://wiki.nlp.ipipan.waw.pl/clarin/Parser%20kategorialny?action=AttachFile&do=view
	&target=JK_reprezentacja_przyimkow.pdf}
	\item[Opracowanie dokumentu:] Jakub Kozakoszczak
\end{description}

\vspace{6pt}
\hrule
\vspace{-14pt}
\renewcommand\contentsname{\normalsize{}}
\tableofcontents
\vspace{20pt}
\hrule
\newpage

\section{Koncepcja} % (fold)

Proponowana reprezentacja rozwiązuje problem modyfikowalności przyimków przez przypisywanie im zmiennej odniesieniowej i nadanie roli tematycznej.
Z uwagi na występowanie modyfikatorów dla przyimków semantycznych (np. ,,dość głęboko w szafie'' , ,,5 km od domu'' (dodatkowy argument ,,5 km'') ,,w szczególnym związku z'' (modyfikacja przyimka złożonego), przyimki te nie mogą być reprezentowane jako predykaty wiążące identyfikator odniesienia swojego podrzędnika z identyfikatorem odniesienia nadrzędnika.
Dlatego z każdym przyimkiem semantycznym wiążemy zmienną reprezentującą jego własne odniesienie i wiążemy ją z zmienną reprezentującą odniesienie nadrzędnika jedną z relacji oznaczających role tematyczne.

Drugim problemem jest wieloznaczność przyimków. Dla każdego przyimka identyfikujemy jego sensy, każdy traktujemy jak oddzielny przyimek (oddzielną jednostkę leksykalną), zaliczamy każdy do klasy podobnych sensów i wyczerpująco opisujemy uwzględniając dla każdego sensu
rolę tematyczną, 
typ semantyczny całej frazy przyimkowej
i preferencje selekcyjne przyimka.

\section{Podział ze względu na rolę tematyczną przyimka} % (fold)
\label{sec:podzia_ze_wzgldu_na_rol_tematyczna}

\subsection{Przyimki niesemantyczne (argumenty)} % (fold)
\label{sub:argument}
\paragraph{dla-9}\label{prep-9}	przyimek wyznaczający obiekt odniesienia komunikowanego sądu, najczęściej osobę, z punktu widzenia której sąd ten jest wydawany
\begin{equation}
\text{Był to dla mnie rok pełen sukcesów.}
\end{equation}
\paragraph{do-12}\label{prep-12}	przyimek komunikujący cel czynności
\begin{equation}
\text{Oddał buty do naprawy.}
\end{equation}
\paragraph{do-17}\label{prep-17}	przyimek wprowadzający składnik zdania rządzony przez czasownik, przymiotnik, rzeczownik lub określający je
\begin{equation}
\text{Przywiązał się do dziadka. Był podobny do siostry.}
\end{equation}
\paragraph{do-18}\label{prep-18}	przyimek wprowadzający samodzielne wypowiedzenia nieoznajmujące, m.in. przekleństwa, formuły grzecznościowe, odezwy
\begin{equation}
\text{Do diabła! Do zobaczenia! Do broni!}
\end{equation}
\paragraph{dokoła-21, dookoła-22}\label{prep-21}	przyimek określający główny temat tego, o czym mowa w zdaniu
\begin{equation}
\text{Wszystkie jego myśli krążyły dokoła rodzinnego domu.}
\end{equation}
\paragraph{ku-35}\label{prep-35}	przyimek wraz z przyłączanym rzeczownikiem komunikujący, że zbliża się dana pora roku lub dnia
\begin{equation}
\text{Dni są coraz dłuższe i ma się ku wiośnie.}
\end{equation}
\paragraph{ku-36}\label{prep-36}	przyimek występujący w zdaniu komunikującym, że ktoś jest zainteresowany inną osobą pod względem erotycznym
\begin{equation}
\text{Nasze dzieci mają się ku sobie.}
\end{equation}
\paragraph{między-40}\label{prep-40}	przyimek komunikujący, iż zależność, o której mowa, dotyczy wymienionych w zdaniu osób, zjawisk, przedmiotów lub faktów
\begin{equation}
\text{Pokrewieństwo między nimi było bardzo dalekie.}
\end{equation}
\paragraph{między-42}\label{prep-42}	przyimek wprowadzający określenie przeznaczenia tego, co stanowi rezultat dokonanego podziału
\begin{equation}
\text{Podzielił majątek między synów.}
\end{equation}
\paragraph{między-44}\label{prep-44}	przyimek komunikujący, iż wybór, przed którym stoi dana osoba, dotyczy wymienionych przedmiotów lub osób
\begin{equation}
\text{Wybieraj między mną a nim.}
\end{equation}
\paragraph{między-45}\label{prep-45}	przyimek wprowadzający określenie zbioru, do którego należą lub spośród którego się wyróżniają charakteryzowane w zdaniu przedmioty lub osoby
\begin{equation}
\text{Słynęła z urody między rówieśnicami.}
\end{equation}
\paragraph{na-50}\label{prep-50}	przyimek tworzący z nazwami czynności wyrażenia określające cel, skutek lub sposób odbywania się tych czynności
\begin{equation}
\text{Składać na pół.}
\end{equation}
\paragraph{na-51}\label{prep-51}	przyimek łączący z nadrzędnymi czasownikami rzeczowniki stanowiące ich uzupełnienie
\begin{equation}
\text{Liczyć na pomoc.}
\end{equation}
\paragraph{na-52}\label{prep-52}	z nadrzędnymi rzeczownikami wyrazy stanowiące ich uzupełnienie
\begin{equation}
\text{Skrzynka na listy.}
\end{equation}
% section podzia_ze_wzgldu_na_rol_tematyczna (end)
\subsection{Beneficient} % (fold)
\label{sub:beneficient}
\paragraph{dla-7} \label{prep-7}	przyimek komunikujący o przeznaczeniu obiektu
\begin{equation}
\text{Film dla młodzieży.}
\end{equation}
% subsection quasi_beneficient (end)
\subsection{Comparison} % (fold)
\label{sub:comparison}
\paragraph{aniżeli-1} \label{prep-1}	przyimek łączący podstawę porównania z wyrażeniem, które komunikuje zwykle wyższy stopień nasilenia cechy
\begin{equation}
\text{Była ładniejsza aniżeli jej koleżanka.}
\end{equation}
\paragraph{co-6} \label{prep-6}	przyimek używany w połączeniach, komunikujących, że przedmioty pod względem danej cechy są tożsame
\begin{equation}
\text{Krawat tego koloru co koszula}
\end{equation}
\paragraph{jak-24} \label{prep-24}	przyimek wprowadzający wyrażenie porównawcze
\begin{equation}
\text{Nos miała jak kartofelek.}
\end{equation}
\subsection{Condition} % (fold)
\label{sub:condition}
\paragraph{dzięki-23} \label{prep-23}	przyimek komunikujący o cechach, faktach lub osobach, które spowodowały, że dana sytuacja jest dla kogoś pomyślna
\begin{equation}
\text{Uratował się dzięki przytomności umysłu.}
\end{equation}
\paragraph{mimo-38} \label{prep-38}	przyimek oznaczający rozbieżność między tym, co się dzieje, a tym, czego się należało spodziewać
\begin{equation}
\text{Mimo późnej pory wyszedł na spacer.}
\end{equation}
\paragraph{na\_skutek-361} \label{prep-361}
\paragraph{pod\_wpływem-400} \label{prep-400}
\paragraph{w\_wyniku-479} \label{prep-479}
\paragraph{z\_powodu-504} \label{prep-504}

\subsection{Duration} % (fold)
\label{sub:duration}
\paragraph{na-48} \label{prep-48}	przyimek tworzący wyrażenia oznaczające trwanie czegoś, okres, termin lub porę dziania się czegoś
\begin{equation}
\text{Wyjechać na weekend.}
\end{equation}
% subsection duration (end)

\subsection{Location} % (fold)
\label{sub:location}
\paragraph{blisko-3} \label{prep-3}	przyimek komunikujący o małej odległości między przedmiotami
\begin{equation}
\text{Mieszka blisko stacji metra.}
\end{equation}
\paragraph{dokoła-20} \label{prep-20}	przyimek komunikujący, że to, o czym mowa w zdaniu, ma miejsce ze wszystkich stron czegoś znajdującego się w środku
\begin{equation}
\text{Dokoła klombów posadzono tulipany.}
\end{equation}
\paragraph{koło-29} \label{prep-29}	przyimek komunikujący o małej odległości między przedmiotami
\begin{equation}
\text{Usiądź koło mnie.}
\end{equation}
\paragraph{koło-31} \label{prep-31}	przyimek przyłączający nazwę przedmiotu będącego obiektem czynności, o której mowa w zdaniu
\begin{equation}
\text{Dłubał coś koło zegara.}
\end{equation}
\paragraph{między-39} \label{prep-39}	przyimek komunikujący, że to, o czym mowa, jest z obu stron otoczone czymś
\begin{equation}
\text{Ulica ciągnęła się między domami.}
\end{equation}
\paragraph{między-43} \label{prep-43}	przyimek ograniczający dwustronnie zbiór cech, które można przypisać nietypowemu obiektowi, o którym mowa
\begin{equation}
\text{Bluzka koloru między żółtym a brązowym.}
\end{equation}
\paragraph{na-46} \label{prep-46}	przyimek tworzący wyrażenia oznaczające miejsce dziania się lub znajdowania się czegoś
\begin{equation}
\text{Pracował na polu.}
\end{equation}
% subsection location (end)

\subsection{Location\_goal} % (fold)
\label{sub:location_goal}
\paragraph{do-13} \label{prep-13}	przyimek komunikujący kierunek ruchu
\begin{equation}
\text{Skoczył do wody.}
\end{equation}
\paragraph{do-15} \label{prep-15}	przyimek wyznaczający górną lub dolną granicę czegoś
\begin{equation}
\text{Inflacja spadła do dziesięciu procent.}
\end{equation}
\paragraph{do-16} \label{prep-16}	przyimek komunikujący o rezultacie procesu lub charakteryzujący stopień nasilenia tego procesu
\begin{equation}
\text{Blacha rozpaliła się do czerwoności.}
\end{equation}
\paragraph{ku-32} \label{prep-32}	przyimek przyłączający określenie miejsca, w kierunku którego ktoś zmierza lub coś prowadzi
\begin{equation}
\text{Odwrócił się nagle i ruszył ku drzwiom.}
\end{equation}
\paragraph{na-47} \label{prep-47}	miejsce lub kierunek, będące celem ruchu, czynności
\begin{equation}
\text{Pojechali na Mazury.}
\end{equation}

\subsection{Extra\_result} % (fold)
\label{sub:manner}
\paragraph{ku-34} \label{prep-34}	przyimek przyłączający określenie czyjegoś stanu wewnętrznego, który wytworzył się w następstwie zdarzenia lub zjawiska, o którym mowa w zdaniu
\begin{equation}
\text{Ku radości maluchów spadł śnieg.}
\end{equation}

\subsection{Measure} % (fold)
\label{sub:measure}
\paragraph{na-49} \label{prep-49}	przyimek tworzący wyrażenia oznaczające miarę, ocenę wielkości oraz zakres ich zastosowania
\begin{equation}
\text{Gruby na palec.}
\end{equation}
% subsection measure (end)

\subsection{Member} % (fold)
\label{sub:member}
\paragraph{bez-2} \label{prep-2}	przyimek komunikujący zwykle brak, nieobecność czegoś lub kogoś
\begin{equation}
\text{Las bez grzybów. Sukienka bez rękawów. Odejść bez pożegnania.}
\end{equation}
% subsection member (end)

\subsection{Math} % (fold)
\label{sub:mod_math}
\paragraph{do-19} \label{prep-19}	przyimek między liczebnikami oznaczający potęgowanie
\begin{equation}
\text{Dwa do piątej.}
\end{equation}
% subsection mod_math (end)

\subsection{Purpose} % (fold)
\label{sub:purpose}
\paragraph{dla-10} \label{prep-10}	przyimek wprowadzający przyczynę jakichś działań
\begin{equation}
\text{Na dziki polowano dla mięsa.}
\end{equation}
\paragraph{dla-8} \label{prep-8}	przyimek komunikujący cel czynności
\begin{equation}
\text{Tupali nogami dla rozgrzewki.}
\end{equation}
\paragraph{do-11} \label{prep-11}	przyimek komunikujący o przeznaczeniu obiektu
\begin{equation}
\text{filiżanka do kawy}
\end{equation}
\paragraph{ku-33} \label{prep-33}	przyimek przyłączający określenie celu czynności, o której mowa w zdaniu
\begin{equation}
\text{Nakręcono ten film ku przestrodze młodzieży.}
\end{equation}
% subsection purpose (end)

\subsection{Quantifier} % (fold)
\label{sub:quantifier}
\paragraph{co-5} \label{prep-5}	przyimek używany w połączeniach charakteryzujących częstość, z jaką coś się powtarza
\begin{equation}
\text{Przystawał co krok.}
\end{equation}
\paragraph{jako-28} \label{prep-28}	przyimek przyłączający liczebnik porządkowy, który wskazuje na kolejność charakteryzowanego zdarzenia względem innych
\begin{equation}
\text{Sekretarz komisji przyszedł na zebranie jako pierwszy.}
\end{equation}
% subsection quantifier (end)

\subsection{Role} % (fold)
\label{sub:role}
\paragraph{jako-25} \label{prep-25}	przyimek wprowadzający określenie roli, jaką pełni jakaś osoba w danych okolicznościach, a także określenie stosunku tej osoby do innych osób
\begin{equation}
\text{Jako gospodarz spotkania starał się być dla wszystkich życzliwy i miły.}
\end{equation}
\paragraph{jako-27} \label{prep-27}	przyimek wprowadzający określenie sposobu potraktowania lub oceny danego obiektu lub stanu rzeczy
\begin{equation}
\text{Takie stany są określane jako depresyjne.}
\end{equation}
% subsection role (end)

\subsection{Time} % (fold)
\label{sub:time}
\paragraph{blisko-4} \label{prep-4}	lub między punktami czasowymi
\begin{equation}
\text{Było już blisko daty wyjazdu.}
\end{equation}
\paragraph{jako-26} \label{prep-26}	przyimek wraz z przyłączanym rzeczownikiem wskazujący na okres życia osoby, o której mowa w zdaniu
\begin{equation}
\text{Jako dziecko przebyła wszystkie choroby zakaźne.}
\end{equation}
\paragraph{koło-30} \label{prep-30}	przyimek poprzedzający określenie liczby, ilości, miary lub czasu, który komunikuje, że jest ono przybliżone
\begin{equation}
\text{Zadzwoniła koło szóstej.}
\end{equation}
\paragraph{lada-37} \label{prep-37}	przyimek łączący się z nazwą jednostki czasu, komunikujący, iż coś stanie się w najbliższym czasie
\begin{equation}
\text{Przyjadą lada godzina.}
\end{equation}
\paragraph{między-41} \label{prep-41}	przyimek wprowadzający określenia granic czasowych, w których miało miejsce dane zdarzenie
\begin{equation}
\text{Wrócę między drugą a trzecią.}
\end{equation}
% subsection time (end)

\subsection{Time\_goal} % (fold)
\label{sub:time_goal}
\paragraph{do-14} \label{prep-14}	przyimek określający czas zakończenia zdarzenia
\begin{equation}
\text{Spał do południa.}
\end{equation}

% subsection time_goal (end)

\section{Indeks przyimków} % (fold)
\label{sec:indeks_przyimkow}
\begin{multicols}{4}
\noindent
\hyperref[prep-1]{aniżeli-1}\\
\hyperref[prep-2]{bez-2}\\
\hyperref[prep-3]{blisko-3}\\
\hyperref[prep-4]{blisko-4}\\
\hyperref[prep-5]{co-5}\\
\hyperref[prep-6]{co-6}\\
\hyperref[prep-7]{dla-7}\\
\hyperref[prep-8]{dla-8}\\
\hyperref[prep-9]{dla-9}\\
\hyperref[prep-10]{dla-10}\\
\hyperref[prep-11]{do-11}\\
\hyperref[prep-12]{do-12}\\
\hyperref[prep-13]{do-13}\\
\hyperref[prep-14]{do-14}\\
\hyperref[prep-15]{do-15}\\
\hyperref[prep-16]{do-16}\\
\hyperref[prep-17]{do-17}\\
\hyperref[prep-18]{do-18}\\
\hyperref[prep-19]{do-19}\\
\hyperref[prep-20]{dokoła-20}\\
\hyperref[prep-21]{dokoła-21}\\
\hyperref[prep-21]{dookoła-22}\\
\hyperref[prep-23]{dzięki-23}\\
\hyperref[prep-24]{jak-24}\\
\hyperref[prep-25]{jako-25}\\
\hyperref[prep-26]{jako-26}\\
\hyperref[prep-27]{jako-27}\\
\hyperref[prep-28]{jako-28}\\
\hyperref[prep-29]{koło-29}\\
\hyperref[prep-30]{koło-30}\\
\hyperref[prep-31]{koło-31}\\
\hyperref[prep-32]{ku-32}\\
\hyperref[prep-33]{ku-33}\\
\hyperref[prep-34]{ku-34}\\
\hyperref[prep-35]{ku-35}\\
\hyperref[prep-36]{ku-36}\\
\hyperref[prep-37]{lada-37}\\
\hyperref[prep-38]{mimo-38}\\
\hyperref[prep-39]{między-39}\\
\hyperref[prep-40]{między-40}\\
\hyperref[prep-41]{między-41}\\
\hyperref[prep-42]{między-42}\\
\hyperref[prep-43]{między-43}\\
\hyperref[prep-44]{między-44}\\
\hyperref[prep-45]{między-45}\\
\hyperref[prep-46]{na-46}\\
\hyperref[prep-47]{na-47}\\
\hyperref[prep-48]{na-48}\\
\hyperref[prep-49]{na-49}\\
\hyperref[prep-50]{na-50}\\
\hyperref[prep-51]{na-51}\\
\hyperref[prep-52]{na-52}\\
\hyperref[prep-361]{na\_skutek-361}\\
\hyperref[prep-400]{pod\_wpływem-400}\\
\hyperref[prep-479]{w\_wyniku-479}\\
\hyperref[prep-504]{z\_powodu-504}\\
\end{multicols}

% section indeks_przyimkow (end)
\end{document}