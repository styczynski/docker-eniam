% \RequirePackage[hyphens]{url}
\documentclass[a4paper, 12pt]{article}
% \documentclass[preprint, a4paper, 11pt]{article}

\usepackage[T1]{fontenc}
\usepackage[english, polish]{babel}
\usepackage[utf8]{inputenc}

\usepackage{fullpage}
\usepackage{amsmath}
\usepackage{amsthm}
\theoremstyle{remark}
\newtheorem{test}{Test}


\usepackage[hidelinks]{hyperref}
% \urlstyle{sf}
\expandafter\def\expandafter\UrlBreaks\expandafter{\UrlBreaks%  save the current one
      \do\\~} % this stops urls from sticking out of the damn screen and the paper. I hate this arrggh.


\usepackage{tocloft}
\cftsetindents{section}{0pt}{2.25em}
\cftsetindents{subsection}{0pt}{2.25em}
\cftsetindents{subsubsection}{0pt}{2.25em}
\addtocontents{toc}{\cftpagenumbersoff{section}}

\usepackage{xcolor}

\newcommand{\eniam}{\textsc{Eniam}}
\newcommand{\experiencer}{\textsc{Experiencer}}
\newcommand{\agent}{\textsc{Agent}}
\newcommand{\theme}{\textsc{Theme}}

\frenchspacing

\title{
\normalsize
\setlength{\topsep}{0pt} 
\noindent
Zleceniodawca: IPI PAN \hfill Dzieło do umowy z dnia: 2.11.2016 \\
Zleceniobiorca: Jakub Kozakoszczak \hfill Termin zakończenia pracy: 25.11.2016 \\
\rule{\linewidth}{2pt} \\[6pt] 
\huge Lista zjawisk semantycznych \\ przeznaczonych do uwzględnienia \\ w reprezentacji semantycznej parsera \eniam  \\[-2pt]}
\author{\Large Wojciech Jaworski, Jakub Kozakoszczak\\[6pt]
\hspace{-5pt}\rule{\linewidth}{2pt}  }
\date{}
% \date{\normalsize Date Submitted}

\begin{document}

\maketitle
		
\vspace{-26pt}
\begin{description}
\setlength{\itemsep}{4pt} 
\setlength{\parskip}{0pt} 
	\item[Zrealizowane w ramach projektu:] „CLARIN -- Polskie wspólne zasoby językowe i infrastruktura technologiczna”
	\item[Tytuł pracy zamówionej:] ,,Opracowanie listy zjawisk semantycznych przeznaczonych do uwzględnienia w reprezentacji semantycznej generowanej przez parser kategorialny''
	\item[Adres dzieła:] \url{http://wiki.nlp.ipipan.waw.pl/clarin/Parser\%20kategorialny?action=AttachFile\&do=view\&target=JK_zjawiska_semantyczne.pdf}
	\item[Opracowanie dokumentu:] Jakub Kozakoszczak
\end{description}

\vspace{6pt}
\hrule
\vspace{-14pt}
\renewcommand\contentsname{\normalsize{}}
\tableofcontents
\vspace{20pt}
\hrule
\newpage

\section{Wstęp} % (fold)
{
\renewcommand\thesection{}
\renewcommand\thesubsection{}
\setcounter{subsection}{-1}
\subsection{Zakres i przeznaczenie listy} % (fold)
}
Lista zbiera zjawiska semantyczne, których spójne opisanie w ramach jednolitej implementowalnej reprezentacji znaczenia o szerokim pokryciu dla języka polskiego wymaga szczegółowych, głębszych badań.
Są to przede wszystkim zjawiska polegające na wprowadzaniu przez wyrażenia elementów znaczenia do zdań w sposób inny podstawowy, czyli przez zawężającą modyfikację znaczenia bezpośredniego nadrzędnika składniowego.
Ten podstawowy sposób nazywamy roboczo \emph{lokalnym wkładem znaczeniowym}.
Lista jest wstępnym rozpoznaniem, które posłuży jako mapa zadań w następnych etapach rozwoju kategorialnego parsera składniowo-semantycznego \eniam.

% section zakres_i_cel_listy (end)

%\chapter{Lista zjawisk semantycznych} % (fold)
{
\section{Lokalność i nielokalność składniowo-semantyczna}
\renewcommand\thesection{}
\renewcommand\thesubsection{}
\setcounter{subsection}{-1}
\subsection{Opis zjawiska} % (fold)
\label{sub:definicja_robocza1}
}

% subsection definicja_robocza (end)
Podzbiorem logiki pierwszego rzędu, który wystarcza do wyrażenia znakomitej większości wypowiedzeń w języku naturalnym, jest logika egzystencjalno-koniunkcyjna (logika $\exists\wedge$). Język reprezentacji znaczenia parsera \eniam{} jest bardziej złożony, bo wzbogacony między innymi o operator \textsc{dscr} wiążący zmienną z formułą oraz pozalogiczne predykaty relacyjne, jak \textsc{type} i \textsc{count} i role tematyczne, jak \agent{} i \theme{}. Istnieją jednak jednostki języka naturalnego, wśród nich kwantyfikatory, operatory logiczne i modyfikatory nieintersektywne, których znaczenie leksykalne (w sensie wkładu w znaczenie zdania) jest niewyrażalne w logice $\exists\wedge$ z operatorem \textsc{dscr}, np. dla języka polskiego
\begin{itemize}
	\item każdy, prawie każdy, każdy z powyższych, co dziesiąty, 
każdy \dots ma inny \dots, nawzajem, razem,
codziennie, nigdy, zawsze, tylko, jakiś, pewien, jedyny, oba, inny
nie tylko \dots lecz także
\end{itemize}
Jednostki te nazywać będziemy \emph{jednostkami o nielokalnym wkładzie składniowo-semantycznym} lub po prostu \emph{operatorami nielokalnymi}. Pozostałe jednostki mają \emph{lokalny wkład składniowo-semantyczny}.

Interesują nas następujące cechy operatorów nielokalnych:
\begin{itemize}
\item konstrukcje składniowe, w których występują,
\item semantyka, czyli wprowadzane zmienne i liczba argumentów, 
\item odwzorowanie konstrukcji składniowych na semantykę.
\end{itemize}
Nie będziemy opatrywać operatorów nielokalnych pełnymi hasłami leksykalnymi, a jedynie zaznaczać ich wystąpienia w formach logicznych
w sposób taki, by potencjalne dodanie znaczenia leksykalnego nie wymagało dalszego zmieniania formy logicznej zdania.

Z uwagi na to, że operatory nielokalne wchodzą ze sobą w interakcje w zdaniu (np. mogą stać w różnej kolejności),
zadane przez nas odwzorowanie konstrukcji składniowych na formę logiczną nie będzie stanowić przepisu na jednoznaczną semantykę,
w szczególności nie będziemy na tym etapie rozwiązywać problemu niejednoznaczności zakresu operatorów nielokalnych.

Osobnym problemem jest reprezentacja nieciągłych operatorów nielokalnych,np. \emph{Każda strona ma inny kolor} i interakcji między operatorami nielokalnymi innych niż zagnieżdżanie się zakresów
np. w przypadku kwatyfikatorów rozgałęzionych w zdaniu Hintikki.

\textbf{Negację} traktujemy jako operator nielokalny mający jeden argument i nie wprowadzający zmiennej.

Spójniki ,,\textbf{jeśli \dots, to \dots}'' oraz~ ,,\textbf{jeśli \dots, \dots}'' użyte w znaczeniu logicznym uznajemy za operatory nielokalne.

% TODO for ENIAM:
% Uwaga zgodnie z formalizmem wprowadzonym w Walentym ,,jeśli'' wiąże zdanie podrzędne z orzeczeniem zdania nadrzędnego za pomocą
% roli tematycznej Condition. Trzeba będzie zbadać użycia roli Condition i określić czy są one uogólnienie intersektywne.
% Jeśli nie są trzeba zbadać w jaki sposób należy je przekształcić do postaci formuły logicznej.

% Rolę Manner też można traktować jako nielokalną.

\subsection{Zjawisko wyciągania kwantyfikatorów poza zakres operatora spowodowane koreferencją elementów argumentów operatora}
\begin{equation}
	\text{Jeśli rolnik ma krowę, to ją doi}
\end{equation}
W powyższym przykładzie ,,jeśli'' traktujemy jako logiczną implikację, ,,rolnik'' i ,,krowa'' są kwantyfikowane egzystencjalnie, a 
,,ją'' i ,,pro'' (podmiot domyślny czasownika ,,doi'') są kwantyfikowane koreferencyjnie ($\iota$). 
Aby rozwiązać koreferencję trzeba przenieść powyższe 4 kwantyfikatory ponad implikację co powoduje
zamianę kwantyfikatorów  ezgzystencjanych przy ,,rolniku'' i ,,krowie'' na uniwersalne. 

Analogiczne zjawisko będzie występować przy wszystkich operatorach nielokalnych mających więcej niż jeden argument.

\section{Niejawne argumenty semantyczne} % (fold)
{
\renewcommand\thesection{}
\renewcommand\thesubsection{}
\setcounter{subsection}{-1}
\subsection{Opis zjawiska} % (fold)
\label{sub:definicja_robocza2}
}
Niezbędne jest rozpoznanie i oznaczenie argumentów niejawnych w leksemach i wypracowanie skutecznej koncepcji ich opisu i przetwarzania.

Za niejawne argumenty semantyczne uznajemy te argumenty, które są niezbędne do ustalenia znaczenia wyrazu mającego ten argument.
Przykłady występowania wyrażeń z różnymi argumentami niejawnymi:
\begin{itemize}
	\item większy -- większy niż słoń \\ Znaczenia wyrazu ,,większy'' nie da się obliczyć bez argumentu ,,niż słoń''.
	\item Biegnie. -- On biegnie. -- Jaś biegnie. \\
	W pierwszym zdaniu inicjatora (\textsc{Initiator}) reprezentujemy oznaczając go niemym zaimmkiem (\textsc{pro}), w drugim jest on wyrażony za pomocą zaimka ,,on'', w trzecim jest wskazany jawnie jako ,,Jaś''.
	Aby obliczyć znaczenie \textsc{pro} i ,,on'' potrzebna jest wartość koreferencyjnego lub deiktycznego argumentu.
	\item Karol chce. -- Karol chce pracować.\\
	Brakującego argumentu składniowego czasownika ,,chcieć'' nie traktujemy jako niejawnego argumentu semantycznego.
\end{itemize}

% subsection definicja_robocza (end)
\subsection{Lista niejawnych argumentów semantycznych} % (fold)
\label{niejawne}
Po wstępnej analizie przyjmujemy istnienie sześciu typów niejawnych argumentów semantycznych:

\begin{enumerate}
\item Okazjonalne \\ Np. \emph{tu}.
\item Deiktyczne\\ Np. \emph{taki}.
\item Porównawcze \\ Leksemy z tym argumentem mogą przyjmować porównawczy argument przyimkowy.
\item Porządkowe \\ Np. \emph{kolejny}.
\item Koreferencyjne \\ Np. \emph{on}.
\item Relacyjne \\ Aby określić znaczenie znaczenie przymiotników i liczebników z argumentem relacyjnym (\emph{duży}, \emph{sporo}) trzeba wiedzieć, jaki predykat jest modyfikowany przez dane słowo.
\label{relac}
\end{enumerate}


\section{Znaczenie relacji składniowych} % (fold)

\subsection{Uogólniona intersektywność} % (fold)
\label{uog_interektywnosc}
Ważną cechą semantyczną, która przysługuje nie wszystkim leksemom, a pozwala na reprezentowanie ich wkładu znaczeniowego jako zawężającego znaczenie szerszej frazy, jest \emph{intersektywność}.
Tradycyjnie rozumiana intersektywność przypisywana jest przymiotnikom i oznacza, że denotacja frazy rzeczownikowej z przymiotnikiem intersektywnym jest podzbiorem denotacji frazy bez niego.

\begin{equation}
\text{Widzę czarnego kota.} \rightarrow \text{Widzę kota.}
\end{equation}


Rozumienie to można uogólnić na przysłówki i mówić o przysłówkach intersektywnych. Zbadania wymaga, w jakich innych częściach mowy lub pozycjach składniowych da się mówić o podziale na wyrażenia nieintersektywne i intersektywne w sensie uogólnionym -- wyrażenia takie powinny spełniać większość lub wszystkie testy:
\begin{test}
Zaczynając od korzenia drzewa rozbioru składniowego dodajemy kolejne węzły i sprawdzamy, czy ze znaczenia drzewa rozszerzonego wynika znaczenie drzewa mniejszego. Jeśli tak, dodawane węzły są intersektywne w sensie uogólnionym.
\end{test}
\begin{test}
Zaczynamy od liści i ucinamy je po jednym, sprawdzając, czy znaczenie zdania staje się bardziej ogólne. Jeśli tak, ucinane węzły są intersektywne w sensie uogólnionym.
\end{test}
\begin{test}
Porównujemy typy semantyczne wyrażenia złożonego i tego samego wyrażenia bez jednego z podrzędników. Jeżeli pierwszy z tych typów jest podtypem drugiego, podrzędnik jest intersektywny w sensie uogólnionym.
\end{test}
\begin{equation}\label{biegnie}
	\text{Jaś biegnie.} \rightarrow \text{Biegnie.}
\end{equation}
\begin{equation}\label{jablka}
	\text{cztery jabłka} \subset \text{jabłka}
\end{equation}
\begin{equation}\label{fontanna}
\text{Mieszkam nieopodal fontanny.} \stackrel{?}{\rightarrow} \text{Mieszkam nieopodal.}
\end{equation}

Przykłady \ref{biegnie} i \ref{fontanna} ilustrują wynikanie zdań, przykład \ref{jablka} mówi o zawieraniu się denotacji.
Ostatni przykład pokazuje, że rozstrzygnięcia wymaga problem relacji znaczenia jednobrzmiących leksemów przysłówkowych i przyimkowych. Pożądaną opcją wymagającą zweryfikowania jest identyczność znaczeniowa wszystkich lub większości takich par \emph{modulo} argument niejawny przysłówka.

\subsection{Rozbiór składniowy w podejściu semantycznym}

Do wyliczenia reprezentacji semantycznej zdania potrzebne są dane wejściowe -- rozbiór składniowy w podejściu semantycznym. Otrzymanie takiego rozbioru wymaga przekształcenia bardziej niskopoziomowego rozbioru w podejściu morfoskładniowym z użyciem informacji o znaczeniu wzajemnych rzędników. Przekształcenie jest potrzebne, gdy bezpośredni podrzędnik węzła $n$ nie spełnia jego ograniczeń selekcyjnych, ale podrzędnik następnego rzędu (wnuczek $n$) je spełnia. Spełnianie ograniczeń selekcyjnych przez leksem polega na tym, że jego typ semantyczny jest podtypem jednego z typów wskazanych w ograniczeniach selekcyjnych. W takim przypadku podrzędnik pierwszego i drugiego rzędu są zamieniane miejscami w drzewie rozbioru.

Typowym przykładem zadania, w którym opisana wyżej procedura jest nietrywialna, jest rozpoznanie, czy wystąpienie przyimka przewidzianego w ramie składniowej nadrzędnego czasownika jest realizacją tej ramy, jak ,,w głowie'', czy jest to przyimek semantyczny, jak w wyrażeniu ,,w samolocie'' w przykładzie:

\begin{equation}
	\text{\textbf{W samolocie} wielu osobom kręci się \textbf{w głowie}.}
\end{equation}
W rozwijanej dla parsera \eniam{} odmianie podejścia semantycznego
o tym, czy bezpośrednim podrzędnikiem czasownika jest przyimek czy rzeczownik,
decyduje ostatecznie to, czy preferencje selekcyjne argumentu ramy uzgadniają się z typem semantycznym rzeczownika czy frazy przyimkowej.
Podobny problem dotyczy liczebników i rzeczowników w użyciu pojemnikowym (patrz sekcja \ref{pojemniki})

\subsection{Zależność roli tematycznej od znaczenia rzędników}

Podrzędniki dzielimy na argumenty i modyfikatory rozumiane semantycznie. Rola tematyczna podrzędnika zależy za każdym razem od 
jego typu zależności składniowej, a ponadto
\begin{itemize}
	\item w przypadku argumentu -- od znaczenia nadrzędnika.
Przykładowo, odniesienie podmiotu czasownika \emph{czuć} to \experiencer{} zdarzenia, czasownika \emph{myśleć} -- to \agent{} zdarzenia, a jego dopełnienia -- to \theme{} zdarzenia.
\item w przypadku modyfikatora -- od jego znaczenia (podrzędnika).
Przykładem jest modyfikator rzeczownikowy w narzędniku. Taki modyfikator może wprowadzać role \textsc{Instrument}, \textsc{Time} lub \textsc{Manner} w zależności od jego znaczenia:
\end{itemize}
\begin{equation}
	\text{Naprawiła szafę \textbf{młotkiem}.} 
\end{equation}
\begin{equation}
	\text{Naprawiła szafę \textbf{wieczorem}.} 
\end{equation}
\begin{equation}
	\text{Naprawiła szafę \textbf{własnym sposobem}.} 
\end{equation}

Oddzielnej uwagi wymagają też rzeczownikowe frazy wolne w celowniku,
przymiotniki, które mają rolę tematyczną inną niż \textsc{Attribute}
i przysłówki, które mają rolę tematyczną inną niż \textsc{Manner}.

\section{Nieintersektywność}

Modyfikatory nieintersektywne mają nielokalny wkład znaczeniowy. Przykładami modyfikatorów nieintersektywnych są
\begin{itemize}
\item ,,były''
\begin{equation}
	\text{były prezydent}
\end{equation}
\item ,,fałszywy''
\begin{equation}
	\text{fałszywy brylant}
\end{equation}
\item ,,co najmniej''
\begin{equation}
	\text{co najmniej cztery dni}
\end{equation}
\end{itemize}

\emph{Były} i \emph{fałszywy} są funktorami, które tworzą wraz ze znaczeniem nadrzędnika pojęcie na tyle zmodyfikowane, że jego denotacja nie musi lub nie może mieścić się w przecięciu oryginalnych denotacji.
Funktor \emph{co najmniej} bierze za argument nazwę liczby bądź miary i zwraca nowe określenie liczby bądź miary, które nie jest zawężeniem określenia oryginalnego.


% \subsection{Nieintersektywność II typu}
% Modyfikuje aspekt znaczenia swojego podrzędnika, generując obiekt innego typu semantycznego niż podrzędnik.
% To, jaki to jest aspekt, jest zdeterminowane znaczeniem funktora.
% \textcolor{red}{(Obie rzeczy są naturalnym zachowaniem nadrzędnika.)}
% Leksem funktora może mieć kilka znaczeń dotyczących różnych aspektów.

% Przykłady:

% \begin{itemize}
% \item przysłówek ,,ledwo'' w znaczeniu czasowym
% \begin{equation}
% 	\text{Słońce ledwo wzeszło}
% \end{equation}
% \begin{equation}
% 	\text{Ledwo zabrał się do pracy, zadzwonił telefon.}
% \end{equation}
% \emph{Ledwo} traktujemy tu jak funktor biorący zdarzenie i generujący czas chwilę po nim. W funkcji 
% spójnika podrzędnego, czas ten określa czas zdarzenia ze zdania nadrzędnego, w funkcji przysłówka 
% czas ten określa czas zdarzenia wynikającego z kontekstu, co można wyrazić jako pro-zdarzenie.

% \item spójnik ,,gdy''
% \begin{equation}
% 	\text{Wszedł, gdy go wpuścili.}
% \end{equation}
% \emph{Gdy} bierze sytuację (zdarzenie) i generuje jej czas.

% \item przyimek ,,pod''
% \begin{equation}
% 	\text{pod stołem}
% \end{equation}
% Funktor biorący obiekt i zwracający miejsce oraz wskazujący, że jest to miejce, gdzie znajduje się referent nadrzędnika.
% \begin{equation}
% 	\text{pod stół}
% \end{equation}
% Funktor biorący obiekt i zwracający miejsce oraz wskazujący, że jest to miejce, do którego zmierza się referent nadrzędnika.
% \item ponieważ
% \begin{equation}
% 	\text{Pada, ponieważ spadło ciśnienie}
% \end{equation}
% Funktor biorący sytuację i generujący obiekt o typie semantycznym \textsf{CZEMU}.
% \end{itemize}

\section{Relacyjność}
{
\renewcommand\thesection{}
\renewcommand\thesubsection{}
\setcounter{subsection}{-1}
\subsection{Opis zjawiska} % (fold)
\label{sub:opis_zjawiska1}
}

Przez relacyjność modyfikatora rozumiemy tradycyjnie tę jego cechę, że jego znaczenie zależy w pewnym ustalonym zakresie od znaczenia jego nadrzędnika. Dogodnym sposobem jej reprezentowania jest relacyjny argument niejawny (punkt \ref{relac} na liście niejawnych argumentów w części \ref{niejawne}).

\begin{itemize}
\item ,,ledwo'' w znaczeniu `prawie nie'
\begin{equation}
	\text{Ledwo zdał egzamin}.
\end{equation}
\textcolor{red}{Mamy tu funktor który modyfikuje pojęcie będące jego składniowym nadrzędnikiem (analogicznie jak leksem ,,były'').}
Funktor \emph{ledwo} ma domyślny argument -- referencyjny porządek, zależny od nadrzędnika.
\item niedużo
\begin{equation}
	\text{niedużo mleka}
\end{equation}
\item dużo
\begin{equation}
	\text{dużo złota}
\end{equation}
Te typowe przykłady przymiotników relacyjnych mają znaczenie jednoznacznie dodeterminowane znaczeniem nadrzędnika. 
,,Dużo złota'' to inna ilość niż ,,dużo węgla''.
\item już
\begin{equation}
	\text{już za daleko}
\end{equation}
\item jeszcze
\begin{equation}
	\text{jeszcze we wtorek}
\end{equation}
\item aż
\begin{equation}
	\text{aż cztery}
\end{equation}
Te funktory mają domyśny argument wskazujący skalę jako czasową, przestrzenną itp. ale niedostarczony przez sam nadrzędnik.
Nadrzędnik ogranicza ich znaczenie, niekoniecznie je determinując. 
\color{red}{(Ew. skala jest jednoznacznie przypisana do znaczenia leksemu
a ich wielość wynika z wielości znaczeń leksemu)
}
\end{itemize}

\section{Przyrematyczność}
{
\renewcommand\thesection{}
\renewcommand\thesubsection{}
\setcounter{subsection}{-1}
\subsection{Opis zjawiska} % (fold)
\label{sub:opis_zjawiska2}
}
Przyrematyczne są te partykuły, które obejmują swoim zakresem zawsze remat zdania, choć może to być niewidoczne na poziomie morfoskładni i szyku, często swobodnego.
\begin{itemize}
\item chyba
\begin{equation}
	\text{\textbf{Oslo} jest chyba stolicą Norwegii.}
\end{equation}
\begin{equation}
	\text{Oslo jest chyba stolicą \textbf{Norwegii}.}
\end{equation}
\item także
\begin{equation}
	\text{Pójdziemy także do kina.}
\end{equation}
\item również
\begin{equation}
	\text{Pojedziemy również do Stanów.}
\end{equation}
\item też
\begin{equation}
	\text{Polecimy też do Berlina.}
\end{equation}
\item nawet
\begin{equation}
	\text{Popłyniemy nawet do Oslo.}
\end{equation}


\end{itemize}


\section{Operatory nieprawdomówne (\emph{nonveridical})}
{
\renewcommand\thesection{}
\renewcommand\thesubsection{}
\setcounter{subsection}{-1}
\subsection{Opis zjawiska} % (fold)
\label{sub:opis_zjawiska3}
}

Szereg wyrażeń języka można definiować jako funktory od argumentów zdaniowych, w tym przysłówki, partykuły i czasowniki przyjmujące dopełnienie zdaniowe. Kluczowym podziałem przy reprezentowaniu ich znaczenia jest to, czy prawdziwość frazy pociąga prawdziwość argumentu (\emph{veridicality}) czy nie (\emph{nonveridicality}).
Przykładem jest większość partykuł i wyrażeń przyimkowych wyrażających modalność epistemiczną, jak
\begin{itemize}
\item ,,chyba''
\begin{equation}
	\text{Chyba idzie zima.}
\end{equation}
\item ,,pewnie''
\begin{equation}
	\text{Pewnie spadnie śnieg.}
\end{equation}
\item ,,może''
\begin{equation}
	\text{Może będzie przymrozek.}
\end{equation}
\item ,,być może''
\begin{equation}
	\text{Być może spadnie grad.}
\end{equation}
\item \textcolor{red}{,,na pewno''}
\begin{equation}
	\text{\textcolor{red}{Na pewno spadnie grad.}}
\end{equation}
\item \textcolor{red}{,,oczywiście ''}
\begin{equation}
	\text{\textcolor{red}{Spadnie oczywiście grad.}}
\end{equation}
\item ,,prawdopodobnie''
\begin{equation}
	\text{Prawdopodobnie będzie ślisko.}
\end{equation}
\end{itemize}
które stanowią ocenę stopnia uzasadnienia sądu. \emph{Prawdopodobnie} może też być partykułą modalną aletyczną wyrażającą możliwość i w tym znaczeniu też jest operatorem nieprawdomównym.

Tryb rozkazujący także jest nieprawdomówny, choć zależy to dodatkowo od znaczenia czasownika: Ze zdania ,,Podaj mi sól'' wynika pragmatycznie, że teraz odbiorca nie podaje soli, ale ze zdania ,,Bądź punktualny'' nie wynika, że odbiorca punktualny nie jest.

\subsection{Niefaktywność} % (fold)
Najlepiej znanym podtypem nieprawdomówności jest niefaktywność, tradycyjnie przypisywana leksemom czasownikowym. W naszej analizie faktywność i niefaktywność przysługuje pozycji w ramie walencyjnej, co uwzględnia odmienne zachowanie podrzędników tego samego leksemu wprowadzanych przez różne spójniki podrzędne.
\begin{itemize}
	\item ,,roić sobie, że\_''
	\begin{equation}
		\text{Anna roi sobie, że pada.} 
	\end{equation}
	\item ,,chcieć, żeby\_''
	\begin{equation}
		\text{Anna chce, żeby nie padało.} 
	\end{equation}
\end{itemize}

Oddzielnej analizy wymaga prawdomówność spójników leksykalnych i faktywność pozycji wypełnionych przez frazy czasownikowe niefinitywne -- bezokolicznikowe, imiesłowowe, odsłownikowe, bezosobnikowe, którym nie przysługuje np. czas gramatyczny.
	
% subsection faktywność (end)

\section{Cechy składniowo-semantyczne jednostek nieopisanych w zasobach} % (fold)
{
\renewcommand\thesection{}
\renewcommand\thesubsection{}
\setcounter{subsection}{-1}
\subsection{Opis problemu} % (fold)
\label{sub:opis_problemu}
}
% subsection opis_problemu (end)
Słownik \emph{Walenty}, który jest podstawowym źródłem sformalizowanej informacji o cechach składniowo-semantycznych jednostek współczesnego języka polskiego, nie obejmuje niektórych klas gramatycznych lub obejmuje je w niedostatecznym stopniu.
Według wstępnego rozeznania wiele z brakujących leksemów rzeczownikowych, przymiotnikowych, przysłówkowych i przyimkowych wnosi wkład znaczeniowy lokalnie, a więc w sposób najbardziej podstawowy. Niezbędny jest jednak ich opis składniowo-semantyczny pozwalający przypisać im odpowiednią rolę tematyczną.
Dalsze potrzeby zasobowe obejmują typ semantyczny leksemów nieopisanych w Słowosieci.

\subsection{Lista cech składniowo-semantycznych} % (fold)
\label{sub:cechy_ala_Walenty}

% subsection cechy_ala_Walenty (end)
Docelowy opis cech składniowo-semantycznych obejmuje następujące cechy:

\begin{enumerate}
\item Rola tematyczna łącząca wyraz z nadrzędnikiem, jeśli leksem podrzędnika ma ją do siebie przypisaną oraz jest uogólnienie intersektywny
\item Struktura składniowa argumentów leksemu (rama składniowa), np. dla przyimka będzie to
\begin{enumerate}
\item Liczba argumentów jawnych przyimka (por. \emph{między\_a}, \emph{między\_i})
\item Okalającość przyimka (por. \emph{między\_a})
\item Pozycyjność przyimka (por. postpozycyjne  \emph{(100 lat) temu})
\item Przypadek wymagany przez przyimek
\end{enumerate}

\item Preferencje selekcyjne i role tematyczne argumentów intersektywnych w sposób uogólniony
\item Typ semantyczny leksemu, jeśli leksem nie występuje w Słowosieci. Dla przyimków i spójników podrzędnych jest to typ semantyczny frazy, których są elementem głównym.
\end{enumerate}

\section{Oznaczanie miary i liczebności referenta} % (fold)
{
\renewcommand\thesection{}
\renewcommand\thesubsection{}
\setcounter{subsection}{-1}
\subsection{Opis zjawiska} % (fold)
\label{sub:opis_zjawiska4}
}
% subsection opis_zjawiska (end)

Szczególnej uwagi wymagają wyrażenia, których typem semantycznym jest \textsf{ILOŚĆ} dla wyrażeń policzalnych lub \textsf{MIARA} dla wyrażeń niepoliczalnych. Ich rolą tematyczną jest odpowiednio \textsc{count} lub \textsc{measure}.


\subsection{Użycie pojemnikowe} % (fold)
\label{pojemniki}
Użycie pojemnikowe rzeczownika to takie, w którym rzeczownik odnosi do miary referenta swojego podrzędnika rzeczownikowego w dopełniaczu w morfoskładniowym (niesemantycznym) podejściu do składni. Przykładem jest ,,łyżka'' w wyrażeniu ,,łyżka cukru''.
Żeby otrzymać rozkład zdania zawierającego rzeczowniki w użyciach pojemnikowych w składni semantycznej, jak Universal Dependencies, traktujemy pojemniki jak podrzędniki swoich tradycyjnych składniowych nadrzędników. Do tego celu niezbędne jest oznaczenie pojemników w języku polskim.

% subsubsection pojemniki (end)

\subsection{Podział liczebników ze względu na dokładność} % (fold)

Liczebniki niewłaściwe (\emph{trochę}, \emph{sporo}) i większość rzeczowników w użyciu oznaczającym miarę nie podaje dokładnej wartości wielkości, którą opisuje. Żeby reprezentować ich znaczenia, trzeba je podzielić ze względu na dokładność, opisać zakres niedokładności i zaproponować spójną koncepcję reprezentacji znaczeń niedokładnych.

% subsubsection podział_liczebników_ze_względu_na_dokładność (end)

\subsection{Miara referentów niepoliczalnych} % (fold)

Do zdefiniowania, które leksemy wprowadzają w języku polskim relację \textsc{Measure} (np. \emph{niedużo}), konieczne jest ich oznaczenie w zasobie typu słownikowego.

% subsubsection miara_referentów_niepoliczalnych (end)


\section{Niedenotatywne funkcje języka} % (fold)
{
\renewcommand\thesection{}
\renewcommand\thesubsection{}
\setcounter{subsection}{-1}
\subsection{Opis zjawiska} % (fold)
\label{sub:opis_zjawiska2}
}
% subsection opis_zjawiska (end)
% Pierwotna nazwa podrozdziału: Metatekstowość
Szczególnej uwagi wymagają wszystkie leksemy i konstrukcje pełniące funkcję fatyczną (\emph{halo}), konatywną (np. tryb rozkazujący), emotywną (np. ekspresywizmy) lub metatekstową (np. \emph{że tak powiem}).
Do opisu ostatniej należą złożone zagadnienia znaczenia spójników, wyrażeń przyrematycznych i przytematycznych.

Przykłady wyrażeń pełniących pierwotnie funkcję niedenotatywną:
\begin{itemize}
	\item ,,halo''
	\item ,,brawo''
	\item ,,brr''
	\item ,,miau''
	\item ,,że tak powiem''
\end{itemize}

\subsection{Metatekstowość}
Wyrażenia metatekstowe, przede wszystkim partykuły i spójniki, odnoszą do fragmentu wypowiedzenia lub wyrażają komentarz na jego temat. Przykładem jest
\begin{itemize}
\item ,,więc''
\begin{equation}
	\text{Padało, więc wyszliśmy tańczyć.} 
\end{equation}
\end{itemize}
Spójnik \emph{więc} informuje, że zachodzi związek quasi-przyczynowy między odniesieniami koordynowanych członów.

\section{Podmiot epistemiczny}
Niektóre wyrażenia i kategorie gramatyczne odnoszą do stanów mentalnych osoby niekoniecznie tożsamej z nadawcą. Należą tu
\begin{itemize}

\item partykuły modalne

	\begin{itemize}
	\item ,,chyba''
	\begin{equation}
		\text{Piotr mówi, że chyba skończyli już układać.}
	\end{equation}
	\item ,,pewnie''
	\begin{equation}
		\text{Piotr mówi, że pewnie skończyli już układać.}
	\end{equation}
	W obu przykładach osoba referencyjna potwierdziłaby modyfikowany sąd, gdyby wiedziała, że pewna brakująca przesłanka jest prawdziwa, ale osoba tą nie jest nadawca zdania.
	\end{itemize}

\item partykuły miriatywne

	\begin{itemize}
	\item ,,nawet''
	\begin{equation}
		\text{Anna narzekała, że nie przyszedł nawet przywitać się z rodzeństwem.}
	\end{equation}
	Modyfikowany sąd dziwi podmiot epistemiczny, który ponownie nie musi być tożsamy z nadawcą.
	\end{itemize}

\item tryb rozkazujący, ponieważ wskazuje na intencje nadawcy
\item spójniki przeciwstawne i wprowadzające okoliczniki przyzwolenia
	\begin{itemize}
		\item ,,chociaż''
		\begin{equation}
			\text{Maria powiedziała, że wyszli tańczyć, chociać padał deszcz.}
		\end{equation}
		\item ,,ale''
		\begin{equation}
			\text{Maria powiedziała, że padał deszcz, ale oni tańczyli.}
		\end{equation}
		\item ,,a''
		\begin{equation}
			\text{Maria powiedziała, że padał deszcz, a oni tańczyli.}
		\end{equation}
	\end{itemize}

\end{itemize}


\end{document}