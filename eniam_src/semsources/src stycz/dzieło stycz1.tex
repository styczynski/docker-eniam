% \RequirePackage[hyphens]{url}
\documentclass[a4paper, 12pt]{article}
% \documentclass[preprint, a4paper, 11pt]{article}

\usepackage[T1]{fontenc}
\usepackage[english, polish]{babel}
\usepackage[utf8]{inputenc}

\usepackage{fullpage}
\usepackage{amsmath}
\usepackage{amsthm}
\usepackage{latexsym}
\usepackage{dsfont}
\theoremstyle{remark}
\newtheorem{test}{Test}

\usepackage{multicol}
\usepackage{graphicx}
\usepackage{rotating}

\usepackage[hidelinks]{hyperref}
% \urlstyle{sf}
\expandafter\def\expandafter\UrlBreaks\expandafter{\UrlBreaks% save the current one
 \do\\~} % this stops urls from sticking out of the damn screen and the paper. I hate this arrggh.


 \usepackage{tocloft}
 \cftsetindents{section}{0pt}{2.25em}
 \cftsetindents{subsection}{0pt}{2.25em}
 \cftsetindents{paragraph}{0pt}{2.25em}
% \addtocontents{toc}{\cftpagenumbersoff{section}}

\usepackage{xcolor}

\newcommand{\eniam}{\textsc{Eniam}}
\newcommand{\experiencer}{\textsc{Experiencer}}
\newcommand{\agent}{\textsc{Agent}}
\newcommand{\theme}{\textsc{Theme}}
\frenchspacing

\title{
\normalsize
\setlength{\topsep}{0pt} 
\noindent
Zleceniodawca: IPI PAN \hfill Dzieło do umowy z dnia: 28.11.2016 \\
Zleceniobiorca: Jakub Kozakoszczak \hfill Termin zakończenia pracy: 15.12.2016 \\
\rule{\linewidth}{2pt} \\[6pt] 
\huge Reprezentacja semantyczna \\ przyimków i ich modyfikatorów \\ dla parsera \eniam \\[-2pt]}
\author{\Large Wojciech Jaworski, Jakub Kozakoszczak\\[6pt]
\hspace{-5pt}\rule{\linewidth}{2pt} }
\date{}
% \date{\normalsize Date Submitted}

\begin{document}
% latex \let\install=y\input{mathstyle.dtx}

\maketitle

\vspace{-26pt}
\begin{description}
	\setlength{\itemsep}{4pt} 
	\setlength{\parskip}{0pt} 
	\item[Zrealizowane w ramach projektu:] „CLARIN -- Polskie wspólne zasoby językowe i infrastruktura technologiczna”
	\item[Tytuł pracy zamówionej:] ,,Koncepcja reprezentacji semantycznej przyimków uwzględniająca ich modyfikowalność''
	\item[Adres dzieła:] \url{http://wiki.nlp.ipipan.waw.pl/clarin/Parser%20kategorialny?action=AttachFile&do=view
	&target=JK_reprezentacja_przyimkow.pdf}
	\item[Opracowanie dokumentu:] Jakub Kozakoszczak
\end{description}

\vspace{6pt}
\hrule
\vspace{-14pt}
\renewcommand\contentsname{\normalsize{}}
\tableofcontents
\vspace{20pt}
\hrule
\newpage

\section{Koncepcja} % (fold)

\subsection{Założenia ogólne} % (fold)
\label{sub:zalożenia_oglne}


Proponowana reprezentacja rozwiązuje problem modyfikowalności przyimków przez przypisywanie im zmiennej odniesieniowej i nadanie roli tematycznej.
Z uwagi na występowanie modyfikatorów dla przyimków semantycznych (np. ,,dość głęboko w szafie'' , ,,5 km od domu'' (dodatkowy argument ,,5 km'') ,,w szczególnym związku z'' (modyfikacja przyimka złożonego), przyimki te nie mogą być reprezentowane jako predykaty wiążące identyfikator odniesienia swojego podrzędnika z identyfikatorem odniesienia nadrzędnika.
Dlatego z każdym przyimkiem semantycznym wiążemy zmienną reprezentującą jego własne odniesienie i wiążemy ją z zmienną reprezentującą odniesienie nadrzędnika jedną z relacji oznaczających role tematyczne.

Drugim problemem jest wieloznaczność przyimków. Dla każdego przyimka identyfikujemy jego sensy, każdy traktujemy jak oddzielny przyimek (oddzielną jednostkę leksykalną), zaliczamy każdy do klasy podobnych sensów i wyczerpująco opisujemy uwzględniając dla każdego sensu
rolę tematyczną, 
typ semantyczny całej frazy przyimkowej
i preferencje selekcyjne przyimka.
Preferencje selekcyjne przyimka semantycznego dotyczą nie tylko frazy rzeczownikowej zależnej, ale są też nakładane na jego nadrzędnik, zwykle czasownik lub rzeczownik. 
% Ew. daj przykład "pod parasolem" vs. "pod pomnikiem"
% subsection założenia_oglne (end)

Tak, jak fizyce korzysta się z modeli, o których wiadomo, że nie opisują wszystkiego i są na nie kontrprzykłady, podobnie przedstawiony tu model semantyki języka nie opisuje wszystkich zjawisk, jakie występują w języku, a w założeniu ma opisywać to, co powszechnie występuje w tekstach języka polskiego. Model ten ma dwa poziomy. Jeden znich wskazuje podział przyimków na sensy i przypiuje im role tematyczne oraz preferencje selekcyjne. Drugi z nich stanowi uzasadnienie poprawności podziału w pierwszym poprzez formalne zdefiniowanie sensów przyimków w założonej ,,ontologii''. Pierwszy poziom jest bardziej użytkowy a drugi podaje model rzeczywistości.

\subsection{Znaczenie przyimka semantycznego} % (fold)
\label{sub:znaczenie_przyimka_semantycznego}

Zgodnie z założeniami reprezentacji wypracowanej w ramach projektu \textsc{clarin 1} reprezentujemy znaczenie przyimka jako jego algorytm znaczeniowy i odniesienie.
Odniesieniami przyimków są relacje, np. jedna z możliwych relacji dla przyimka ,,blisko'' jest zdefiniowana jako

\begin{equation}
v(\text{blisko})=\{(X,Y):\text{dist}(X,Y) \leq e\}
\end{equation}

gdzie e=0,01 a ,,dist'' jest to polimorficza metryka zdefiniowana jak poniżej:
\begin{enumerate}
\item gdy $X$ i $Y$ są punktami w przestrzeni liniowej, jest to metryka Euklidesowa,
\item gdy $X$ i $Y$ są podzbiorami przestrzeni liniowej, jest to metryka Hausdorffa nad metryką Euklidesową.
W szczególności, gdy $X$ i $Y$ są liczbami, wtedy
\begin{equation}
 \text{dist}(X,Y) = |X - Y|,
\end{equation}
\item gdy $X$ i $Y$ są parametryzowane przez czas, czyli traktujemy je jak funkcje takie, że $X(t)$ jest punktem lub podzbiorem przestrzeni liniowej, wtedy
\begin{equation}
 \text{dist}(X,Y) = sup_t(\text{dist}(X(t),Y(t)))
\end{equation}
\end{enumerate}

Algorytm znaczeniowy ,,blisko'' zwraca \texttt{true} dla tych i tylko tych obiektów, które są relacją definiowalną w postaci 
\begin{equation}\label{alg_dla_blisko}
R=\{(X,Y):dist(X,Y) \leq e\}
\end{equation}
 gdzie ,,dist'' jest metryką, a $e$ liczbą rzeczywistą większą od 0.

% subsection znaczenie_przyimka_semantycznego (end)

\subsection{Człony relacji oznaczanej przez przyimek} % (fold)
\label{sub:czlony_relacji_oznaczanej_przez_przyimek}

\begin{itemize}
\item Przez $B$ oznaczymy odniesienie \textbf{podrzędnika przyimka}.
\item Przez $P$ oznaczymy odniesienie \textbf{wyrażenia przyimkowego przestrzennego}.
\item Przez $R$ oznaczymy odniesienie \textbf{przyimka}.
\item Przez $A$ oznaczymy odniesienie \textbf{nadrzędnika wyrażenia przyimkowego}.
\item W przypadku określeń czasu nadrzędnikiem przyimka jest zazwyczaj nazwa zdarzenia lub stanu.
\item W przypadku określeń czasu podrzędnikiem przyimka jest zazwyczaj nazwa przedziału czasowego albo nazwa zdarzenia.
\item Odniesieniami \textbf{nazw przedziałów czasowych }są interwały (zaczepione w ustalonych punktach na osi czasu). 
\item Odniesieniami \textbf{nazw zdarzeń }są zdarzenia.
\item Odniesieniami \textbf{wyrażeń przyimkowych będących określeniami czasu }są interwały.
\item Przez $.beg$ będziemy oznaczać początek interwału, przez $.end$ koniec, a przez $.len$ długość.
\item Przez $.time$ będziemy oznaczać czas zdarzenia/stanu, przez $.loc$ będziemy oznaczać parametryzowane przez czas miejsce zdarzenia/stanu/przedmiotu.
\end{itemize}

% subsection czlony_relacji_oznaczanej_przez_przyimek (end)

\subsection{Role tematyczne wyrażeń przyimkowych} % (fold)
\label{sub:role_tematyczne_przyimkw_semantycznych}

Role tematyczne dla przyimków definiujemy jako trójargumentowe relacje, których pierwszym argumentem jest odniesienie nadrzędnika przyimka, drugim jest odniesienie przyimka, a trzecim odniesienie podrzędnika przyimka.

We wzorach poniżej relacja $X\sqsubset Y$ jest uogólnieniem zawierania zbiorów zdefiniowanym jak poniżej:
\begin{itemize}
	\item gdy $X$ i $Y$ są punktami w przestrzeni liniowej, wtedy
	\begin{equation}
	X\sqsubset Y \text{\ wtw\ } X \text{\ jest podzbiorem otoczki wypukłej\ }Y
	\end{equation}
	\item gdy $X$ i $Y$ są parametryzowane przez czas, wtedy
	\begin{equation}
	X\sqsubset Y \text{\ wtw\ } \forall t: X(t) \text{\ jest podzbiorem otoczki wypukłej\ }Y(t))
	\end{equation}
\end{itemize}


\paragraph{Definicje ról tematycznych} % (fold)
\label{par:role_tematyczne}
\begin{enumerate}
\item \textbf{Loc}
\begin{equation}
\text{Loc}(A,R,B) \equiv \exists P: \forall t: A.loc(t) \sqsubset P(t) \wedge (P,B.loc) \in R
\end{equation}
\item \textbf{LocGoal}
\begin{align}
\nonumber\text{LocGoal}(A,R,B) \equiv \exists P: A.loc(A.end) \sqsubset P(A.end) \\ \wedge A.loc(A.beg) \not\sqsubset P(A.beg) \wedge (P,B.loc) \in R
\end{align}
\item \textbf{LocSrc}
\begin{align}
\nonumber\text{LocSrc}(A,R,B) \equiv \exists P: A.loc(A.beg) \sqsubset P(A.beg) \\ \wedge A.loc(A.end) \not\sqsubset P(A.end) \wedge (P,B.loc) \in R
\end{align}
\item \textbf{Time}
Gdy $B$ jest interwałem
\begin{equation}
\text{Time}(A,R,B) \equiv (A.time,B) \in R
\end{equation}
% \item \textbf{Time}
wpp.
\begin{equation}
\text{Time}(A,R,B) \equiv (A.time,B.time) \in R
\end{equation}
\item \textbf{Count}
\begin{equation}
\text{Count}(A,R,B) \equiv (A.count,B) \in R
\end{equation}
\begin{equation}
\text{Count}(A,B) \equiv A.count = B
\end{equation}
\item \textbf{Order}
\begin{align}
\nonumber\text{Order}(A,R,B) \equiv
context.order \text{ jest porządkiem indukowanym} \\ \text{przez algorytmy znaczeniowe A i B} \wedge (A,B) \in R
\end{align}
\item \textbf{Cond}
\begin{align}
\nonumber\text{Cond}(A,R,B) \equiv context.order \text{ jest porządkiem oddającym} \\ \text{zależności przyczynowo-skutkowe} \wedge (A,B) \in R
\end{align}
\end{enumerate}
% \begin{equation}
% #Order(A,R,B) \equiv \exists P: A.order=porządek indukowany przez P \wedge R(,B) \in R
% \end{equation}
% paragraph role_tematyczne (end)

% subsection role_tematyczne_przyimkw_semantycznych (end)

Np. dla zdania ,,Strzał padł koło południa'' mamy $\text{type}(A,\text{paść})$, 
$\text{type}(R,\text{koło}_1)$, \newline $\text{type}(B,\text{południe})$ oraz wiążącą je rolę tematyczną $\text{Time}(A,R,B)$.
Zadanie opisania semantyki przyimków sprowadza się teraz do wskazania ich (przyimków) 
\begin{enumerate}
\item lematu i schematu składniowego, 
\item preferencji selekcyjnych nałożonych na nadrzędnik, 
\item preferencji selekcyjnych nałożonych na podrzędnik, 
\item roli tematycznej, 
\item sensu przyimka -- nazwy algorytmu znaczeniowego,
\item ewentualnego numeru sensu w istniejących słownikach języka polskiego,
\item przykładu
\end{enumerate}
Oprócz tego potrzebujemy wskazać sensy identyczne oraz zdefiniować poszczególne algorytmy znaczeniowe.

Przyjmujemy, że przyimki mogą mieć wiele sensów.
Oznaczamy je poprzez dodawanie do lematów przyimków kolejnych liczb naturalnych. 
Sensy różnych przyimków mogą być identyczne. Mają wtedy ten sam algorytm znaczeniowy, jak np. w przypadku bliskoznacznych przyimków przestrzennych
koło$_1$ = blisko$_1$ = około$_1$ = niedaleko$_1$ = opodal$_1$ = pod$_2$ = obok$_1$.
Algorytmy znaczeniowe przyimków definiujemy tak, by były niezależne od ról tematycznych. Przykładowa definicja algorytmu znaczeniowego przyimka ,,blisko'' podana jest powyżej \hyperref[alg_dla_blisko]{(definicja \ref*{alg_dla_blisko})}, kolejne dwie dla ,,w$_1$'' (zarazem ,,do$_1$'', ,,z$_1$'' i ,,podczas$_1$'') i ,,pod$_1$'' podane są poniżej:

\begin{equation}
	\text{w}_1(R) = true \text{\ wtw R jest relacją definiowalną w postaci\ }  R = \{(X,Y): X \sqsubset Y\}
\end{equation}
\begin{align} 
\nonumber \text{pod}_1(R) = true \text{\ wtw R jest relacją definiowalną w postaci:}\\
 \{(X,Y) \in \textsc{miejsce} \times \textsc{miejsce}: X.height < Y.height\}\\
 \nonumber \cup \{(X,Y) \in \text{I\!R} \times \text{I\!R}: X < Y\} 
 \cup \{(X,Y): X \prec Y\},
\end{align}
 gdzie $\prec\;=context.order$.
   \newline Jeśli X i Y są parametryzowane przez czas, to
\begin{equation}
	R=\{(X,Y)\!:\ \forall t\!: (X(t),Y(t)) \in R\}
\end{equation}

\section{Przyimki polskie} % (fold)
\label{sec:przyimki_polskie}
% Wersja z typami semantycznymi:

% \begin{sidewaystable}
% \centering
% {% \begin{tabular}{|p{2.5cm}|p{2.5cm}|p{2.5cm}|p{2.5cm}|p{1.5cm}|p{2cm}|p{0.5cm}|p{5cm}|} \hline lemat i~składnia & pref. sel. na nadrz. & typ semantyczny & pref. sel. na podrz. & rola tematyczna & sens (algorytm) & nu\-mer w SJP & przykład \\ \hline koło:gen &  & CZAS & CZAS & Time & koło$_1$ & 2 & Zadzwoniła koło szóstej.\\ około:gen &  & CZAS & CZAS & Time & około$_1$ & & \\ blisko:gen &  & POŁOŻENIE & PRZED. FIZ. & Loc & blisko$_1$ & a & Mieszka blisko stacji metra. \\ koło:gen &  & POŁOŻENIE & PRZED. FIZ. & Loc & koło$_1$ &  1& \\ niedaleko:gen &  & POŁOŻENIE & PRZED. FIZ. & Loc & niedaleko$_1$ & & Mieszkali niedaleko parku. \\ opodal:gen &  & POŁOŻENIE & PRZED. FIZ. & Loc & opodal$_1$ & & Konie pasły się opodal lasu. \\ pod:inst &  & ? & PRZED. FIZ. & Loc & pod$_2$ &  3& \\ obok:gen &  & POŁOŻENIE & PRZED. FIZ. & Loc & obok$_1$ &  1& \\ pod:acc &  & ? & PRZED. FIZ. & LocGoal & pod$_2$ & & \\ koło:gen &  & ? & LICZBA & Count & koło$_1$ & & Ma koło pięciu lat. \\ blisko:gen &  & ? & ? & ? & blisko$_1$ & & Obracał się blisko prezesa \\ podczas:gen &  & KIEDY & SYTUACJA & Time & podczas$_1$ & & \\ w:loc &  & POŁOŻENIE & PRZED. FIZ. & Loc & w$_1$ & 1 & \\ w:acc &  & ? & PRZED. FIZ. & LocGoal & w$_1$ & 2,4 & \\ do:gen &  & ? & PRZED. FIZ. & LocGoal & do$_1$ & 3 & \\ z:gen &  & ? & PRZED. FIZ. & LocSrc & z$_1$ & 1b & \\ po:loc &  & KIEDY & SYTUACJA & Time & po$_1$ & 6 & \\ po:loc &  & CZAS & CZAS & Time & po$_1$ & & \\ po:loc &  & ? & PRZED. FIZ. & Loc & po$_1$ & 3 & \\ pod:acc &  & ? & ? & LocGoal & pod$_1$ & 1,2 & \\ pod:acc &  & ? & ? & Count & pod$_1$ & 14 & \\ pod:inst &  & POŁOŻENIE & ? & Loc & pod$_1$ & 1 & Kot spał pod biurkiem. \\ pod:inst &  & ? & ? & Order & pod$_1$ & 16 & Pod wiceministrem było pięciu dyrektorów departamentu. \\ pod:inst &  & ? & ? & Cond & pod$_1$ & 8,17 & Drzwi otworzyły się pod naporem tłumu. Kazano mu opuścić lokal pod groźbą eksmisji. \\ \hline \end{tabular}%
% }
% \end{sidewaystable}
% % subsection hasla_leksykalne (end)

\begin{sidewaystable}
\centering
{%
\begin{tabular}{|p{2.5cm}|p{2.5cm}|p{2.5cm}|p{2cm}|p{2cm}|p{2cm}|p{9cm}|}
\hline
lemat i~składnia & pref. sel. na nadrz. & pref. sel. na podrz. & rola tematyczna & sens (algorytm) & numer w SJP & przykład \\ \hline
koło:gen & zdarzenie-2 & CZAS & Time & koło$_1$ & 2 & Zadzwoniła koło szóstej.\\
około:gen & zdarzenie-2 & CZAS & Time & około$_1$ & & \\
blisko:gen & SYTUACJA & PRZED. FIZ. & Loc & blisko$_1$ & a & Mieszka blisko stacji metra. \\
koło:gen & SYTUACJA & PRZED. FIZ. & Loc & koło$_1$ &  1& \\
niedaleko:gen & SYTUACJA & PRZED. FIZ. & Loc & niedaleko$_1$ & & Mieszkali niedaleko parku. \\
opodal:gen & SYTUACJA & PRZED. FIZ. & Loc & opodal$_1$ & & Konie pasły się opodal lasu. \\
pod:inst & SYTUACJA & PRZED. FIZ. & Loc & pod$_2$ &  3& \\
obok:gen & SYTUACJA & PRZED. FIZ. & Loc & obok$_1$ &  1& \\
pod:acc & zdarzenie-2 & PRZED. FIZ. & LocGoal & pod$_2$ & & \\
koło:gen & SYTUACJA & LICZBA & Count & koło$_1$ & & Ma koło pięciu lat. \\
blisko:gen & SYTUACJA & LUDZIE & <undefined> & blisko$_1$ & & Obracał się blisko prezesa \\
podczas:gen & SYTUACJA & CZAS & Time & podczas$_1$ & & \\
w:loc & SYTUACJA & PRZED. FIZ. & Loc & w$_1$ & 1 & \\
w:acc & zdarzenie-2 & PRZED. FIZ. & LocGoal & w$_1$ & 2,4 & \\
do:gen & zdarzenie-2 & PRZED. FIZ. & LocGoal & do$_1$ & 3 & \\
z:gen & zdarzenie-2 & PRZED. FIZ. & LocSrc & z$_1$ & 1b & \\
po:loc & SYTUACJA & ZDARZENIE & Time & po$_1$ & 6 & \\
po:loc & SYTUACJA & CZAS & Time & po$_1$ & & \\
po:loc & zdarzenie-2 & PRZED. FIZ. & Loc & po$_1$ & 3 & \\
pod:acc & zdarzenie-2 & PRZED. FIZ. & LocGoal & pod$_1$ & 1,2 & \\
pod:acc & SYTUACJA & ILOŚĆ & Count & pod$_1$ & 14 & \\
pod:inst & SYTUACJA & PRZED. FIZ. & Loc & pod$_1$ & 1 & Kot spał pod biurkiem. \\
pod:inst & SYTUACJA & LUDZIE & Order & pod$_1$ & 16 & Pod wiceministrem było pięciu dyrektorów departamentu. \\
pod:inst & zdarzenie-2 & CZEMU & Cond & pod$_1$ & 8,17 & Drzwi otworzyły się pod naporem tłumu. Kazano mu opuścić lokal pod groźbą eksmisji. \\
\hline
\end{tabular}%
}
\end{sidewaystable}
% subsection hasla_leksykalne (end)

\subsection{Wybrane częste sensy i hasła leksykalne} % (fold)
\label{sec:podzia_ze_wzgldu_na_rol_tematyczna}


\paragraph{Przyimki niesemantyczne (argumenty)} % (fold)
\label{sub:argument}
\paragraph{dla-9}\label{prep-9}	przyimek wyznaczający obiekt odniesienia komunikowanego sądu, najczęściej osobę, z punktu widzenia której sąd ten jest wydawany
\begin{equation}
\text{Był to dla mnie rok pełen sukcesów.}
\end{equation}
\paragraph{do-12}\label{prep-12}	przyimek komunikujący cel czynności
\begin{equation}
\text{Oddał buty do naprawy.}
\end{equation}
\paragraph{do-17}\label{prep-17}	przyimek wprowadzający składnik zdania rządzony przez czasownik, przymiotnik, rzeczownik lub określający je
\begin{equation}
\text{Przywiązał się do dziadka. Był podobny do siostry.}
\end{equation}
\paragraph{do-18}\label{prep-18}	przyimek wprowadzający samodzielne wypowiedzenia nieoznajmujące, m.in. przekleństwa, formuły grzecznościowe, odezwy
\begin{equation}
\text{Do diabła! Do zobaczenia! Do broni!}
\end{equation}
\paragraph{dokoła-21, dookoła-22}\label{prep-21}	przyimek określający główny temat tego, o czym mowa w zdaniu
\begin{equation}
\text{Wszystkie jego myśli krążyły dokoła rodzinnego domu.}
\end{equation}
\paragraph{ku-35}\label{prep-35}	przyimek wraz z przyłączanym rzeczownikiem komunikujący, że zbliża się dana pora roku lub dnia
\begin{equation}
\text{Dni są coraz dłuższe i ma się ku wiośnie.}
\end{equation}
\paragraph{ku-36}\label{prep-36}	przyimek występujący w zdaniu komunikującym, że ktoś jest zainteresowany inną osobą pod względem erotycznym
\begin{equation}
\text{Nasze dzieci mają się ku sobie.}
\end{equation}
\paragraph{między-40}\label{prep-40}	przyimek komunikujący, iż zależność, o której mowa, dotyczy wymienionych w zdaniu osób, zjawisk, przedmiotów lub faktów
\begin{equation}
\text{Pokrewieństwo między nimi było bardzo dalekie.}
\end{equation}
\paragraph{między-42}\label{prep-42}	przyimek wprowadzający określenie przeznaczenia tego, co stanowi rezultat dokonanego podziału
\begin{equation}
\text{Podzielił majątek między synów.}
\end{equation}
\paragraph{między-44}\label{prep-44}	przyimek komunikujący, iż wybór, przed którym stoi dana osoba, dotyczy wymienionych przedmiotów lub osób
\begin{equation}
\text{Wybieraj między mną a nim.}
\end{equation}
\paragraph{między-45}\label{prep-45}	przyimek wprowadzający określenie zbioru, do którego należą lub spośród którego się wyróżniają charakteryzowane w zdaniu przedmioty lub osoby
\begin{equation}
\text{Słynęła z urody między rówieśnicami.}
\end{equation}
\paragraph{na-50}\label{prep-50}	przyimek tworzący z nazwami czynności wyrażenia określające cel, skutek lub sposób odbywania się tych czynności
\begin{equation}
\text{Składać na pół.}
\end{equation}
\paragraph{na-51}\label{prep-51}	przyimek łączący z nadrzędnymi czasownikami rzeczowniki stanowiące ich uzupełnienie
\begin{equation}
\text{Liczyć na pomoc.}
\end{equation}
\paragraph{na-52}\label{prep-52}	z nadrzędnymi rzeczownikami wyrazy stanowiące ich uzupełnienie
\begin{equation}
\text{Skrzynka na listy.}
\end{equation}
% section podzia_ze_wzgldu_na_rol_tematyczna (end)
\paragraph{Beneficient} % (fold)
\label{sub:beneficient}
\paragraph{dla-7} \label{prep-7}	przyimek komunikujący o przeznaczeniu obiektu
\begin{equation}
\text{Film dla młodzieży.}
\end{equation}
% paragraph quasi_beneficient (end)
\paragraph{Comparison} % (fold)
\label{sub:comparison}
\paragraph{aniżeli-1} \label{prep-1}	przyimek łączący podstawę porównania z wyrażeniem, które komunikuje zwykle wyższy stopień nasilenia cechy
\begin{equation}
\text{Była ładniejsza aniżeli jej koleżanka.}
\end{equation}
\paragraph{co-6} \label{prep-6}	przyimek używany w połączeniach, komunikujących, że przedmioty pod względem danej cechy są tożsame
\begin{equation}
\text{Krawat tego koloru co koszula}
\end{equation}
\paragraph{jak-24} \label{prep-24}	przyimek wprowadzający wyrażenie porównawcze
\begin{equation}
\text{Nos miała jak kartofelek.}
\end{equation}
\paragraph{Condition} % (fold)
\label{sub:condition}
\paragraph{dzięki-23} \label{prep-23}	przyimek komunikujący o cechach, faktach lub osobach, które spowodowały, że dana sytuacja jest dla kogoś pomyślna
\begin{equation}
\text{Uratował się dzięki przytomności umysłu.}
\end{equation}
\paragraph{mimo-38} \label{prep-38}	przyimek oznaczający rozbieżność między tym, co się dzieje, a tym, czego się należało spodziewać
\begin{equation}
\text{Mimo późnej pory wyszedł na spacer.}
\end{equation}
\paragraph{na\_skutek-361} \label{prep-361}
\paragraph{pod\_wpływem-400} \label{prep-400}
\paragraph{w\_wyniku-479} \label{prep-479}
\paragraph{z\_powodu-504} \label{prep-504}

\paragraph{Duration} % (fold)
\label{sub:duration}
\paragraph{na-48} \label{prep-48}	przyimek tworzący wyrażenia oznaczające trwanie czegoś, okres, termin lub porę dziania się czegoś
\begin{equation}
\text{Wyjechać na weekend.}
\end{equation}
% paragraph duration (end)

\paragraph{Location} % (fold)
\label{sub:location}
\paragraph{blisko-3} \label{prep-3}	przyimek komunikujący o małej odległości między przedmiotami
\begin{equation}
\text{Mieszka blisko stacji metra.}
\end{equation}
\paragraph{dokoła-20} \label{prep-20}	przyimek komunikujący, że to, o czym mowa w zdaniu, ma miejsce ze wszystkich stron czegoś znajdującego się w środku
\begin{equation}
\text{Dokoła klombów posadzono tulipany.}
\end{equation}
\paragraph{koło-29} \label{prep-29}	przyimek komunikujący o małej odległości między przedmiotami
\begin{equation}
\text{Usiądź koło mnie.}
\end{equation}
\paragraph{koło-31} \label{prep-31}	przyimek przyłączający nazwę przedmiotu będącego obiektem czynności, o której mowa w zdaniu
\begin{equation}
\text{Dłubał coś koło zegara.}
\end{equation}
\paragraph{między-39} \label{prep-39}	przyimek komunikujący, że to, o czym mowa, jest z obu stron otoczone czymś
\begin{equation}
\text{Ulica ciągnęła się między domami.}
\end{equation}
\paragraph{między-43} \label{prep-43}	przyimek ograniczający dwustronnie zbiór cech, które można przypisać nietypowemu obiektowi, o którym mowa
\begin{equation}
\text{Bluzka koloru między żółtym a brązowym.}
\end{equation}
\paragraph{na-46} \label{prep-46}	przyimek tworzący wyrażenia oznaczające miejsce dziania się lub znajdowania się czegoś
\begin{equation}
\text{Pracował na polu.}
\end{equation}
% paragraph location (end)

\paragraph{Location\_goal} % (fold)
\label{sub:location_goal}
\paragraph{do-13} \label{prep-13}	przyimek komunikujący kierunek ruchu
\begin{equation}
\text{Skoczył do wody.}
\end{equation}
\paragraph{do-15} \label{prep-15}	przyimek wyznaczający górną lub dolną granicę czegoś
\begin{equation}
\text{Inflacja spadła do dziesięciu procent.}
\end{equation}
\paragraph{do-16} \label{prep-16}	przyimek komunikujący o rezultacie procesu lub charakteryzujący stopień nasilenia tego procesu
\begin{equation}
\text{Blacha rozpaliła się do czerwoności.}
\end{equation}
\paragraph{ku-32} \label{prep-32}	przyimek przyłączający określenie miejsca, w kierunku którego ktoś zmierza lub coś prowadzi
\begin{equation}
\text{Odwrócił się nagle i ruszył ku drzwiom.}
\end{equation}
\paragraph{na-47} \label{prep-47}	miejsce lub kierunek, będące celem ruchu, czynności
\begin{equation}
\text{Pojechali na Mazury.}
\end{equation}

\paragraph{Extra\_result} % (fold)
\label{sub:manner}
\paragraph{ku-34} \label{prep-34}	przyimek przyłączający określenie czyjegoś stanu wewnętrznego, który wytworzył się w następstwie zdarzenia lub zjawiska, o którym mowa w zdaniu
\begin{equation}
\text{Ku radości maluchów spadł śnieg.}
\end{equation}

\paragraph{Measure} % (fold)
\label{sub:measure}
\paragraph{na-49} \label{prep-49}	przyimek tworzący wyrażenia oznaczające miarę, ocenę wielkości oraz zakres ich zastosowania
\begin{equation}
\text{Gruby na palec.}
\end{equation}
% paragraph measure (end)

\paragraph{Member} % (fold)
\label{sub:member}
\paragraph{bez-2} \label{prep-2}	przyimek komunikujący zwykle brak, nieobecność czegoś lub kogoś
\begin{equation}
\text{Las bez grzybów. Sukienka bez rękawów. Odejść bez pożegnania.}
\end{equation}
% paragraph member (end)

\paragraph{Math} % (fold)
\label{sub:mod_math}
\paragraph{do-19} \label{prep-19}	przyimek między liczebnikami oznaczający potęgowanie
\begin{equation}
\text{Dwa do piątej.}
\end{equation}
% paragraph mod_math (end)

\paragraph{Purpose} % (fold)
\label{sub:purpose}
\paragraph{dla-10} \label{prep-10}	przyimek wprowadzający przyczynę jakichś działań
\begin{equation}
\text{Na dziki polowano dla mięsa.}
\end{equation}
\paragraph{dla-8} \label{prep-8}	przyimek komunikujący cel czynności
\begin{equation}
\text{Tupali nogami dla rozgrzewki.}
\end{equation}
\paragraph{do-11} \label{prep-11}	przyimek komunikujący o przeznaczeniu obiektu
\begin{equation}
\text{filiżanka do kawy}
\end{equation}
\paragraph{ku-33} \label{prep-33}	przyimek przyłączający określenie celu czynności, o której mowa w zdaniu
\begin{equation}
\text{Nakręcono ten film ku przestrodze młodzieży.}
\end{equation}
% paragraph purpose (end)

\paragraph{Quantifier} % (fold)
\label{sub:quantifier}
\paragraph{co-5} \label{prep-5}	przyimek używany w połączeniach charakteryzujących częstość, z jaką coś się powtarza
\begin{equation}
\text{Przystawał co krok.}
\end{equation}
\paragraph{jako-28} \label{prep-28}	przyimek przyłączający liczebnik porządkowy, który wskazuje na kolejność charakteryzowanego zdarzenia względem innych
\begin{equation}
\text{Sekretarz komisji przyszedł na zebranie jako pierwszy.}
\end{equation}
% paragraph quantifier (end)

\paragraph{Role} % (fold)
\label{sub:role}
\paragraph{jako-25} \label{prep-25}	przyimek wprowadzający określenie roli, jaką pełni jakaś osoba w danych okolicznościach, a także określenie stosunku tej osoby do innych osób
\begin{equation}
\text{Jako gospodarz spotkania starał się być dla wszystkich życzliwy i miły.}
\end{equation}
\paragraph{jako-27} \label{prep-27}	przyimek wprowadzający określenie sposobu potraktowania lub oceny danego obiektu lub stanu rzeczy
\begin{equation}
\text{Takie stany są określane jako depresyjne.}
\end{equation}
% paragraph role (end)

\paragraph{Time} % (fold)
\label{sub:time}
\paragraph{blisko-4} \label{prep-4}	lub między punktami czasowymi
\begin{equation}
\text{Było już blisko daty wyjazdu.}
\end{equation}
\paragraph{jako-26} \label{prep-26}	przyimek wraz z przyłączanym rzeczownikiem wskazujący na okres życia osoby, o której mowa w zdaniu
\begin{equation}
\text{Jako dziecko przebyła wszystkie choroby zakaźne.}
\end{equation}
\paragraph{koło-30} \label{prep-30}	przyimek poprzedzający określenie liczby, ilości, miary lub czasu, który komunikuje, że jest ono przybliżone
\begin{equation}
\text{Zadzwoniła koło szóstej.}
\end{equation}
\paragraph{lada-37} \label{prep-37}	przyimek łączący się z nazwą jednostki czasu, komunikujący, iż coś stanie się w najbliższym czasie
\begin{equation}
\text{Przyjadą lada godzina.}
\end{equation}
\paragraph{między-41} \label{prep-41}	przyimek wprowadzający określenia granic czasowych, w których miało miejsce dane zdarzenie
\begin{equation}
\text{Wrócę między drugą a trzecią.}
\end{equation}
% paragraph time (end)

\paragraph{Time\_goal} % (fold)
\label{sub:time_goal}
\paragraph{do-14} \label{prep-14}	przyimek określający czas zakończenia zdarzenia
\begin{equation}
\text{Spał do południa.}
\end{equation}

% subsection time_goal (end)

\section{Indeks przyimków} % (fold)
\label{sec:indeks_przyimkow}
\begin{multicols}{4}
\noindent
\hyperref[prep-1]{aniżeli-1}\\
\hyperref[prep-2]{bez-2}\\
\hyperref[prep-3]{blisko-3}\\
\hyperref[prep-4]{blisko-4}\\
\hyperref[prep-5]{co-5}\\
\hyperref[prep-6]{co-6}\\
\hyperref[prep-7]{dla-7}\\
\hyperref[prep-8]{dla-8}\\
\hyperref[prep-9]{dla-9}\\
\hyperref[prep-10]{dla-10}\\
\hyperref[prep-11]{do-11}\\
\hyperref[prep-12]{do-12}\\
\hyperref[prep-13]{do-13}\\
\hyperref[prep-14]{do-14}\\
\hyperref[prep-15]{do-15}\\
\hyperref[prep-16]{do-16}\\
\hyperref[prep-17]{do-17}\\
\hyperref[prep-18]{do-18}\\
\hyperref[prep-19]{do-19}\\
\hyperref[prep-20]{dokoła-20}\\
\hyperref[prep-21]{dokoła-21}\\
\hyperref[prep-21]{dookoła-22}\\
\hyperref[prep-23]{dzięki-23}\\
\hyperref[prep-24]{jak-24}\\
\hyperref[prep-25]{jako-25}\\
\hyperref[prep-26]{jako-26}\\
\hyperref[prep-27]{jako-27}\\
\hyperref[prep-28]{jako-28}\\
\hyperref[prep-29]{koło-29}\\
\hyperref[prep-30]{koło-30}\\
\hyperref[prep-31]{koło-31}\\
\hyperref[prep-32]{ku-32}\\
\hyperref[prep-33]{ku-33}\\
\hyperref[prep-34]{ku-34}\\
\hyperref[prep-35]{ku-35}\\
\hyperref[prep-36]{ku-36}\\
\hyperref[prep-37]{lada-37}\\
\hyperref[prep-38]{mimo-38}\\
\hyperref[prep-39]{między-39}\\
\hyperref[prep-40]{między-40}\\
\hyperref[prep-41]{między-41}\\
\hyperref[prep-42]{między-42}\\
\hyperref[prep-43]{między-43}\\
\hyperref[prep-44]{między-44}\\
\hyperref[prep-45]{między-45}\\
\hyperref[prep-46]{na-46}\\
\hyperref[prep-47]{na-47}\\
\hyperref[prep-48]{na-48}\\
\hyperref[prep-49]{na-49}\\
\hyperref[prep-50]{na-50}\\
\hyperref[prep-51]{na-51}\\
\hyperref[prep-52]{na-52}\\
\hyperref[prep-361]{na\_skutek-361}\\
\hyperref[prep-400]{pod\_wpływem-400}\\
\hyperref[prep-479]{w\_wyniku-479}\\
\hyperref[prep-504]{z\_powodu-504}\\
\end{multicols}

% section indeks_przyimkow (end)
\end{document}