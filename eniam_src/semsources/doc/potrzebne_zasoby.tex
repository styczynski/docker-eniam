\documentclass[a4paper,12pt]{article}
\usepackage[utf8]{inputenc}

\usepackage{amsmath}

\usepackage[T1]{fontenc}
\usepackage[english, polish]{babel}
\usepackage[utf8]{inputenc}

\title{Zasoby potrzebne do generowania reprezentacji semantycznej}
\author{Wojciech Jaworski}
\date{}

\begin{document}

\maketitle

Rzutowanie Słowosieci na SUMO

\section{Wyrażenia wielosłowne}
Wyrażenia wielosłowne rozpoznawane będą za pomocą programu Mewex (jeśli on da się zastosować), oraz istniejących list
takich wyrażeń (SEJF, SEJFIK, SAWA) uzupełnionych o wielosłowne sensy ze Słowosieci
i ew. pojęcia z dbPedii.
Frazeologizmy (wyrażenia wielosłowne o luźnym szyku) pobrane zostaną z Walentego.

% Dezambiguacja wyrażeń wielosłownych odbywać się będzie zgodnie z obserwacją, 
% że 

\section{Nazwy własne}
Nazwy własne rozpoznawane będą za pomocą programu ..., oraz listy wydobytej z SGJP i Polimorfa.
Rozpoznawane są też heurystycznie na podstawie wystąpienia w tekście słów zaczynających się od wielkiej litery.

NELexicon 2.0

Liner2

\section{Sensy słów}
Sensy słów wykrywane są za pomocą Słowosieci i dezambiguowane za pomocą programu WoSeDon oraz na podstawie spełniania preferencji selekcyjnych z Walentego.

\section{Role tematyczne}
Argumenty czasownika: Walenty

Wykrywanie ról semantycznych wewnątrz frazy NP: NPSemRel, MLNpSemRel

\section{Relacja przestrzenne}
Rozpoznawanie wyrażeń przestrzeennych (SpatialPL)

\section{Relacja czasowe}
Wykrywane za pomocą programu Timex, oraz rozpoznawane przez gramatykę

\section{Zdarzenia, stany, procesy}
Rozpoznawane za pomocą Słowosieci i programu ...

Wykrywanie czasowników z podmiotem domyślnym (Minos)

Rozpoznawanie wyznaczników sytuacji (EventsPL)

\section{Liczebniki}
Liczebniki określane są relacją Count albo Measure, a liczebniki główne dodatkowo opisane są za pomocą 
zapisanej za pomocą cyfr arabskich liczby wyrażającej ich znaczenie.
Oprócz tego wskazane są typy niejawnych argumentów semantycznych dla poszczególnych leksemów.
%TODO dać Kubie do sprawdzenia

\section{Przyimki}

\section{Spójniki podrzędne i zaimki względne}

\section{Spójniki współrzędne}

\section{Kubliki}

\section{Wykrzykniki}

\begin{itemize}
\item lista jednostek leksykalnych przeznaczonych do szczegółowej analizy (załączona do tego dokumentu)
\item lista leksemów okazjonalnych, deiktycznych, koreferencyjnych
\item lista modyfikatorów nieintersektywnych
\item lista pojemników, jednostek miar, rzeczoników znumeralizowanych itp.
\item lista kwantyfikatorów i okresleń częstości
\item lista rzeczowników wyrażających nazwy cech
\item lista niefaktywnych argumentów czasowników
\item lista 
\item lista 
\item lista przyimków i spójników podrzędnych z rolami tematycznymi


\end{itemize}

\end{document}
