\documentclass[a4paper,12pt]{article}
\usepackage[utf8]{inputenc}

\usepackage{amsmath}

\usepackage[T1]{fontenc}
\usepackage[english, polish]{babel}
\usepackage[utf8]{inputenc}
\usepackage{tikz}
\usetikzlibrary{conceptgraph}

\newcommand{\sg}{{\it sg} }
\newcommand{\pl}{{\it pl} }
\newcommand{\mass}{{\it mass} }
\newcommand{\ind}{{\it indexical} }
\newcommand{\corf}{{\it coreferential} }
\newcommand{\deict}{{\it deictic} }
\newcommand{\interr}{{\it interrogative} }

\title{Model świata}
\author{Wojciech Jaworski}
\date{}

\begin{document}

\maketitle

Przymiotnik określający rzeczownik definiuje wartość, cechy a nazwa tej cechy wynika 
z wartości i można ją określić na podstawie Słowosieci.

Słowosieć definiuje typy bytów dla naszego modelu. 
Wprowadza też ontologię najbardziej ogólnych pojęć, 
w oparciu o które będziemy definiować język reprezentacji znaczenia.

Zgodnie ze Słowosiecią 
przymiotniki dzielimy na 
jakościowe (jasny +orzecznik, +ość, prepozycja, +bardzo), relacyjne (brzegowy -orzecznik,-ość, pospozycja, -bardzo) i materiałowe (drewniany +orzecznik, -ość, prepozycja, -bardzo).


Podstawowe typy bytów i ich główne podtypy wg. Słowosieci to (w nawiasach liczba hiponimów):
\begin{itemize}
\item istota 1 - człowiek, duch, zwierzę - każde żywe stworzenie lub byt myślący (19299)
  \begin{itemize}
  \item humanoid 1 (249)
  \item istota fantastyczna 1 (198)
  \item istota żywa 1 (12382)
    \begin{itemize}
    \item krewniak 2 (105)
    \item stworzenie oceniane negatywnie 1 (662)
    \item zwierzę 1 (11413)
    \item istota nadprzyrodzona 1 (81)
    \end{itemize}
  \item osoba 4 (7642)
    \begin{itemize}
    \item człowiek 1 (7622)
      \begin{itemize}
      \item człowiek ze względu na płeć 1  (942)
      \item przedstawiciel 3  (1618)
      \item człowiek, który coś robi 1  (3381)
      \item persona 1  (434)
      \item człowiek posiadający jakieś nadprzyrodzone zdolności 1 (55)
      \item człowiek, który coś zrobił 1  (264)
      \item człowiek charakteryzowany ze względu na miejsce pobytu 1  (610)
      \item człowiek ze względu na pochodzenie etniczne 1  (367)
      \item człowiek ze względu na pełnioną funkcję 1  (1028)
      \item człowiek charakteryzowany ze względu na kwalifikacje 1  (1340)
      \item człowiek charakteryzowany pod względem wieku 1  (1010)
      \item człowiek ze względu na relacje społeczne 1  (3060)
      \item nazwa człowieka uwzględniająca jego cechy 1  (2695)
      \item człowiek prowadzący określony tryb życia 1  (68)
      \end{itemize}
    \end{itemize}
  \end{itemize}
\item całość 1 - wszystkie części czegoś, ogół, komplet, wszystko (11185)
  \begin{itemize}
  \item zbiór 1 (9868)
    \begin{itemize}
    \item grupa istot 1 (4721)
    \item zbiór rzeczy 1 (797)
    \item zespół cech 1 (2030)
    \item kompleks 3 (631)
    \end{itemize}
  \item układ 1 (984)
  \item jednostka 1 (493)
  \end{itemize}
\item jakiś 1 (312) %kwantyfikatory itp?
\item część 3 (4616)
  \begin{itemize}
  \item część ciała 1 (550)
  \item miejsce wyróżniające się z całości obiektu 1 (478)
  \item element 3 (1252)
  \item składnik 1 (2005)
  \end{itemize}
\item podmiot 3 (1121)
  \begin{itemize}
  \item osoba prawna 1 (676)
  \item osoba fizyczna 1 (110)
  \item właściciel 1 (162)
  \end{itemize}
\item byt 1 (30)
\item czyn 1  (1111)
  \begin{itemize}
  \item zabieg 1  (166)
  \item umowa 1  (105)
  \item niesprawiedliwość 2  (98)
  \item praktyka 1  (405)
  \item zagrywka 1  (115)
  \item postępek 2  (142)
  \end{itemize}
\item  właściwość 2 (255)
  \begin{itemize}
  \item wielkość 6  (251)
    \begin{itemize}
    \item wielkość fizyczna 1  (227)
    \end{itemize}
  \end{itemize}
\item ewenement 1  (110)
  \begin{itemize}
  \item osobliwość 1  (100)
  \end{itemize}
\item przedmiot 4  (132)
  \begin{itemize}
  \item  pośmiewisko 2 (60)
  \item  adresat 1 (44)
  \end{itemize}
\item środek 1  (3464)
  \begin{itemize}
  \item środek płatniczy 1  (194)
  \item środek lokomocji 1  (1048)
  \item narzędzie 1  (2153)
  \end{itemize}
\item ilość 1  (2266)
  \begin{itemize}
  \item część 1  (1137)
  \item zawartość 3 (237) %pojemniki
  \item kwota 1  (398)
  \item porcja 1 - ilość żywności itp przeznaczona do jednorazowego spożycia (253)
  \end{itemize}
\item ciąg 4  (2209)
  \begin{itemize}
  \item  ciąg wydarzeń 1  (2196)
    \begin{itemize}
    \item  proces 1 (2191)
    \end{itemize}
  \end{itemize}
\item materia 3  (5105)
  \begin{itemize}
  \item ciało 5  (781)
    \begin{itemize}
    \item ciało lotne 1  (76)
    \item ciecz 1  (577)
    \item ciało stałe 1  (69)
    \item ciało niebieskie 1  (71)
    \end{itemize}
  \item substancja 1  (4897)
    \begin{itemize}
    \item substancja określana ze względu na właściwości fizyczne lub chemiczne 1  (240)
    \item substancja określana ze względu na rolę chemiczną 1  (351)
    \item substancja określana ze względu na stan skupienia 1  (652)
    \item substancja pochodzenia naturalnego 1  (97)
    \item substancja określana ze względu na działanie lub funkcję 1  (1883)
    \item  mieszanina 1 (608)
    \item tworzywo 1  (1148)
    \item substancja chemiczna 1  (1733)
    \end{itemize}
  \end{itemize}
\item zależność 3 - związek pomiędzy rzeczami, gdy jedna warunkuje drugą (483)
  \begin{itemize}
  \item relacja 2  (420)
    \begin{itemize}
    \item związek 3 (256)
    \end{itemize}
  \end{itemize}
\item rzecz 2  (192)
  \begin{itemize}
  \item sprawa 1  (85)
  \item dowód 1  (102)
  \end{itemize}
\item środek 1  (3464)
  \begin{itemize}
  \item środek lokomocji 1  (1048)
  \item narzędzie 1  (2153)
  \end{itemize}
\item CZASOWNIK 1  (21020)
  \begin{itemize}
  \item CZASOWNIK należący do określonego pola leksykalnego 1  (18337)
    \begin{itemize}
    \item CZASOWNIK oznaczający sytuację związaną z wytwarzaniem czegoś 1  (470)
      \begin{itemize}
      \item CZASOWNIK $\tensor(\text{PROCES}\with\text{DZIAŁANIE})\tensor(\text{NDK}\with\text{DK}\tensor$ oznaczający wytwarzanie czegoś 1
      \end{itemize}
    \item CZASOWNIK oznaczający sytuację związaną ze współżyciem w społeczeństwie 1  (3397)
%   <node name="CZASOWNIK - DZIAŁANIE NDK oznaczający sytuację związaną (lub sytuacje związane) ze współżyciem w społeczeństwie 1" visited="true" size="137" abstract="true"/>
%   <node name="CZASOWNIK - CZYNNOŚĆ NDK oznaczający sytuację związaną (lub sytuacje związane) ze współżyciem w społeczeństwie 1" visited="true" size="1308" abstract="true"/>
%   <node name="CZASOWNIK - ZDARZENIE NDK oznaczający sytuację związaną ze współżyciem w społeczeństwie 1" visited="true" size="58" abstract="true"/>
%       
%   <node name="CZASOWNIK - WYPADEK oznaczający zmianę sytuacji związanych (lub sytuacje związane) ze współżyciem w społeczeństwie 1" visited="true" size="21" abstract="true"/>
%   <node name="CZASOWNIK - PROCES NDK oznaczający zmianę sytuacji związanych (lub sytuacje związane) ze współżyciem w społeczeństwie 1" visited="true" size="13" abstract="true"/>
%   <node name="CZASOWNIK - PROCES DK oznaczający zmianę sytuacji związanych (lub sytuacje związane) ze współżyciem w społeczeństwie 1" visited="true" size="19" abstract="true"/>
%   <node name="CZASOWNIK - DZIAŁANIE DK oznaczający zmianę sytuacji związanych (lub sytuacje związane) ze współżyciem w społeczeństwie 1" visited="true" size="104" abstract="true"/>
%   <node name="CZASOWNIK - AKT oznaczający zmianę sytuacji związanych (lub sytuacje związane) ze współżyciem w społeczeństwie 1" visited="true" size="897" abstract="true"/>
%   
%   <node name="CZASOWNIK - ZDARZENIE NDK oznaczający powodowanie zmiany sytuacji związanych (lub sytuacje związane) ze współżyciem w społeczeństwie 1" visited="true" size="7" abstract="true"/>
%   <node name="CZASOWNIK - WYPADEK oznaczający spowodowanie zmiany sytuacji związanych (lub sytuacje związane) ze współżyciem w społeczeństwie 1" visited="true" size="190" abstract="true"/>
%   <node name="CZASOWNIK - DZIAŁANIE DK oznaczający spowodowanie zmiany sytuacji związanych (lub sytuacje związane) ze współżyciem w społeczeństwie 1" visited="true" size="178" abstract="true"/>
%   <node name="CZASOWNIK - CZYNNOŚĆ NDK oznaczający powodowanie zmiany sytuacji związanych (lub sytuacji związanych) ze współżyciem w społeczeństwie 1" visited="true" size="183" abstract="true"/>
%   
%   <node name="CZASOWNIK - PROCES DK oznaczający spowodowanie zmiany sytuacji związanych  ze współżyciem w społeczeństwie 1" visited="true" size="5" abstract="true"/>  
%   <node name="CZASOWNIK - PROCES NDK oznaczający powodowanie zmiany sytuacji związanych ze współżyciem w społeczeństwie 1" visited="true" size="50" abstract="true"/>
%   <node name="CZASOWNIK - DZIAŁANIE NDK oznaczający powodowanie zmiany sytuacji związanych ze współżyciem w społeczeństwie 1" visited="true" size="145" abstract="true"/>
%   
%   <node name="CZASOWNIK - PROCES NDK oznaczający zmianę w obrębie społeczeństwa 1" visited="true" size="33" abstract="true"/>
%   <node name="CZASOWNIK - PROCES DK oznaczający zmianę w obrębie społeczeństwa 1" visited="true" size="24" abstract="true"/>
%   
%   <node name="CZASOWNIK - PROCES NDK oznaczający powodowanie zmian w obrębie społeczeństwa 1" visited="true" size="2" abstract="true"/>
%   <node name="CZASOWNIK - PROCES DK oznaczający spowodowanie zmian w obrębie społeczeństwa 1" visited="true" size="2" abstract="true"/>
%   <node name="CZASOWNIK - AKT oznaczający spowodowanie zmian w obrębie społeczeństwa 1" visited="true" size="130" abstract="true"/>    
      \begin{itemize}
      \item   ()
      \item   ()
      \item   ()
      \item   ()
      \end{itemize}
    \item CZASOWNIK oznaczający sytuacje związane z relacjami abstrakcyjnymi 1  (1284)
      \begin{itemize}
      \item   ()
      \item   ()
      \item   ()
      \item   ()
      \end{itemize}
    \item CZASOWNIK oznaczający sytuację związaną z położeniem (relacjami przestrzennymi) lub zmianą położenia (relacji przestrzennych) 1  (3563)
      \begin{itemize}
      \item   ()
      \item   ()
      \item   ()
      \item   ()
      \end{itemize}
    \item CZASOWNIK oznaczający sytuację związaną z posiadaniem 1  (1122)
      \begin{itemize}
      \item   ()
      \item   ()
      \item   ()
      \item   ()
      \end{itemize}
    \item CZASOWNIK oznaczający sytuację związaną ze stanem mentalnym lub emocjonalnym albo reakcją emocjonalno-fizjologiczną 1  (3235)
      \begin{itemize}
      \item   ()
      \item   ()
      \item   ()
      \item   ()
      \end{itemize}
    \item CZASOWNIK oznaczający sytuację związaną ze zjawiskiem fizycznym 1  (2257)
      \begin{itemize}
      \item   ()
      \item   ()
      \item   ()
      \item   ()
      \end{itemize}
    \item CZASOWNIK oznaczający sytuacje związane z łańcuchem przyczynowo-skutkowym 1  (3226)
      \begin{itemize}
      \item   ()
      \item   ()
      \item   ()
      \item   ()
      \end{itemize}
    \item CZASOWNIK oznaczający sytuację związaną z następstwem czasowym zdarzeń 1  (419)
      \begin{itemize}
      \item   ()
      \item   ()
      \item   ()
      \item   ()
      \end{itemize}
    \item CZASOWNIK oznaczający sytuację związaną z kontaktem fizycznym 1  (1971)
      \begin{itemize}
      \item   ()
      \item   ()
      \item   ()
      \item   ()
      \end{itemize}
    \item CZASOWNIK oznaczający sytuację związaną z reakcją organizmu lub czynnością fizjologiczną 1  (555)
      \begin{itemize}
      \item   ()
      \item   ()
      \item   ()
      \item   ()
      \end{itemize}
    \end{itemize}
  \item CZASOWNIK STANOWY NDK 1  (822)
    \begin{itemize}
    \item CZASOWNIK - STAN NDK oznaczający stan psychiczny intelektualny 1  (54)
    \item CZASOWNIK - STAN NDK oznaczający stan emocjonalny 1  (152)
    \item CZASOWNIK - STAN NDK oznaczający stan fizyczny 1  (73)
    \item być 3  (86)
    \item CZASOWNIK - STAN NDK oznaczający położenie lub relacje przestrzenne 1  (66)
    \item CZASOWNIK - STAN NDK oznaczający stan mentalny 1  (6)
    \item CZASOWNIK - STAN NDK oznaczający sytuacje związane z percepcją lub odczuwaniem 1  (150)
    \item znajdować się 2  (69)
    \item istnieć 1  (57)
    \item CZASOWNIK - STAN NDK oznaczający relacje ponadczasowe i abstrakcyjne 1  (128)
    \end{itemize}
  \item CZASOWNIK DYNAMICZNY (AKCJA) 1  (20231)
    \begin{itemize}
    \item CZASOWNIK DYNAMICZNY ZMIENNOSTANOWY 1  (14759)
      \begin{itemize}
      \item   ()
      \item   ()
      \item   ()
      \item   ()
      \end{itemize}
    \item CZASOWNIK DYNAMICZNY NIEZMIENNOSTANOWY 1  (5926)
       \begin{itemize}
      \item   ()
      \item   ()
      \item   ()
      \item   ()
      \end{itemize}
   \item robić 1  (5484)
      \begin{itemize}
      \item   ()
      \item   ()
      \item   ()
      \item   ()
      \end{itemize}
    \end{itemize}
  \end{itemize}
\item   ()
  \begin{itemize}
  \item   ()
  \item   ()
  \item   ()
  \item   ()
  \end{itemize}
\item   ()
  \begin{itemize}
  \item   ()
  \item   ()
  \item   ()
  \item   ()
  \end{itemize}
\item 
\item 
\item 
\item 
\item 
\item 
\item 
\end{itemize}



\end{document}





