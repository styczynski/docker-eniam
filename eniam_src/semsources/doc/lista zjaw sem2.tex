% \RequirePackage[hyphens]{url}
\documentclass[a4paper, 12pt]{article}
% \documentclass[preprint, a4paper, 11pt]{article}

\usepackage[T1]{fontenc}
\usepackage[english, polish]{babel}
\usepackage[utf8]{inputenc}

\usepackage{fullpage}
\usepackage{lineno}
\usepackage{amsmath}
\usepackage{amsthm}
\theoremstyle{remark}
\newtheorem{test}{Test}

\usepackage[hidelinks]{hyperref}
% \urlstyle{sf}
\expandafter\def\expandafter\UrlBreaks\expandafter{\UrlBreaks%  save the current one
      \do\\~} % this stops urls from sticking out of the damn screen and the paper. I hate this arrggh.

\usepackage{tocloft}
\cftsetindents{section}{0pt}{1.5em}
\cftsetindents{subsection}{1.5em}{3em}
\cftsetindents{subsubsection}{1.5em}{3em}

\newcommand{\eniam}{\textsc{Eniam}}
\newcommand{\experiencer}{\textsc{Experiencer}}
\newcommand{\agent}{\textsc{Agent}}
\newcommand{\theme}{\textsc{Theme}}

\frenchspacing

\title{
\normalsize
\setlength{\topsep}{0pt} 
\noindent
Zleceniodawca: IPI PAN \hfill Dzieło do umowy z dnia: 2.11.2016 \\
Zleceniobiorca: Jakub Kozakoszczak \hfill Termin zakończenia pracy: 25.11.2016 \\
\rule{\linewidth}{2pt} \\[6pt] 
\huge Wstępna lista zjawisk semantycznych przeznaczonych do szczegółowej analizy \\[-2pt]}
\author{\Large Wojciech Jaworski, Jakub Kozakoszczak\\[6pt]
\hspace{-5pt}\rule{\linewidth}{2pt}  }
\date{}
% \date{\normalsize Date Submitted}

\begin{document}

\maketitle
		
\vspace{-26pt}
\begin{description}
\setlength{\itemsep}{4pt} 
\setlength{\parskip}{0pt} 
	\item[Zrealizowane w ramach projektu:] „CLARIN -- Polskie wspólne zasoby językowe i infrastruktura technologiczna”
	\item[Tytuł pracy zamówionej:] ,,Opracowanie wstępnej listy zjawisk semantycznych, które podlegać będą szczegółowej analizie''
	\item[Adres dzieła:] \url{http://wiki.nlp.ipipan.waw.pl/clarin/Parser\%20kategorialny?action=AttachFile\&do=view\&target=JK_zjawiska_semantyczne.pdf}
	\item[Opracowanie dokumentu:] Jakub Kozakoszczak
\end{description}

\vspace{6pt}
\hrule
\vspace{-14pt}
\renewcommand\contentsname{\normalsize{}}
\tableofcontents
\vspace{20pt}
\hrule
\newpage

\linenumbers
\section{Zakres i przeznaczenie listy} % (fold)
Lista zbiera zjawiska semantyczne, których spójne opisanie w ramach jednolitej implementowalnej reprezentacji znaczenia o szerokim pokryciu dla języka polskiego wymaga szczegółowych, głębszych badań.
 Są to przede wszystkim zjawiska polegające na wprowadzaniu przez wyrażenia elementów znaczenia do zdań w sposób inny podstawowy, czyli przez zawężającą modyfikację znaczenia bezpośredniego nadrzędnika składniowego.
Ten podstawowy sposób nazywamy roboczo \emph{lokalnym wkładem znaczeniowym}.
Lista jest wstępnym rozpoznaniem, które posłuży jako mapa zadań w następnych etapach rozwoju kategorialnego parsera składniowo-semantycznego \eniam.

% section zakres_i_cel_listy (end)

%\chapter{Lista zjawisk semantycznych} % (fold)

\section{Operatorowość}
Takie leksemy, czy konstrukcje, do wyrażenia których w logice nie wystarcza koniunkcja, predykaty relacyjne, dscr i kwantyfikator egzystencjalny,
tzn. nie są wyrażalne w logice egzytencjalno koniunkcyjnej rozzerzonejo dscr.

Tzn. kwantyfikatory, albo operatory nielokalnie zmieniające strukturę zdania.
Zwane dalej qwantyfikatory.

Przykłady: każdy, prawie każdy, każdy z powyższych, co dziesiąty, 
każda strona ma inny kolor, nawzajem, razem,
codziennie, nigdy, zawsze, tylko, jakiś, pewien, jedyny, oba, inny
nie tylko \dots lecz także

Interesują nas następujące cechy qwantyfikatorów:
\begin{itemize}
\item konstrukcje składniowe, w których występują,
\item semantyka, czyli wprowadzane zmienne i liczba argumentów, 
\item odwzorowanie konstrukcji składniowych na semantykę.
\end{itemize}
Nie będziemy definiować qwantyfikatorów a jedynie zaznaczać ich wystąpienia w formach logicznych
w sposób taki, by po zdefiniowaniu qwantyfikatora znaczenie formy logicznej było zgodne ze znaczeniem zdania.

Z uwagi na to, że qwantyfikatory będące w zdaniu wchodzą ze sobą w interakcje (np. mogą stać w różnej kolejności)
zadane przez nas odwzorowanie konstrukcji składniowych na semantykę nie będzie stanowić przepisu na jednoznaczną semantykę,
w szczególności nie będziemy na tym etapie rozwiązywać problemu niejednoznaczności zakresu qwantyfikatorów.

Osobnym problemem jest reprezentacja qwantyfikatorów wprowadzonych przez kilka leksemów znajdujących się w różnych częściach
zdania (np Każda strona ma inny kolor.) i interakcji między qwantyfikatorami innej niż wpływanie na swój zakres
np. kwatyfikatory rozgałęzione w zdaniu Hintikki.

Negację traktujemy jako qwantyfikator mający jeden argument i nie wprowadzający zmiennej.

``jeśli \dots, to \dots'' oraz ``jeśli \dots, \dots'' użyte w znaczeniu logicznym uznajemy ze qwantyfikator.
Uwaga zgodnie z formalizmem wprowadzonym w Walentym ``jeśli'' wiąże zdanie podrzędne z orzeczeniem zdania nadrzędnego za pomocą
roli tematycznej Condition. Trzeba będzie zbadać użycia roli Condition i określić czy są one uogólnienie intersektywne.
Jeśli nie są trzeba zbadać w jaki sposób należy je przekształcić do postaci formuły logicznej.

Rolę Manner też można traktować jako nielokalną.

\subsection{Zjawisko wyciągania kwantyfikatorów poza zakres opertora spowodowane koreferencją pomiędzy argumentami elementami argumentów operatora}
\begin{equation}
	\text{Jeśli rolnik ma krowę, to ją doi}
\end{equation}
W powyższym przykładzie ``jeśli traktujemy jako logiczną implikację, ''rolnik`` i ''krowa`` są kwantyfikowane egzystencjalnie, a 
''ją`` i ''pro`` (niemy podmiot ''doi``) są kwantyfikowane koreferencjnie (jota?). 
Aby rozwiązać koreferencję trzeba przenieść powyższe 4 kwantyfikatory ponad implikację co powoduje
zamianę kwantyfikatorów  ezgzystencjanych przy ''rolniku`` i ''krowie`` na uniwersalne. 

Analogiczne zjawisko będzie występować przy wszystkich qwantyfikatorach mających więcej niż jeden argument.

\section{Niejawne argumenty semantyczne} % (fold)

Po wstępnej analizie przyjmujemy istnienie sześciu typów niejawnych argumentów semantycznych:

\begin{enumerate}
\item Okazjonalne \\ Np. \emph{tu}.
\item Deiktyczne\\ Np. \emph{taki}.
\item Porównawcze \\ Leksemy z tym argumentem mogą przyjmować porównawczy argument przyimkowy.
\item Porządkowe \\ Np. \emph{kolejny}.
\item Koreferencyjne \\ Np. \emph{on}.
\item Relacyjne \\ Aby określić znaczenie znaczenie przymiotników i liczebników z argumentem relacyjnym (\emph{duży}, \emph{sporo}) trzeba wiedzieć, jaki predykat jest modyfikowany przez dane słowo.
\end{enumerate}

Niezbędne jest rozpoznanie i oznaczenie argumentów niejawnych w leksemach i wypracowanie skutecznej koncepcji ich opisu i przetwarzania.

Za niejawne argumenty semantyczne uznajemy te argumenty, które są niezbędne do ustalenia znaczenia wyrazu mającego ten argument.
Przykłady występowania wyrażeń z różnymi argumentami niejawnymi:
\begin{itemize}
	\item większy -- większy niż słoń \\ Znaczenia wyrazu ,,większy'' nie da się obliczyć bez argumentu ,,niż słoń''.
	\item Biegnie. -- On biegnie. -- Jaś biegnie. \\
	W pierwszym zdaniu inicjatora (\textsc{Initiator}) reprezentujemy oznaczając go niemym zaimmkiem (\textsc{pro}), w drugim jest on wyrażony za pomocą zaimka ,,on'', w trzecim jest wskazany jawnie jako ,,Jaś''.
	Aby obliczyć znaczenie \textsc{pro} i ,,on'' potrzebna jest wartość koreferencyjnego lub deiktycznego argumentu.
	\item Karol chce. -- Karol chce pracować.\\
	Brakującego argumentu składniowego czasownika ,,chcieć'' nie traktujemy jako niejawnego argumentu semantycznego.
\end{itemize}

\section{Znaczenie relacji składniowych} % (fold)

\subsection{Uogólniona intersektywność} % (fold)
Ważną cechą semantyczną, która przysługuje nie wszystkim leksemom, a pozwala na reprezentowanie ich wkładu znaczeniowego jako zawężającego znaczenie szerszej frazy, jest \emph{intersektywność}.
Tradycyjnie rozumiana intersektywność przypisywana jest przymiotnikom i oznacza, że denotacja frazy rzeczownikowej z przymiotnikiem intersektywnym jest podzbiorem denotacji frazy bez niego.

\begin{equation}
\text{Widzę czarnego kota.} \rightarrow \text{Widzę kota.}
\end{equation}


Rozumienie to można uogólnić na przysłówki i mówić o przysłówkach intersektywnych. Zbadania wymaga, w jakich innych częściach mowy lub pozycjach składniowych da się mówić o podziale na wyrażenia nieintersektywne i intersektywne w sensie uogólnionym -- wyrażenia takie powinny spełniać większość lub wszystkie testy:
\begin{test}
Zaczynając od korzenia drzewa rozbioru składniowego dodajemy kolejne węzły i sprawdzamy, czy ze znaczenia drzewa rozszerzonego wynika znaczenie drzewa mniejszego. Jeśli tak, dodawane węzły są intersektywne w sensie uogólnionym.
\end{test}
\begin{test}
Zaczynamy od liści i ucinamy je po jednym, sprawdzając, czy znaczenie zdania staje się bardziej ogólne. Jeśli tak, ucinane węzły są intersektywne w sensie uogólnionym.
\end{test}
\begin{test}
Porównujemy typy semantyczne wyrażenia złożonego i tego samego wyrażenia bez jednego z podrzędników. Jeżeli pierwszy z tych typów jest podtypem drugiego, podrzędnik jest intersektywny w sensie uogólnionym.
\end{test}
\begin{equation}\label{biegnie}
	\text{Jaś biegnie.} \rightarrow \text{Biegnie.}
\end{equation}
\begin{equation}\label{jablka}
	\text{cztery jabłka} \subset \text{jabłka}
\end{equation}
\begin{equation}\label{fontanna}
\text{Mieszkam nieopodal fontanny.} \stackrel{?}{\rightarrow} \text{Mieszkam nieopodal.}
\end{equation}

Przykłady \ref{biegnie} i \ref{fontanna} ilustrują wynikanie zdań, przykład \ref{jablka} mówi o zawieraniu się denotacji.
Ostatni przykład pokazuje, że rozstrzygnięcia wymaga problem relacji znaczenia jednobrzmiących leksemów przysłówkowych i przyimkowych. Pożądaną opcją wymagającą zweryfikowania jest identyczność znaczeniowa wszystkich lub większości takich par \emph{modulo} argument niejawny przysłówka.

\subsection{Rozbiór składniowy w podejściu semantycznym}

Do wyliczenia reprezentacji semantycznej zdania potrzebne są dane wejściowe -- rozbiór składniowy w podejściu semantycznym. Otrzymanie takiego rozbioru wymaga przekształcenia bardziej niskopoziomowego rozbioru w podejściu morfoskładniowym z użyciem informacji o znaczeniu wzajemnych rzędników. Przekształcenie jest potrzebne, gdy bezpośredni podrzędnik węzła $n$ nie spełnia jego ograniczeń selekcyjnych, ale podrzędnik następnego rzędu (wnuczek $n$) je spełnia. Spełnianie ograniczeń selekcyjnych przez leksem polega na tym, że jego typ semantyczny jest podtypem jednego z typów wskazanych w ograniczeniach selekcyjnych. W takim przypadku podrzędnik pierwszego i drugiego rzędu są zamieniane miejscami w drzewie rozbioru.

Typowym przykładem zadania, w którym opisana wyżej procedura jest nietrywialna, jest rozpoznanie, czy wystąpienie przyimka przewidzianego w ramie składniowej nadrzędnego czasownika jest realizacją tej ramy, jak ,,w głowie'', czy jest to przyimek semantyczny, jak w wyrażeniu ,,w samolocie'' w przykładzie:

\begin{equation}
	\text{\textbf{W samolocie} wielu osobom kręci się \textbf{w głowie}.}
\end{equation}
W rozwijanej dla parsera \eniam{} odmianie podejścia semantycznego
o tym, czy bezpośrednim podrzędnikiem czasownika jest przyimek czy rzeczownik,
decyduje ostatecznie to, czy preferencje selekcyjne argumentu ramy uzgadniają się z typem semantycznym rzeczownika czy frazy przyimkowej.
Podobny problem dotyczy liczebników i rzeczowników w użyciu pojemnikowym (patrz sekcja \ref{pojemniki})

\subsection{Zależność roli tematycznej od cech pozaskładniowych}

Podrzędniki dzielimy na argumenty i modyfikatory rozumiane semantycznie, czyli tak, że rola tematyczna argumentu zależy przede wszystkim od znaczenia nadrzędnika i typu zależności składniowej między nim a podrzędnikiem.
\begin{equation}
	\text{były prezydent}
\end{equation}
Przykładowo, odniesienie podmiotu \emph{myśleć} to \agent{} zdarzenia, odniesienie dopełnienia to jego \theme, a podmiot \emph{czuć} to \experiencer. 
W przypadku modyfikatorów, ich rola tematyczna nie zależy od znaczenia nadrzędnika, a od typu zależności i znaczenia podrzędnika.
% W pewnych wypadkach rola zależy dodatkowo od formy fleksyjnej modyfikatora rzeczownikowego, mimo że rodzaj zależności składniowej nie dostarcza o niej informacji.
Przykładem jest modyfikator rzeczownikowy w narzędniku. Taki modyfikator może wprowadzać role \textsc{Instrument}, \textsc{Time} lub \textsc{Manner}.

\begin{equation}
	\text{Naprawiła szafę \textbf{młotkiem}.} 
\end{equation}
\begin{equation}
	\text{Naprawiła szafę \textbf{wieczorem}.} 
\end{equation}
\begin{equation}
	\text{Naprawiła szafę \textbf{własnym sposobem}.} 
\end{equation}

Podobnie oddzielnej uwagi wymagają rzeczownikowe frazy wolne w celowniku,
przymiotniki, które mają rolę tematyczną inną niż \textsc{Attribute}
i przysłówki, które mają rolę tematyczną inną niż \textsc{Manner}.

\section{Cechy semantyczne wyrażeń nieintersektywnych}

\subsection{Funktorowatość zmieniająca pojęcie}
\begin{itemize}
\item ``były'' traktujemy jako funktor, który modyfikuje pojęcie składniowego nadrzędnika, tworzą nowe.
\begin{equation}
	\text{były prezydent}
\end{equation}
\item ``ledwo'' ma dwa znaczenia:
2. określające położenie pojęcia na jakiejś skali porównawczej, w jakimś porządku np.
\begin{equation}
	\text{Ledwo zdał egzamin}
\end{equation}
Mamy tu funktor który modyfikuje pojęcie będące jego składniowym nadrzędnikiem (analogicznie jak leksem ``były'').
Funktor ten ma też domyślny argument (położenie w jakimś porządku) zależny od nadrzędnika

\item co najmniej: funktor biorący określenie liczby bądź miary i generujący nowe określenie liczby bądź miary.

\end{itemize}
\subsection{Funktorowatość wyłuskująca cechę pojęcia}
Modyfikuje aspekt znaczenia swojego podrzędnika, generując obiekt innego typu semantycznego niż podrzędnik.
To jaki to jest aspekt jest zdaterminowane znaczeniem funktora.
Leksem funktora może mieć kilka znaczeń dotyczących różnych aspektów.
\begin{itemize}
\item ``ledwo'' ma dwa znaczenia:
1. czasowe
\begin{equation}
	\text{Słońce ledwo wzeszło}
\end{equation}
\begin{equation}
	\text{Ledwo zabrał się do pracy, zadzwonił telefon.}
\end{equation}
``ledwo'' traktujemy tu jak funktor biorący zdarzenie i generujący czas chwilę po nim. W funkcji 
spójnika podrzędnego, czas ten określa czas zdarzenia ze zdania nadrzędnego, w funkcji przysłówka 
czas ten określa czas zdarzenia wynikającego z kontekstu, co można wyrazić jako pro-zdarzenie.
\item ``gdy'' czasowe bierze sytuację (zdarzenie) i generuje jej czas.
\begin{equation}
	\text{Wszedł, gdy go wpuścili.}
\end{equation}

\item pod
\begin{equation}
	\text{pod stołem}
\end{equation}
Funktor biorący obiekt i zwracający miejsce, wskazujący, że jest to miejce, gdzie znajduje się nadrzędnik.
\begin{equation}
	\text{pod stół}
\end{equation}
Funktor biorący obiekt i zwracający miejsce, wskazujący, że jest to miejce, do którego zmierza się nadrzędnik.
\item ponieważ: funktor biorący sytuację i generujący CZEMU

\end{itemize}

\subsection{Zależność znaczenia pojęcia od typu semantycznego nadrzędnika - relacyjność}
\begin{itemize}
\item ``ledwo'' ma dwa znaczenia:
2. określające położenie pojęcia na jakiejś skali porównawczej, w jakimś porządku np.
\begin{equation}
	\text{Ledwo zdał egzamin}
\end{equation}
Mamy tu funktor który modyfikuje pojęcie będące jego składniowym nadrzędnikiem (analogicznie jak leksem ``były'').
Funktor ten ma też domyślny argument (położenie w jakimś porządku) zależny od nadrzędnika
\item dużo, niedużo: aż mają domyśny argument zależny od nadrzędnika
``dużo złota'' to inna ilość niż ``dużo węgla''.
\item już, jeszcze, aż mają domyśny argument wskazujący skalę jako czasową, przestrzenną itp.
nadrzędnik  ogranicza znaczenie, niekoniecznie je determinując. (ew. skala jest jednoznacznie przypisana do znaczenia leksemu
a ich wielość wynika z wielości znaczeń leksemu)

\end{itemize}

\subsection{Przyrematyczność}
\begin{itemize}
\item także, również, też: są przyrematyczne, stwierdzają, że istnieją inne obiekty oprócz wskazanego, które spełniają resztę drzewa
\item nawet - sąd dziwi osobę referencyjną, przyrematyczny, miriatywny

\end{itemize}

\subsection{Faktywność} % (fold)
(napisać czym jest, czemu przysługuje, jakie pytania sobie z nią związane zadajemy)

podrzędniki zdaniowe rzeczowników i czasowników z Walentego - faktywność przysługuje pozycji w ramie walencyjnej

spójniki zdaniowe - faktywność przysługuje spójnikowi (zdaniu podrzędnemu). 

faktywność fraz niefinitywnych (bezokolicznikowych, praet, ger, imps, ...), jak poradzić sobie z brakiem czasu gramatycznego


% subsection faktywność (end)

\subsection{Nieweredyczność}
\begin{itemize}
\item ``prawdopodobnie'' modalność aletyczna. (ew. jest to funktor, który wyciąga z sytuacji jej odległość od świata aktualnego)

\item chyba, pewnie, prawdopodobnie, może, być może, na pewno, oczywiście stanowią ocenę stopnia uzasadnienia sądu, modalność epistemiczna

\item tryb rozkazujący
Ze zdania ``Bądź bardziej punktualny'' wynika pragmatycznie, że teraz słuchacz punktualny nie jest.
\end{itemize}

\subsection{Metajęzykowość}
Metajęzykowość jest to orzekanie czegoś o sądzie.
\begin{itemize}
\item więc: zachodzi związek quasiprzyczynowy między koniunktami (sytuacjami)
\end{itemize}
\subsubsection{Odwołanie do osoby referencyjnej, subiektywność}
\begin{itemize}
\item nawet - sąd dziwi osobę referencyjną, przyrematyczny, miriatywny
\item chyba, pewnie: osoba referencyjna potwierdziłaby modyfikowany sąd, gdyby wiedziała, że pewna kontekstualna 
przesłanka jest prawdziwa
\item tryb rozkazujący wskazuje intencje mówcy
\item chociaż, ale, a są miriatywne,poza tym są zwykłą koniunkcją faktów
\end{itemize}

\section{Cechy składniowo-semantyczne jednostek nieopisanych w zasobach} % (fold)

Słownik \emph{Walenty}, który jest podstawowym źródłem sformalizowanej informacji o cechach składniowo-semantycznych jednostek współczesnego języka polskiego, nie obejmuje niektórych klas gramatycznych lub obejmuje je w niedostatecznym stopniu.
Według wstępnego rozeznania wiele z brakujących leksemów rzeczownikowych, przymiotnikowych, przysłówkowych i przyimkowych wnosi wkład znaczeniowy lokalnie, a więc w sposób najbardziej podstawowy. Niezbędny jest jednak ich opis składniowo-semantyczny pozwalający przypisać im odpowiednią rolę tematyczną.
Dalsze potrzeby zasobowe obejmują typ semantyczny leksemów nieopisanych w Słowosieci.
Docelowy \textbf{opis cech składniowo-semantycznych} obejmuje następujące cechy:

\begin{enumerate}
\item Rola tematyczna łącząca wyraz z nadrzędnikiem, jeśli leksem podrzędnika ma ją do siebie przypisaną oraz jest uogólnienie intersektywny
\item Struktura składniowa argumentów leksemu (rama składniowa), np. dla przyimka będzie to
\begin{enumerate}
\item Liczba argumentów jawnych przyimka (por. \emph{między\_a}, \emph{między\_i})
\item Okalającość przyimka (por. \emph{między\_a})
\item Pozycyjność przyimka (por. postpozycyjne  \emph{(100 lat) temu})
\item Przypadek wymagany przez przyimek
\end{enumerate}

\item Preferencje selekcyjne i role tematyczne argumentów intersektywnych w sposób uogólniony
\item Typ semantyczny leksemu, jeśli leksem nie występuje w Słowosieci. Dla przyimków i spójników podrzędnych jest to typ semantyczny frazy, których są elementem głównym.
\end{enumerate}



\section{Oznaczanie miary i liczebności referenta} % (fold)

Szczególnej uwagi wymagają wyrażenia, których typem semantycznym jest \textsf{ILOŚĆ} dla wyrażeń policzalnych lub \textsf{MIARA} dla wyrażeń niepoliczalnych. Ich rolą tematyczną jest odpowiednio \textsc{count} lub \textsc{measure}.


\subsection{Użycie pojemnikowe} % (fold)
\label{pojemniki}
Użycie pojemnikowe rzeczownika to takie, w którym rzeczownik odnosi do miary referenta swojego podrzędnika rzeczownikowego w dopełniaczu w morfoskładniowym (niesemantycznym) podejściu do składni. Przykładem jest ,,łyżka'' w wyrażeniu ,,łyżka cukru''.
Żeby otrzymać rozkład zdania zawierającego rzeczowniki w użyciach pojemnikowych w składni semantycznej, jak Universal Dependencies, traktujemy pojemniki jak podrzędniki swoich tradycyjnych składniowych nadrzędników. Do tego celu niezbędne jest oznaczenie pojemników w języku polskim.

% subsubsection pojemniki (end)

\subsection{Podział liczebników ze względu na dokładność} % (fold)

Liczebniki niewłaściwe (\emph{trochę}, \emph{sporo}) i większość rzeczowników w użyciu oznaczającym miarę nie podaje dokładnej wartości wielkości, którą opisuje. Żeby reprezentować ich znaczenia, trzeba je podzielić ze względu na dokładność, opisać zakres niedokładności i zaproponować spójną koncepcję reprezentacji znaczeń niedokładnych.

% subsubsection podział_liczebników_ze_względu_na_dokładność (end)

\subsection{Miara referentów niepoliczalnych} % (fold)

Do zdefiniowania, które leksemy wprowadzają w języku polskim relację \textsc{Measure} (np. \emph{niedużo}), konieczne jest ich oznaczenie w zasobie typu słownikowego.

% subsubsection miara_referentów_niepoliczalnych (end)


\section{Niedenotatywne funkcje języka} % (fold)
% Pierwotna nazwa podrozdziału: Metatekstowość
Szczególnej uwagi wymagają wszystkie leksemy i konstrukcje pełniące funkcję fatyczną (\emph{halo}), konatywną (np. tryb rozkazujący), emotywną (np. ekspresywizmy) lub metatekstową (np. \emph{że tak powiem}).
Do opisu ostatniej należą złożone zagadnienia znaczenia spójników, wyrażeń przyrematycznych i przytematycznych.

halo, brawo, brr, miau, "że tak powiem"

% temat, remat, implikatura konwencjonalna


% \subsection{Opis przykładów, które nie są uogólnienie intersektywne} % (fold)
% Bycie kwantyfikatorem / nielokalne zmienianie zależności semantycznych
% każdy
% prawdopodobnie
% może
% oczywiście
% tylko
% jeśli w znaczeniu logicznym, 
% Jeśli występuje nielokalne zmienianie zależności semantycznych, to nie ma uogólnionej intersektywności.
% powyższe stwierdzenie niedziała dla metatekstowego "oczywiście", czy jest prawdziwe po pominięciu metatekstowości?
% Reprezentacja semantyczna partykuł.
% % subsection opis_przykładów_które_nie_są_uogólnienie_intersektywne (end)


\section{Zdarzenia, procesy (zmiany stanu rzeczy w czasie)} % (fold)

czym się różni zdarzenie od sytuacji i od sądu
koreferencyjność zdarzeń: ``Zosia ukryła piłkę wkładając ją do szafy'' - dwa czasowniki opisujące to samo zdarzenie.

% subsection zdarzenia_procesy_zmiany_stanu_rzeczy_w_czasie_ (end)

\section{Relacje czasowe i przestrzenne} % (fold)
Relacje czasowe (role tematyczne Time i Duration, pytanie kiedy?) i przestrzenne (role tematyczne Location i Path, pytania gdzie?, skąd?, dokąd?, którędy?)



10. Relacje przyczynowe, celowe, warunkowe i przyzwoleniowe (role Purpose i Condition)



Czy rozróżnienie pomiędzy Purpose i Condition to rozróżnienie pomiędzy intencjonalnymi i nieintecjonalnymi okolicznościami wpływającymi na podjęcie akcji.

% subsection relacje_czasowe_i_przestrzenne (end)



-------------------------------------------------------------------------------------------------------------------------



Dodać w części wyraźnie oddzielonej możliwość przebadania też zjawisk elipsy i metonimii i podobnych - patrz. rozdz. ,,Typy znaczeń wariantowych'' w R. Grzegorczykowa ,,Wprowadzenie do semantyki językoznawczej'' ss. 44-47. Ta część raportu musi być opatrzona zastrzeżeniem, że będzie wykonana tylko w razie, gdy zostanie czasu i środków.

% section lista_zjawisk_semantycznych (end)

\end{document}