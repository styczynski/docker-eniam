\documentclass[a4paper,12pt]{article}
\usepackage[utf8]{inputenc}

\usepackage{amsmath}
\usepackage{amssymb}

\usepackage[T1]{fontenc}
\usepackage[english, polish]{babel}
\usepackage{tikz}
\usetikzlibrary{conceptgraph}
%\DeclareMathSymbol{\vDash}        {\mathrel}{AMSa}{"0F}

\newcommand{\sg}{{\it sg} }
\newcommand{\pl}{{\it pl} }
\newcommand{\mass}{{\it mass} }
\newcommand{\ind}{{\it indexical} }
\newcommand{\corf}{{\it coreferential} }
\newcommand{\deict}{{\it deictic} }
\newcommand{\interr}{{\it interrogative} }

\newcommand{\type}[2]{\text{{\sc type}}(#1,\text{#2})}
\newcommand{\hasName}[2]{\text{{\sc hasName}}(#1,\text{'#2'})}
\newcommand{\dscr}[2]{\text{{\sc dscr}}(#1,#2)}
\newcommand{\init}[2]{\text{{\sc init}}(#1,#2)}
\newcommand{\pres}[1]{\text{{\sc pres}}(#1)}
\newcommand{\indexic}{\text{{\sc indexical}}}
\newcommand{\deictic}{\text{{\sc deictic}}}
\newcommand{\coreferential}{\text{{\sc coreferential}}}

\title{Reprezentacja przydawki dopełniaczowej}
\author{Wojciech Jaworski}
%\date{}

\begin{document}

\maketitle
 
{\it x y.gen}\\
{\sc poss}(x,y)\\
y.has=x\\
x.poss=y\\
has - uogólnienie wszystkich (wielu) atrybutów (podobnie jak relacja ``cecha'')\\
typ semantyczny x i y wyznacza atrybut
 

Algorytm znaczeniowy x definiuje znaczenie relacji poss.\\
x ze swojej natury musi być w tej relacji z jakimś innym bytem.\\
y ze swojej natury też powinno być w tej relacji z jakimś innym bytem.\\
{\it kolor piłki}\\
x=y.kolor\\
${\bf kolor}=\lambda x\; (x.poss).kolor=x$\\
{\it ojciec Józefa}\\ 
y.ojciec=x\\
{\it dziadek Konrada}\\
y.ojciec.ojciec=x\\
{\it student polonistyki}\\
y.student=x\\
{\it student Politechniki}\\
y.student=x\\
{\it waga Konrada}\\
y.waga=x\\
{\it promień koła}\\
y.promień\\
{\it faza ewolucji}\\
y.faza=x\\
{\it kierownictwo budowy}\\

Relacja bycia właścicielem.\\
Nie jest obowiązkowa.\\
{\it piłka Stasia}\\ 
x=y.własność\\
{\it bielizna Ani}\\
y.strój=x\\

Wydarzenia, czynności, czyny - relacja odpowiada roli tematycznej czasownika. - podobne do pierwszej sytuacji.
{\it wizja św. A.}

Użycie pojemnikowe, semantyczny nadrzędnikiem jest y\\
{\it łyżka cukru}\\
y.measure=x
 
{\it suma kwadratów trzech liczb}\\

{\it zbiór osób}\\
 
{\it silnik volkswagena}\\
 
{\it sekwencja zdarzeń}\\
 
{\it graf hiponimii rzeczowników}\\
 
{\it fizyka cieczy}\\
 
  
\section{Relacja quasi-posiadania i identyczności}
Predykat Poss symbolizuje relację posiadania ({\it piłka chłopca}) oraz pozostałe 
relacje wskazywane przez modyfikator rzeczownikowy w dopełniaczu (przydawkę dopełniaczową)
({\it sposób wykonania}, {\it prezeska organizacji}, {\it obrzeża Warszawy}).
Poss jest domyślną rolą dla przydawki dopełniaczowej. W konkretnych przypadkach np. przy użyciu pojemnikowym jest zastępowana inną relacją.
%W toku dalszych badań Poss zostanie podzielona na podklasy
\[\begin{tikzpicture}
\node[concept] (a) {\sg piłka};
\node[relation, right=1cm of a] (b) {Poss};
\node[concept, right=1cm of b] (c) {\sg chłopiec};
\edge {a} {b};
\edge {b} {c};
\end{tikzpicture}\]

Predykat Poss wykorzystujemy również w sytuacjach, gdy pojęcia mają charakter funkcji biorących odniesienie jednego ze 
swych podrzędników i określających swoje odniesienie na tej podstawie, np odniesieniem frazy 
%{\it pod stołem} będzie miejsce znajdujące się poniżej jakiegoś {\it stołu}.
%Podobnie przy frazie 
{\it kolor piłki} mamy {\it piłkę}, z której wyłuskujemy cechę.
% \[\begin{tikzpicture}
% \node[concept] (a) {pod};
% \node[relation, right=1cm of a] (b) {Poss};
% \node[concept, right=1cm of b] (c) {\sg stół};
% \edge {a} {b};
% \edge {b} {c};
% \end{tikzpicture}\]
\[\begin{tikzpicture}
\node[concept] (a) {\sg kolor};
\node[relation, right=1cm of a] (b) {Poss};
\node[concept, right=1cm of b] (c) {\sg piłka};
\edge {a} {b};
\edge {b} {c};
\end{tikzpicture}\]

\section{Testy}
\begin{enumerate}
\item test has
\item ontologiczna obowiązkowość relacji
\item co jest semantycznym nadrzędnikiem, uzgadnia się z preferencjami selekcyjnymi nadrzędnika
\item położenie x i y w ontologii
\item 
\item 
\end{enumerate}


\end{document}