% tikzlibrary.code.tex
%
% Copyright 2010-2011 by Laura Dietz
% Copyright 2013 by Wojciech Jaworski
%
% This file may be distributed and/or modified
%
% 1. under the LaTeX Project Public License and/or
% 2. under the GNU General Public License.
%
% See the files LICENSE_LPPL and LICENSE_GPL for more details.

% Load other libraries
\usetikzlibrary{shapes}
\usetikzlibrary{fit}
\usetikzlibrary{chains}
\usetikzlibrary{arrows}

% Concept
\tikzstyle{concept} = [rectangle,fill=white,draw=black,inner sep=1pt,
minimum size=20pt, font=\fontsize{10}{10}\selectfont, node distance=1]
% Concept
\tikzstyle{virtualconcept} = [font=\fontsize{10}{10}\selectfont, node distance=1]
% Conceptual relation
\tikzstyle{relation} = [circle,fill=white,draw=black,inner sep=1pt,
minimum size=20pt, font=\fontsize{10}{10}\selectfont, node distance=1]
% Context
\tikzstyle{context} = [draw, rectangle, fit=#1]
% Negation
\tikzstyle{negcontext} = [draw, rectangle, rounded corners, fit=#1]

% Constant node
%\tikzstyle{const} = [rectangle, inner sep=0pt, node distance=1]
% Factor node
%\tikzstyle{factor} = [rectangle, fill=black,minimum size=5pt, inner
%sep=0pt, node distance=0.4]
% Deterministic node
%\tikzstyle{det} = [relation, diamond]

% Invisible wrapper node
\tikzstyle{wrap} = [inner sep=0pt, fit=#1]
% Gate
\tikzstyle{gate} = [draw, rectangle, dashed, fit=#1]

% Caption node
\tikzstyle{caption} = [font=\footnotesize, node distance=0] %
\tikzstyle{context caption} = [caption, node distance=0, inner sep=0pt,
above right=5pt and 0pt of #1.north west] %
\tikzstyle{negcontext caption} = [caption, node distance=0, inner sep=0pt,
above right=5pt and 0pt of #1.north west] %

\tikzset{>={triangle 45}}

% \edge [options] {inputs} {outputs}
\newcommand{\edge}[3][]{ %
  % Connect all nodes #2 to all nodes #3.
  \foreach \x in {#2} { %
    \foreach \y in {#3} { %
      \draw[->,#1] (\x) -- (\y) ;%
    } ;
  } ;
}

\newcommand{\coref}[3][]{ %
  % Connect all nodes #2 to all nodes #3.
  \foreach \x in {#2} { %
    \foreach \y in {#3} { %
      \draw[-,dashed,#1] (\x) -- (\y) ;%
    } ;
  } ;
}

% \context [options] {name} {fitlist} {caption}
\newcommand{\context}[4][]{ %
  \node[wrap=#3] (#2-wrap) {}; %
  \node[context caption=#2-wrap] (#2-caption) {#4}; %
  \node[context=(#2-wrap)(#2-caption), #1] (#2) {}; %
}

% \negcontext [options] {name} {fitlist} {caption}
\newcommand{\negcontext}[4][]{ %
  \node[wrap=#3] (#2-wrap) {}; %
  \node[negcontext caption=#2-wrap] (#2-caption) {#4}; %
  \node[negcontext=(#2-wrap)(#2-caption), #1] (#2) {}; %
}


