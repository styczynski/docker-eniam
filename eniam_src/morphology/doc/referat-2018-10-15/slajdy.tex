\documentclass{beamer}
\usepackage[polish]{babel}
\usepackage[T1]{fontenc}
\usepackage[utf8]{inputenc}
\usepackage{amsmath}
\usepackage{amsfonts}
%\usepackage{amsthm}
\usepackage[mathscr]{eucal}
\usepackage{graphicx}
\usepackage[T1]{tipa}
\usepackage{longtable}
\usepackage{multicol}

%\usepackage{amssymb}
\usepackage{bussproofs}
\usepackage{cmll}
\usepackage{xcolor}

\newcommand{\form}{{\it form}}

\DeclareUnicodeCharacter{3B5}{\ensuremath{\varepsilon}}
\DeclareUnicodeCharacter{3C9}{\ensuremath{\omega}}
\DeclareUnicodeCharacter{3BA}{\ensuremath{\kappa}}
\DeclareUnicodeCharacter{3B4}{\ensuremath{\delta}}
\DeclareUnicodeCharacter{2B2}{\textipa{\super{j}}}
\DeclareUnicodeCharacter{2032}{\ensuremath{'}}
\DeclareUnicodeCharacter{1EF}{\textipa{\v{Z}}}

% for themes, etc.
%\mode<presentation>
%{ \usetheme{Boadilla} }
%\definecolor{mycolor}{rgb}{0.4,0.2,0.5}
%\usecolortheme[named=mycolor]{structure}

\mode<presentation>
{ \usetheme{Boadilla} }
%\definecolor{mycolor}{rgb}{0.0,0.4,0.0}
\definecolor{mycolor}{rgb}{0.31,0.2,0.6}
\definecolor{mycolor2}{rgb}{1.0,1.0,1.0}
\usecolortheme[named=mycolor]{structure}
 \pgfdeclareverticalshading{beamer@headfade}{\paperwidth}
  {
    color(0cm)=(mycolor2);
    color(1.25cm)=(mycolor)
  }
\addtoheadtemplate{\pgfuseshading{beamer@headfade}\vskip-1.25cm}{}

\usepackage{times}  

\newcommand{\tensor}{\bullet}
\newcommand{\forward}{\operatorname{/}}
\newcommand{\backward}{\operatorname{\backslash}}
\newcommand{\both}{\mid}
\newcommand{\plus}{\oplus}
\newcommand{\zero}{0}
\newcommand{\one}{1}
\newcommand{\maybe}{?}


\author{Wojciech Jaworski, Szymon Rutkowski}
\title[Wielowarstwowy regułowy model fleksji języka polskiego]{Wielowarstwowy regułowy model fleksji języka polskiego}
\institute[MIM UW, IPI PAN]{Instytut Informatyki Uniwersytetu Warszawskiego\\ $\cdot$\\
Instytut Podstaw Informatyki Polskiej Akademii Nauk}
\date{15 października 2018}

% note: do NOT include a \maketitle line; also note that this title
% material goes BEFORE the \begin{document}

% have this if you'd like a recurring outline
\AtBeginSection[]  % "Beamer, do the following at the start of every section"
{
\begin{frame}<beamer> 
\frametitle{Spis treści} % make a frame titled óutline"
\tableofcontents[currentsection]  % show TOC and highlight current section
\end{frame}
}

\begin{document}

\frame{\titlepage}

\begin{frame}
\frametitle{Model}
\begin{itemize}
\item Model reprezentuje zasady morfologiczne języka polskiego
jako zestaw operacji wykonywanych na obserwowanej formie słowa
prowadzących do przekształcenia jej w lemat i zestaw cech morfoskładniowych.
\item Celem jest stworzenie reprezentacji polskiej fleksji, która
\begin{itemize}
\item jest zwarta i zrozumiała dla człowieka,
\item odzwierciedla strukturę języka,
\item jest precyzyjna w sposób umożliwiający jej bezpośrednią implementację
w postaci {\it odgadywacza} (ang. {\it guesser}) oraz generatora form.
\end{itemize}
\item Model został opracowany na podstawie Słownika Gramatycznego Języka Polskiego 
w wersji z 30.07.2017.
%Guesser jest to program, który ma za zadanie przypisywanie („odgadywanie”) 
%cech morfosyntaktycznych segmentu na podstawie jego formy, to znaczy przede wszystkim rozpoznawalnych afiksów. 
%Widząc, powiedzmy, napis \textit{burachnajami}, można od razu wziąć pod uwagę, 
%że formą podstawową wyrazu jest \textit{burachnaj}, a do niej dodano końcówkę narzędnika liczby mnogiej \textit{-ami}.
%Różni się on od {\it analizatora morfologicznego} \cite{Morfeusz} tym, że jego działanie nie jest ograniczone do 
%zamkniętego zbioru słów pochodzących z dostarczonego programowi słownika gramatycznego.
%Próbie interpretacji może zostać poddany dowolny napis.
\end{itemize}
\end{frame}

\begin{frame}
\frametitle{Zakres analizy}
\begin{itemize}
\item Tworząc model skupiliśmy się na produktywnej części polskiej fleksji,
by uchwycić odmianę słów nowych, nieznanych, nie należących do słownika.
\item Model nie obejmuje nieregularnych czasowników oraz niewielkiej liczby słów 
należących do innych części mowy o nieregularnej odmianie.
\begin{itemize}
\item Wynika to stąd, że znany zamknięty zbiór słów można 
zawrzeć w słowniczku załączonym do modelu. 
\end{itemize}
\item Model nie analizuje również form które nie mają widocznych cech fleksyjnych
takich jak 
\begin{itemize}
\item znaki interpunkcyjne,
\item liczby, daty, itp.  zapisane cyframi,
\item skróty.
\end{itemize}
\item Model obejmuje 
\begin{itemize}
\item odmianę akronimów,
\item odmianę słów o niepolskiej ortografii,
\item niektóre formy gwarowe.
\end{itemize}
\end{itemize}
\end{frame}

\begin{frame}
\frametitle{Niejednoznaczność}
\begin{itemize}
\item Zadania lematyzacji i anotacji morfosyntaktycznej 
nie da się wykonać w sposób jednoznaczny jedynie na podstawie 
obserwacji pojedynczej, wyrwanej z kontekstu formy.
\item Guesser określa z pomocą swoich reguł jedynie zbiór możliwych interpretacji.
\item Mogą stanowić one dane wejściowe dla {\it taggera}
przeprowadzającego dezambiguację morfosyntaktyczną na podstawie modeli statystycznych.
\end{itemize}
\end{frame}

\begin{frame}
\frametitle{Warstwy}
\begin{enumerate}
\item Warstwa \textbf{ortograficzno-fonetyczna} abstrahuje od polskiej ortografii przez przekonwertowanie formy segmentu do wewnętrznej reprezentacji, odzwierciedlającej prawidłowości morfonologiczne języka.
\item Warstwa \textbf{analityczna} generuje lemat oraz określa występujące afiksy.
\item Warstwa \textbf{interpretacji} nadaje segmentowi interpretację morfosyntaktyczną na podstawie wykrytych afiksów.
\item Warstwa korygująca wygenerowane formy i lematy zawierające wygłos.
\end{enumerate}
\end{frame}

\begin{frame}
\frametitle{Warstwa ortograficzno-fonetyczna}
%dla standardowej ortografii
\begin{itemize}
\item Zadania warstwy ortograficzno-fonetycznej to:
\begin{itemize}
\item wprowadzenie zasady {\it jeden znak --- jeden dźwięk},
\item wprowadzenie operatora palatalizacji,
\item ujednolicenie ortografii, przykładowo:
\begin{itemize}
\item w słowie „Franz” piszemy przez „z” na końcu, \\czytamy „c” i odmieniamy tak, jak słowa kończące się na „c”
\item w słowie „ZOZ” piszemy przez „z” na końcu, \\czytamy „z” i odmieniamy tak, jak słowa kończące się na „z”
\item w słowie „NFZ” piszemy przez „z” na końcu, \\czytamy „zet” i odmieniamy tak, jak słowa kończące się na „t”
\end{itemize}
\end{itemize}
\item Konwersja jest odwracalna, ale nie jest jednoznaczna.
\item Celem przeprowadzenia tej konwersji jest uproszczenie kolejnych reguł, które mogą korzystać z uogólnień dokonanych już przez tę warstwę.
\end{itemize}
\end{frame}
%marznąć - wyjątek

\begin{frame}
\frametitle{}
\begin{itemize}
\item W polskim zapisie ortograficznym formy zawierające ten sam rdzeń często różnią się.
\item Widać to na przykład w ciągu wyrazów: \textit{pani}, \textit{pań}, \textit{panie}.
\item Za pomocą reguł 
\begin{center}
\begin{tabular}{l|l}
reguła & prawy kontekst \\
\hline
n$'$ $\leftarrow$ ni & a ą e ę o ó u\\
n$'$ $\leftarrow$ n & i {\it sylabotwórcze}\\
n$'$ $\leftarrow$ ń & {\it spółgłoska lub wygłos}\\
\end{tabular}
\end{center}

można je przekształcić do postaci: \texttt{pan$'$i}, \texttt{pan$'$}, \texttt{pan$'$e},\\
gdzie dobrze widoczny jest wspólny rdzeń (\texttt{pan$'$}).
\item Domyślna reguła przepisuje znak wejściowy bez zmian; uruchamia się ona, kiedy żadna z innych reguł nie znajduje zastosowania.
\item Stosowalność reguł wymaga
\begin{itemize}
\item dopasowania ciągu znaków podlegającego przekształceniu, 
\item dopasowania prawego kontekstu (ciągu znaków następującego bezpośrednio po ciągu przekształcanym).
\end{itemize}
\end{itemize}
\end{frame}

\begin{frame}
\frametitle{Wybrane reguły ortograficzno-fonetyczne}
\begin{center}
\begin{tabular}{l|l|l|l}
reguła & reguła & reguła & prawy kontekst \\
\hline
b$'$ $\leftarrow$ bi & p$'$ $\leftarrow$ pi & m$'$ $\leftarrow$ mi & a ą e ę o ó u\\
b$'$ $\leftarrow$ b & p$'$ $\leftarrow$ p & m$'$ $\leftarrow$ m & i {\it sylabotwórcze}\\
\hline
v$'$ $\leftarrow$ wi & f$'$ $\leftarrow$ fi &  & a ą e ę o ó u\\
v$'$ $\leftarrow$ w & f$'$ $\leftarrow$ f &  & i {\it sylabotwórcze}\\
v $\leftarrow$ w &  & x $\leftarrow$ ch & {\it litera inna niż } i \\
\hline
d$'$ $\leftarrow$ dzi & t$'$ $\leftarrow$ ci & n$'$ $\leftarrow$ ni & a ą e ę o ó u\\
d$'$ $\leftarrow$ dz & t$'$ $\leftarrow$ c & n$'$ $\leftarrow$ n & i {\it sylabotwórcze}\\
d$'$ $\leftarrow$ dź & t$'$ $\leftarrow$ ć & n$'$ $\leftarrow$ ń & {\it spółgłoska lub wygłos}\\
\hline
z$'$ $\leftarrow$ zi & s$'$ $\leftarrow$ si &  & a ą e ę o ó u\\
z$'$ $\leftarrow$ z & s$'$ $\leftarrow$ s &  & i {\it sylabotwórcze}\\
z$'$ $\leftarrow$ ź & s$'$ $\leftarrow$ ś &  & {\it spółgłoska lub wygłos}\\
\hline
\textipa{\v{Z}} $\leftarrow$ dż & č $\leftarrow$ cz & \textipa{Z} $\leftarrow$ dz & {\it litera inna niż } i\\
\hline
ž $\leftarrow$ ż & š $\leftarrow$ sz & ř $\leftarrow$ rz & {\it litera inna niż } i\\
\hline
g$'$ $\leftarrow$ gi & k$'$ $\leftarrow$ ki &  & a ą e ę o ó u\\
g$'$ $\leftarrow$ g & k$'$ $\leftarrow$ k &  & i {\it sylabotwórcze}\\
\end{tabular}
\end{center}
\end{frame}

\begin{frame}
\frametitle{Podział głosek}
\begin{itemize}
\item Głoski dzielimy na funkcjonalnie miękkie i funkcjonalnie twarde.
\item Funkcjonalnie twarde to takie, które można zmiękczyć,\\ należą do nich:
\begin{center}
b, x, d, f, g, h, k, ł, m, n, p, r, s, t, v, z
\end{center}
\item Funkcjonalnie miękkie to takie, których zmiękczyć się nie da,\\ należą do nich:
\begin{center}
b$'$, t$'$, d$'$, f$'$, m$'$, n$'$, p$'$, s$'$, v$'$, z$'$, l, c, č, \textipa{Z}, \textipa{\v{Z}}, ř, š, ž
\end{center}
\item Dalszą analizę będziemy wykonywać osobno dla słów mających funkcjonalnie twardą ostatnią głoskę rdzenia (np. {\it pan}, {\it gwiazda})
i tych, które mają ją funkcjonalnie miękką (np. {\it pani}, {\it kość}).
\item Mają one bowiem różne paradygmaty odmiany ze względu na możliwość wystąpienia sufiksów zmiękczających.
\end{itemize}
\end{frame}

\begin{frame}
\frametitle{Sufiksy i alternacje}
W poniższej tabeli znajdują się wybrane formy rzeczowników {\it gwiazda}, {\it sąsiad}, {\it szpieg} i {\it waga} oraz przymiotników {\it rudy} i {\it nagi} uporządkowane według końcówek.

\begin{center}
\begin{tabular}{l|l|l|l|l|l|l}
-a & gwiazda & sąsiada & ruda & szpiega & waga & naga \\
-om & gwiazdom & sąsiadom &  & szpiegom & wagom &  \\
%ów &  & sąsiadów &  & szpiegów &  &  \\
%owi &  & sąsiadowi &  & szpiegowi &  &  \\
-ą & gwiazdą &  & rudą &  & wagą & nagą \\
%ę & gwiazdę &  &  &  & wagę &  \\
%o & gwiazdo &  &  &  & wago &  \\
%u &  &  &  & szpiegu &  &  \\
-e &  &  & rude & & & nag{\color{red}i}e \\
%ego &  &  & rudego & & & nagiego \\
%ej &  &  & rudej & & & nagiej \\
-em &  & sąsiadem & & szpieg{\color{red}i}em &  &  \\
%emu &  &  & rudemu & & & nagiemu \\
-y & gwiazdy & sąsiady & rudy & szpieg{\color{red}i} & wag{\color{red}i} & nag{\color{red}i} \\
%ych &  &  & rudych &  &  & nagich \\
%ym &  &  & rudym &  &  & nagim \\
%ymi &  &  & rudymi &  &  & nagimi \\
-i &  & sąsi{\color{red}edz}i & ru{\color{red}dz}i & szpie{\color{red}dzy} &  & na{\color{red}dzy} \\
-ie & gwi{\color{red}eźdz}ie & sąsi{\color{red}edz}ie &  &  & wa{\color{red}dz}e &  \\
- & gwiazd & sąsiad &  & szpieg & wag &  \\
\end{tabular}
\end{center}
Na czerwono zaznaczone są alternacje rdzenia i zmiany końcówki.
\end{frame}

\begin{frame}
\frametitle{Sufiksy i alternacje}
Zastosowanie reguł warstwy ortograficzno-fonetycznej upraszcza alternacje.

\begin{center}
\begin{tabular}{l|l|l|l|l|l|l}
-a & gv$'$azda & sąs$'$ada & ruda & szp$'$ega & waga & naga \\
-om & gv$'$azdom & sąs$'$adom &  & szp$'$egom & wagom &  \\
-ą & gv$'$azdą &  & rudą &  & wagą & nagą \\
-e &  &  & rude & & & nag{\color{red}$'$}e \\
-em &  & sąs$'$adem & & szp$'$eg{\color{red}$'$}em &  &  \\
-y & gv$'$azdy & sąs$'$ady & rudy & szp$'$eg{\color{red}$'$i} & wag{\color{red}$'$i} & nag{\color{red}$'$i} \\
-$'$i &  & sąs$'${\color{red}e}d$'$i & rud$'$i & szp$'$e{\color{red}\textipa{Z}y} &  & na{\color{red}\textipa{Z}y} \\
-$'$e & gv$'${\color{red}ez$'$}d$'$e & sąs$'${\color{red}e}d$'$e &  &  & wa{\color{red}\textipa{Z}}e &  \\
-$\varepsilon$ & gv$'$azd & sąs$'$ad &  & szp$'$eg & wag &  \\
\end{tabular}
\end{center}
\end{frame}

\begin{frame}
\frametitle{Rodzaje sufiksów}
\begin{itemize}
\item Ze względu na występujące alternacje końcówki możemy podzielić na 
\begin{itemize}
\item neutralne: -a, -ami, -ach, -om, -o, -u 
\item zmiękczające głoski g i k: -e, -ego, -ej, -em, -emu
\item -y występujące czasami jako -i: -y, -ych, -ym, -ymi
\item zmiękczające -$'$i
\item zmiękczające -$'$e
\item wygłos -$\varepsilon$
\end{itemize}
\item W przypadku głoski g występują dwa rodzaje zmiękczenia: zamiana na g$'$ oraz zamiana na \textipa{Z}.
\item Z uwagi na to, że w formie adj:pl:nom:m1:pos w przypadku głoski g następuje zamiana na \textipa{Z}y, a w przypadku 
innych głosek mamy tu typową palatalizację, uznajemy \textipa{Z}y za efekt działania zmiękczającego -$'$i.
\item Analogicznie postępujemy w przypadku paradygmatów rzeczownikowych zmiękczającego -$'$i oraz zmiękczającego -$'$e.
\end{itemize}
\end{frame}

\begin{frame}
\frametitle{Grupy alternacyjne}
Dla każdej końcówki możemy wypisać występujące przy nich alternacje:

\begin{longtable}{r|rrrrrr}
 & \boldmath$\alpha${\rm y} & \boldmath$\alpha${\rm e} & \boldmath$\alpha$ & \boldmath$\alpha${\rm i} & \boldmath$\alpha${\rm ie} & \boldmath$\alpha\varepsilon$\\
\hline
d & dy $\rightarrow$ d & de $\rightarrow$ d & d $\rightarrow$ d & ed$'$i $\rightarrow$ ad & ed$'$e $\rightarrow$ ad & d $\rightarrow$ d\\
 &  &  &  &  & ez$'$d$'$e $\rightarrow$ azd & \\
\hline
g & g$'$i $\rightarrow$ g & g$'$e $\rightarrow$ g & g $\rightarrow$ g & \textipa{Z}y $\rightarrow$ g & \textipa{Z}e $\rightarrow$ g & g $\rightarrow$ g\\
\end{longtable}
\begin{itemize}
\item W nagłówku tabeli umieszczone są nazwy grup alternacji.
\item Nazwy składają się z 
\begin{itemize}
\item symbolu \boldmath$\alpha$ oznaczającego głoskę funkcjonalnie twardą oraz
\item jednej lub dwu liter oznaczających sufiks.
\end{itemize}
\item Zaznaczone w nazwie litery sufiksu są włączone do alternacji.
\end{itemize}
\end{frame}

\begin{frame}
\frametitle{Tabela alternacji dla głosek funkcjonalnie twardych}
\begin{scriptsize}\begin{longtable}{r|rrrrrr}
 & \boldmath$\alpha${\rm y} & \boldmath$\alpha${\rm e} & \boldmath$\alpha$ & \boldmath$\alpha${\rm i} & \boldmath$\alpha${\rm ie} & \boldmath$\alpha\varepsilon$\\
\hline
x & xy $\rightarrow$ x & xe $\rightarrow$ x & x $\rightarrow$ x & s$'$i $\rightarrow$ x & še $\rightarrow$ x & x $\rightarrow$ x\\
 &  &  &  &  &  & ex $\rightarrow$ x\\
% &  &  &  &  &  & x $\rightarrow$ ks\\
\hline
d & dy $\rightarrow$ d & de $\rightarrow$ d & d $\rightarrow$ d & d$'$i $\rightarrow$ d & d$'$e $\rightarrow$ d & d $\rightarrow$ d\\
 &  &  &  & ed$'$i $\rightarrow$ ad & z$'$d$'$e $\rightarrow$ zd & ed $\rightarrow$ d\\
 &  &  &  &  & ed$'$e $\rightarrow$ ad & ód $\rightarrow$ od\\
 &  &  &  &  & ed$'$e $\rightarrow$ od & ąd $\rightarrow$ ęd\\
 &  &  &  &  & ez$'$d$'$e $\rightarrow$ azd & \\
\hline
f & fy $\rightarrow$ f & fe $\rightarrow$ f & f $\rightarrow$ f & f$'$i $\rightarrow$ f & f$'$e $\rightarrow$ f & f $\rightarrow$ f\\
\hline
h & hy $\rightarrow$ h & he $\rightarrow$ h & h $\rightarrow$ h & z$'$i $\rightarrow$ h & še $\rightarrow$ h & h $\rightarrow$ h\\
 &  &  &  &  & že $\rightarrow$ h & \\
\hline
m & my $\rightarrow$ m & me $\rightarrow$ m & m $\rightarrow$ m & m$'$i $\rightarrow$ m & m$'$e $\rightarrow$ m & m $\rightarrow$ m\\
 &  &  &  & s$'$m$'$i $\rightarrow$ sm & s$'$m$'$e $\rightarrow$ sm & em $\rightarrow$ m\\
\hline
r & ry $\rightarrow$ r & re $\rightarrow$ r & r $\rightarrow$ r & řy $\rightarrow$ r & ře $\rightarrow$ r & r $\rightarrow$ r\\
 &  &  &  &  & eře $\rightarrow$ ar & er $\rightarrow$ r\\
 &  &  &  &  & etře $\rightarrow$ atr & $'$er $\rightarrow$ r\\
 &  &  &  &  & ře $\rightarrow$ rr & ór $\rightarrow$ or\\
 &  &  &  &  &  & $\star$cer $\rightarrow$ kr\\
 &  &  &  &  &  & óbr $\rightarrow$ obr\\
 &  &  &  &  &  & óstr $\rightarrow$ ostr\\
\hline
k & k$'$i $\rightarrow$ k & k$'$e $\rightarrow$ k & k $\rightarrow$ k & cy $\rightarrow$ k & ce $\rightarrow$ k & k $\rightarrow$ k\\
 &  &  &  &  &  & ek $\rightarrow$ k\\
 &  &  &  &  &  & ąk $\rightarrow$ ęk\\
\hline
\dots & \dots
\end{longtable}\end{scriptsize}
\end{frame}

\begin{frame}
\frametitle{Reguły analityczne}
\begin{itemize}
\item Możemy teraz zdefiniować reguły opisujące zmiany następujące podczas dodawania sufiksu do rdzenia.
\item Każda reguła składa się z opisu modyfikacji wykonywanych na przetwarzanej formie oraz zbioru definiowanych atrybutów.
\item Reguły te są parametryzowane przez grupy alternacyjne.
\item Przykładowa reguła ucinająca końcówkę „ego” u przymiotników:
\begin{center}
\begin{tabular}{ll}
$-$\boldmath$\alpha${\rm e} go & flex:=ego palat:=n cat:=adj\\
\end{tabular}
\end{center}
\item Po zastąpieniu $\boldmath\alpha{\rm e}$ przez kolejne alternacje należące do tej grupy otrzymujemy reguły
\begin{center}
\begin{tabular}{ll}
dego $\rightarrow$ d & flex:=ego palat:=n con:=d cat:=adj\\
g$'$ego $\rightarrow$ g & flex:=ego palat:=n con:=g cat:=adj
\end{tabular}
\end{center}
zamieniające {\it rudego} na {\it rud} oraz {\it nag$'$ego} na {\it nag}.
\item Wartość atrybutu con jest dodawana podczas rozwijania reguły na podstawie wybranej alternacji.
\end{itemize}
\end{frame}

\begin{frame}
\frametitle{Reguły ucinające sufiks formy i dodające sufiks lematu rzeczownika z głoską funkcjonalnie twardą}
\begin{scriptsize}\[
\left[\begin{array}{ll}
-\text{\boldmath$\alpha${\rm y}} & \text{flex}:=\text{y}, \downarrow, \text{noun}\\
%-\text{\boldmath$\alpha${\rm y}x} & \text{flex}:=\text{ych}, \downarrow, \text{noun}\\
%-\text{\boldmath$\alpha${\rm y}m} & \text{flex}:=\text{ym}, \downarrow, \text{noun}\\
%-\text{\boldmath$\alpha${\rm y}m$'$i} & \text{flex}:=\text{ymi}, \downarrow, \text{noun}\\
-\text{\boldmath$\alpha${\rm e}} & \text{flex}:=\text{e}, \downarrow, \text{noun}\\
%-\text{\boldmath$\alpha${\rm e}go} & \text{flex}:=\text{ego}, \downarrow, \text{noun}\\
%-\text{\boldmath$\alpha${\rm e}j} & \text{flex}:=\text{ej}, \downarrow, \text{noun}\\
-\text{\boldmath$\alpha${\rm e} m} & \text{flex}:=\text{em}, \downarrow, \text{noun}\\
%-\text{\boldmath$\alpha${\rm e}mu} & \text{flex}:=\text{emu}, \downarrow, \text{noun}\\
-\text{\boldmath$\alpha$ a} & \text{flex}:=\text{a}, \downarrow, \text{noun}\\
-\text{\boldmath$\alpha$ ax} & \text{flex}:=\text{ach}, \downarrow, \text{noun}\\
-\text{\boldmath$\alpha$ am$'$i} & \text{flex}:=\text{ami}, \downarrow, \text{noun}\\
-\text{\boldmath$\alpha$ ą} & \text{flex}:=\text{ą}, \downarrow, \text{noun}\\
-\text{\boldmath$\alpha$ ę} & \text{flex}:=\text{ę}, \downarrow, \text{noun}\\
-\text{\boldmath$\alpha$ o} & \text{flex}:=\text{o}, \downarrow, \text{noun}\\
-\text{\boldmath$\alpha$ om} & \text{flex}:=\text{om}, \downarrow, \text{noun}\\
-\text{\boldmath$\alpha$ ov$'$i} & \text{flex}:=\text{owi}, \downarrow, \text{noun}\\
-\text{\boldmath$\alpha$ ov$'$e} & \text{flex}:=\text{owie}, \downarrow, \text{noun}\\
-\text{\boldmath$\alpha$ óv} & \text{flex}:=\text{ów}, \downarrow, \text{noun}\\
-\text{\boldmath$\alpha$ u} & \text{flex}:=\text{u}, \downarrow, \text{noun}\\
-\text{\boldmath$\alpha$ um} & \text{flex}:=\text{um}, \downarrow, \text{noun}\\
-\text{\boldmath$\alpha${\rm i}} & \text{flex}:=\text{i}, \downarrow, \text{noun}\\
-\text{\boldmath$\alpha${\rm ie}} & \text{flex}:=\text{ie}, \downarrow, \text{noun}\\
-\text{\boldmath$\alpha\varepsilon$} & \text{flex}:=\text{$\varepsilon$}, \downarrow, \text{noun}\\
\star-\text{\boldmath$\alpha\varepsilon$ m$'$i} & \text{flex}:=\text{ami}, \downarrow, \text{noun}\\
\end{array}\right] \otimes \left[\begin{array}{ll}
+\text{\boldmath$\alpha${\rm y}} & \text{lemma}:=\text{y}\\
+\text{\boldmath$\alpha${\rm e}} & \text{lemma}:=\text{e}\\
+\text{\boldmath$\alpha$ a} & \text{lemma}:=\text{a}\\
+\text{\boldmath$\alpha$ o} & \text{lemma}:=\text{o}\\
+\text{\boldmath$\alpha$ ov$'$e} & \text{lemma}:=\text{owie}\\
+\text{\boldmath$\alpha$ um} & \text{lemma}:=\text{um}\\
\star+\text{\boldmath$\alpha$ us} & \text{lemma}:=\text{us}\\
+\text{\boldmath$\alpha${\rm i}} & \text{lemma}:=\text{i}\\
+\text{\boldmath$\alpha\varepsilon$} & \text{lemma}:=\text{$\varepsilon$}
\end{array}\right]
\]\end{scriptsize}\\
Reguły dla końcówek ych, ym, ymi, ego, ej, emu zostały pominięte.\\
Symbol $+$ oznacza, że reguła przykleja sufiks.\\
\end{frame}

\begin{frame}
\frametitle{Rozpakowywanie reguł}
Rozpatrzmy alternacje
\begin{scriptsize}\begin{longtable}{r|rrrrrr}
 & \boldmath$\alpha${\rm y} & \boldmath$\alpha${\rm e} & \boldmath$\alpha$ & \boldmath$\alpha${\rm i} & \boldmath$\alpha${\rm ie} & \boldmath$\alpha\varepsilon$\\
\hline
d & dy $\rightarrow$ d & de $\rightarrow$ d & d $\rightarrow$ d & ed$'$i $\rightarrow$ ad & ed$'$e $\rightarrow$ ad & d $\rightarrow$ d\\
 &  &  &  &  & ez$'$d$'$e $\rightarrow$ azd & \\
\hline
g & g$'$i $\rightarrow$ g & g$'$e $\rightarrow$ g & g $\rightarrow$ g & \textipa{Z}y $\rightarrow$ g & \textipa{Z}e $\rightarrow$ g & g $\rightarrow$ g\\
\end{longtable}\end{scriptsize}
oraz reguły analityczne
\begin{scriptsize}\[
\left[\begin{array}{ll}
-\text{\boldmath$\alpha$ a} & \text{flex}:=\text{a}, \downarrow, \text{noun}\\
-\text{\boldmath$\alpha${\rm e} m} & \text{flex}:=\text{em}, \downarrow, \text{noun}\\
-\text{\boldmath$\alpha${\rm ie}} & \text{flex}:=\text{ie}, \downarrow, \text{noun}\\
\end{array}\right] \otimes \left[\begin{array}{ll}
+\text{\boldmath$\alpha$ a} & \text{lemma}:=\text{a}\\
+\text{\boldmath$\alpha\varepsilon$} & \text{lemma}:=\text{$\varepsilon$}
\end{array}\right]
\]\end{scriptsize}
Po rozwinięciu alternacji otrzymamy reguły:
\begin{scriptsize}\[
\left[\begin{array}{ll}
\text{da $\rightarrow$ d} & \text{flex}:=\text{a}, \downarrow, \text{con}:=\text{d}, \text{noun}\\
\text{ga $\rightarrow$ g} & \text{flex}:=\text{a}, \downarrow, \text{con}:=\text{g}, \text{noun}\\
\text{dem $\rightarrow$ d} & \text{flex}:=\text{em}, \downarrow, \text{con}:=\text{d}, \text{noun}\\
\text{g$'$em $\rightarrow$ g} & \text{flex}:=\text{em}, \downarrow, \text{con}:=\text{g}, \text{noun}\\
\text{ed$'$e $\rightarrow$ ad} & \text{flex}:=\text{ie}, \downarrow, \text{con}:=\text{d}, \text{noun}\\
\text{ez$'$d$'$e $\rightarrow$ azd} & \text{flex}:=\text{ie}, \downarrow, \text{con}:=\text{d}, \text{noun}\\
\text{\textipa{Z}e $\rightarrow$ g} & \text{flex}:=\text{ie}, \downarrow, \text{con}:=\text{g}, \text{noun}\\
\end{array}\right] \otimes \left[\begin{array}{ll}
\text{d $\rightarrow$ da} & \text{lemma}:=\text{a}\\
\text{g $\rightarrow$ ga} & \text{lemma}:=\text{a}\\
\text{d $\rightarrow$ d} & \text{lemma}:=\text{$\varepsilon$}\\
\text{g $\rightarrow$ g} & \text{lemma}:=\text{$\varepsilon$}
\end{array}\right]
\]\end{scriptsize}
\end{frame}

\begin{frame}
\frametitle{Rozpakowywanie reguł cd.}
\begin{scriptsize}\[
\left[\begin{array}{ll}
\text{da $\rightarrow$ d} & \text{flex}:=\text{a}, \downarrow, \text{con}:=\text{d}, \text{noun}\\
\text{ga $\rightarrow$ g} & \text{flex}:=\text{a}, \downarrow, \text{con}:=\text{g}, \text{noun}\\
\text{dem $\rightarrow$ d} & \text{flex}:=\text{em}, \downarrow, \text{con}:=\text{d}, \text{noun}\\
\text{g$'$em $\rightarrow$ g} & \text{flex}:=\text{em}, \downarrow, \text{con}:=\text{g}, \text{noun}\\
\text{ed$'$e $\rightarrow$ ad} & \text{flex}:=\text{ie}, \downarrow, \text{con}:=\text{d}, \text{noun}\\
\text{ez$'$d$'$e $\rightarrow$ azd} & \text{flex}:=\text{ie}, \downarrow, \text{con}:=\text{d}, \text{noun}\\
\text{\textipa{Z}e $\rightarrow$ g} & \text{flex}:=\text{ie}, \downarrow, \text{con}:=\text{g}, \text{noun}\\
\end{array}\right] \otimes \left[\begin{array}{ll}
\text{d $\rightarrow$ da} & \text{lemma}:=\text{a}\\
\text{g $\rightarrow$ ga} & \text{lemma}:=\text{a}\\
\text{d $\rightarrow$ d} & \text{lemma}:=\text{$\varepsilon$}\\
\text{g $\rightarrow$ g} & \text{lemma}:=\text{$\varepsilon$}
\end{array}\right]
\]\end{scriptsize}
Teraz łączymy reguły z pierwszej kolumny z tymi z kolumny drugiej:
\begin{scriptsize}\[
\begin{array}{ll}
\text{da $\rightarrow$ da} & \text{flex}:=\text{a}, \downarrow, \text{con}:=\text{d}, \text{lemma}:=\text{a}, \text{noun}\\
\text{ga $\rightarrow$ ga} & \text{flex}:=\text{a}, \downarrow, \text{con}:=\text{g}, \text{lemma}:=\text{a}, \text{noun}\\
\text{dem $\rightarrow$ da} & \text{flex}:=\text{em}, \downarrow, \text{con}:=\text{d}, \text{lemma}:=\text{a}, \text{noun}\\
\text{g$'$em $\rightarrow$ ga} & \text{flex}:=\text{em}, \downarrow, \text{con}:=\text{g}, \text{lemma}:=\text{a}, \text{noun}\\
\text{ed$'$e $\rightarrow$ ada} & \text{flex}:=\text{ie}, \downarrow, \text{con}:=\text{d}, \text{lemma}:=\text{a}, \text{noun}\\
\text{ez$'$d$'$e $\rightarrow$ azda} & \text{flex}:=\text{ie}, \downarrow, \text{con}:=\text{d}, \text{lemma}:=\text{a}, \text{noun}\\
\text{\textipa{Z}e $\rightarrow$ ga} & \text{flex}:=\text{ie}, \downarrow, \text{con}:=\text{g}, \text{lemma}:=\text{a}, \text{noun}\\
\text{da $\rightarrow$ d} & \text{flex}:=\text{a}, \downarrow, \text{con}:=\text{d}, \text{lemma}:=\text{$\varepsilon$}, \text{noun}\\
\text{ga $\rightarrow$ g} & \text{flex}:=\text{a}, \downarrow, \text{con}:=\text{g}, \text{lemma}:=\text{$\varepsilon$}, \text{noun}\\
\text{dem $\rightarrow$ d} & \text{flex}:=\text{em}, \downarrow, \text{con}:=\text{d}, \text{lemma}:=\text{$\varepsilon$}, \text{noun}\\
\text{g$'$em $\rightarrow$ g} & \text{flex}:=\text{em}, \downarrow, \text{con}:=\text{g}, \text{lemma}:=\text{$\varepsilon$}, \text{noun}\\
\text{ed$'$e $\rightarrow$ ad} & \text{flex}:=\text{ie}, \downarrow, \text{con}:=\text{d}, \text{lemma}:=\text{$\varepsilon$}, \text{noun}\\
\text{ez$'$d$'$e $\rightarrow$ azd} & \text{flex}:=\text{ie}, \downarrow, \text{con}:=\text{d}, \text{lemma}:=\text{$\varepsilon$}, \text{noun}\\
\text{\textipa{Z}e $\rightarrow$ g} & \text{flex}:=\text{ie}, \downarrow, \text{con}:=\text{g}, \text{lemma}:=\text{$\varepsilon$}, \text{noun}\\
\end{array}
\]\end{scriptsize}
\end{frame}

\begin{frame}
\frametitle{Rozpakowywanie reguł cd.}
\begin{scriptsize}\[
\begin{array}{ll}
\text{da $\rightarrow$ da} & \text{flex}:=\text{a}, \downarrow, \text{con}:=\text{d}, \text{lemma}:=\text{a}, \text{noun}\\
\text{ga $\rightarrow$ ga} & \text{flex}:=\text{a}, \downarrow, \text{con}:=\text{g}, \text{lemma}:=\text{a}, \text{noun}\\
\text{dem $\rightarrow$ da} & \text{flex}:=\text{em}, \downarrow, \text{con}:=\text{d}, \text{lemma}:=\text{a}, \text{noun}\\
\text{g$'$em $\rightarrow$ ga} & \text{flex}:=\text{em}, \downarrow, \text{con}:=\text{g}, \text{lemma}:=\text{a}, \text{noun}\\
\text{ed$'$e $\rightarrow$ ada} & \text{flex}:=\text{ie}, \downarrow, \text{con}:=\text{d}, \text{lemma}:=\text{a}, \text{noun}\\
\text{ez$'$d$'$e $\rightarrow$ azda} & \text{flex}:=\text{ie}, \downarrow, \text{con}:=\text{d}, \text{lemma}:=\text{a}, \text{noun}\\
\text{\textipa{Z}e $\rightarrow$ ga} & \text{flex}:=\text{ie}, \downarrow, \text{con}:=\text{g}, \text{lemma}:=\text{a}, \text{noun}\\
\text{da $\rightarrow$ d} & \text{flex}:=\text{a}, \downarrow, \text{con}:=\text{d}, \text{lemma}:=\text{$\varepsilon$}, \text{noun}\\
\text{ga $\rightarrow$ g} & \text{flex}:=\text{a}, \downarrow, \text{con}:=\text{g}, \text{lemma}:=\text{$\varepsilon$}, \text{noun}\\
\text{dem $\rightarrow$ d} & \text{flex}:=\text{em}, \downarrow, \text{con}:=\text{d}, \text{lemma}:=\text{$\varepsilon$}, \text{noun}\\
\text{g$'$em $\rightarrow$ g} & \text{flex}:=\text{em}, \downarrow, \text{con}:=\text{g}, \text{lemma}:=\text{$\varepsilon$}, \text{noun}\\
\text{ed$'$e $\rightarrow$ ad} & \text{flex}:=\text{ie}, \downarrow, \text{con}:=\text{d}, \text{lemma}:=\text{$\varepsilon$}, \text{noun}\\
\text{ez$'$d$'$e $\rightarrow$ azd} & \text{flex}:=\text{ie}, \downarrow, \text{con}:=\text{d}, \text{lemma}:=\text{$\varepsilon$}, \text{noun}\\
\text{\textipa{Z}e $\rightarrow$ g} & \text{flex}:=\text{ie}, \downarrow, \text{con}:=\text{g}, \text{lemma}:=\text{$\varepsilon$}, \text{noun}\\
\end{array}
\]\end{scriptsize}
Rozpakowane reguły możemy użyć do lematyzacji form:
\begin{scriptsize}\[
\begin{array}{ll}
\text{gv$'$azda $\rightarrow$ gv$'$azda} & \text{flex}:=\text{a}, \downarrow, \text{con}:=\text{d}, \text{lemma}:=\text{a}, \text{noun}\\
\text{gv$'$azda $\rightarrow$ gv$'$azd} & \text{flex}:=\text{a}, \downarrow, \text{con}:=\text{d}, \text{lemma}:=\text{$\varepsilon$}, \text{noun}\\
\text{szp$'$eg$'$em $\rightarrow$ szp$'$ega} & \text{flex}:=\text{em}, \downarrow, \text{con}:=\text{g}, \text{lemma}:=\text{a}, \text{noun}\\
\text{szp$'$eg$'$em $\rightarrow$ szp$'$eg} & \text{flex}:=\text{em}, \downarrow, \text{con}:=\text{g}, \text{lemma}:=\text{$\varepsilon$}, \text{noun}\\
\text{gv$'$ez$'$d$'$e $\rightarrow$ gv$'$azda} & \text{flex}:=\text{ie}, \downarrow, \text{con}:=\text{d}, \text{lemma}:=\text{a}, \text{noun}\\
\text{gv$'$ez$'$d$'$e $\rightarrow$ gv$'$azd} & \text{flex}:=\text{ie}, \downarrow, \text{con}:=\text{d}, \text{lemma}:=\text{$\varepsilon$}, \text{noun}\\
\text{wa\textipa{Z}e $\rightarrow$ waga} & \text{flex}:=\text{ie}, \downarrow, \text{con}:=\text{g}, \text{lemma}:=\text{a}, \text{noun}\\
\text{wa\textipa{Z}e $\rightarrow$ wag} & \text{flex}:=\text{ie}, \downarrow, \text{con}:=\text{g}, \text{lemma}:=\text{$\varepsilon$}, \text{noun}\\
\end{array}
\]\end{scriptsize}

\end{frame}

\begin{frame}
\frametitle{Warstwa interpretacji}
\begin{itemize}
\item Warstwa interpretacji zawiera reguły przypisujące interpretację morfosyntaktyczną na podstawie wartości atrybutów.
\item Warstwa ta dokonuje selekcji kandydatów powstałych w wyniku działania warstwy analitycznej (wprowadzając jednocześnie kolejną niejednoznaczność).
\begin{scriptsize}\[
\begin{array}{lll}
\text{flex}:=\text{a}, \downarrow, \text{lemma}:=\text{a}, \text{noun} & \rightarrow & \text{subst:sg:nom:m1.m2.f}\\
\text{flex}:=\text{a}, \downarrow, \text{lemma}:=\text{a}, \text{noun} & \rightarrow & \text{subst:pl:nom.acc.voc:n:pt}\\
\text{flex}:=\text{a}, \downarrow, \text{lemma}:=\text{$\varepsilon$}, \text{noun} & \rightarrow & \text{subst:sg:gen.acc:m1.m2}\\
\text{flex}:=\text{a}, \downarrow, \text{lemma}:=\text{$\varepsilon$}, \text{noun} & \rightarrow & \text{subst:sg:gen:m3}\\
\text{flex}:=\text{em}, \downarrow, \text{lemma}:=\text{$\varepsilon$}, \text{noun} & \rightarrow & \text{subst:sg:inst:m1.m2.m3}\\
\end{array} 
\]\end{scriptsize}
\item Dla rzeczowników jest to najmniej ustrukturalizowana warstwa.
\item W przypadku czasowników, przymiotników i przysłówków to odwzorowanie jest dość jednoznaczne.
\item Reguły interpretacji zostały wytworzone półautomatycznie na podstawie SGJP.
\end{itemize}
\end{frame}

\begin{frame}
\frametitle{Działanie warstwy interpretacji}
Reguły interpretacji
\begin{scriptsize}\[
\begin{array}{lll}
\text{flex}:=\text{a}, \downarrow, \text{lemma}:=\text{a}, \text{noun} & \rightarrow & \text{subst:sg:nom:m1.m2.f}\\
\text{flex}:=\text{a}, \downarrow, \text{lemma}:=\text{a}, \text{noun} & \rightarrow & \text{subst:pl:nom.acc.voc:n:pt}\\
\text{flex}:=\text{a}, \downarrow, \text{lemma}:=\text{$\varepsilon$}, \text{noun} & \rightarrow & \text{subst:sg:gen.acc:m1.m2}\\
\text{flex}:=\text{a}, \downarrow, \text{lemma}:=\text{$\varepsilon$}, \text{noun} & \rightarrow & \text{subst:sg:gen:m3}\\
\text{flex}:=\text{em}, \downarrow, \text{lemma}:=\text{$\varepsilon$}, \text{noun} & \rightarrow & \text{subst:sg:inst:m1.m2.m3}\\
\end{array} 
\]\end{scriptsize}
przypiszą formom
\begin{scriptsize}\[
\begin{array}{ll}
\text{gv$'$azda $\rightarrow$ gv$'$azda} & \text{flex}:=\text{a}, \downarrow, \text{con}:=\text{d}, \text{lemma}:=\text{a}, \text{noun}\\
\text{gv$'$azda $\rightarrow$ gv$'$azd} & \text{flex}:=\text{a}, \downarrow, \text{con}:=\text{d}, \text{lemma}:=\text{$\varepsilon$}, \text{noun}\\
\text{szp$'$eg$'$em $\rightarrow$ szp$'$ega} & \text{flex}:=\text{em}, \downarrow, \text{con}:=\text{g}, \text{lemma}:=\text{a}, \text{noun}\\
\text{szp$'$eg$'$em $\rightarrow$ szp$'$eg} & \text{flex}:=\text{em}, \downarrow, \text{con}:=\text{g}, \text{lemma}:=\text{$\varepsilon$}, \text{noun}\\
\end{array}
\]\end{scriptsize}
następujące interpretacje morfosyntaktyczne:
\begin{scriptsize}\[
\begin{array}{ll}
\text{gv$'$azda $\rightarrow$ gv$'$azda} & \text{subst:sg:nom:m1.m2.f}\\
\text{gv$'$azda $\rightarrow$ gv$'$azda} & \text{subst:pl:nom.acc.voc:n:pt}\\
\text{gv$'$azda $\rightarrow$ gv$'$azd} & \text{subst:sg:gen.acc:m1.m2}\\
\text{gv$'$azda $\rightarrow$ gv$'$azd} & \text{subst:sg:gen:m3}\\
\text{szp$'$eg$'$em $\rightarrow$ szp$'$eg} & \text{subst:sg:inst:m1.m2.m3}\\
\end{array}
\]\end{scriptsize}


\end{frame}

\begin{frame}
\frametitle{Quasi-paradygmaty odmiany}
\begin{itemize}
\item Reguły przypisujące interpretacje można pogrupować ze względu na wartość atrybutu lemma i rodzaj rzeczownika generowany przez regułę. 
\item Uzyskujemy w ten sposób „quasi-paradygmaty” odmiany rzeczowników.
\item Należy jednak pamiętać, że dany lemat nie jest do takich ,,paradygmatów'' sztywno przypisany:
\begin{itemize}
\item nie musi on mieć form pochodzących tylko z jednego paradygmatu i
\item nie musi mieć wszystkich form występujących w danym paradygmacie.
\end{itemize}
\end{itemize}
\end{frame}

\begin{frame}
\frametitle{Rzeczowniki z wygłosem w lemacie}
\vspace{-3mm}
\begin{scriptsize}
\begin{multicols}{2}
cat=noun lemma=$\varepsilon$ gender:=f\\
\begin{tabular}{l|l}
sg:nom.acc & $\varepsilon\uparrow$\\
sg:gen.dat.loc.voc|pl:gen & y$\uparrow$\\
sg:inst & ą$\uparrow$\\
pl:nom.acc.voc & y$\uparrow$ e$\uparrow$\\
pl:dat & om$\uparrow$\\
pl:inst & ami$\uparrow$\\
pl:loc & ach$\uparrow$\\
\end{tabular}\\
cat=noun lemma=$\varepsilon$ gender:=m1\\
\begin{tabular}{l|l}
sg:nom & $\varepsilon$\\
sg:gen & $\star$y\\
sg:gen.acc & a\\
sg:dat & owi $\star$u\\
sg:dat.loc & $\star$y\\
sg:acc & $\star$y\\
sg:inst & em $\star$ą\\
sg:loc & $\star$u $\star$ie\\
sg:loc.voc & u$\uparrow$ u$\downarrow\rightarrow$ ie$\downarrow\leftarrow$\\
sg:voc & cze$\uparrow$ $\star$y $\star$ie\\
pl:nom.voc & y$\uparrow$ i$\downarrow$ e$\uparrow$ owie $\star$ie\\
pl:gen.acc & ów y$\uparrow$\\
pl:dat & om\\
pl:inst & ami\\
pl:loc & ach\\
depr & y$\downarrow$ e$\uparrow$\\
\end{tabular}\\
cat=noun lemma=$\varepsilon$ gender:=m2\\
\begin{tabular}{l|l}
sg:nom & $\varepsilon$\\
sg:gen.acc & a\\
sg:dat & owi $\star$u\\
sg:inst & em\\
sg:loc.voc & u$\uparrow$ u$\downarrow\rightarrow$ ie$\downarrow\leftarrow$\\
pl:nom.acc.voc & y$\downarrow$ e$\uparrow$ $\star$e\\
pl:gen & ów y$\uparrow$\\
pl:dat & om\\
pl:inst & ami\\
pl:loc & ach\\
\end{tabular}\\
cat=noun lemma=$\varepsilon$ gender:=m3\\
\begin{tabular}{l|l}
sg:nom.acc & $\varepsilon$\\
sg:gen & u a\\
sg:dat & $\star$u$\downarrow$ owi\\
sg:inst & em\\
sg:loc & $\star$ie\\
sg:loc.voc & u$\uparrow$ u$\downarrow\rightarrow$ ie$\downarrow\leftarrow$\\
sg:voc & $\star$ie\\
pl:nom.acc.voc & y$\downarrow$ e$\uparrow$ $\star$e $\star$a\\
pl:gen & ów y$\uparrow$\\
pl:dat & om\\
pl:inst & ami\\
pl:loc & ach\\
\end{tabular}\\
\end{multicols}\end{scriptsize}
\end{frame}

\begin{frame}
\frametitle{Rzeczowniki z kończące się na „a” w lemacie}
\vspace{-3mm}
\begin{scriptsize}
\begin{multicols}{2}
cat=noun lemma=a gender:=f\\
\begin{tabular}{l|l}
sg:nom & a\\
sg:gen & y $\star\varepsilon$\\
sg:gen.dat.loc & ej\\
sg:dat.loc & y$\uparrow$ ie$\downarrow$\\
sg:acc & ę ą\\
sg:inst & ą\\
sg:voc & u$\uparrow$ o a\\
pl:nom.acc.voc & y$\downarrow$ e\\
pl:gen & $\varepsilon$ y$\uparrow$\\
pl:gen.loc & ych\\
pl:dat & ym om\\
pl:inst & ymi ami\\
pl:loc & ach\\
\end{tabular}\\
\vfill\null
\columnbreak
cat=noun lemma=a gender:=m1\\
\begin{tabular}{l|l}
sg:nom & a\\
sg:gen & y $\star$ego\\
sg:gen.acc & $\star$ego\\
sg:dat & $\star$emu\\
sg:dat.loc & y$\uparrow$ ie$\downarrow$\\
sg:acc & ę\\
sg:inst & ą $\star$ym\\
sg:loc & $\star$ym\\
sg:voc & o $\star$u\\
pl:nom.voc & y$\uparrow$ i$\downarrow$ owie $\star$e\\
pl:gen.acc & ów $\star\varepsilon$\\
pl:dat & om\\
pl:inst & ami\\
pl:loc & ach\\
depr & y e\\
\end{tabular}
\end{multicols}\end{scriptsize}
\end{frame}

\begin{frame}
\frametitle{Lista nieobecności}
\begin{itemize}
\item Głoski funcjonalnie miękkie.
\item Leksemy typu „-cja”, „-pia”, „-dia”, „-rium”.
%leksemy z tematem kończącym się na samogłoskę
\item Słowa pisane z użyciem obcej ortografii. 
% nie ma alternacji głębszych niż przy ostatniej głosce więc to co jest wcześniej można zapisać w dowolny sposób co jest wykorzystywane by zagwarantować odwracalność ujednolicenia ortografii, podobnie akronimy
\item Odmiana akronimów
%NFZ-ecie, ZOZ-ie - w warstwie fonetycznej translacja jest odwracalna ale niejednoznaczna.
\item Odmiana (stopniowanie) przymiotników i przysłówków
\item Odmiana czasowników
%i ich grupy
\item Postprocessing wygłosu
\item Formy gwarowe
\end{itemize}
\end{frame}
%pani, i pozotałe grupy std noun odmiany m1 m3, n - zarówno funkcjonalnie miękkie jak i funkcjonalnie twarde
%pokazać przekształcenia fonetyczne
%pokazać występujące w nich alternacje
%zestaw wszystkich końcówek rzeczownikowych
%działanie reguł interpretacyjnych dla tych słów

\begin{frame}
\frametitle{Reguły operacyjne}
\begin{itemize}
\item Model składa się z
\begin{itemize}
\item 723 reguł warstwy ortograficzno-fonetycznej
\item 748 alternacji
\item 367 reguł analitycznych
\item 960 reguł przypisujących interpretację
\end{itemize}
\item W celu wytworzenia wydajnego systemu reguły te zostały złączone ze sobą:
\begin{itemize}
\item do każdej możliwej sekwencji reguł analitycznych
\item zostały dopasowane reguły przypisujące interpretację;
\item następnie zostały przekonwertowane na standardową ortografię.
\end{itemize}
\item W wyniku tego procesu powstało ok. 10 000 000 reguł operacyjnych.
\item Następnie została dokonana selekcja reguł polegająca na wyborze tych, których użycie jest poświadczone w SGJP uzupełnionym o przykładowe formy gwarowe i dodatkowe odmienione akronimy. 
\item Reguł operacyjnych jest 31122.
%\item jest ich dużo ale są generowane automatycznie, dzięki czemu zgodne z modelem i poprawne.
\end{itemize}
\end{frame}

\begin{frame}
\frametitle{Reguły operacyjne}
Liczbę reguł z podziałem na ich typy i części mowy:
\begin{center}

\begin{tabular}{l|r|r|r|r|r}
                              &  noun &  adj & adv &  verb &  suma \\
\hline
	           produktywne    &  7534 & 1501 & 150 &  9107 & 18292 \\
\hline
	\textbf{*} nieproduktywne &   209 &  389 & --- &  3701 &  4299 \\
\hline
	\textbf{A} obce           &  1275 &  --- & --- &   --- &  1275 \\
\hline
	\textbf{B} obce           &   206 &  --- & --- &   --- &   206 \\
\hline
	\textbf{C} akronimy       &   557 &  --- & --- &   --- &   557 \\
\hline
	\textbf{D} gwarowe        &  2639 &  380 & --- &  3474 &  6493 \\
\hline
	         suma             & 12420 & 2270 & 150 & 16282 & 31122 \\
\end{tabular}
\end{center}
\begin{itemize}
\item Grupa ,,obcych A'' dotyczy słów o obcej ortografii, w których pierwotna postać rdzenia jest zawarta w obserwowanej formie.
\item W wypadku ,,obcych B'' pierwotna postać rdzenia nie jest zawarta w obserwowanej formie i musi zostać odgadnięta (np. dopełniacz \textit{Chiraka} od lematu \textit{Chirac}).
\item Wszystkim regułom towarzyszą informacje o frekwencji --- liczba form ze słownika lematyzowalnych za pomocą danej reguły. 
\end{itemize}
\end{frame}

\begin{frame}
\frametitle{Pokrycie modelu}
\begin{itemize}
\item Reguły produkcyjne opisują fleksję
\begin{itemize}
\item $\frac{143643}{143643+343}$ = 99,76\% lematów rzeczownikowych,
\item $\frac{66426}{66426+26}$ = 99,96\% lematów przymiotnikowych,
\item $\frac{25839}{25839+422}$ = 98,39\% lematów przysłówkowych,
\item $\frac{28571}{28571+1229}$ = 95,88\% lematów czasownikowych.
\end{itemize}
\item Po usunięciu lematów czasownikowych, które powstały poprzez dodanie prefiksu wartość wzrasta do $\frac{13852}{13852+167}$ = 98,81\%.
\item Takie wartości wskazują, że opisywany model w sposób poprawny i pełny opisuje zawartą w SGJP fleksję języka polskiego.
\item Leksemy niepokryte przez model odmieniają się w sposób nieregularny -- powinny one stanowić zamknięty zbiór.
\item Jest to szczególnie istotne przy czasownikach,
gdzie 167 nieregularnych rdzeni generuje, po uzupełnieniu o prefiksy, 1229 nieregularnych leksemów.
\item W przypadku przysłówków, na 422 niepokryte przez model leksemy składają się zasadniczo przysłówki niestopniowalne i niepochodzące od przymiotników.
\end{itemize}
\end{frame}

\begin{frame}
\frametitle{Dezambiguacja}
\begin{itemize}
\item Rezultaty zwracane przez model są zazwyczaj wysoce niejednoznaczne.
\item W celu ich ujednoznacznienia można podjąć następujące kroki
\begin{itemize}
\item konfrontacja wyniku z SGJP
\item weryfikacja za pomocą listy znanych lematów
\item dezambiguacja statystyczna wykonywana przez tager
\end{itemize}
\end{itemize}
\end{frame}

\begin{frame}
\frametitle{Konfrontacja z SGJP}
\begin{itemize}
\item Reguły produkcyjne mają swoje identyfikatory. 
\item Na podstawie SGJP została wytworzona lista rdzeni wraz przypisanymi im identyfikatorami reguł właściwych dla danego rdzenia.
\item Interpretacje potwierdzone przez listę zostają opatrzone statusem „LemmaVal”.
\item Formy z SJGP niepokryte przez model zostały umieszczone w osobnym słowniczku.
\item Interpretacje uzyskane za pomocą tego słowniczka są opatrzone statusem „LemmaAlt”.
\item Pozostałe interpretacje są oznaczone jako „LemmNotVal”.
\item Jeśli odgadywacz nie znajdzie żadnej interpretacji dla danej formy zwracają ze statusem „TokNotFound”.
\item W ten sposób odgadywacz uzyskuje pełne pokrycie na SGJP i funkcjonalność analizatora morfologicznego.
\end{itemize}
\end{frame}

\begin{frame}
\frametitle{Bazy form podstawowych słów}
\begin{itemize}
\item SGJP
\begin{itemize}
\item  ponad 333000 lematów
\end{itemize}
\item SAWA
\item TERYT
\begin{itemize}
\item 304 powiaty
\item 38889 miejscowości
\item 24508 części miejscowości 
\item 42871 ulice (11272 z nich mają osobowego patrona)
\end{itemize}
\item nazwiska-polskie.pl
\begin{itemize}
\item ponad 220000 nazwisk
\end{itemize}
\item Wikipedia/DBpedia
\item Geonames - nazwy geograficzne
\item KRS - nazwy organizacji
\end{itemize}
\end{frame}
%Słownictwo specjalistyczne (biologiczne, chemiczne) itd. 



\begin{frame}
\frametitle{Algorytm dezambiguacji symbolicznej}
\begin{itemize}
\item Algorytm polega na przypisaniu interpretacjom priorytetów i wyborze tych interpretacji, które mają najmniejszy priorytet.
\item Kryteria wyboru priorytetu:
\end{itemize}
\begin{center}\begin{scriptsize}
\begin{tabular}{|l|l|c|c|c|c|c|c|c|c|}
\hline
1 & lemat jest na liście znanych lematów & \multicolumn{6}{c|}{+} & \multicolumn{2}{c|}{-} \\
\hline
2 & lemat jest w SGJP & \multicolumn{3}{c|}{+} & \multicolumn{3}{c|}{-} & \multicolumn{2}{c|}{} \\
\hline
3 & lematyzacja przeprowadzona zgodnie z SGJP & + & \multicolumn{2}{|c|}{-} & \multicolumn{3}{c|}{} & + & - \\
\hline
4 & tag „no-sgjp” & & + & - & \multicolumn{3}{|c|}{} & & \\
\hline
5 & forma nieodmienna & & & & \multicolumn{2}{c|}{+} & - & & \\
\hline
6 & tag „poss-ndm” & & & & + & - & & & \\
\hline
 & priorytet & 1 & 1 & R & 1 & R & 1 & 2 & R \\
\hline
\end{tabular}\end{scriptsize}
\end{center}
\begin{itemize}
\item Interpretacje z priorytetem oznaczonym „R” są odrzucane, gdy spełniony jest przynajmniej jeden z warunków:
\begin{itemize}
\item forma została wydzielona z tekstu przy z odciętym aglutynatem,
\item forma została zlematyzowana ze zmienioną  wielkością liter,
\item forma została zlematyzowana za pomocą reguły typu {\bf B}.
\end{itemize}
\item Jeśli interpretacja z priorytetem oznaczonym „R” nie zostaje odrzucona otrzymuje priorytet 3.
\end{itemize}
\end{frame}

\begin{frame}
\frametitle{Struktura form słownych w NKJP1M}
\begin{scriptsize}
\begin{tabular}{l|rrrr}
 & Liczba & Liczba & Procent & Procent\\
 & unikalnych & form & unikalnych & form\\
 & form & & form & \\
\hline
%SGJP-EXACT & 132264 & 813399 & 72,2056\% & 66,9479\%\\
%SGJP-LMM-UNCAPITAL & 21320 & 80045 & 11,6390\% & 6,5882\%\\
%SGJP-LMM-CAPITAL & 194 & 295 & 0,1059\% & 0,0243\%\\
%SGJP-LMM-LOWER & 899 & 9058 & 0,4908\% & 0,7455\%\\
%SGJP-BTH-LOWER & 1888 & 3716 & 1,0307\% & 0,3059\%\\
%SGJP łącznie
lematyzowane przez SGJP & 156565 & 906513 & 85,4720\% & 74,6117\%\\
%SYMB
symbole & 5796 & 250926 & 3,1642\% & 20,6528\%\\
\hline
%CORR
poprawne spoza SGJP & 16581 & 42195 & 9,0519\% & 3,4729\%\\
%COMPD
formy z dywizem i apostrofem & 659 & 783 & 0,3598\% & 0,0644\%\\
%PLTAN
pt lematyzowane do sg przez SGJP & 168 & 461 & 0,0917\% & 0,0379\%\\
%TAGD
tag inny niż proponowany przez SGJP & 1151 & 11020 & 0,6284\% & 0,9070\%\\
%DIAL
formy gwarowe bądź archaiczne & 132 & 166 & 0,0721\% & 0,0137\%\\
\hline
%CERR
powszechny błąd & 156 & 393 & 0,0852\% & 0,0323\%\\
%PHON
zapis fonetyczny & 166 & 191 & 0,0906\% & 0,0157\%\\
%ERR
literówka & 1415 & 1728 & 0,7725\% & 0,1422\%\\
%TAGE
niepoprawny tag & 383 & 593 & 0,2091\% & 0,0488\%\\
%TERR 
błąd tokenizacji & 5 & 5 & 0,0027\% & 0,0004\%\\
\hline
cały korpus & 183177 & 1214974 & 100,0000\% & 100,0000\%\\
\end{tabular}
\end{scriptsize}
\begin{itemize}
\item Pierwsze dwie kategorie oraz ostatnie pięć nie stanowi ciekawych danych do testowania odgadywacza.
\item Pozostałe pięć kategorii wykorzystaliśmy do przeprowadzenia walidacji.
\end{itemize}
\end{frame}

\begin{frame}
\frametitle{Walidacja}
\begin{itemize}
\item Odgadywacz został porównany z następującymi programami:
\begin{itemize}
\item Analizator morfologiczny SAM (1996)
\item TaKIPI (2007)
\end{itemize}
\item Wygrywa to porównanie niejako walkowerem z uwagi na to że:
\begin{itemize}
\item SAM korzysta z innego tagsetu niż wszystkie obecne narzędzia (m.in. nie rozróżnia fleksemów form czasownika i segmentów nieodmiennych.
\item TaKIPI wymaga Morfeusza w wersji SIaT (rzuca wyjątek, gdy biblioteka libmorfeusz zwróci tag morfosyntaktyczny comp, interj, brev lub burk).
\item SAM generuje segmentation fault dla niektórych segmentów, np.: „Samotrzeciej”, „samoprzyznaniem”, „samorozwiązania”, „samorozwiązanie”, „zekowaniem”.
\item TaKIPI zmienia wielkość liter przy lematyzacji, np. lematyzuje „XVII-wieczny” jako „xvii-wieczny”.
\item SAM zmienia wielkość liter i usuwa myślniki przy lematyzacji, np. lematyzuje „XVII-wieczny” jako „xviiwieczny”.

\end{itemize}
\end{itemize}
\end{frame}

\begin{frame}
\frametitle{Walidacja: formy poprawne spoza SGJP}
%\textbf{nasze/walid\_xCORR.tab}
\begin{center}\begin{scriptsize}
\begin{tabular}{l|rrrr}
 & Liczba & Liczba & Procent & Procent\\
 & unikalnych & form & unikalnych & form\\
 & form & & form & \\
\hline
OK & 14747 & 38816 & 88,9338\% & 91,9898\% \\
OK CC & 207 & 231 & 1,2483\% & 0,5474\% \\
GOODPOS & 151 & 177 & 0,9106\% & 0,4195\% \\
GOODPOS CC & 364 & 474 & 2,1952\% & 1,1233\% \\
LEMMA & 790 & 1383 & 4,7642\% & 3,2776\% \\
LEMMA CC & 181 & 935 & 1,0915\% & 2,2158\% \\
FAIL & 142 & 180 & 0,8564\% & 0,4266\% \\
\hline
cały korpus & 16582 & 42196 & 100,0000\% & 100,0000\%\\
\end{tabular}
\end{scriptsize}\end{center}
Oznaczenia:
\begin{itemize}
\item OK --- przykład poprawnie przetworzony
\item GOODPOS --- zgodność lematu i części mowy
\item LEMMA --- zgodność lematu
\item FAIL --- brak zgodności
\item CC --- ignorowanie wielkości liter przy porównywaniu lematów
\end{itemize}
\end{frame}

\begin{frame}
\frametitle{Walidacja: formy z dywizem i apostrofem}
%\textbf{nasze/walid\_xCOMPD.tab}
\begin{center}\begin{scriptsize}
\begin{tabular}{l|rrrr}
 & Liczba & Liczba & Procent & Procent\\
 & unikalnych & form & unikalnych & form\\
 & form & & form & \\
\hline
OK & 459 & 576 & 69,6510\% & 73,5632\% \\
OK CC & 3 & 3 & 0,4552\% & 0,3831\% \\
GOODPOS & 15 & 15 & 2,2762\% & 1,9157\% \\
GOODPOS CC & 2 & 2 & 0,3035\% & 0,2554\% \\
LEMMA & 23 & 23 & 3,4901\% & 2,9374\% \\
FAIL & 157 & 164 & 23,8240\% & 20,9451\% \\
\hline
cały korpus & 659 & 783 & 100,0000\% & 100,0000\%\\
\end{tabular}
\end{scriptsize}\end{center}
Oznaczenia:
\begin{itemize}
\item OK --- przykład poprawnie przetworzony
\item GOODPOS --- zgodność lematu i części mowy
\item LEMMA --- zgodność lematu
\item FAIL --- brak zgodności
\item CC --- ignorowanie wielkości liter przy porównywaniu lematów
\end{itemize}


\end{frame}

\begin{frame}
\frametitle{Walidacja: pt lematyzowane do sg przez SGJP}
%\textbf{nasze/walid\_xPLTAN.tab}
\begin{center}\begin{scriptsize}
\begin{tabular}{l|rrrr}
 & Liczba & Liczba & Procent & Procent\\
 & unikalnych & form & unikalnych & form\\
 & form & & form & \\
\hline
OK & 130 & 371 & 77,3810\% & 80,4772\% \\
GOODPOS & 13 & 13 & 7,7381\% & 2,8200\% \\
GOODPOS CC & 8 & 57 & 4,7619\% & 12,3644\% \\
LEMMA CC & 2 & 2 & 1,1905\% & 0,4338\% \\
FAIL & 15 & 18 & 8,9286\% & 3,9046\% \\
\hline
cały korpus & 168 & 461 & 100,0000\% & 100,0000\%\\
\end{tabular}
\end{scriptsize}\end{center}
Oznaczenia:
\begin{itemize}
\item OK --- przykład poprawnie przetworzony
\item GOODPOS --- zgodność lematu i części mowy
\item LEMMA --- zgodność lematu
\item FAIL --- brak zgodności
\item CC --- ignorowanie wielkości liter przy porównywaniu lematów
\end{itemize}


\end{frame}

\begin{frame}
\frametitle{Walidacja: tag inny niż proponowany przez SGJP}
%\textbf{nasze/walid\_xTAGD.tab}
\begin{center}\begin{scriptsize}
\begin{tabular}{l|rrrr}
 & Liczba & Liczba & Procent & Procent\\
 & unikalnych & form & unikalnych & form\\
 & form & & form & \\
\hline
OK & 537 & 6261 & 46,6551\% & 56,8149\% \\
OK CC & 25 & 45 & 2,1720\% & 0,4083\% \\
GOODPOS & 61 & 157 & 5,2997\% & 1,4247\% \\
GOODPOS CC & 14 & 46 & 1,2163\% & 0,4174\% \\
LEMMA & 332 & 3512 & 28,8445\% & 31,8693\% \\
LEMMA CC & 105 & 708 & 9,1225\% & 6,4247\% \\
FAIL & 77 & 291 & 6,6898\% & 2,6407\% \\
\hline
cały korpus & 1151 & 11020 & 100,0000\% & 100,0000\%\\
\end{tabular}
\end{scriptsize}\end{center}
Oznaczenia:
\begin{itemize}
\item OK --- przykład poprawnie przetworzony
\item GOODPOS --- zgodność lematu i części mowy
\item LEMMA --- zgodność lematu
\item FAIL --- brak zgodności
\item CC --- ignorowanie wielkości liter przy porównywaniu lematów
\end{itemize}

\end{frame}

\begin{frame}
\frametitle{Walidacja: formy gwarowe bądź archaiczne}
%\textbf{nasze/walid\_xDIAL.tab}
\begin{center}\begin{scriptsize}
\begin{tabular}{l|rrrr}
 & Liczba & Liczba & Procent & Procent\\
 & unikalnych & form & unikalnych & form\\
 & form & & form & \\
\hline
OK & 25 & 28 & 18,9394\% & 16,8675\% \\
GOODPOS & 5 & 5 & 3,7879\% & 3,0120\% \\
GOODPOS CC & 1 & 1 & 0,7576\% & 0,6024\% \\
LEMMA & 8 & 10 & 6,0606\% & 6,0241\% \\
LEMMA CC & 1 & 1 & 0,7576\% & 0,6024\% \\
FAIL & 92 & 121 & 69,6970\% & 72,8916\% \\
\hline
cały korpus & 132 & 166 & 100,0000\% & 100,0000\%\\
\end{tabular}
\end{scriptsize}\end{center}
Oznaczenia:
\begin{itemize}
\item OK --- przykład poprawnie przetworzony
\item GOODPOS --- zgodność lematu i części mowy
\item LEMMA --- zgodność lematu
\item FAIL --- brak zgodności
\item CC --- ignorowanie wielkości liter przy porównywaniu lematów
\end{itemize}
\end{frame}


\begin{frame}
\frametitle{Porównanie odgadywaczy: formy poprawne spoza SGJP}
\begin{center}\begin{scriptsize}
\begin{tabular}{l|rrr|rrr}
& \multicolumn{3}{c|}{Procent unikalnych form} &
\multicolumn{3}{c}{Procent form} \\
\hline
& ENIAM & TaKIPI & SAM & ENIAM & TaKIPI & SAM \\
\hline
OK & 88,93\% & 5,07\% & 0,00\% & 91,99\% & 48,12\% & 0,00\% \\
OK CC & 1,25\% & 13,10\% & 0,00\% & 0,55\% & 6,70\% & 0,00\% \\
GOODPOS & 0,91\% & 6,46\% & 8,45\% & 0,42\% & 2,96\% & 3,80\% \\
GOODPOS\_NONINFL & 0,00\% & 0,00\% & 0,00\% & 0,00\% & 0,00\% & 0,00\% \\
GOODPOS\_VERB & 0,00\% & 0,00\% & 0,00\% & 0,00\% & 0,00\% & 0,00\% \\
GOODPOS CC & 2,20\% & 27,66\% & 54,06\% & 1,12\% & 13,85\% & 26,39\% \\
GOODPOS\_NONINFL CC & 0,00\% & 0,00\% & 0,00\% & 0,00\% & 0,00\% & 0,00\% \\
GOODPOS\_VERB CC & 0,00\% & 0,00\% & 0,00\% & 0,00\% & 0,00\% & 0,00\% \\
LEMMA & 4,76\% & 6,69\% & 5,44\% & 3,28\% & 3,53\% & 5,93\% \\
LEMMA CC & 1,09\% & 7,18\% & 1,32\% & 2,22\% & 3,36\% & 0,56\% \\
FAIL & 0,86\% & 33,23\% & 30,69\% & 0,43\% & 21,23\% & 63,30\% \\
CRASH & 0,00\% & 0,00\% & 0,03\% & 0,00\% & 0,00\% & 0,01\% \\
\end{tabular}
\end{scriptsize}\end{center}
\begin{scriptsize}\begin{itemize}
\item OK --- przykład poprawnie przetworzony
\item GOODPOS --- zgodność lematu i części mowy
\item GOODPOS\_NONINFL --- zgodność lematu i tego, że część mowy jest nieodmienna
\item GOODPOS\_VERB --- zgodność lematu i tego, że część mowy jest czasownikiem
\item LEMMA --- zgodność lematu
\item FAIL --- brak zgodności
\item CRASH --- runtime error
\item CC --- ignorowanie wielkości liter przy porównywaniu lematów
\end{itemize}\end{scriptsize}
\end{frame}

\begin{frame}
\frametitle{Porównanie odgadywaczy: formy z dywizem i apostrofem}
\begin{center}\begin{scriptsize}
\begin{tabular}{l|rrr|rrr}
& \multicolumn{3}{c|}{Procent unikalnych form} &
\multicolumn{3}{c}{Procent form} \\
\hline
& ENIAM & TaKIPI & SAM & ENIAM & TaKIPI & SAM \\
\hline
OK & 69,65\% & 19,42\% & 0,00\% & 73,56\% & 18,26\% & 0,00\% \\
OK CC & 0,46\% & 3,49\% & 0,00\% & 0,38\% & 3,70\% & 0,00\% \\
GOODPOS & 2,28\% & 21,24\% & 0,61\% & 1,92\% & 21,58\% & 0,51\% \\
GOODPOS\_NONINFL & 0,00\% & 0,00\% & 0,00\% & 0,00\% & 0,00\% & 0,00\% \\
GOODPOS\_VERB & 0,00\% & 0,00\% & 0,00\% & 0,00\% & 0,00\% & 0,00\% \\
GOODPOS CC & 0,30\% & 3,34\% & 7,28\% & 0,26\% & 2,94\% & 6,39\% \\
GOODPOS\_NONINFL CC & 0,00\% & 0,00\% & 0,00\% & 0,00\% & 0,00\% & 0,00\% \\
GOODPOS\_VERB CC & 0,00\% & 0,00\% & 0,00\% & 0,00\% & 0,00\% & 0,00\% \\
LEMMA & 3,49\% & 0,91\% & 0,00\% & 2,94\% & 0,77\% & 0,00\% \\
LEMMA CC & 0,00\% & 0,15\% & 0,00\% & 0,00\% & 0,13\% & 0,00\% \\
FAIL & 23,82\% & 46,43\% & 92,11\% & 20,95\% & 48,40\% & 93,10\% \\
CRASH & 0,00\% & 0,00\% & 0,00\% & 0,00\% & 0,00\% & 0,00\% \\
\end{tabular}
\end{scriptsize}\end{center}
\begin{scriptsize}\begin{itemize}
\item OK --- przykład poprawnie przetworzony
\item GOODPOS --- zgodność lematu i części mowy
\item GOODPOS\_NONINFL --- zgodność lematu i tego, że część mowy jest nieodmienna
\item GOODPOS\_VERB --- zgodność lematu i tego, że część mowy jest czasownikiem
\item LEMMA --- zgodność lematu
\item FAIL --- brak zgodności
\item CRASH --- runtime error
\item CC --- ignorowanie wielkości liter przy porównywaniu lematów
\end{itemize}\end{scriptsize}
\end{frame}

\begin{frame}
\frametitle{Porównanie odgadywaczy: pt lematyzowane do sg przez SGJP}
\begin{center}\begin{scriptsize}
\begin{tabular}{l|rrr|rrr}
& \multicolumn{3}{c|}{Procent unikalnych form} &
\multicolumn{3}{c}{Procent form} \\
\hline
& ENIAM & TaKIPI & SAM & ENIAM & TaKIPI & SAM \\
\hline
OK & 77,38\% & 1,19\% & 0,00\% & 80,48\% & 0,43\% & 0,00\% \\
OK CC & 0,00\% & 0,60\% & 0,00\% & 0,00\% & 0,22\% & 0,00\% \\
GOODPOS & 7,74\% & 0,60\% & 25,00\% & 2,82\% & 0,22\% & 13,23\% \\
GOODPOS\_NONINFL & 0,00\% & 0,00\% & 0,00\% & 0,00\% & 0,00\% & 0,00\% \\
GOODPOS\_VERB & 0,00\% & 0,00\% & 0,00\% & 0,00\% & 0,00\% & 0,00\% \\
GOODPOS CC & 4,76\% & 0,00\% & 25,00\% & 12,36\% & 0,00\% & 11,50\% \\
GOODPOS\_NONINFL CC & 0,00\% & 0,00\% & 0,00\% & 0,00\% & 0,00\% & 0,00\% \\
GOODPOS\_VERB CC & 0,00\% & 0,00\% & 0,00\% & 0,00\% & 0,00\% & 0,00\% \\
LEMMA & 0,00\% & 1,19\% & 0,60\% & 0,00\% & 0,43\% & 0,22\% \\
LEMMA CC & 1,19\% & 0,00\% & 0,00\% & 0,43\% & 0,00\% & 0,00\% \\
FAIL & 8,93\% & 96,43\% & 49,40\% & 3,90\% & 98,70\% & 75,05\% \\
CRASH & 0,00\% & 0,00\% & 0,00\% & 0,00\% & 0,00\% & 0,00\% \\
\end{tabular}
\end{scriptsize}\end{center}
\begin{scriptsize}\begin{itemize}
\item OK --- przykład poprawnie przetworzony
\item GOODPOS --- zgodność lematu i części mowy
\item GOODPOS\_NONINFL --- zgodność lematu i tego, że część mowy jest nieodmienna
\item GOODPOS\_VERB --- zgodność lematu i tego, że część mowy jest czasownikiem
\item LEMMA --- zgodność lematu
\item FAIL --- brak zgodności
\item CRASH --- runtime error
\item CC --- ignorowanie wielkości liter przy porównywaniu lematów
\end{itemize}\end{scriptsize}
\end{frame}

\begin{frame}
\frametitle{Porównanie odgadywaczy: tag inny niż proponowany przez SGJP}
\begin{center}\begin{scriptsize}
\begin{tabular}{l|rrr|rrr}
& \multicolumn{3}{c|}{Procent unikalnych form} &
\multicolumn{3}{c}{Procent form} \\
\hline
& ENIAM & TaKIPI & SAM & ENIAM & TaKIPI & SAM \\
\hline
OK & 46,66\% & 9,73\% & 0,00\% & 56,81\% & 27,30\% & 0,00\% \\
OK CC & 2,17\% & 0,00\% & 0,00\% & 0,41\% & 0,00\% & 0,00\% \\
GOODPOS & 5,30\% & 13,12\% & 13,03\% & 1,42\% & 8,08\% & 1,77\% \\
GOODPOS\_NONINFL & 0,00\% & 0,00\% & 0,00\% & 0,00\% & 0,00\% & 0,00\% \\
GOODPOS\_VERB & 0,00\% & 0,00\% & 0,00\% & 0,00\% & 0,00\% & 0,00\% \\
GOODPOS CC & 1,22\% & 0,09\% & 0,00\% & 0,42\% & 0,01\% & 0,00\% \\
GOODPOS\_NONINFL CC & 0,00\% & 0,00\% & 0,00\% & 0,00\% & 0,00\% & 0,00\% \\
GOODPOS\_VERB CC & 0,00\% & 0,00\% & 0,00\% & 0,00\% & 0,00\% & 0,00\% \\
LEMMA & 28,84\% & 35,27\% & 41,79\% & 31,87\% & 54,49\% & 64,95\% \\
LEMMA CC & 9,12\% & 0,00\% & 0,87\% & 6,42\% & 0,00\% & 0,59\% \\
FAIL & 6,69\% & 41,70\% & 44,31\% & 2,64\% & 10,11\% & 32,70\% \\
CRASH & 0,00\% & 0,00\% & 0,00\% & 0,00\% & 0,00\% & 0,00\% \\
\end{tabular}
\end{scriptsize}\end{center}
\begin{scriptsize}\begin{itemize}
\item OK --- przykład poprawnie przetworzony
\item GOODPOS --- zgodność lematu i części mowy
\item GOODPOS\_NONINFL --- zgodność lematu i tego, że część mowy jest nieodmienna
\item GOODPOS\_VERB --- zgodność lematu i tego, że część mowy jest czasownikiem
\item LEMMA --- zgodność lematu
\item FAIL --- brak zgodności
\item CRASH --- runtime error
\item CC --- ignorowanie wielkości liter przy porównywaniu lematów
\end{itemize}\end{scriptsize}
\end{frame}

\begin{frame}
\frametitle{Porównanie odgadywaczy: formy gwarowe bądź archaiczne}
\begin{center}\begin{scriptsize}
\begin{tabular}{l|rrr|rrr}
& \multicolumn{3}{c|}{Procent unikalnych form} &
\multicolumn{3}{c}{Procent form} \\
\hline
& ENIAM & TaKIPI & SAM & ENIAM & TaKIPI & SAM \\
\hline
OK & 18,94\% & 2,27\% & 0,00\% & 16,87\% & 1,81\% & 0,00\% \\
OK CC & 0,00\% & 0,76\% & 0,00\% & 0,00\% & 0,60\% & 0,00\% \\
GOODPOS & 3,79\% & 4,55\% & 4,55\% & 3,01\% & 4,82\% & 4,82\% \\
GOODPOS\_NONINFL & 0,00\% & 0,00\% & 0,00\% & 0,00\% & 0,00\% & 0,00\% \\
GOODPOS\_VERB & 0,00\% & 0,00\% & 0,00\% & 0,00\% & 0,00\% & 0,00\% \\
GOODPOS CC & 0,76\% & 0,76\% & 0,00\% & 0,60\% & 1,20\% & 0,00\% \\
GOODPOS\_NONINFL CC & 0,00\% & 0,00\% & 0,00\% & 0,00\% & 0,00\% & 0,00\% \\
GOODPOS\_VERB CC & 0,00\% & 0,00\% & 0,00\% & 0,00\% & 0,00\% & 0,00\% \\
LEMMA & 6,06\% & 1,52\% & 5,30\% & 6,02\% & 1,20\% & 4,22\% \\
LEMMA CC & 0,76\% & 0,00\% & 0,76\% & 0,60\% & 0,00\% & 1,20\% \\
FAIL & 69,70\% & 90,15\% & 89,39\% & 72,89\% & 90,36\% & 89,76\% \\
CRASH & 0,00\% & 0,00\% & 0,00\% & 0,00\% & 0,00\% & 0,00\% \\
\end{tabular}
\end{scriptsize}\end{center}
\begin{scriptsize}\begin{itemize}
\item OK --- przykład poprawnie przetworzony
\item GOODPOS --- zgodność lematu i części mowy
\item GOODPOS\_NONINFL --- zgodność lematu i tego, że część mowy jest nieodmienna
\item GOODPOS\_VERB --- zgodność lematu i tego, że część mowy jest czasownikiem
\item LEMMA --- zgodność lematu
\item FAIL --- brak zgodności
\item CRASH --- runtime error
\item CC --- ignorowanie wielkości liter przy porównywaniu lematów
\end{itemize}\end{scriptsize}
\end{frame}

\begin{frame}
\frametitle{Wersja demonstracyjna}
\begin{itemize}
\item Przedstawiony w artykule model został zaimplementowany i 
stanowi fragment kategorialnego parsera składniowo-semantycznego „ENIAM”.
\item Internetowa wersja demonstracyjna guessera dostępna jest pod adresem:\\ {\tt http://eniam.nlp.ipipan.waw.pl/morphology.html}.
\item Internetowa wersja demonstracyjna generatora form dostępna jest pod adresem:\\ {\tt http://eniam.nlp.ipipan.waw.pl/morphology2.html}.
\end{itemize}
\end{frame}

\begin{frame}
\frametitle{Kod źródłowy i zasoby}
\begin{itemize}
\item Kod źródłowy, dane modelu i otagowana lista frekwencyjna NKJP1M znajdują się w repozytorium:\\ {\tt http://git.nlp.ipipan.waw.pl/\\wojciech.jaworski/ENIAM}
\item Odpowiednio w 
\begin{itemize}
\item katalogu {\tt morphology},
\item katalogu {\tt morphology/data} i
\item pliku {\tt resources/NKJP1M/\\NKJP1M-tagged-frequency-26.07.2017.tab}
\end{itemize}
\item Definicja tagsetu listy frekwencyjnej znajduje się w pliku {\tt resources/NKJP1M/\\NKJP-tagged-frequency-tagset.txt}
\end{itemize}
\end{frame}

\end{document}


\begin{frame}
\frametitle{}
\begin{itemize}
\item 
\item 
\item 
\end{itemize}
\end{frame}

lematyzacja (w szczególności odgadywanie lematu) ma sens wtedy, gdy jest pod ręką zasób, który coś o lemacie mówi - np. nadaje mu kategorię semantyczną. Bez tego lemat wnosi tyle samo informacji do dalszego przetwarzania co forma obserwowana w tekście.”



prezentacja interfejsu



\begin{frame}
\frametitle{}
\begin{itemize}
\item przykład z terytu nazwiska odmienionego na różne sposoby i tego, że odgadywacz wykrywa poprawną formę podstawową
\item lematyzacja jest potrzebna żeby była forma kanoniczna
\item różne sposoby odmiany w bazie TERYT sugerują, że inny użytkownicy języka mogą słowo odmieniać na różne sposoby.
\end{itemize}
\end{frame}

guesser nie zmienia wielkości liter, 
natomiast można mu podawać różne wersje tokenu i obserwować wyniki.

czasowniki są żadkie stanowią jedynie x% form spoza SGJP, y% form w korpusie.

