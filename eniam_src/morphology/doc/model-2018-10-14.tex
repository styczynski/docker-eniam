\documentclass{article}
\usepackage{amsmath}
\usepackage{amssymb}
\usepackage[T1]{fontenc}
\usepackage[utf8]{inputenc}
\usepackage[polish]{babel}
% \usepackage{tikz}
% \usetikzlibrary{conceptgraph}
\usepackage{amsthm}
\usepackage[T1]{tipa}
\usepackage{longtable}
\usepackage{multicol}

\parindent 0pt
\parskip 4pt

\newcommand{\form}{{\it form}}

\DeclareUnicodeCharacter{3B5}{\ensuremath{\varepsilon}}
\DeclareUnicodeCharacter{3C9}{\ensuremath{\omega}}
\DeclareUnicodeCharacter{3BA}{\ensuremath{\kappa}}
\DeclareUnicodeCharacter{3B4}{\ensuremath{\delta}}
\DeclareUnicodeCharacter{2B2}{\textipa{\super{j}}}
\DeclareUnicodeCharacter{2032}{\ensuremath{'}}
\DeclareUnicodeCharacter{1EF}{\textipa{\v{Z}}}

\renewcommand*{\thefootnote}{(\roman{footnote})}

\title{Model fleksji języka polskiego}
\author{Wojciech Jaworski, Szymon Rutkowski}
\date{}

\begin{document}
\maketitle

Niniejszy artykuł opisuje zasady morfologiczne języka polskiego
jako zestaw wykonywanych na obserwowanej formie słowa operacji
prowadzących do przekształcenia jej w lemat i zestaw cech morfoskładniowych.
Reguły opracowane zostały na podstawie Słownika Gramatycznego Języka Polskiego 
w wersji z 30.07.2017 \cite{SGJP}.

Celem jest stworzenie reprezentacji polskiej fleksji, która
z jednej strony jest zwarta i zrozumiała dla człowieka, 
a z drugiej będzie precyzyjna w sposób umożliwiający jej bezpośrednią implementację
w postaci {\it guessera}. 
Guesser jest to program, który ma za zadanie przypisywanie („odgadywanie”) 
cech morfosyntaktycznych segmentu na podstawie jego formy, to znaczy przede wszystkim rozpoznawalnych afiksów. 
Widząc, powiedzmy, napis \textit{burachnajami}, można od razu wziąć pod uwagę, 
że formą podstawową wyrazu jest \textit{burachnaj}, a do niej dodano końcówkę narzędnika liczby mnogiej \textit{-ami}.
Różni się on od {\it analizatora morfologicznego} \cite{Morfeusz} tym, że jego działanie nie jest ograniczone do 
zamkniętego zbioru słów pochodzących z dostarczonego programowi słownika gramatycznego.
Próbie interpretacji może zostać poddany dowolny napis.

Przedstawiony w artykule model został zaimplementowany i 
stał się fragmentem kategorialnego parsera składniowo-semantycznego „ENIAM” \cite{ENIAM}.
Internetowa wersja demonstracyjna guessera dostępna jest w Internecie\footnote{ {\tt http://eniam.nlp.ipipan.waw.pl/morphology.html}}.

Tworząc model skupiliśmy się na produktywnej części polskiej fleksji,
by uchwycić odmianę słów nowych, nieznanych, nie należących do słownika.
Model nie obejmuje nieregularnych czasowników oraz niewielkiej liczby słów 
należących do innych części mowy o nieregularnej odmianie.
Wynika to stąd, że znany zamknięty zbiór słów można 
zawrzeć w słowniczku załączonym do modelu. 

Zadania lematyzacji i anotacji morfosyntaktycznej oczywiście 
nie da się wykonać w sposób jednoznaczny jedynie na podstawie 
obserwacji pojedynczej, wyrwanej z kontekstu formy.
Dlatego guesser określa z pomocą swoich reguł jedynie zbiór możliwych interpretacji.
Mogą stanowić one dane wejściowe dla {\it taggera} \cite{Concraft}, czyli programu
przeprowadzającego dezambiguację morfosyntaktyczną na podstawie modeli statystycznych.

W literaturze pojawiały się dotąd próby ujmowania polskiej fleksji w systemy reguł przystosowane dla komputerów,
ale zwykle nie powstawały one z myślą o klasyfikacji słów nieznanych.
Przykładem może być tutaj praca Teresy Rokickiej \cite{rok}.
Ukoronowaniem tego rodzaju wysiłków są wzory paradygmatyczne Słownika Gramatycznego Języka Polskiego.

Bezpośrednie przechowywanie reguł fleksyjnych jest zwykle mało ekonomiczne, także w przedstawianiu wiedzy lingwistycznej, ponieważ struktura takich reguł jest powtarzalna.
Na przykład o alternacje głosek następujące na granicy morfemów zachowują się zazwyczaj podobnie niezależnie od konkretnej końcówki.
Ukazuje to, jak problem, przed którym staje guesser, angażuje różne poziomy systemu języka:
fonologiczny, ortograficzny, morfonologiczny, morfologiczny.

Model rozdziela konceptualnie owe poziomy %%%Celem modelu jest opisanie w zwarty sposób produktywnych reguł morfologicznych wykorzystywanych we fleksji języka polskiego.
poprzez wyróżnienie trzech warstw reguł: ortograficzno-fonetycznej, analitycznej i intepretacji.
Można o nich myśleć jako o kolejnych warstwach wykonujących swoje zadania ,,po kolei''. W takim ujęciu:
\begin{itemize}
	\item zadaniem warstwy \textbf{ortograficzno-fonetycznej} jest wyabstrahowanie od polskiej ortografii przez przekonwertowanie formy segmentu do wewnętrznej reprezentacji, odzwierciedlającej prawidłowości morfonologiczne języka;
	\item warstwa \textbf{analityczna} generuje lemat oraz określa występujące afiksy;
	\item wreszcie warstwa \textbf{interpretacji} nadaje segmentowi interpretację morfosyntaktyczną na podstawie wykrytych afiksów.
\end{itemize}

\section{Warstwa ortograficzno-fonetyczna}\label{sec_phon}

\begin{table}
  \centering
  \begin{tabular}{l|l|l|l|l|l}
1 & 2 & 3 & 4 & 5 & przykład \\
	  d$'$ $\leftarrow$ dzi & d$'$ $\leftarrow$ dz &  & \textipa{Z} $\leftarrow$ dz & d$'$ $\leftarrow$ dź & \textit{gwieździe} \\
	  k$'$ $\leftarrow$ ki & k$'$ $\leftarrow$ k & k$'$j $\leftarrow$ ki &  &  & \textit{wielkim} \\
	  m$'$ $\leftarrow$ mi & m$'$ $\leftarrow$ m & m$'$j $\leftarrow$ mi &  &  & \textit{miara} \\
	  n$'$ $\leftarrow$ ni & n$'$ $\leftarrow$ n & n$'$j $\leftarrow$ ni &  & n$'$ $\leftarrow$ ń & \textit{panią} \\
	  r\textipa{\super{j}} $\leftarrow$ ri & r\textipa{\super{j}} $\leftarrow$ r & rj $\leftarrow$ ri & ř $\leftarrow$ rz &  & \textit{rzeka}  \\
	  t$'$ $\leftarrow$ ci & t$'$ $\leftarrow$ c &  &  & t$'$ $\leftarrow$ ć & \textit{kwiecie} \\
	  v$'$ $\leftarrow$ wi & v$'$ $\leftarrow$ w & vj $\leftarrow$ wi & v $\leftarrow$ w &  & \textit{wierzą}\\
	  ž\textipa{\super{j}} $\leftarrow$ żi & ž\textipa{\super{j}} $\leftarrow$ ż & žj $\leftarrow$ żi & ž $\leftarrow$ ż &  & \textit{ażiotaż} \\
  \end{tabular}
	\caption{Wybrane reguły ortograficzno-fonetyczne.
	Odpowiednie przekształcenia są realizowane zawsze przy obecności odpowiedniego prawego kontekstu, który jest wspólny dla każdej kolumny.
	\textit{Kolumna 1.} wymaga jednego z następujących ciągów bezpośrednio po prawej ($\varepsilon$ oznacza koniec segmentu): a ą e ę o ó u; \textit{kolumna 2.}: ib ic ić id if ig ih ii ij ik il ił im in iń ip ir is iś it iw iz iź iż i- i$\varepsilon$; \textit{kolumna 3.}: a$\varepsilon$ ach$\varepsilon$ ami$\varepsilon$ ą$\varepsilon$ e$\varepsilon$ ę$\varepsilon$ i$\varepsilon$ o$\varepsilon$ om$\varepsilon$ on$\varepsilon$ um$\varepsilon$; \textit{kolumna 4.}: a ą b c ć d e ę f g h k l ł m n ń o ó p r s ś t u w y z ź ż - $\varepsilon$; \textit{kolumna 5.}:  b c ć d f g h k l ł m n ń p r s ś t w z ź ż - $\varepsilon$.\label{table:orto}}
\end{table}
\begin{table}
  \centering
  \begin{tabular}{c|c|c}
	  wyraz & przekształcenie & użyte reguły / prawy kontekst \\
	  \hline
	  \textit{wielkimi} & \texttt{v$'$elk$'$im$'$i} & v$'$ $\leftarrow$ wi/e, k$'$ $\leftarrow$ k/im, m$'$ $\leftarrow$ m/i$\varepsilon$ \\
	  \textit{panią} & \texttt{pan$'$ą} & n$'$ $\leftarrow$ ni/ą \\
	  \textit{kwiecie} & \texttt{kv$'$et$'$e} & v$'$ $\leftarrow$ wi/e, t$'$ $\leftarrow$ ci/e \\
	  \textit{przemierzyli} & \texttt{přem$'$eřyli} & ř $\leftarrow$ rz/e, m$'$ $\leftarrow$ mi/e, ř $\leftarrow$ rz/y  \\
  \end{tabular}
	\caption{Przykłady przekształceń przy użyciu reguł ortograficzno-fonetycznych.\label{table:ortoprz}}
\end{table}

Warstwa ortograficzno-fonetyczna odpowiada za przekształcenie formy ortograficznej segmentu, otrzymanej na wejściu, na specjalnie zaprojektowaną reprezentację wewnętrzną.
Przypomina ona uproszczony zapis fonetyczny, chociaż zachowuje rozróżnienia (takich jak 'u'-'ó', 'ch'-'h', 'rz'-'ż') istotne dla przewidywania alternacji występujących w tematach wyrazów.
Celem przeprowadzenia tej konwersji jest uproszczenie kolejnych reguł, które mogą korzystać z uogólnień dokonanych już przez tę warstwę.

Reguły opracowano ręcznie na podstawie danych ze Słownika Gramatyczny Języka Polskiego.
Pełen ich spis znajduje się w Dodatku (\textbf{6.1}).

W polskim zapisie ortograficznym formy zawierające ten sam rdzeń często różnią się.
Widać to na przykład w ciągu wyrazów: \textit{pani}, \textit{pań}, \textit{panie}.
Nasze reguły przekształcają je do postaci: \texttt{pan$'$i}, \texttt{pan$'$}, \texttt{pan$'$e}, gdzie dobrze widoczny jest wspólny rdzeń (\texttt{pan$'$}).
Odpowiednie reguły przedstawia Tablica ~\ref{table:orto}.

Fonologiczne przekształcenia tematu mogą być bardziej skomplikowane.
Widać to w wyrazach \textit{niedźwiadek}, \textit{niedźwiedzica}, \textit{niedźwiedź}. Obejmują je reguły dla 'd'.
Pokrewieństwa tych rzeczowników nie wykorzystujemy bezpośrednio w modelu (ze względu na przynależność do różnych części mowy i odmienne lematyzacje), ale jesteśmy w stanie dostrzec wspólność ich tematów.
Dalsze przykłady przedstawia Tablica ~\ref{table:ortoprz}.

Domyślna reguła przepisuje znak wejściowy bez zmian; uruchamia się ona, kiedy żadna z innych reguł nie znajduje zastosowania. Stosowalność reguł wymaga po pierwsze dopasowania ciągu znaków podlegającego przekształceniu, po drugie dopasowania prawego kontekstu (ciągu znaków następującego bezpośrednio po ciągu przekształcanym).

Znak '$'$' służy jako stały operator palatalizacji, do którego redukują się zarówno polskie diakrytyki z ,,kreską'', jak i zapisy ze zmiękczającym 'i'.
Dzięki temu nie muszą istnieć reguły szczegółowe obsługujące każdy z tych przypadków oddzielnie.
Z kolei operator '\textipa{\super{j}}' służy wyrażeniu zmiękczającej roli 'i' tam, gdzie nie powoduje ono pełnej palatalizacji, np. formy \textit{tiara} i \textit{Diana}
zostaną zapisane jako \texttt{t\textipa{\super{j}}ara} i \texttt{D\textipa{\super{j}}ana}.
Analogicznie traktujemy zjawiska występujące w ostatnich zgłoskach wyrazów takich jak \textit{palatalizacja}, \textit{inwazja} czy \textit{Leokadia}.

Jeśli pominąć operator palatalizacji  '$'$', zapis wynikowy realizuje zasadę jeden znak--jeden dźwięk.
Istotnym wymaganiem, jakie stawiamy naszym regułom, jest odwracalność: z sekwencji liter powstającej
na skutek stosowania reguł zawsze da się wydedukować pierwotną postać.

W wypadku słów o rdzeniu zapisywanym według obcej ortografii, które mimo wszystko odmieniają się według zasad polskiej fleksji
(np. nazwisko prezydenta \textit{Giscarda d'Estainga}), prowadzi to do zastosowania specjalnych znaczników w nawiasach klamrowych.
Na przykład reguła

\begin{equation}
\{ng\}n \leftarrow ng
\end{equation}

(stosowana przy dowolnym prawym kontekście) pozwala uwzględnić wymowę 'ng' jako 'n' przy operacjach regułowych. Dzięki temu wydedukowany później rdzeń zbiegnie się z tym uzyskanym z formy miejscownika \textit{d’Estainie}, raportowanej przez Słownik Gramatyczny Języka Polskiego. Jednocześnie jednak kolejne warstwy reguł pozostają ,,świadome'' powiązania 'ng' z 'n', przez co możliwy jest powrót do oryginalnej formy rdzenia. Oczywiście, na mocy pustej reguły ,,domyślnej'', model będzie brał \textit{także} pod uwagę możliwość 'ng' będącego po prostu 'ng', tak jak w słowie \textit{mustang}.

Bardziej typowy przykład przedstawia słowo \textit{Franz}, które dzięki regule

\begin{equation}
\{z\}c \leftarrow z
\end{equation}

(również stosowanej przy dowolnym prawym kontekście) zostanie przekształcone do formy \texttt{Fran\{z\}c}.
Dzięki temu słowo to będzie mogło się odmieniać tak jak \textit{pajac}, a nie jak \textit{markiz} --
stąd w miejscowniku i wołaczu liczby pojedynczej nie pojawi się forma \textit{Franzie}, tylko \textit{Francu}.
Ponieważ w praktyce reguły dotyczące afiksów działają tylko przy początku i końcu wyrazów, element \texttt{\{z\}} nie wywiera wpływu na ich działanie i służy jako techniczne przypomnienie o początkowej formie wyrazu.

Część słów o obcej ortografii wymusza niestety złamanie zasady prostej odwracalności i oddzielne formułowanie specjalnych reguł odwrotnych.
Służą one późniejszemu przejściu od zapisu konwencjonalnego, ustalanego na podstawie stosowania reguł, do zapisu ortograficznego lematu ,,odgadniętego'' przez guesser. Spis owych reguł, obejmujących szczególne przypadki, znajduje się również w Dodatku (\textbf{6.2}).

\section{Warstwa analityczna}

\begin{table}
\vspace*{-1cm}
\hspace*{-2cm}
  \centering
  \begin{tabular}{r|rrrrrrrr}
& \boldmath$\alpha'${\bf y} & \boldmath$\alpha'$ & \boldmath$\alpha'\varepsilon$ & \boldmath$\alpha${\bf y} & \boldmath$\alpha${\bf e} & \boldmath$\alpha\varepsilon$ & \boldmath$\beta${\bf li} \\
\hline
& & & & & & & li $\rightarrow$ \\
\hline
a & & & & & & & $\star$eli $\rightarrow$ a \\
\hline
d$'$, d & d$'$i $\rightarrow$ d$'$ & d$'$ $\rightarrow$ d$'$ & d$'$ $\rightarrow$ d$'$ & dy $\rightarrow$ d & de $\rightarrow$ d & d $\rightarrow$ d & $\star$edli $\rightarrow$ ad  \\

& & & ód$'$ $\rightarrow$ od$'$ & & & ed $\rightarrow$ d & \\

& & & ąd$'$ $\rightarrow$ ęd$'$ & & & ód $\rightarrow$ od & \\

& & & óz$'$d$'$ $\rightarrow$ oz$'$d$'$ & & & ąd $\rightarrow$ ęd & \\

& & & & & & & \\
\hline

k, k$'$ & & k$'$ $\rightarrow$ k$'$ & & k$'$i $\rightarrow$ k & k$'$e $\rightarrow$ k & k $\rightarrow$ k & \\

& & & & & & ek $\rightarrow$ k & \\

& & & & & & ąk $\rightarrow$ ęk & \\
\hline

m$'$, m & m$'$i $\rightarrow$ m$'$ & m$'$ $\rightarrow$ m$'$ & m $\rightarrow$ m$'$ & my $\rightarrow$ m & me $\rightarrow$ m & m $\rightarrow$ m & \\

& & & & & & em $\rightarrow$ m & \\
\hline

n$'$, n & n$'$i $\rightarrow$ n$'$ & n$'$ $\rightarrow$ n$'$ & n$'$ $\rightarrow$ n$'$ & ny $\rightarrow$ n & ne $\rightarrow$ n & n $\rightarrow$ n & \\

& & & en$'$ $\rightarrow$ n$'$ & & & en $\rightarrow$ n & \\

& & & $'$en$'$ $\rightarrow$ n$'$ & & & $'$en $\rightarrow$ n & \\

& & & $\star$d$'$en$'$ $\rightarrow$ edn$'$ & & & & \\

& & & & & & & \\

& & & & & & & \\
\hline

ř, r & řy $\rightarrow$ ř & ř $\rightarrow$ ř & ř $\rightarrow$ ř & ry $\rightarrow$ r & re $\rightarrow$ r & r $\rightarrow$ r & \\

& & & eř $\rightarrow$ ř & & & er $\rightarrow$ r & \\

& & & $'$eř $\rightarrow$ ř & & & $'$er $\rightarrow$ r & \\

& & & óř $\rightarrow$ oř & & & ór $\rightarrow$ or & \\

& & & ójř $\rightarrow$ ojř & & & $\star$cer $\rightarrow$ kr & \\

& & & & & & óbr $\rightarrow$ obr & \\

& & & & & & óstr $\rightarrow$ ostr & \\
\hline

t$'$, t & t$'$i $\rightarrow$ t$'$ & t$'$ $\rightarrow$ t$'$ & t$'$ $\rightarrow$ t$'$ & ty $\rightarrow$ t & te $\rightarrow$ t & t $\rightarrow$ t & $\star$etli $\rightarrow$ ot \\

& & & ót$'$ $\rightarrow$ ot$'$ & & & et $\rightarrow$ t & \\

& & & et$'$ $\rightarrow$ t$'$ & & & ót $\rightarrow$ ot & \\

& & & $'$et$'$ $\rightarrow$ t$'$ & & & ąt $\rightarrow$ ęt & \\

& & & & & & & \\
\hline

v$'$, v & v$'$i $\rightarrow$ v$'$ & v$'$ $\rightarrow$ v$'$ & v $\rightarrow$ v$'$ & vy $\rightarrow$ v & ve $\rightarrow$ v & v $\rightarrow$ v & \\

& & & ev $\rightarrow$ v$'$ & & & ev $\rightarrow$ v & \\

& & & $'$ev $\rightarrow$ v$'$ & & & $'$ev $\rightarrow$ v & \\

& & & óv $\rightarrow$ ov$'$ & & & óv $\rightarrow$ ov & \\

& & & & & \\

& & & & & & & \\
\hline

ž & žy $\rightarrow$ ž & ž $\rightarrow$ ž & ž $\rightarrow$ ž & & & & \\

& & & ež $\rightarrow$ ž & & & & \\

& & & óž $\rightarrow$ ož & & & & \\

& & & ąž $\rightarrow$ ęž & & & & \\
\hline
\end{tabular}
\caption{Wybrane grupy alternacyjne (tytuł kolumny stanowi oznaczenie danej grupy). Po lewej stronie każdej reguły znajduje się ciag obecny w wyrazie przetworzonym przez reguły ortograficzno-fonetyczne, po prawej ciąg obecny w lemacie przewidywanym przez model. Reguły analityczne dotyczą zawsze całej grupy alternacyjnej (a więc całej kolumny). Symbol $\star$ oznacza, że reguła jest nieproduktywna, czyli działa na zamkniętej grupie słów.\label{table:altern}}
\end{table}

Warstwa analityczna działa na ciągach wyjściowych z warstwy ortograficzno-fonetycznej.
To na tym etapie nasz system reguł wykonuje większość swojej faktycznej pracy.
Wydziela on afiksy i rdzeń oraz określa parametry, które ulegną potem mechanicznej intepretacji w kolejnej warstwie.
Wynikiem pracy warstwy analitycznej jest także uzyskanie formy lematu w konwencjonalnym zapisie ortograficzno-fonetycznym. Z niej później zostanie uzyskana forma podstawowa słowa w zwyczajnej ortografii.

Reguły analityczne korzystają z dodatkowej abstrakcji \textbf{grup alternacyjnych}. Tworzą je grupy podreguł zawierających po lewej stronie (przekształcanej) ciąg znaków występujący w formie, a po prawej (docelowej) stronie ciąg występujący w lemacie.
Poszczególne podreguły odnoszą się do konkretnych zjawisk alternacji w języku polskim
-- alternację rozumiemy tutaj jako wymianę głosek występujących przy końcu rdzenia.

Każda taka grupa podreguł jest wydzielana ze względu na przyjmowanie afiksów w podobny sposób i traktowana łącznie przez właściwe reguły analityczne.

\begin{figure}
	\centering
	\textbf{'v$'$elk$'$im$'$i'}
\begin{scriptsize}\[
\left[\begin{array}{ll}
\star-\text{žkolv$'$ek} & \text{suf}:=\text{žkolv$'$ek}\\
\star-\text{žekolv$'$ek} & \text{suf}:=\text{žkolv$'$ek}\\
\star-\text{s$'$kolv$'$ek} & \text{suf}:=\text{s$'$kolv$'$ek}\\
\star-\text{kolv$'$ek} & \text{suf}:=\text{kolv$'$ek}\\
\star-\text{ž} & \text{suf}:=\text{ž}\\
\star-\text{že} & \text{suf}:=\text{ž}\\
\star-\text{ž} & \text{suf}:=\text{že}\\
\star-\text{že} & \text{suf}:=\text{že}\\
\star-\text{s$'$t$'$is$'$} & \text{suf}:=\text{s$'$t$'$is$'$}\\
\star-\text{t$'$is$'$} & \text{suf}:=\text{t$'$is$'$}\\
\star-\text{s$'$} & \text{suf}:=\text{s$'$}\\
\star-\text{s$'$t$'$i} & \text{suf}:=\text{s$'$t$'$i}\\
\star-\text{s$'$ik} & \text{suf}:=\text{sik}\\
\star-\text{s$'$i} & \text{suf}:=\text{si}\\
	\underline{-\varepsilon} & 
\end{array}\right] \otimes \left[\begin{array}{ll}
-\text{\boldmath$\alpha${\bf y}} & \text{flex}:=\text{y}, \downarrow, \text{adj}\\
-\text{\boldmath$\alpha${\bf y}x} & \text{flex}:=\text{ych}, \downarrow, \text{adj}\\
-\text{\boldmath$\alpha${\bf y}m} & \text{flex}:=\text{ym}, \downarrow, \text{adj}\\
	\underline{-\text{\boldmath$\alpha${\bf y}m$'$i}} & \underline{\text{flex}:=\text{ymi}, \downarrow, \text{adj}}\\
-\text{\boldmath$\alpha${\bf e}} & \text{flex}:=\text{e}, \downarrow, \text{adj}\\
-\text{\boldmath$\alpha${\bf e}go} & \text{flex}:=\text{ego}, \downarrow, \text{adj}\\
-\text{\boldmath$\alpha${\bf e}j} & \text{flex}:=\text{ej}, \downarrow, \text{adj}\\
-\text{\boldmath$\alpha${\bf e}mu} & \text{flex}:=\text{emu}, \downarrow, \text{adj}\\
-\text{\boldmath$\alpha$a} & \text{flex}:=\text{a}, \downarrow, \text{adj}\\
-\text{\boldmath$\alpha$ą} & \text{flex}:=\text{ą}, \downarrow, \text{adj}\\
-\text{\boldmath$\alpha$o} & \text{flex}:=\text{o}, \downarrow, \text{adj}\\
-\text{\boldmath$\alpha$u} & \text{flex}:=\text{u}, \downarrow, \text{adj}\\
-\text{\boldmath$\alpha${\bf i}} & \text{flex}:=\text{i}, \downarrow, \text{adj}\\
\star-\text{\boldmath$\alpha\varepsilon$} & \text{flex}:=\text{$\varepsilon$}, \downarrow, \text{adj}\\
D-\text{\boldmath$\alpha${\bf e}m} & \text{flex}:=\text{ym}, \downarrow, \text{adj}\\
D-\text{\boldmath$\alpha${\bf e}mi} & \text{flex}:=\text{ymi}, \downarrow, \text{adj}\\
	& (...)
\end{array}\right] \otimes \left[\begin{array}{ll}
	\underline{+\text{\boldmath$\alpha${\bf y}}} & \underline{\text{lemma}:=\text{y}}\\
\star+\text{\boldmath$\alpha\varepsilon$} & \text{lemma}:=\text{$\varepsilon$}
\end{array}\right]
\]\end{scriptsize}

	\textbf{'pan$'$ą'}
\begin{scriptsize}\[
\left[\begin{array}{ll}
-\text{\boldmath$\alpha'${\bf y}} & \text{flex}:=\text{y}, \uparrow, \text{noun}\\
-\text{\boldmath$\alpha'${\bf y}x} & \text{flex}:=\text{ych}, \uparrow, \text{noun}\\
-\text{\boldmath$\alpha'${\bf y}m} & \text{flex}:=\text{ym}, \uparrow, \text{noun}\\
-\text{\boldmath$\alpha'${\bf y}m$'$i} & \text{flex}:=\text{ymi}, \uparrow, \text{noun}\\
-\text{\boldmath$\alpha'$e} & \text{flex}:=\text{e}, \uparrow, \text{noun}\\
-\text{\boldmath$\alpha'$ego} & \text{flex}:=\text{ego}, \uparrow, \text{noun}\\
-\text{\boldmath$\alpha'$ej} & \text{flex}:=\text{ej}, \uparrow, \text{noun}\\
-\text{\boldmath$\alpha'$em} & \text{flex}:=\text{em}, \uparrow, \text{noun}\\
-\text{\boldmath$\alpha'$emu} & \text{flex}:=\text{emu}, \uparrow, \text{noun}\\
-\text{\boldmath$\alpha'$a} & \text{flex}:=\text{a}, \uparrow, \text{noun}\\
-\text{\boldmath$\alpha'$ax} & \text{flex}:=\text{ach}, \uparrow, \text{noun}\\
-\text{\boldmath$\alpha'$am$'$i} & \text{flex}:=\text{ami}, \uparrow, \text{noun}\\
	\underline{-\text{\boldmath$\alpha'$ą}} & \text{flex}:=\text{ą}, \uparrow, \text{noun}\\
-\text{\boldmath$\alpha'$ę} & \text{flex}:=\text{ę}, \uparrow, \text{noun}\\
-\text{\boldmath$\alpha'$o} & \text{flex}:=\text{o}, \uparrow, \text{noun}\\
-\text{\boldmath$\alpha'$om} & \text{flex}:=\text{om}, \uparrow, \text{noun}\\
-\text{\boldmath$\alpha'$ov$'$i} & \text{flex}:=\text{owi}, \uparrow, \text{noun}\\
	& (...)  
\end{array}\right] \otimes \left[\begin{array}{ll}
\underline{+\text{\boldmath$\alpha'${\bf y}}} & \text{lemma}:=\text{y}\\
+\text{\boldmath$\alpha'$e} & \text{lemma}:=\text{e}\\
+\text{\boldmath$\alpha'$a} & \text{lemma}:=\text{a}\\
+\text{\boldmath$\alpha'$o} & \text{lemma}:=\text{o}\\
+\text{\boldmath$\alpha'$ov$'$e} & \text{lemma}:=\text{owie}\\
+\text{\boldmath$\alpha'$um} & \text{lemma}:=\text{um}\\
+\text{\boldmath$\alpha'\varepsilon$} & \text{lemma}:=\text{$\varepsilon$}
\end{array}\right]
\]\end{scriptsize}

	\textbf{'přem$'$eřyli'}
\begin{scriptsize}\[\hspace{-2cm}
\left[\begin{array}{ll}
	\underline{\varepsilon-} & \underline{\text{pref}:=\text{$\varepsilon$}}\\
\text{n$'$e}- & \text{pref}:=\text{nie}
\end{array}\right] \otimes \left[\begin{array}{ll}
-\varepsilon & \text{flex}:=\text{$\varepsilon$}, \text{flex2}:=\text{$\varepsilon$}\\
-\text{my} & \text{flex}:=\text{my}, \text{flex2}:=\text{$\varepsilon$}\\
-\text{t$'$e} & \text{flex}:=\text{cie}, \text{flex2}:=\text{$\varepsilon$}\\
-\text{š} & \text{flex}:=\text{sz}, \text{flex2}:=\text{$\varepsilon$}\\
 & (...) \\
-\text{\boldmath$\beta${\bf ł}šy} & \text{flex}:=\text{szy}, \text{flex2}:=\text{ł}\\
-\text{\boldmath$\beta${\bf ł}} & \text{flex}:=\text{$\varepsilon$}, \text{flex2}:=\text{ł}\\
-\text{ł} & \text{flex}:=\text{$\varepsilon$}, \text{flex2}:=\text{ł}\\
-\text{ła} & \text{flex}:=\text{a}, \text{flex2}:=\text{ł}\\
-\text{ło} & \text{flex}:=\text{o}, \text{flex2}:=\text{ł}\\
-\text{ły} & \text{flex}:=\text{y}, \text{flex2}:=\text{ł}\\
\underline{-\text{\boldmath$\beta${\bf li}}} & \text{flex}:=\text{i}, \text{flex2}:=\text{ł}\\
-\text{tyx} & \text{flex}:=\text{ych}, \text{flex2}:=\text{t}\\
-\text{tym} & \text{flex}:=\text{ym}, \text{flex2}:=\text{t}\\
-\text{tym$'$i} & \text{flex}:=\text{ymi}, \text{flex2}:=\text{t}\\
-\text{te} & \text{flex}:=\text{e}, \text{flex2}:=\text{t}\\
-\text{ty} & \text{flex}:=\text{y}, \text{flex2}:=\text{t}\\
-\text{tą} & \text{flex}:=\text{ą}, \text{flex2}:=\text{t}\\
	& (...)
\end{array}\right] \otimes \left[\begin{array}{ll}
 \underline{-\text{\boldmath$\alpha'${\bf y}}} & \text{group}:=\text{y}, \text{verb}\\
-\text{\boldmath$\alpha'$e} & \text{group}:=\text{e}, \text{verb}\\
-\text{\boldmath$\alpha'$eje} & \text{group}:=\text{eje}, \text{verb}\\
-\text{\boldmath$\alpha'$ej} & \text{group}:=\text{ej}, \text{verb}\\
-\text{\boldmath$\alpha'${\bf y}j} & \text{group}:=\text{yj}, \text{verb}\\
-\text{\boldmath$\alpha'\varepsilon$} & \text{group}:=\text{$\varepsilon$}, \text{verb}\\
-\text{\boldmath$\alpha'$a} & \text{group}:=\text{a}, \text{verb}\\
D-\text{\boldmath$\alpha'$o} & \text{group}:=\text{a}, \text{verb}
\end{array}\right] \otimes \left[\begin{array}{ll}
 \underline{+\text{\boldmath$\alpha$'{\bf y}}\text{t$'$}} & \text{lemma}:=\text{palat-ć}\\
+\text{\boldmath$\alpha$'}\text{et$'$} & \text{lemma}:=\text{palat-eć}
\end{array}\right]
\]\end{scriptsize}
	\caption{Przykłady reguł analitycznych stosowanych do form uzyskanych z reguł warstwy ortograficzno-fonetycznej. Wykorzystane reguły są podkreślone.\label{figure:analitprz}}
	\end{figure}

Patrząc niejako z lotu ptaka, można dostrzec następujący układ grup alternacyjnych.
\begin{itemize}
	\item Spółgłoski (funkcjonalnie) miękkie, gdzie mieszczą się zarówno zwyczajne palatalizacje, jak 's$'$' (z 's'),
		jak i spółgłoski pełniące analogiczną formę dla spółgłosek twardych z powodów historycznych (tak jak 't$'$' dla 't', powstające z 'ci' lub 'ć' na mocy reguł ortograficzno-fonetycznych).
		Zbiór ten tworzą grupy alternacyjne o symbolach zawierających $\boldsymbol{\alpha'}$ (znak $'$ zwraca uwagę na zmiękczenie).
	\item Spółgłoski (funkcjonalnie) twarde, zazwyczaj posiadające swoje (funkcjonalne) zmiękczenia. Mieszczą się tutaj grupy alternacyjne o symbolu $\boldsymbol{\alpha}$ bez zmiękczenia. Wiele leksemów wymienia w swoich różnych formach spółgłoski funkcjonalnie miękkie na twarde (por. np. mianownik \textit{niebo}, miejscownik \textit{niebie}).
	\item Podczas gdy powyższe grupy opisują zjawiska występujące we właściwie wszystkich częściach mowy polszczyzny, istnieją też grupy dotyczące zjawisk specyficznych.
		Podreguła \textit{el $\rightarrow$ ał} pozwala nam przejść od formy \textit{bielszy} do lematu \textit{biały}.
		Ma ona oznaczenie grupy alternacyjnej $\boldsymbol{\kappa}$, odnoszące się do stopniowania przymiotników, podobnie jak $\boldsymbol{\lambda}$. $\boldsymbol{\xi}$ dotyczy odmiany przysłówków.
		Pozostałe grupy alternacyjne w większości dotyczą specyficznej odmiany czasowników.
\end{itemize}

Podobny podział wśród polskich spółgłosek przy zachowaniach fleksyjnych zauważał Jan Tokarski \cite{tok1}\cite{tok2}, ale tego rodzaju obserwacji dokonywano w polskim językoznawstwie już co najmniej w początkach XIX wieku \cite{mrozny}; pewnym źródłem inspiracji był dla nas system opublikowany w Internecie przez Grzegorza Jagodzińskiego \cite{jagoda}.

Przykładowy wybór grup alternacyjnych przedstawia Tabela ~\ref{table:altern}.
Pełna lista grup alternacyjnych znajduje się w Dodatku (\textbf{6.3}). Jej zawartość została wygenerowana na podstawie SGJP w wersji z 30.07.2017.
	
Sposób opisu alternacji staje się jaśniejszy, kiedy weźmiemy pod uwagę ich zastosowanie jako budulec właściwych reguł analitycznych.
Poniższa reguła znajduje i usuwa sufiks składający się z wyniku zastosowania alternacji \boldmath$\alpha'$ oraz ciągu \textit{ego},
przypisując całemu segmentowi parametry wypisane z prawej strony.

\[
	-\text{\boldmath$\alpha'$ego} \>\>\> \text{flex}:=\text{ego}, \uparrow, \text{adj}
\]

Przykładami pasujących sufiksów są te zawarte w wyrazach \textit{bliźniego} czy \textit{Idziego}.

Reguły są zasadniczo zbudowane z oznaczenia grupy alternacyjnej i afiksu oraz wyszczególnienia przypiswanych atrybutów.
Reguła może ucinać prefiks (postać \textbf{x-}), ucinać sufiks (postać \textbf{-x}) lub dodawać sufiks (postać \textbf{+x}).
Znak minus oznacza, że przekształcenie opisane w grupie alternacyjnej odbywa się zgodnie z kierunkiem strzałki (a więc w formie musi się znajdować zbitka znajdująca się z lewej strony i zostaje ona zamieniona na tę po prawej).
Plus oznacza przekształcenie odwrotne.

Każdy wyraz, żeby jego identyfikacja się udała, musi pasować do jakiegoś schematu grup reguł analitycznych wykonywanych po kolei, od lewej do prawej.
Wybór schematów pasujących do kilku przykładowych wyrazów przedstawia Rysunek ~\ref{figure:analitprz}. Wewnątrz schematu grupy są zawarte w dużych nawiasach kwadratowych.

Zwróćmy uwagę, jak model poradziłby sobie z uzyskanym wcześniej ciągiem \texttt{v$'$elk$'$im$'$i} (\textit{wielkimi}).
Wśród pierwszej grupy reguł w schemacie, najbardziej po lewej, segment zostaje uchwycony przez ,,pustą'' regułę wykrywającą znacznik końca wyrazu ($\varepsilon$).
Z drugiej grupy pasuje następująca reguła analityczna:

\[
-\text{\boldmath$\alpha${\bf y}m$'$i} \>\>\> \text{flex}:=\text{ymi}, \downarrow, \text{adj}
\]

Jest tak, ponieważ reguła \textit{k$'$i $\rightarrow$ k} należy do grupy alternacyjnej \boldmath$\alpha${\bf y}
-- przy sprawdzaniu stosowalności reguły analitycznej jej symbol (w tym wypadku \boldmath$\alpha${\bf y}) wskazuje na miejsce zadziałania alternacji.
Reguła ucina sufiks, zastosowuje zawartą w nim regułę alternacyjną i przypisuje segmentowi atrybuty podane po prawej stronie.
Teraz ma więc on postać \texttt{v$'$elk} i parametry \texttt{flex:=ymi, adj}.

Obecność alternacji \boldmath$\alpha${\bf y} na końcu powoduje również zastosowanie reguły z ostatniej grupy (najbardziej na prawo w schemacie):

\[
+\text{\boldmath$\alpha${\bf y}} \>\>\> \text{lemma}:=\text{y}
\]

Ponieważ jest to grupa ze znakiem $+$ (dodająca sufiks), reguły alternacyjne działają w przeciwną stronę.
Odwrócona reguła \textit{k$'$i $\leftarrow$ k} przekształca segment do postaci \texttt{v$'$elk'i}.
Jest to hipoteza modelu co do lematu (i okazuje się ona słuszna: lemat \textit{wielki}).

Nieco podobnie w wypadku segmentu \texttt{pan$'$ą} osiągamy (zgodnie z odpowiednim schematem z Rysunku ~\ref{figure:analitprz})
najpierw postać \texttt{pan$'$} na mocy reguły alternacyjnej \textit{n$'$ $\rightarrow$ n$'$}
   i analitycznej
	[ $-\text{\boldmath$\alpha'$ą}$ $\>\>\>$ $\text{flex}:=\text{ą}, \uparrow, \text{noun}$ ].
Do uzyskania poprawnego lematu \texttt{pan$'$i} służy następnie reguła
\[
	+\text{\boldmath$\alpha'${\bf y}} \>\>\> \text{lemma}:=\text{y}
	\]
	dzięki odwróceniu alternacji \textit{n$'$i $\rightarrow$ n$'$} należącej do grupy $\text{\boldmath$\alpha'${\bf y}}$.

Nieco bardziej skomplikowany proces analizy morfologicznej prześledzić można na przykładzie segmentu \texttt{přem$'$eřyli} (\textit{przemierzyli}). Na pierwszym etapie przykładamy regułę 

\[
-\text{\boldmath$\beta${\bf li}} \>\>\> \text{flex}:=\text{i}, \text{flex2}:=\text{ł}
  \]

  -- o obecności alternacji \textbf{$\beta${\bf li}} (nb. jest to w całości wytłuszczony symbol alternacji!) decyduje reguła \textit{li $\rightarrow$ }, gdzie po prawej stronie znajduje się pusty napis. Wynikiem jest forma \texttt{přem$'$eřy}.
  Następna reguła w schemacie, posługująca się alternacją \textit{řy $\rightarrow$ ř} należącą do grupy $\alpha'${\bf y},
  wykrywa przynależność czasownika do jednej z grup morfologicznych i przypisuje segmentowi parametry \texttt{group:=y, verb}. Możemy wówczas przejść do określenia postaci podstawowej lematu, za co odpowiada reguła

\[
+\boldmath\alpha'{\bf y}\text{t$'$} \>\>\> \text{lemma}:=\text{palat-ć}.
  \]

Przywraca ona dłuższą zbitkę \texttt{řy} i dodaje końcówkę \texttt{t$'$}, prowadząc do formy \texttt{přem$'$eřyt$'$} (\textit{przemierzyć}). Wynik to zatem znowu forma podstawowa opatrzona zbiorem parametrów, które posłużą następnej warstwie do rekonstrukcji oznaczeń morfosyntaktycznych.

\section{Lematy kończące się wygłosem}
Kiedy przewidywany lemat kończy się wygłosem, często występują dodatkowe zjawiska, które kształtują jego ostateczną postać.
Na przykład gdy przewidujemy lemat \textit{męż}, w istocie pojawi się on jako \textit{mąż}: 'ę' wymieni się tutaj na 'ą'
w ostatniej samogłosce wyrazu.

Z myślą o tego typu przypadkach wytworzyliśmy dodatkowy zestaw reguł dla rzeczowników, których lemat kończy się wygłosem, a nie samogłoską.
Próbują one przewidzieć, jakie dodatkowe zjawiska mogą wystąpić na końcu takiego lematu.
Z materiału dostępnego w SGJP wydobyliśmy listę zamieszczoną w Dodatku (\textbf{6.5}).
Zawiera ona zaobserwowane formy wygłosowe rzeczownika parametryzowane przez dwie ostatnie głoski tematu.

Dla wygenerowanych przez model lematów sprawdzamy, czy nie pasują one do jakichś pozycji z listy. Jeżeli tak, opcje zasugerowane przez owe pozycje są podawane jako odpowiedź modelu.
Jeśli żadna wersja lematu nie pasuje, selekcja nie jest wykonywana (pozostawiamy sam lemat wyjściowy).

\begin{table}
  \centering
  \begin{tabular}{p{7cm}|l|l|l}
interpretacja & cat & flex & lemma\\
\hline
adj:sg:inst.loc:m1.m2.m3.n:pos adj:pl:dat:m1.m2.m3.f.n:pos & adj & ym & y\\
adj:pl:nom.voc:m1:pos & adj & i & y\\
adj:pl:gen.loc:m1.m2.m3.f.n:pos adj:pl:acc:m1:pos & adj & ych & y\\
adj:pl:inst:m1.m2.m3.f.n:pos & adj & ymi & y\\
adjp & adj & u & y\\
adja & adj & o & y\\
\hline
\end{tabular}
\caption{Próbka reguł warstwy intepretacji. Pierwsza kolumna zawiera zapis otagowania morfosyntaktycznego, a pozostałe wymagane dla nich wartości parametrów o nazwach podanych w tytule kolumny (czasami, jak w wypadku \texttt{adj:pl:gen.loc:m1.m2.m3.f.n:pos} i \texttt{adj:pl:acc:m1:pos}, takie same zestawy atrybutów mogą pasować do kilku zbiorów tagów).\label{table:interp}}
\end{table}

\section{Warstwa interpretacji}

Warstwa interpretacji przypisuje segmentowi interpretację morfosyntaktyczną (oznaczenie z tagsetu SGJP \cite{sgjp}) na podstawie wartości atrybutów
określonych przez warstwę analityczną.
Pełen spis tych reguł znajduje się w Dodatku (\textbf{6.6}).

Niewielką próbkę przedstawia poza tym Tabela ~\ref{table:interp}. Jak pamiętamy, dla segmentu \textit{wielkimi} uzyskaliśmy hipotezę lematu \texttt{v$'$elk'i} (\textit{wielki}) oraz zestaw atrybutów \texttt{flex:=ymi, adj, lemma:=ymi}. Wartość \texttt{adj} wskazuje na parametr \texttt{cat} (dla niektórych parametrów o zamkniętym zbiorze wartości pomijamy często ich nazwę, bo jest określona jednoznacznie). Jak wskazuje Tabela ~\ref{table:interp}, pozwala to wywnioskować otagowanie \texttt{adj:pl:inst:m1.m2.m3.f.n:pos}.

Atrybuty dostarczają strukturalnej informacji o morfologii, które mogą być wykorzystane do dalszej dezambiguacji morfosyntaktycznej.

Wszystkie możliwe przypisywane parametry to:
\begin{description}
\item{\texttt{cat:}} kategoria (legalne wartości to \texttt{noun}, \texttt{adj}, \texttt{adj:grad}, \texttt{adv}, \texttt{verb}, \texttt{ndm});
\item{\texttt{flex:}} końcówka fleksyjna;
\item{\texttt{flex2:}} afiks wyznaczający fleksem czasownikowy;
\item{\texttt{group:}} afiks związany ze schematem odmiany czasownika;
\item{\texttt{grad:}} afiks stopnia wyższego;
\item{\texttt{pref:}} prefiks (\texttt{naj}, \texttt{nie} lub pusty);
\item{\texttt{lemma:}} końcówka lematu;
\item{\texttt{con:}} ostatnia głoska tematu formy;
\item{\texttt{con2:}} rozróżnienie grupy alternacyjnej formy, gdy \texttt{con} nie określa jej jednoznacznie;
\item{\texttt{lcon:}} ostatnia głoska tematu lematu (równa \texttt{con});
\item{\texttt{lcon2:}} rozróżnienie grupy alternacyjnej lematu, gdy \texttt{lcon} nie określa jej jednoznacznie;
\item{\texttt{agl:}} określenie grupy alternacyjnej imiesłowu biernego, czasownika w czasie przeszłym, imiesłowu uprzedniego;
\item{\texttt{agl2:}} rozróżnienie grupy alternacyjnej we wspomnianych wypadkach, jeżeli \texttt{lcon} nie określa jej jednoznacznie;
\item{\texttt{palat:}} atrybut zdefiniowany tylko dla rzeczowników i przymiotników w stopniu równym; ma wartość \texttt{t}, gdy \texttt{con} jest funkcjonalnie miękkie i \texttt{n}, gdy \texttt{con} jest funkcjonalnie twarde;
\item{\texttt{velar:}} atrybut zdefiniowany tylko dla rzeczowników z \texttt{palat:=n} oraz \texttt{flex:=ie} bądź \texttt{flex:=u}; ma wartość \texttt{t}, gdy \texttt{con} $\in \{$\texttt{x}, \texttt{h}, \texttt{g}, \texttt{k}$\}$ oraz i \texttt{n} w przeciwnym wypadku;
\item{\texttt{orth:}} atrybut zdefiniowany tylko dla reguł z udziałem obcej ortografii, wymagających odgadnięcia dokładnego zapisu lematu; wartością jest dodany przez regułę sufiks o obcej ortografii.
\end{description}
Atrybuty nieokreślone dla danego wyrazu lub reguły mają wartości puste.

Reguły przypisujące interpretacje zestawom atrybutów układają się w quasi-paradygmaty odmiany,
gdzie np. wartości atrybutów \texttt{cat}, \texttt{lemma}, \texttt{gender} pozostają stałe,
a zmienia się tylko \texttt{flex} (wskazujący na konkretny przypadek czy osobę).
Należy jednak pamiętać, że dany lemat nie jest do takich ,,paradygmatów'' sztywno przypisany;
nie musi on mieć form pochodzących tylko z jednego paradygmatu
i nie musi mieć wszystkich form występujących w danym paradygmacie.

Nie próbujemy przewidywać aspektu czasowników.

\section{Generowanie reguł}
W ramach technicznej optymalizacji na podstawe powyższego modelu, formułującego \textbf{reguły szczegółowe}, generowane są ujednolicone \textbf{reguły operacyjne}.
Reguły te wykonują pracę wszystkich opisanych warstw za jednym zamachem.
Operują one bezpośrednio na otrzymywanej formie: ucinają sufiks i ewentualnie prefiks, przypisują sufiks formy bazowej-lematu oraz nadają interpretację morfosyntaktyczną.
W ten sposób dokonują lematyzacji i interpretacji morfosyntaktycznej dla słów spełniających wymagania całego ciągu reguł z kolejnych warstw.

Reguł tych uzyskujemy łącznie 30983. Zostały wygenerowane na podstawie słownika uzupełnionego o przykładowe formy gwarowe i dodatkowe odmienione akronimy.

Poniższa tabela przedstawia liczbę reguł z podziałem na ich typy i części mowy:
\begin{center}
\begin{tabular}{l|r|r|r|r|r}
                              &  noun &  adj & adv &  verb &  suma \\
\hline
	           produktywne    &  7534 & 1501 & 150 &  9107 & 18292 \\
\hline
	\textbf{*} nieproduktywne &   209 &  389 & --- &  3701 &  4299 \\
\hline
	\textbf{A} obce           &  1275 &  --- & --- &   --- &  1275 \\
\hline
	\textbf{B} obce           &   206 &  --- & --- &   --- &   206 \\
\hline
	\textbf{C} akronimy       &   557 &  --- & --- &   --- &   557 \\
\hline
	\textbf{D} gwarowe        &  2639 &  380 & --- &  3474 &  6493 \\
\hline
	         suma             & 12420 & 2270 & 150 & 16282 & 31122 \\
\end{tabular}
\end{center}
Grupa ,,obcych A'' dotyczy słów o obcej ortografii, w których pierwotna postać rdzenia jest zawarta w obserwowanej formie.
W wypadku ,,obcych B'' pierwotna postać rdzenia nie jest zawarta w obserwowanej formie i musi zostać odgadnięta (np. dopełniacz \textit{Chiraka} od lematu \textit{Chirac}).
Powoduje to, że reguły typu B wprowadzają znaczną niejednoznaczność.

Wszystkim regułom towarzyszą informacje o frekwencji, wskazujące liczbę form ze słownika lematyzowalnych za pomocą danej reguły. Umożliwia to w praktyce wykorzystywanie informacji, jak bardzo jest ona ,,pospolita''.

% WORKNOTE Na czym to jest sprawdzane? Na SGJP? -> Tak

%Wykomentowane są wartości wyliczone bezpośrednio z modelu (bez generowania freq_rules)
%Reguły dla rzeczowników poprawnie opisują fleksję 142366/(142366+1257)=99,12\% lematów rzeczownikowych.
Reguły dla rzeczowników poprawnie opisują fleksję $\frac{143643}{143643+343}$ = 99,76\% lematów rzeczownikowych. %FIXME: uaktualnić
%Reguły dla przymiotników poprawnie opisują fleksję 66390/(66390+26) = 99,96\% lematów przymiotnikowych.
Reguły dla przymiotników poprawnie opisują fleksję $\frac{66426}{66426+26}$ = 99,96\% lematów przymiotnikowych.
%Reguły dla przysłówków poprawnie opisują fleksję 25816/(25816+421) = 98,40\% lematów przysłówkowych. 
Reguły dla przysłówków poprawnie opisują fleksję $\frac{25839}{25839+422}$ = 98,39\% lematów przysłówkowych. 
%Reguły dla czasowników poprawnie opisują fleksję 28558/(28558+1238) = 95,85\% lematów czasownikowych,
%a gdy usuniemy lematy, które powstały poprzez dodanie prefiksu wartość ta wzrasta do 13848/(13848+171) = 98,78\%.
Reguły dla czasowników poprawnie opisują fleksję $\frac{28571}{28571+1229}$ = 95,88\% lematów czasownikowych,
a gdy usuniemy lematy, które powstały poprzez dodanie prefiksu wartość ta wzrasta do $\frac{13852}{13852+167}$ = 98,81\%.
Takie wartości wskazują, że opisywany model w sposób poprawny i pełny opisuje zawartą w SGJP fleksję języka polskiego.

Leksemy niepokryte przez model odmieniają się w sposób nieregularny -- powinny one stanowić zamknięty zbiór.
Jest to szczególnie istotne przy czasownikach,
gdzie 167 nieregularnych rdzeni generuje, po uzupełnieniu o prefiksy, 1229 nieregularnych leksemów.
W przypadku przysłówków,
na 422 niepokryte przez model leksemy składają się zasadniczo przysłówki niestopniowalne i niepochodzące od przymiotników.
% \begin{scriptsize}

\section{Podsumowanie}

Przedstawiony model stanowi naszym zdaniem istotny krok
w organizowaniu empirycznej wiedzy na temat polskiej morfonologii
i w wykorzystywaniu tej wiedzy przez komputer.
Struktura warstwowa pozwala na znaczną oszczędność i powiększenie, na ile to możliwe, jasności opisu złożonych zjawisk.
Warstwa ortograficzno-fonetyczna usuwa niespójności wynikające ze sposobu zapisu wyrazów,
warstwa analityczna wykrywa alternacje i wydziela końcówki,
zaś warstwa interpretacji przegląda zebrane informacje i przyporządkowuje segmentowi ustandardyzowane tagi.

System reguł pozwala w formalny sposób wytworzyć zbiór potencjalnych interpretacji słowa, który często okazuje się duży.
Następnym zadaniem, przed którym staje program komputerowy albo użytkownik języka, jest wybranie tej interpretacji,
jaką ostatecznie zdecyduje się przypisać wyrazowi. Jest to już jednak odmienny problem, wymagający znajomości
przynajmniej najbliższego kontekstu frazy.
Wykracza tym samym poza zakres zainteresowań modelu fleksji.

\begin{thebibliography}{1}
	\bibitem{jagoda} http://grzegorj.private.pl/gram/pl/wymiany.html (dostęp 9.05.2018)
	\bibitem{ENIAM} Wojciech Jaworski, Jakub Kozakoszczak 2016: \textit{ENIAM: Categorial Syntactic-Semantic Parser for Polish},
		w \textit{Proceedings of COLING 2016, the 26th International Conference on Computational Linguistics},
		s. 243-247.
	\bibitem{Morfeusz} Witold Kieraś, Marcin Woliński 2017: \textit{Morfeusz 2 – analizator i generator fleksyjny dla języka polskiego},
		,,Język Polski'', 2017, 97, 1, s. 75-83.
	\bibitem{mrozny} Józef Mroziński 1822: \textit{Pierwsze zasady grammatyki języka polskiego}. Warszawa. (s. 30-31)
	\bibitem{rok} Teresa Rokicka 2002: \textit{Komputerowy model alternacji tematu fleksyjnego polskich fleksemów odmiennych}. Kraków.
	\bibitem{SGJP} Zygmunt Saloni, Marcin Woliński, Robert Wołosz, Włodzimierz Gruszczyński, Danuta Skowrońska,
		\textit{Grammatical Dictionary of Polish — online version},
		http://sgjp.pl (dostęp: 16.07.2018).
	\bibitem{tok1} Jan Tokarski 1951: \textit{Czasownik polski}. Warszawa. (s. 53)
	\bibitem{tok2} Jan Tokarski 2001: \textit{Fleksja polska}. Warszawa. (s. 60)
	\bibitem{Concraft} Jakub Waszczuk 2012: \textit{Harnessing the CRF complexity with domain-specific constraints. The case of morphosyntactic tagging of a highly inflected language},
		w \textit{Proceedings of the 24th International Conference on Computational Linguistics (COLING 2012)},
		s. 2789–2804.
\end{thebibliography}

\section{Dodatek}
\subsection{Pełne reguły warstwy ortograficzno-fonetycznej}

W poniższej tabeli oznaczenia kolumn identyfikują prawy kontekst; oznacza to, że regułę z danej kolumny można zastosować, gdy po rozpoznawanej sekwencji znaków znajduje się:

\begin{enumerate}
\item a ą e ę o ó u
\item ib ic ić id if ig ih ii ij ik il ił im in iń ip ir is iś it iw iz iź iż i- i$\varepsilon$
\item a$\varepsilon$ ach$\varepsilon$ ami$\varepsilon$ ą$\varepsilon$ e$\varepsilon$ ę$\varepsilon$ i$\varepsilon$ o$\varepsilon$ om$\varepsilon$ on$\varepsilon$ um$\varepsilon$
\item a ą b c ć d e ę f g h k l ł m n ń o ó p r s ś t u w y z ź ż - $\varepsilon$
\item b c ć d f g h k l ł m n ń p r s ś t w z ź ż - $\varepsilon$
\item i
\end{enumerate}

Symbol $\varepsilon$ oznacza koniec segmentu.

\begin{center}
\begin{longtable}{l|l|l|l|l|l}
1 & 2 & 3 & 4 & 5 & 6 \\
\hline
 &  &  &  &  & a\textipa{\super{j}} $\leftarrow$ a\\
b$'$ $\leftarrow$ bi & b$'$ $\leftarrow$ b & bj $\leftarrow$ bi &  &  & \\
d$'$ $\leftarrow$ dzi & d$'$ $\leftarrow$ dz &  & \textipa{Z} $\leftarrow$ dz & d$'$ $\leftarrow$ dź & \\
d\textipa{\super{j}} $\leftarrow$ di & d\textipa{\super{j}} $\leftarrow$ d & dj $\leftarrow$ di &  &  & \\
 &  &  &  &  & e\textipa{\super{j}} $\leftarrow$ e\\
f$'$ $\leftarrow$ fi & f$'$ $\leftarrow$ f & fj $\leftarrow$ fi &  &  & \\
g$'$ $\leftarrow$ gi & g$'$ $\leftarrow$ g & g$'$j $\leftarrow$ gi &  &  & \\
h\textipa{\super{j}} $\leftarrow$ hi & h\textipa{\super{j}} $\leftarrow$ h &  &  &  & \\
 &  &  &  &  & i\textipa{\super{j}} $\leftarrow$ i\\
k$'$ $\leftarrow$ ki & k$'$ $\leftarrow$ k & k$'$j $\leftarrow$ ki &  &  & \\
l\textipa{\super{j}} $\leftarrow$ li &  & lj $\leftarrow$ li &  &  & \\
m$'$ $\leftarrow$ mi & m$'$ $\leftarrow$ m & m$'$j $\leftarrow$ mi &  &  & \\
n$'$ $\leftarrow$ ni & n$'$ $\leftarrow$ n & n$'$j $\leftarrow$ ni &  & n$'$ $\leftarrow$ ń & \\
 &  &  &  &  & o\textipa{\super{j}} $\leftarrow$ o\\
p$'$ $\leftarrow$ pi & p$'$ $\leftarrow$ p & pj $\leftarrow$ pi &  &  & \\
r\textipa{\super{j}} $\leftarrow$ ri & r\textipa{\super{j}} $\leftarrow$ r & rj $\leftarrow$ ri & ř $\leftarrow$ rz &  & \\
s$'$ $\leftarrow$ si & s$'$ $\leftarrow$ s &  &  & s$'$ $\leftarrow$ ś & \\
t$'$ $\leftarrow$ ci & t$'$ $\leftarrow$ c &  &  & t$'$ $\leftarrow$ ć & \\
t\textipa{\super{j}} $\leftarrow$ ti & t\textipa{\super{j}} $\leftarrow$ t & tj $\leftarrow$ ti &  &  & \\
 &  &  &  &  & u\textipa{\super{j}} $\leftarrow$ u\\
v$'$ $\leftarrow$ wi & v$'$ $\leftarrow$ w & vj $\leftarrow$ wi & v $\leftarrow$ w &  & \\
x\textipa{\super{j}} $\leftarrow$ chi & x\textipa{\super{j}} $\leftarrow$ ch & x\textipa{\super{j}}j $\leftarrow$ chi & x $\leftarrow$ ch &  & \\
 &  &  &  &  & y\textipa{\super{j}} $\leftarrow$ y\\
z$'$ $\leftarrow$ zi & z$'$ $\leftarrow$ z &  &  & z$'$ $\leftarrow$ ź & \\
 &  &  &  &  & ó\textipa{\super{j}} $\leftarrow$ ó\\
č\textipa{\super{j}} $\leftarrow$ czi & č\textipa{\super{j}} $\leftarrow$ cz &  & č $\leftarrow$ cz &  & \\
š\textipa{\super{j}} $\leftarrow$ szi & š\textipa{\super{j}} $\leftarrow$ sz &  & š $\leftarrow$ sz &  & \\
ž\textipa{\super{j}} $\leftarrow$ żi & ž\textipa{\super{j}} $\leftarrow$ ż & žj $\leftarrow$ żi & ž $\leftarrow$ ż &  & \\
\textipa{\v{Z}}\textipa{\super{j}} $\leftarrow$ dżi & \textipa{\v{Z}}\textipa{\super{j}} $\leftarrow$ dż & \textipa{\v{Z}}j $\leftarrow$ dżi & \textipa{\v{Z}} $\leftarrow$ dż &  & \\
\end{longtable}
\end{center}

Oto kolejne cztery reguły przydatne w wypadku sekwencji 'rz' i 'ei':
\begin{center}
\begin{tabular}{l|l}
reguła & prawy kontekst \\
\hline
r	 $\leftarrow$ r	& zi \\
mar	 $\leftarrow$ mar	& z\\
m$'$er	 $\leftarrow$ mier	& z\\
n$'$e	 $\leftarrow$ nie	& i\\
\end{tabular}
\end{center}

Kolejne reguły dotyczą \textbf{słów obcych}, przede wszystkim nazw własnych. W ich wypadku interpretacja nazw kolumn przedstawia się następująco:

\begin{enumerate}
\item a ą e ę o ó u
\item ib ic ić id if ig ih ii ij ik il ił im in iń ip ir is iś it iw iz iź iż i- iε iv ix iq
\item a ą b c ć d e ę f g h k l ł m n ń o ó p r s ś t u w y z ź ż - ε v x q
\item iε
\end{enumerate}

\begin{center}
\begin{longtable}{l|l|l|l|l|l}
	1 & 2 & 3 & 4 \\
\hline
\{ay\}aj $\leftarrow $ay & \{dh\}dʲ $\leftarrow $dh & \{dh\}d $\leftarrow $dh & \{ni\}n′ $\leftarrow $ni \\
\{ey\}ej $\leftarrow $ey & \{gh\}g′ $\leftarrow $gh & \{gh\}g $\leftarrow $gh & \{ri\}rʲ $\leftarrow $ri \\
\{oy\}oj $\leftarrow $oy & \{kh\}k′ $\leftarrow $kh & \{kh\}k $\leftarrow $kh & \{ny\}n′ $\leftarrow $ny \\
\{ai\}aj $\leftarrow $ai & \{nh\}n′ $\leftarrow $nh & \{nh\}n $\leftarrow $nh &   \\
\{dh\}dʲ $\leftarrow $dhi & \{th\}tʲ $\leftarrow $th & \{th\}t $\leftarrow $th &   \\
\{gh\}g′ $\leftarrow $ghi &   &   &   \\
\{kh\}k′ $\leftarrow $khi &   &   &   \\
\{nh\}n′ $\leftarrow $nhi &   &   &   \\
\{th\}tʲ $\leftarrow $thi &   &   &   \\
\end{longtable}
\end{center}

Następujące reguły mają zastosowanie przy dowolnym prawym kontekście:
\begin{multicols}{3}\begin{itemize}
\item \{ch\}š $\leftarrow$ ch
\item \{q\}k $\leftarrow$ q
\item \{ng\}n $\leftarrow$ ng
\item \{tch\}č $\leftarrow$ tch
\item \{sh\}š $\leftarrow$ sh
\item \{w\}ł $\leftarrow$ w
\item \{leigh\}l $\leftarrow$ leigh
\item \{au\}ał $\leftarrow$ au
\item \{sch\}š $\leftarrow$ sch
\item \{tsch\}č $\leftarrow$ tsch
\item \{z\}c $\leftarrow$ z
\end{itemize}\end{multicols}

Następujące reguły mają zastosowanie jedynie na końcu segmentu:
\begin{multicols}{3}\begin{itemize}
\item \{zs\}ž $\leftarrow$ zs
\item \{cs\}č $\leftarrow$ cs
\item \{ay\}aj $\leftarrow$ ay
\item \{ey\}ej $\leftarrow$ ey
\item \{oy\}oj $\leftarrow$ oy
\item \{ai\}aj $\leftarrow$ ai
\item \{dieu\}dʲe $\leftarrow$ dieu
\item \{dieu\}dʲi $\leftarrow$ dieu
\item \{quieu\}k′e $\leftarrow$ quieu
\item \{quieu\}k′i $\leftarrow$ quieu
\item \{lieu\}lʲe $\leftarrow$ lieu
\item \{lieu\}lʲi $\leftarrow$ lieu
\item \{rie\}rʲe $\leftarrow$ rie
\item \{rie\}rʲi $\leftarrow$ rie
\item \{gie\}ǯʲe $\leftarrow$ gie
\item \{gie\}ǯʲi $\leftarrow$ gie
\item \{kie\}k′e $\leftarrow$ kie
\item \{kie\}k′i $\leftarrow$ kie
\item \{tie\}tʲe $\leftarrow$ tie
\item \{tie\}tʲi $\leftarrow$ tie
\item \{pie\}p′e $\leftarrow$ pie
\item \{pie\}p′i $\leftarrow$ pie
\item \{die\}dʲe $\leftarrow$ die
\item \{die\}dʲi $\leftarrow$ die
\item \{bee\}b′e $\leftarrow$ bee
\item \{bee\}b′i $\leftarrow$ bee
\item \{chais\}še $\leftarrow$ chais
\item \{lais\}le $\leftarrow$ lais
\item \{nais\}ne $\leftarrow$ nais
\item \{rès\}re $\leftarrow$ rès
\item \{rés\}re $\leftarrow$ rés
\item \{ré\}re $\leftarrow$ ré
\item \{mée\}me $\leftarrow$ mée
\item \{ge\}g′e $\leftarrow$ ge
\item \{ke\}k′e $\leftarrow$ ke
\item \{by\}b′i $\leftarrow$ by
\item \{dy\}dʲi $\leftarrow$ dy
\item \{dí\}dʲi $\leftarrow$ dí
\item \{phy\}f′i $\leftarrow$ phy
\item \{guy\}g′i $\leftarrow$ guy
\item \{ky\}k′i $\leftarrow$ ky
\item \{my\}m′i $\leftarrow$ my
\item \{li\}li $\leftarrow$ li
\item \{ly\}li $\leftarrow$ ly
\item \{ry\}rʲi $\leftarrow$ ry
\item \{sy\}sʲi $\leftarrow$ sy
\item \{cy\}sʲi $\leftarrow$ cy
\item \{şi\}sʲi $\leftarrow$ şi
\item \{thy\}tʲi $\leftarrow$ thy
\item \{de\}d $\leftarrow$ de
\item \{fe\}f $\leftarrow$ fe
\item \{phe\}f $\leftarrow$ phe
\item \{ge\}ǯ $\leftarrow$ ge
\item \{ges\}ǯ $\leftarrow$ ges
\item \{gue\}g $\leftarrow$ gue
\item \{gues\}g $\leftarrow$ gues
\item \{ke\}k $\leftarrow$ ke
\item \{que\}k $\leftarrow$ que
\item \{ques\}k $\leftarrow$ ques
\item \{le\}l $\leftarrow$ le
\item \{les\}l $\leftarrow$ les
\item \{me\}m $\leftarrow$ me
\item \{ne\}n $\leftarrow$ ne
\item \{gne\}n′ $\leftarrow$ gne
\item \{re\}r $\leftarrow$ re
\item \{rue\}r $\leftarrow$ rue
\item \{se\}s $\leftarrow$ se
\item \{ce\}s $\leftarrow$ ce
\item \{che\}š $\leftarrow$ che
\item \{te\}t $\leftarrow$ te
\item \{the\}t $\leftarrow$ the
\item \{ve\}v $\leftarrow$ ve
\item \{we\}ł $\leftarrow$ we
\item \{se\}z $\leftarrow$ se
\item \{ge\}ž $\leftarrow$ ge
\item \{oe\}oł $\leftarrow$ oe
\end{itemize}\end{multicols}
% \end{scriptsize}

\subsection{Reguły ortograficzno-fonetyczne -- odwrotne}
\begin{longtable}{l|p{10cm}}
reguła & prawy kontekst \\
\hline
\{A\}a $\rightarrow$ A & \\
a $\rightarrow$ A & $\varepsilon$\\
\{B\}b $\rightarrow$ B & \\
\{C\}k $\rightarrow$ C & \\
\{C\}c $\rightarrow$ C & \\
\{D\}d $\rightarrow$ D & \\
\{E\}e $\rightarrow$ E & \\
\{F\}f $\rightarrow$ F & \\
\{G\}g $\rightarrow$ G & \\
\{H\}h $\rightarrow$ H & \\
\{I\}j $\rightarrow$ I & \\
\{j\}jot $\rightarrow$ J & \\
\{J\}jot $\rightarrow$ J & \\
\{J\}j $\rightarrow$ J & \\
\{K\}k $\rightarrow$ K & \\
\{L\}l $\rightarrow$ L & \\
\{M\}m $\rightarrow$ M & \\
\{N\}n $\rightarrow$ N & \\
\{O\}o $\rightarrow$ O & \\
\{P\}p $\rightarrow$ P & \\
\{R\}r $\rightarrow$ R & \\
\{S\}s $\rightarrow$ S & \\
\{T\}t $\rightarrow$ T & \\
\{U\}u $\rightarrow$ U & \\
\{v\}v $\rightarrow$ V & \\
\{V\}v $\rightarrow$ V & \\
\{W\}v $\rightarrow$ W & \\
\{x\}ks $\rightarrow$ X & \\
\{X\}ks $\rightarrow$ X & \\
\{Y\}y $\rightarrow$ Y & \\
\{Z\}zet $\rightarrow$ Z & \\
\{Z\}z $\rightarrow$ Z & \\
a $\rightarrow$ a & j\\
a $\rightarrow$ a & $\varepsilon$\\
a $\rightarrow$ ah & $\varepsilon$\\
\{ai\}aj $\rightarrow$ ai & $\varepsilon$\\
\{ai\}aj $\rightarrow$ ai & a e o u \{eu\} ó ą ę\\
\{au\}ał $\rightarrow$ au & \\
\{ay\}aj $\rightarrow$ ay & $\varepsilon$\\
\{ay\}aj $\rightarrow$ ay & a e o u \{eu\} ó ą ę\\
aja $\rightarrow$ ayah & $\varepsilon$\\
b $\rightarrow$ b & $\varepsilon$\\
\{bee\}b$'$i $\rightarrow$ bee & $\varepsilon$\\
\{bee\}b$'$e $\rightarrow$ bee & $\varepsilon$\\
b $\rightarrow$ bes & $\varepsilon$\\
\{by\}b$'$i $\rightarrow$ by & $\varepsilon$\\
k $\rightarrow$ c & a\\
k $\rightarrow$ c & o\\
k $\rightarrow$ c & $\varepsilon$\\
kk $\rightarrow$ cc & o\\
s $\rightarrow$ ce & $\varepsilon$\\
\{ce\}s $\rightarrow$ ce & $\varepsilon$\\
\{ch\}š $\rightarrow$ ch & \\
\{chais\}še $\rightarrow$ chais & $\varepsilon$\\
\{che\}š $\rightarrow$ che & $\varepsilon$\\
k $\rightarrow$ cq & $\varepsilon$\\
k $\rightarrow$ cques & $\varepsilon$\\
\{cs\}č $\rightarrow$ cs & $\varepsilon$\\
kt $\rightarrow$ ct & $\varepsilon$\\
\{cy\}s\textipa{\super{j}}i $\rightarrow$ cy & $\varepsilon$\\
d $\rightarrow$ d & a\\
d $\rightarrow$ d & $\varepsilon$\\
d $\rightarrow$ de & $\varepsilon$\\
\{de\}d $\rightarrow$ de & $\varepsilon$\\
d $\rightarrow$ dh & a\\
\{dh\}d\textipa{\super{j}} $\rightarrow$ dh & i- ib ic id if ig ih ii ij ik il im in ip ir is it iv iw ix iz i\{ ić ič ił iń iř iś iš iź iž i\textipa{\v{Z}} i\textipa{Z} i\textipa{\super{j}} i$\varepsilon$\\
\{dh\}d $\rightarrow$ dh & - a b c d e f g h i k l m n o p r s t u v w x y z \{ ó ą ć č ę ł ń ř ś š ź ž \textipa{\v{Z}} \textipa{Z} $\varepsilon$\\
d $\rightarrow$ dh & $\varepsilon$\\
\{dh\}d\textipa{\super{j}} $\rightarrow$ dhi & a e o u \{eu\} ó ą ę\\
\{die\}d\textipa{\super{j}}i $\rightarrow$ die & $\varepsilon$\\
\{die\}d\textipa{\super{j}}e $\rightarrow$ die & $\varepsilon$\\
\{dieu\}d\textipa{\super{j}}i $\rightarrow$ dieu & $\varepsilon$\\
\{dieu\}d\textipa{\super{j}}e $\rightarrow$ dieu & $\varepsilon$\\
\{dy\}d\textipa{\super{j}}i $\rightarrow$ dy & $\varepsilon$\\
\{dí\}d\textipa{\super{j}}i $\rightarrow$ dí & $\varepsilon$\\
e $\rightarrow$ e & j\\
ej $\rightarrow$ ey & $\varepsilon$\\
\{ey\}ej $\rightarrow$ ey & $\varepsilon$\\
\{ey\}ej $\rightarrow$ ey & a e o u \{eu\} ó ą ę\\
f $\rightarrow$ f & $\varepsilon$\\
f $\rightarrow$ fe & $\varepsilon$\\
\{fe\}f $\rightarrow$ fe & $\varepsilon$\\
g $\rightarrow$ g & a\\
g $\rightarrow$ g & $\varepsilon$\\
g$'$e $\rightarrow$ ge & $\varepsilon$\\
\{ge\}ž $\rightarrow$ ge & $\varepsilon$\\
\{ge\}\textipa{\v{Z}} $\rightarrow$ ge & $\varepsilon$\\
\{ge\}g$'$e $\rightarrow$ ge & $\varepsilon$\\
g$'$el $\rightarrow$ gel & $\varepsilon$\\
\{ges\}\textipa{\v{Z}} $\rightarrow$ ges & $\varepsilon$\\
g $\rightarrow$ gh & a\\
\{gh\}g$'$ $\rightarrow$ gh & i- ib ic id if ig ih ii ij ik il im in ip ir is it iv iw ix iz i\{ ić ič ił iń iř iś iš iź iž i\textipa{\v{Z}} i\textipa{Z} i\textipa{\super{j}} i$\varepsilon$\\
\{gh\}g $\rightarrow$ gh & - a b c d e f g h i k l m n o p r s t u v w x y z \{ ó ą ć č ę ł ń ř ś š ź ž \textipa{\v{Z}} \textipa{Z} $\varepsilon$\\
\{gh\}g$'$ $\rightarrow$ ghi & a e o u \{eu\} ó ą ę\\
\{gie\}\textipa{\v{Z}}\textipa{\super{j}}i $\rightarrow$ gie & $\varepsilon$\\
\{gie\}\textipa{\v{Z}}\textipa{\super{j}}e $\rightarrow$ gie & $\varepsilon$\\
\{gne\}n$'$ $\rightarrow$ gne & $\varepsilon$\\
g $\rightarrow$ gue & $\varepsilon$\\
\{gue\}g $\rightarrow$ gue & $\varepsilon$\\
g $\rightarrow$ gues & $\varepsilon$\\
\{gues\}g $\rightarrow$ gues & $\varepsilon$\\
\{guy\}g$'$i $\rightarrow$ guy & $\varepsilon$\\
k $\rightarrow$ k & \\
k $\rightarrow$ k & $\varepsilon$\\
k $\rightarrow$ ke & $\varepsilon$\\
k$'$e $\rightarrow$ ke & $\varepsilon$\\
\{ke\}k $\rightarrow$ ke & $\varepsilon$\\
\{ke\}k$'$e $\rightarrow$ ke & $\varepsilon$\\
k $\rightarrow$ kh & a\\
\{kh\}k$'$ $\rightarrow$ kh & i- ib ic id if ig ih ii ij ik il im in ip ir is it iv iw ix iz i\{ ić ič ił iń iř iś iš iź iž i\textipa{\v{Z}} i\textipa{Z} i\textipa{\super{j}} i$\varepsilon$\\
\{kh\}k $\rightarrow$ kh & - a b c d e f g h i k l m n o p r s t u v w x y z \{ ó ą ć č ę ł ń ř ś š ź ž \textipa{\v{Z}} \textipa{Z} $\varepsilon$\\
\{kh\}k$'$ $\rightarrow$ khi & a e o u \{eu\} ó ą ę\\
k$'$i $\rightarrow$ kie & $\varepsilon$\\
\{kie\}k$'$i $\rightarrow$ kie & $\varepsilon$\\
\{kie\}k$'$e $\rightarrow$ kie & $\varepsilon$\\
k$'$i $\rightarrow$ kij & $\varepsilon$\\
k$'$i $\rightarrow$ koj & $\varepsilon$\\
ks $\rightarrow$ kx & $\varepsilon$\\
k$'$i $\rightarrow$ ky & $\varepsilon$\\
\{ky\}k$'$i $\rightarrow$ ky & $\varepsilon$\\
k$'$i $\rightarrow$ kyj & $\varepsilon$\\
\{lj\}lj $\rightarrow$ lj & a\\
\{lj\}lj $\rightarrow$ lj & e\\
l $\rightarrow$ l & i\\
\{lais\}le $\rightarrow$ lais & $\varepsilon$\\
\{le\}l $\rightarrow$ le & $\varepsilon$\\
\{leigh\}l $\rightarrow$ leigh & \\
\{les\}l $\rightarrow$ les & $\varepsilon$\\
\{li\}li $\rightarrow$ li & $\varepsilon$\\
\{lieu\}l\textipa{\super{j}}i $\rightarrow$ lieu & $\varepsilon$\\
\{lieu\}l\textipa{\super{j}}e $\rightarrow$ lieu & $\varepsilon$\\
li $\rightarrow$ ly & $\varepsilon$\\
\{ly\}li $\rightarrow$ ly & $\varepsilon$\\
m $\rightarrow$ m & $\varepsilon$\\
m $\rightarrow$ me & $\varepsilon$\\
\{me\}m $\rightarrow$ me & $\varepsilon$\\
\{my\}m$'$i $\rightarrow$ my & $\varepsilon$\\
\{mée\}me $\rightarrow$ mée & $\varepsilon$\\
n $\rightarrow$ n & a\\
n $\rightarrow$ n & $\varepsilon$\\
\{nais\}ne $\rightarrow$ nais & $\varepsilon$\\
n $\rightarrow$ ne & $\varepsilon$\\
\{ne\}n $\rightarrow$ ne & $\varepsilon$\\
n $\rightarrow$ nes & $\varepsilon$\\
\{ng\}n $\rightarrow$ ng & \\
n $\rightarrow$ ng & $\varepsilon$\\
n $\rightarrow$ nh & a\\
\{nh\}n$'$ $\rightarrow$ nh & i- ib ic id if ig ih ii ij ik il im in ip ir is it iv iw ix iz i\{ ić ič ił iń iř iś iš iź iž i\textipa{\v{Z}} i\textipa{Z} i\textipa{\super{j}} i$\varepsilon$\\
\{nh\}n $\rightarrow$ nh & - a b c d e f g h i k l m n o p r s t u v w x y z \{ ó ą ć č ę ł ń ř ś š ź ž \textipa{\v{Z}} \textipa{Z} $\varepsilon$\\
n $\rightarrow$ nh & $\varepsilon$\\
\{nh\}n$'$ $\rightarrow$ nhi & a e o u \{eu\} ó ą ę\\
\{ni\}n$'$ $\rightarrow$ ni & i$\varepsilon$\\
\{ny\}n$'$ $\rightarrow$ ny & i$\varepsilon$\\
o $\rightarrow$ o & j\\
\{oe\}oł $\rightarrow$ oe & $\varepsilon$\\
\{oy\}oj $\rightarrow$ oy & $\varepsilon$\\
\{oy\}oj $\rightarrow$ oy & a e o u \{eu\} ó ą ę\\
oja $\rightarrow$ oya & $\varepsilon$\\
\{pj\}pj $\rightarrow$ pj & e\\
f $\rightarrow$ ph & $\varepsilon$\\
f $\rightarrow$ phe & $\varepsilon$\\
\{phe\}f $\rightarrow$ phe & $\varepsilon$\\
\{phy\}f$'$i $\rightarrow$ phy & $\varepsilon$\\
\{pie\}p$'$i $\rightarrow$ pie & $\varepsilon$\\
\{pie\}p$'$e $\rightarrow$ pie & $\varepsilon$\\
\{q\}k $\rightarrow$ q & \\
k $\rightarrow$ q & $\varepsilon$\\
k $\rightarrow$ que & $\varepsilon$\\
\{que\}k $\rightarrow$ que & $\varepsilon$\\
k $\rightarrow$ ques & $\varepsilon$\\
\{ques\}k $\rightarrow$ ques & $\varepsilon$\\
\{quieu\}k$'$i $\rightarrow$ quieu & $\varepsilon$\\
\{quieu\}k$'$e $\rightarrow$ quieu & $\varepsilon$\\
r $\rightarrow$ r & a\\
r $\rightarrow$ r & $\varepsilon$\\
r $\rightarrow$ re & $\varepsilon$\\
\{re\}r $\rightarrow$ re & $\varepsilon$\\
r $\rightarrow$ res & $\varepsilon$\\
r $\rightarrow$ rh & a\\
r $\rightarrow$ rh & $\varepsilon$\\
\{ri\}r\textipa{\super{j}} $\rightarrow$ ri & i$\varepsilon$\\
\{rie\}r\textipa{\super{j}}i $\rightarrow$ rie & $\varepsilon$\\
\{rie\}r\textipa{\super{j}}e $\rightarrow$ rie & $\varepsilon$\\
r $\rightarrow$ rs & $\varepsilon$\\
\{rue\}r $\rightarrow$ rue & $\varepsilon$\\
\{ry\}r\textipa{\super{j}}i $\rightarrow$ ry & $\varepsilon$\\
\{rès\}re $\rightarrow$ rès & $\varepsilon$\\
\{ré\}re $\rightarrow$ ré & $\varepsilon$\\
\{rés\}re $\rightarrow$ rés & $\varepsilon$\\
s $\rightarrow$ s & k$'$i\\
s $\rightarrow$ s & $\varepsilon$\\
\{sch\}š $\rightarrow$ sch & \\
s $\rightarrow$ se & $\varepsilon$\\
\{se\}z $\rightarrow$ se & $\varepsilon$\\
\{se\}s $\rightarrow$ se & $\varepsilon$\\
\{sh\}š $\rightarrow$ sh & \\
\{sy\}s\textipa{\super{j}}i $\rightarrow$ sy & $\varepsilon$\\
sk$'$i $\rightarrow$ szky & $\varepsilon$\\
t $\rightarrow$ t & a\\
t $\rightarrow$ t & $\varepsilon$\\
\{tch\}č $\rightarrow$ tch & \\
t $\rightarrow$ te & $\varepsilon$\\
\{te\}t $\rightarrow$ te & $\varepsilon$\\
t $\rightarrow$ tes & $\varepsilon$\\
t $\rightarrow$ th & a\\
\{th\}t\textipa{\super{j}} $\rightarrow$ th & i- ib ic id if ig ih ii ij ik il im in ip ir is it iv iw ix iz i\{ ić ič ił iń iř iś iš iź iž i\textipa{\v{Z}} i\textipa{Z} i\textipa{\super{j}} i$\varepsilon$\\
\{th\}t $\rightarrow$ th & - a b c d e f g h i k l m n o p r s t u v w x y z \{ ó ą ć č ę ł ń ř ś š ź ž \textipa{\v{Z}} \textipa{Z} $\varepsilon$\\
t $\rightarrow$ th & $\varepsilon$\\
s $\rightarrow$ th & $\varepsilon$\\
t $\rightarrow$ the & $\varepsilon$\\
\{the\}t $\rightarrow$ the & $\varepsilon$\\
t $\rightarrow$ thes & $\varepsilon$\\
\{th\}t\textipa{\super{j}} $\rightarrow$ thi & a e o u \{eu\} ó ą ę\\
\{thy\}t\textipa{\super{j}}i $\rightarrow$ thy & $\varepsilon$\\
\{tie\}t\textipa{\super{j}}i $\rightarrow$ tie & $\varepsilon$\\
\{tie\}t\textipa{\super{j}}e $\rightarrow$ tie & $\varepsilon$\\
\{tsch\}č $\rightarrow$ tsch & \\
t $\rightarrow$ tt & $\varepsilon$\\
\{v\}v$'$ $\rightarrow$ v & i- ib ic id if ig ih ii ij ik il im in ip ir is it iv iw ix iz i\{ ić ič ił iń iř iś iš iź iž i\textipa{\v{Z}} i\textipa{Z} i\textipa{\super{j}} i$\varepsilon$\\
\{v\}v $\rightarrow$ v & - a b c d e f g h i k l m n o p r s t u v w x y z \{ ó ą ć č ę ł ń ř ś š ź ž \textipa{\v{Z}} \textipa{Z} $\varepsilon$\\
\{v\}v $\rightarrow$ ve & $\varepsilon$\\
\{ve\}v $\rightarrow$ ve & $\varepsilon$\\
\{v\}vj $\rightarrow$ vi & am$'$i$\varepsilon$ ax$\varepsilon$ a$\varepsilon$ e$\varepsilon$ i$\varepsilon$ om$\varepsilon$ on$\varepsilon$ o$\varepsilon$ um$\varepsilon$ ą$\varepsilon$ ę$\varepsilon$\\
\{v\}v$'$ $\rightarrow$ vi & a e o u \{eu\} ó ą ę\\
\{w\}ł $\rightarrow$ w & \\
\{we\}ł $\rightarrow$ we & $\varepsilon$\\
ks $\rightarrow$ x & a\\
\{x\}ks$'$ $\rightarrow$ x & i- ib ic id if ig ih ii ij ik il im in ip ir is it iv iw ix iz i\{ ić ič ił iń iř iś iš iź iž i\textipa{\v{Z}} i\textipa{Z} i\textipa{\super{j}} i$\varepsilon$\\
\{x\}ks $\rightarrow$ x & - a b c d e f g h i k l m n o p r s t u v w x y z \{ ó ą ć č ę ł ń ř ś š ź ž \textipa{\v{Z}} \textipa{Z} $\varepsilon$\\
ks $\rightarrow$ x & $\varepsilon$\\
\{x\}ks$'$ $\rightarrow$ xi & a e o u \{eu\} ó ą ę\\
\{z\}zet $\rightarrow$ z & \\
\{z\}c $\rightarrow$ z & \\
\{zs\}ž $\rightarrow$ zs & $\varepsilon$\\
\{Ć\}t$'$ $\rightarrow$ Ć & \\
\{Ł\}ł $\rightarrow$ Ł & \\
\{ři\}r\textipa{\super{j}}i $\rightarrow$ ři & $\varepsilon$\\
\{Ś\}s$'$ $\rightarrow$ Ś & \\
\{şi\}s\textipa{\super{j}}i $\rightarrow$ şi & $\varepsilon$\\
\{ż\}žet $\rightarrow$ Ż & \\
\{Ż\}žet $\rightarrow$ Ż & \\
\{Ż\}ž $\rightarrow$ Ż & \\
\end{longtable}

\subsection{Grupy alternacyjne warstwy analitycznej}

\begin{longtable}{r|rrr}
 & \boldmath$\alpha'${\bf y} & \boldmath$\alpha'$ & \boldmath$\alpha'\varepsilon$\\
\hline
b$'$ & b$'$i $\rightarrow$ b$'$ & b$'$ $\rightarrow$ b$'$ & b $\rightarrow$ b$'$\\
 &  &  & ąb $\rightarrow$ ęb$'$\\
 &  &  & ób $\rightarrow$ ob$'$\\
\hline
d$'$ & d$'$i $\rightarrow$ d$'$ & d$'$ $\rightarrow$ d$'$ & d$'$ $\rightarrow$ d$'$\\
 &  &  & ód$'$ $\rightarrow$ od$'$\\
 &  &  & ąd$'$ $\rightarrow$ ęd$'$\\
 &  &  & óz$'$d$'$ $\rightarrow$ oz$'$d$'$\\
\hline
f$'$ & f$'$i $\rightarrow$ f$'$ & f$'$ $\rightarrow$ f$'$ & f $\rightarrow$ f$'$\\
\hline
m$'$ & m$'$i $\rightarrow$ m$'$ & m$'$ $\rightarrow$ m$'$ & m $\rightarrow$ m$'$\\
\hline
n$'$ & n$'$i $\rightarrow$ n$'$ & n$'$ $\rightarrow$ n$'$ & n$'$ $\rightarrow$ n$'$\\
 &  &  & en$'$ $\rightarrow$ n$'$\\
 &  &  & $'$en$'$ $\rightarrow$ n$'$\\
 &  &  & $\star$d$'$en$'$ $\rightarrow$ edn$'$\\
\hline
p$'$ & p$'$i $\rightarrow$ p$'$ & p$'$ $\rightarrow$ p$'$ & p $\rightarrow$ p$'$\\
\hline
s$'$ & s$'$i $\rightarrow$ s$'$ & s$'$ $\rightarrow$ s$'$ & s$'$ $\rightarrow$ s$'$\\
 &  &  & $'$es$'$ $\rightarrow$ s$'$\\
\hline
t$'$ & t$'$i $\rightarrow$ t$'$ & t$'$ $\rightarrow$ t$'$ & t$'$ $\rightarrow$ t$'$\\
 &  &  & ót$'$ $\rightarrow$ ot$'$\\
 &  &  & et$'$ $\rightarrow$ t$'$\\
 &  &  & $'$et$'$ $\rightarrow$ t$'$\\
\hline
v$'$ & v$'$i $\rightarrow$ v$'$ & v$'$ $\rightarrow$ v$'$ & v $\rightarrow$ v$'$\\
 &  &  & ev $\rightarrow$ v$'$\\
 &  &  & $'$ev $\rightarrow$ v$'$\\
 &  &  & óv $\rightarrow$ ov$'$\\
\hline
z$'$ & z$'$i $\rightarrow$ z$'$ & z$'$ $\rightarrow$ z$'$ & z$'$ $\rightarrow$ z$'$\\
 &  &  & óz$'$ $\rightarrow$ oz$'$\\
 &  &  & ąz$'$ $\rightarrow$ ęz$'$\\
\hline
l & li $\rightarrow$ l & l $\rightarrow$ l & l $\rightarrow$ l\\
 &  &  & el $\rightarrow$ l\\
 &  &  & $'$el $\rightarrow$ l\\
 &  &  & ól $\rightarrow$ ol\\
 &  &  & ódl $\rightarrow$ odl\\
\hline
c & cy $\rightarrow$ c & c $\rightarrow$ c & c $\rightarrow$ c\\
 &  &  & ec $\rightarrow$ c\\
 &  &  & $'$ec $\rightarrow$ c\\
 &  &  & řec $\rightarrow$ rc\\
 &  &  & $\star$n$'$ec $\rightarrow$ $'$en$'$c\\
\hline
č & čy $\rightarrow$ č & č $\rightarrow$ č & č $\rightarrow$ č\\
 &  &  & eč $\rightarrow$ č\\
 &  &  & óč $\rightarrow$ oč\\
\hline
\textipa{Z} & \textipa{Z}y $\rightarrow$ \textipa{Z} & \textipa{Z} $\rightarrow$ \textipa{Z} & \textipa{Z} $\rightarrow$ \textipa{Z}\\
 &  &  & ó\textipa{Z} $\rightarrow$ o\textipa{Z}\\
\hline
\textipa{\v{Z}} & \textipa{\v{Z}}y $\rightarrow$ \textipa{\v{Z}} & \textipa{\v{Z}} $\rightarrow$ \textipa{\v{Z}} & \textipa{\v{Z}} $\rightarrow$ \textipa{\v{Z}}\\
\hline
ř & řy $\rightarrow$ ř & ř $\rightarrow$ ř & ř $\rightarrow$ ř\\
 &  &  & eř $\rightarrow$ ř\\
 &  &  & $'$eř $\rightarrow$ ř\\
 &  &  & óř $\rightarrow$ oř\\
 &  &  & ójř $\rightarrow$ ojř\\
\hline
š & šy $\rightarrow$ š & š $\rightarrow$ š & š $\rightarrow$ š\\
 &  &  & eš $\rightarrow$ š\\
\hline
ž & žy $\rightarrow$ ž & ž $\rightarrow$ ž & ž $\rightarrow$ ž\\
 &  &  & ež $\rightarrow$ ž\\
 &  &  & óž $\rightarrow$ ož\\
 &  &  & ąž $\rightarrow$ ęž\\
\hline
\textipa{\super{j}} & \textipa{\super{j}}i $\rightarrow$ \textipa{\super{j}} & \textipa{\super{j}} $\rightarrow$ \textipa{\super{j}} & \\
 & \textipa{\super{j}}i $\rightarrow$ j &  & \\
\hline
j & ji $\rightarrow$ j & j $\rightarrow$ j & j $\rightarrow$ j\\
 &  &  & ój $\rightarrow$ oj\\
 &  &  & yj $\rightarrow$ j\\
 &  &  & ij $\rightarrow$ j\\
 &  &  & $'$ij $\rightarrow$ j\\
\hline
g$'$ &  & g$'$ $\rightarrow$ g$'$ & \\
\hline
k$'$ &  & k$'$ $\rightarrow$ k$'$ & \\
\hline
a & a\textipa{\super{j}}i $\rightarrow$ a & a $\rightarrow$ a & \\
\hline
e & e\textipa{\super{j}}i $\rightarrow$ e & e $\rightarrow$ e & \\
\hline
\end{longtable}

\begin{longtable}{r|rrrrrr}
 & \boldmath$\alpha${\bf y} & \boldmath$\alpha${\bf e} & \boldmath$\alpha$ & \boldmath$\alpha${\bf i} & \boldmath$\alpha${\bf ie} & \boldmath$\alpha\varepsilon$\\
\hline
b & by $\rightarrow$ b & be $\rightarrow$ b & b $\rightarrow$ b & b$'$i $\rightarrow$ b & b$'$e $\rightarrow$ b & b $\rightarrow$ b\\
 &  &  &  &  &  & eb $\rightarrow$ b\\
 &  &  &  &  &  & ób $\rightarrow$ ob\\
 &  &  &  &  &  & ąb $\rightarrow$ ęb\\
 &  &  &  &  &  & ós$'$b $\rightarrow$ os$'$b\\
 &  &  &  &  &  & óz$'$b $\rightarrow$ oz$'$b\\
\hline
x & xy $\rightarrow$ x & xe $\rightarrow$ x & x $\rightarrow$ x & s$'$i $\rightarrow$ x & še $\rightarrow$ x & x $\rightarrow$ x\\
 &  &  &  &  &  & ex $\rightarrow$ x\\
% &  &  &  &  &  & x $\rightarrow$ ks\\
\hline
d & dy $\rightarrow$ d & de $\rightarrow$ d & d $\rightarrow$ d & d$'$i $\rightarrow$ d & d$'$e $\rightarrow$ d & d $\rightarrow$ d\\
 &  &  &  & ed$'$i $\rightarrow$ ad & z$'$d$'$e $\rightarrow$ zd & ed $\rightarrow$ d\\
 &  &  &  &  & ed$'$e $\rightarrow$ ad & ód $\rightarrow$ od\\
 &  &  &  &  & ed$'$e $\rightarrow$ od & ąd $\rightarrow$ ęd\\
 &  &  &  &  & ez$'$d$'$e $\rightarrow$ azd & \\
\hline
f & fy $\rightarrow$ f & fe $\rightarrow$ f & f $\rightarrow$ f & f$'$i $\rightarrow$ f & f$'$e $\rightarrow$ f & f $\rightarrow$ f\\
\hline
h & hy $\rightarrow$ h & he $\rightarrow$ h & h $\rightarrow$ h & z$'$i $\rightarrow$ h & še $\rightarrow$ h & h $\rightarrow$ h\\
 &  &  &  &  & že $\rightarrow$ h & \\
\hline
ł & ły $\rightarrow$ ł & łe $\rightarrow$ ł & ł $\rightarrow$ ł & li $\rightarrow$ ł & le $\rightarrow$ ł & ł $\rightarrow$ ł\\
 &  &  &  & eli $\rightarrow$ oł & s$'$le $\rightarrow$ sł & eł $\rightarrow$ ł\\
 &  &  &  & s$'$li $\rightarrow$ sł & z$'$le $\rightarrow$ zł & $'$eł $\rightarrow$ ł\\
 &  &  &  & z$'$li $\rightarrow$ zł & ele $\rightarrow$ ał & el $\rightarrow$ oł\\
 &  &  &  &  & ele $\rightarrow$ oł & ół $\rightarrow$ oł\\
 &  &  &  &  & etle $\rightarrow$ atł & $'$oł $\rightarrow$ ł\\
 &  &  &  &  & lle $\rightarrow$ łł & řeł $\rightarrow$ rł\\
\hline
m & my $\rightarrow$ m & me $\rightarrow$ m & m $\rightarrow$ m & m$'$i $\rightarrow$ m & m$'$e $\rightarrow$ m & m $\rightarrow$ m\\
 &  &  &  & s$'$m$'$i $\rightarrow$ sm & s$'$m$'$e $\rightarrow$ sm & em $\rightarrow$ m\\
\hline
n & ny $\rightarrow$ n & ne $\rightarrow$ n & n $\rightarrow$ n & n$'$i $\rightarrow$ n & n$'$e $\rightarrow$ n & n $\rightarrow$ n\\
 &  &  &  & en$'$i $\rightarrow$ on & en$'$e $\rightarrow$ on & en $\rightarrow$ n\\
 &  &  &  & $\star$cen$'$i $\rightarrow$ t$'$on & s$'$n$'$e $\rightarrow$ sn & $'$en $\rightarrow$ n\\
 &  &  &  & $\star$\textipa{Z}en$'$i $\rightarrow$ d$'$on & z$'$n$'$e $\rightarrow$ zn & \\
 &  &  &  & s$'$n$'$i $\rightarrow$ sn &  & \\
 &  &  &  & z$'$n$'$i $\rightarrow$ zn &  & \\
\hline
p & py $\rightarrow$ p & pe $\rightarrow$ p & p $\rightarrow$ p & p$'$i $\rightarrow$ p & p$'$e $\rightarrow$ p & p $\rightarrow$ p\\
 &  &  &  &  &  & ep $\rightarrow$ p\\
 &  &  &  &  &  & $'$ep $\rightarrow$ p\\
 &  &  &  &  &  & óp $\rightarrow$ op\\
\hline
r & ry $\rightarrow$ r & re $\rightarrow$ r & r $\rightarrow$ r & řy $\rightarrow$ r & ře $\rightarrow$ r & r $\rightarrow$ r\\
 &  &  &  &  & eře $\rightarrow$ ar & er $\rightarrow$ r\\
 &  &  &  &  & etře $\rightarrow$ atr & $'$er $\rightarrow$ r\\
 &  &  &  &  & ře $\rightarrow$ rr & ór $\rightarrow$ or\\
 &  &  &  &  &  & $\star$cer $\rightarrow$ kr\\
 &  &  &  &  &  & óbr $\rightarrow$ obr\\
 &  &  &  &  &  & óstr $\rightarrow$ ostr\\
\hline
s & sy $\rightarrow$ s & se $\rightarrow$ s & s $\rightarrow$ s & s$'$i $\rightarrow$ s & s$'$e $\rightarrow$ s & s $\rightarrow$ s\\
 &  &  &  &  & es$'$e $\rightarrow$ as & $'$es $\rightarrow$ s\\
\hline
t & ty $\rightarrow$ t & te $\rightarrow$ t & t $\rightarrow$ t & t$'$i $\rightarrow$ t & t$'$e $\rightarrow$ t & t $\rightarrow$ t\\
 &  &  &  & s$'$t$'$i $\rightarrow$ st & et$'$e $\rightarrow$ at & et $\rightarrow$ t\\
 &  &  &  & et$'$i $\rightarrow$ ot & et$'$e $\rightarrow$ ot & ót $\rightarrow$ ot\\
 &  &  &  &  & s$'$t$'$e $\rightarrow$ st & ąt $\rightarrow$ ęt\\
 &  &  &  &  & es$'$t$'$e $\rightarrow$ ast & \\
\hline
v & vy $\rightarrow$ v & ve $\rightarrow$ v & v $\rightarrow$ v & v$'$i $\rightarrow$ v & v$'$e $\rightarrow$ v & v $\rightarrow$ v\\
 &  &  &  &  &  & ev $\rightarrow$ v\\
 &  &  &  &  &  & $'$ev $\rightarrow$ v\\
 &  &  &  &  &  & óv $\rightarrow$ ov\\
\hline
z & zy $\rightarrow$ z & ze $\rightarrow$ z & z $\rightarrow$ z & z$'$i $\rightarrow$ z & z$'$e $\rightarrow$ z & z $\rightarrow$ z\\
 &  &  &  &  &  & ez $\rightarrow$ z\\
 &  &  &  &  &  & $'$ez $\rightarrow$ z\\
 &  &  &  &  &  & óz $\rightarrow$ oz\\
 &  &  &  &  &  & ąz $\rightarrow$ ęz\\
\hline
g & g$'$i $\rightarrow$ g & g$'$e $\rightarrow$ g & g $\rightarrow$ g & \textipa{Z}y $\rightarrow$ g & \textipa{Z}e $\rightarrow$ g & g $\rightarrow$ g\\
 &  &  &  &  &  & eg $\rightarrow$ g\\
 &  &  &  &  &  & óg $\rightarrow$ og\\
 &  &  &  &  &  & órg $\rightarrow$ org\\
 &  &  &  &  &  & ąg $\rightarrow$ ęg\\
\hline
k & k$'$i $\rightarrow$ k & k$'$e $\rightarrow$ k & k $\rightarrow$ k & cy $\rightarrow$ k & ce $\rightarrow$ k & k $\rightarrow$ k\\
 &  &  &  &  &  & ek $\rightarrow$ k\\
 &  &  &  &  &  & ąk $\rightarrow$ ęk\\
\hline
o & oy $\rightarrow$ o & oe $\rightarrow$ o & o $\rightarrow$ o &  & o\textipa{\super{j}}i $\rightarrow$ o & o\textipa{\super{j}}i $\rightarrow$ o\\
\hline
u & uy $\rightarrow$ u & ue $\rightarrow$ u & u $\rightarrow$ u &  & u\textipa{\super{j}}i $\rightarrow$ u & u\textipa{\super{j}}i $\rightarrow$ u\\
\hline
\end{longtable}

\begin{longtable}{r|rrr}
 & \boldmath$\kappa'$ & \boldmath$\lambda'$ & \boldmath$\xi'$\\
\hline
m$'$ &  &  & m$'$ $\rightarrow$ m$'$\\
\hline
n$'$ & n$'$ $\rightarrow$ n$'$ & n$'$ $\rightarrow$ n$'$ & n$'$ $\rightarrow$ n$'$\\
\hline
p$'$ & p $\rightarrow$ p$'$ &  & p$'$ $\rightarrow$ p$'$\\
\hline
c & ęt $\rightarrow$ ąc &  & ęc $\rightarrow$ ąc\\
\hline
č &  &  & č $\rightarrow$ č\\
\hline
ž & ž $\rightarrow$ ž &  & ž $\rightarrow$ ž\\
\hline
\end{longtable}

\begin{longtable}{r|rrr}
 & \boldmath$\kappa$ & \boldmath$\lambda$ & \boldmath$\xi$\\
\hline
b & b $\rightarrow$ b &  & \\
\hline
x & x $\rightarrow$ x &  & \\
\hline
d & d $\rightarrow$ d & d$'$ $\rightarrow$ d & \\
 & ed $\rightarrow$ ad &  & \\
\hline
h & h $\rightarrow$ h &  & \\
\hline
ł & l $\rightarrow$ ł & l $\rightarrow$ ł & \\
 & $\star$el $\rightarrow$ oł & s$'$l $\rightarrow$ sł & \\
 & el $\rightarrow$ ał & z$'$l $\rightarrow$ zł & \\
\hline
m & m $\rightarrow$ m & m$'$ $\rightarrow$ m & \\
\hline
n & n$'$ $\rightarrow$ n & n$'$ $\rightarrow$ n & $'$en$'$ $\rightarrow$ on\\
 & en$'$ $\rightarrow$ an & s$'$n$'$ $\rightarrow$ sn & \\
 & en$'$ $\rightarrow$ on & z$'$n$'$ $\rightarrow$ zn & \\
 & $'$en$'$ $\rightarrow$ on &  & \\
\hline
p & p $\rightarrow$ p &  & \\
\hline
r & r $\rightarrow$ r & ř $\rightarrow$ r & \\
 & $\star$ędr $\rightarrow$ ądr &  & \\
\hline
t & t $\rightarrow$ t & t$'$ $\rightarrow$ t & \\
 &  & s$'$t$'$ $\rightarrow$ st & \\
\hline
v & v $\rightarrow$ v & v$'$ $\rightarrow$ v & \\
\hline
g & ž $\rightarrow$ g &  & ž $\rightarrow$ g\\
\hline
k & b $\rightarrow$ bk & t$'$ $\rightarrow$ k & c $\rightarrow$ kk\\
 & b $\rightarrow$ bok & $\star$ž $\rightarrow$ ekk & c $\rightarrow$ tk\\
 & d $\rightarrow$ dk &  & t$'$ $\rightarrow$ k\\
 & k $\rightarrow$ k &  & t$'$ $\rightarrow$ tk\\
 & k $\rightarrow$ kk &  & č $\rightarrow$ k\\
 & el $\rightarrow$ ałk &  & \textipa{Z} $\rightarrow$ dk\\
 & $\star$l $\rightarrow$ lek &  & d$'$ $\rightarrow$ dk\\
 & n$'$ $\rightarrow$ nk &  & ž $\rightarrow$ žk\\
 & p $\rightarrow$ pk &  & n$'$ $\rightarrow$ nk\\
 & r $\rightarrow$ rok &  & b$'$ $\rightarrow$ bok\\
 & ž $\rightarrow$ sk &  & l $\rightarrow$ lek\\
 & ž $\rightarrow$ sok &  & ř $\rightarrow$ rok\\
 & ęž $\rightarrow$ ąsk &  & ęz$'$ $\rightarrow$ ąsk\\
 & t $\rightarrow$ tk &  & ž $\rightarrow$ ekk\\
 & t $\rightarrow$ ck &  & ž $\rightarrow$ sk\\
 & ž $\rightarrow$ žk &  & ž $\rightarrow$ sok\\
\hline
\end{longtable}

\begin{longtable}{r|rrrrrr}
 & {\bf š} & {\bf ši} & {\bf č} & \boldmath$\iota$ & \boldmath$\nu$ & {\bf a}\\
\hline
c &  &  & č $\rightarrow$ c &  &  & \\
\hline
š & š $\rightarrow$ š & s$'$i $\rightarrow$ š &  &  &  & \\
\hline
a &  &  &  & a $\rightarrow$ a & a $\rightarrow$ a & a $\rightarrow$ a\\
\hline
e &  &  &  &  &  & $\star$e $\rightarrow$ e\\
\hline
i &  &  &  & i $\rightarrow$ i &  & \\
\hline
o &  &  &  &  & o $\rightarrow$ o & \\
 &  &  &  &  & e $\rightarrow$ o & \\
 &  &  &  &  & ce $\rightarrow$ t$'$o & \\
\hline
y &  &  &  & y $\rightarrow$ y &  & \\
\hline
\end{longtable}

\begin{longtable}{r|rrrrrrr}
 & \boldmath$\beta${\bf ł} & \boldmath$\beta${\bf li} & \boldmath$\beta${\bf t}$'$ & \boldmath$\gamma\varepsilon$ & \boldmath$\gamma$ & \boldmath$\eta$ & \boldmath$\zeta$\\
\hline
 &  & li $\rightarrow$  &  &  &  &  & \\
\hline
l &  &  &  & l $\rightarrow$ l & l $\rightarrow$ l & l $\rightarrow$ l & \\
\hline
c &  &  &  &  &  & c $\rightarrow$ c & \\
\hline
č &  &  &  &  &  & č $\rightarrow$ č & \\
\hline
ř &  &  &  &  &  & ř $\rightarrow$ ř & \\
\hline
ž & žł $\rightarrow$ ž &  &  &  &  & ž $\rightarrow$ ž & \\
\hline
j &  &  &  & aj $\rightarrow$ aj & aj $\rightarrow$ aj & j $\rightarrow$ j & \\
\hline
a &  & $\star$eli $\rightarrow$ a &  &  &  &  & \\
\hline
b & bł $\rightarrow$ b &  & s$'$t$'$ $\rightarrow$ b & b $\rightarrow$ b & b$'$ $\rightarrow$ b & b $\rightarrow$ b & \\
 & ąbł $\rightarrow$ ęb &  &  &  &  &  & \\
\hline
x & xł $\rightarrow$ x &  &  &  & $\star$š $\rightarrow$ x & x $\rightarrow$ x & \\
 & $\star$exł $\rightarrow$ x &  &  &  &  &  & \\
\hline
d & dł $\rightarrow$ d & $\star$edli $\rightarrow$ ad & s$'$t$'$ $\rightarrow$ d & ž\textipa{\v{Z}} $\rightarrow$ zd & ž\textipa{\v{Z}} $\rightarrow$ zd & d $\rightarrow$ d & \textipa{Z} $\rightarrow$ d$'$\\
 & adł $\rightarrow$ ed &  & es$'$t$'$ $\rightarrow$ ad &  &  &  & ž\textipa{\v{Z}} $\rightarrow$ z$'$d$'$\\
 & ódł $\rightarrow$ od &  & es$'$t$'$ $\rightarrow$ od &  &  &  & \\
 & ądł $\rightarrow$ ęd &  & ós$'$t$'$ $\rightarrow$ od &  &  &  & \\
 &  &  & ąs$'$t$'$ $\rightarrow$ ęd &  &  &  & \\
\hline
f &  &  &  &  &  & f $\rightarrow$ f & \\
\hline
h & hł $\rightarrow$ h &  &  &  &  & h $\rightarrow$ h & \\
\hline
ł & łł $\rightarrow$ ł &  &  &  & $\star$s$'$l $\rightarrow$ sł &  & \\
\hline
m &  &  &  & m $\rightarrow$ m & m$'$ $\rightarrow$ m & m $\rightarrow$ m & \\
\hline
n &  &  &  &  &  &  & n $\rightarrow$ n$'$\\
\hline
p & pł $\rightarrow$ p &  &  & p $\rightarrow$ p & p$'$ $\rightarrow$ p & p $\rightarrow$ p & \\
\hline
r & rł $\rightarrow$ r &  &  & ř $\rightarrow$ r & ř $\rightarrow$ r & r $\rightarrow$ r & \\
 &  &  &  & óř $\rightarrow$ or &  &  & \\
\hline
s & sł $\rightarrow$ s & $\star$s$'$li $\rightarrow$ s & s$'$t$'$ $\rightarrow$ s & š $\rightarrow$ s & š $\rightarrow$ s & s$'$ $\rightarrow$ s & š $\rightarrow$ s$'$\\
 & ósł $\rightarrow$ os & $\star$es$'$li $\rightarrow$ os & es$'$t$'$ $\rightarrow$ os &  &  & s $\rightarrow$ s & \\
 & ąsł $\rightarrow$ ęs &  & ąs$'$t$'$ $\rightarrow$ ęs &  &  &  & \\
 &  &  & ós$'$t$'$ $\rightarrow$ os &  &  &  & \\
\hline
t & tł $\rightarrow$ t & $\star$etli $\rightarrow$ ot & s$'$t$'$ $\rightarrow$ t & č $\rightarrow$ t & c $\rightarrow$ t & t $\rightarrow$ t & č $\rightarrow$ t$'$\\
 & ótł $\rightarrow$ ot &  & es$'$t$'$ $\rightarrow$ ot & šč $\rightarrow$ st & č $\rightarrow$ t &  & c $\rightarrow$ t$'$\\
 &  &  &  &  & šč $\rightarrow$ st &  & šč $\rightarrow$ s$'$t$'$\\
\hline
v &  &  &  &  & $\star$v$'$ $\rightarrow$ v & v $\rightarrow$ v & \\
 &  &  &  &  & $\star$v $\rightarrow$ v &  & \\
\hline
z & zł $\rightarrow$ z & $\star$z$'$li $\rightarrow$ z & z$'$t$'$ $\rightarrow$ z & ž $\rightarrow$ z & ž $\rightarrow$ z & z$'$ $\rightarrow$ z & ž $\rightarrow$ z$'$\\
 & ózł $\rightarrow$ oz & $\star$ez$'$li $\rightarrow$ az & ez$'$t$'$ $\rightarrow$ az &  &  & z $\rightarrow$ z & \\
 & ązł $\rightarrow$ ęz & $\star$ez$'$li $\rightarrow$ oz & ez$'$t$'$ $\rightarrow$ oz &  &  &  & \\
 &  &  & ąz$'$t$'$ $\rightarrow$ ęz &  &  &  & \\
\hline
g & gł $\rightarrow$ g &  & c $\rightarrow$ g & ž $\rightarrow$ g & ž $\rightarrow$ g & g $\rightarrow$ g & \\
 & ógł $\rightarrow$ og &  & óc $\rightarrow$ og & ž\textipa{\v{Z}} $\rightarrow$ zg & ž\textipa{\v{Z}} $\rightarrow$ zg &  & \\
 & ągł $\rightarrow$ ęg &  & ąc $\rightarrow$ ęg &  &  &  & \\
 & ązgł $\rightarrow$ ęzg &  &  &  &  &  & \\
\hline
k & kł $\rightarrow$ k &  & c $\rightarrow$ k & č $\rightarrow$ k & č $\rightarrow$ k & k $\rightarrow$ k & \\
 & ókł $\rightarrow$ ok &  & ąc $\rightarrow$ ęk & šč $\rightarrow$ sk & šč $\rightarrow$ sk &  & \\
 & ąkł $\rightarrow$ ęk &  &  &  &  &  & \\
\hline
\end{longtable}

\subsection{Reguły warstwy analitycznej}
Dla oszczędności miejsca nie podajemy nazw atrybutów, gdzie wartości atrybutów jednoznacznie na nie wskazują.
W ten sposób w domyśle pozostają nazwy atrybutów:
\begin{itemize}
\item cat mającego wartości ndm, adj, adj:grad, adv, noun, verb
\item palat mającego wartości $\uparrow$ (dla funkcjonalnie miękkich) i $\downarrow$ (dla funkcjonalnie twardych)
\item velar mającego wartości $\leftarrow$ (dla b,d,f,ł,m,n,p,r,s,t,v,z,o,u) i $\rightarrow$ (dla x,h,g,k)
\end{itemize}

Oznaczenie $\star$ odnosi się do odmian, które uznaliśmy za nieproduktywne (niewystępujące poza zamkniętą listą lematów). Oznaczenie $D$ odnosi się do odmian gwarowych i w inny sposób niestandardowych, które zostały dodane z powodu ich obecności w podkorpusie milionowym Narodowego Korpusu Języka Polskiego.

\begin{scriptsize}\[
\left[\begin{array}{ll}
-\varepsilon & \text{ndm}
\end{array}\right]
\]\end{scriptsize}

\begin{scriptsize}\[
\left[\begin{array}{ll}
\star-\text{žkolv$'$ek} & \text{suf}:=\text{žkolv$'$ek}\\
\star-\text{žekolv$'$ek} & \text{suf}:=\text{žkolv$'$ek}\\
\star-\text{s$'$kolv$'$ek} & \text{suf}:=\text{s$'$kolv$'$ek}\\
\star-\text{kolv$'$ek} & \text{suf}:=\text{kolv$'$ek}\\
\star-\text{ž} & \text{suf}:=\text{ž}\\
\star-\text{že} & \text{suf}:=\text{ž}\\
\star-\text{ž} & \text{suf}:=\text{že}\\
\star-\text{že} & \text{suf}:=\text{že}\\
\star-\text{s$'$t$'$is$'$} & \text{suf}:=\text{s$'$t$'$is$'$}\\
\star-\text{t$'$is$'$} & \text{suf}:=\text{t$'$is$'$}\\
\star-\text{s$'$} & \text{suf}:=\text{s$'$}\\
\star-\text{s$'$t$'$i} & \text{suf}:=\text{s$'$t$'$i}\\
\star-\text{s$'$ik} & \text{suf}:=\text{sik}\\
\star-\text{s$'$i} & \text{suf}:=\text{si}\\
-\varepsilon & 
\end{array}\right] \otimes \left[\begin{array}{ll}
-\text{\boldmath$\alpha'${\bf y}} & \text{flex}:=\text{y}, \uparrow, \text{adj}\\
-\text{\boldmath$\alpha'${\bf y}x} & \text{flex}:=\text{ych}, \uparrow, \text{adj}\\
-\text{\boldmath$\alpha'${\bf y}m} & \text{flex}:=\text{ym}, \uparrow, \text{adj}\\
-\text{\boldmath$\alpha'${\bf y}m$'$i} & \text{flex}:=\text{ymi}, \uparrow, \text{adj}\\
-\text{\boldmath$\alpha'$e} & \text{flex}:=\text{e}, \uparrow, \text{adj}\\
-\text{\boldmath$\alpha'$ego} & \text{flex}:=\text{ego}, \uparrow, \text{adj}\\
-\text{\boldmath$\alpha'$ej} & \text{flex}:=\text{ej}, \uparrow, \text{adj}\\
-\text{\boldmath$\alpha'$emu} & \text{flex}:=\text{emu}, \uparrow, \text{adj}\\
-\text{\boldmath$\alpha'$a} & \text{flex}:=\text{a}, \uparrow, \text{adj}\\
-\text{\boldmath$\alpha'$ą} & \text{flex}:=\text{ą}, \uparrow, \text{adj}\\
-\text{\boldmath$\alpha'$o} & \text{flex}:=\text{o}, \uparrow, \text{adj}\\
\star-\text{\boldmath$\alpha'\varepsilon$} & \text{flex}:=\text{$\varepsilon$}, \uparrow, \text{adj}\\
-\text{{\bf ši}} & \text{flex}:=\text{i}, \text{palat}:=\text{sz}, \text{adj}\\
D-\text{\boldmath$\alpha'$em} & \text{flex}:=\text{ym}, \uparrow, \text{adj}\\
D-\text{\boldmath$\alpha'$emi} & \text{flex}:=\text{ymi}, \uparrow, \text{adj}\\
D-\text{\boldmath$\alpha'$o} & \text{flex}:=\text{ą}, \uparrow, \text{adj}\\
D-\text{\boldmath$\alpha'$om} & \text{flex}:=\text{ą}, \uparrow, \text{adj}
\end{array}\right] \otimes \left[\begin{array}{ll}
+\text{\boldmath$\alpha'${\bf y}} & \text{lemma}:=\text{y}\\
\star+\text{\boldmath$\alpha'\varepsilon$} & \text{lemma}:=\text{$\varepsilon$}
\end{array}\right]
\]\end{scriptsize}

\begin{scriptsize}\[
\left[\begin{array}{ll}
\star-\text{žkolv$'$ek} & \text{suf}:=\text{žkolv$'$ek}\\
\star-\text{žekolv$'$ek} & \text{suf}:=\text{žkolv$'$ek}\\
\star-\text{s$'$kolv$'$ek} & \text{suf}:=\text{s$'$kolv$'$ek}\\
\star-\text{kolv$'$ek} & \text{suf}:=\text{kolv$'$ek}\\
\star-\text{ž} & \text{suf}:=\text{ž}\\
\star-\text{že} & \text{suf}:=\text{ž}\\
\star-\text{ž} & \text{suf}:=\text{že}\\
\star-\text{že} & \text{suf}:=\text{že}\\
\star-\text{s$'$t$'$is$'$} & \text{suf}:=\text{s$'$t$'$is$'$}\\
\star-\text{t$'$is$'$} & \text{suf}:=\text{t$'$is$'$}\\
\star-\text{s$'$} & \text{suf}:=\text{s$'$}\\
\star-\text{s$'$t$'$i} & \text{suf}:=\text{s$'$t$'$i}\\
\star-\text{s$'$ik} & \text{suf}:=\text{sik}\\
\star-\text{s$'$i} & \text{suf}:=\text{si}\\
-\varepsilon & 
\end{array}\right] \otimes \left[\begin{array}{ll}
-\text{\boldmath$\alpha${\bf y}} & \text{flex}:=\text{y}, \downarrow, \text{adj}\\
-\text{\boldmath$\alpha${\bf y}x} & \text{flex}:=\text{ych}, \downarrow, \text{adj}\\
-\text{\boldmath$\alpha${\bf y}m} & \text{flex}:=\text{ym}, \downarrow, \text{adj}\\
-\text{\boldmath$\alpha${\bf y}m$'$i} & \text{flex}:=\text{ymi}, \downarrow, \text{adj}\\
-\text{\boldmath$\alpha${\bf e}} & \text{flex}:=\text{e}, \downarrow, \text{adj}\\
-\text{\boldmath$\alpha${\bf e}go} & \text{flex}:=\text{ego}, \downarrow, \text{adj}\\
-\text{\boldmath$\alpha${\bf e}j} & \text{flex}:=\text{ej}, \downarrow, \text{adj}\\
-\text{\boldmath$\alpha${\bf e}mu} & \text{flex}:=\text{emu}, \downarrow, \text{adj}\\
-\text{\boldmath$\alpha$a} & \text{flex}:=\text{a}, \downarrow, \text{adj}\\
-\text{\boldmath$\alpha$ą} & \text{flex}:=\text{ą}, \downarrow, \text{adj}\\
-\text{\boldmath$\alpha$o} & \text{flex}:=\text{o}, \downarrow, \text{adj}\\
-\text{\boldmath$\alpha$u} & \text{flex}:=\text{u}, \downarrow, \text{adj}\\
-\text{\boldmath$\alpha${\bf i}} & \text{flex}:=\text{i}, \downarrow, \text{adj}\\
\star-\text{\boldmath$\alpha\varepsilon$} & \text{flex}:=\text{$\varepsilon$}, \downarrow, \text{adj}\\
D-\text{\boldmath$\alpha${\bf e}m} & \text{flex}:=\text{ym}, \downarrow, \text{adj}\\
D-\text{\boldmath$\alpha${\bf e}mi} & \text{flex}:=\text{ymi}, \downarrow, \text{adj}\\
D-\text{\boldmath$\alpha$o} & \text{flex}:=\text{ą}, \downarrow, \text{adj}\\
D-\text{\boldmath$\alpha$om} & \text{flex}:=\text{ą}, \downarrow, \text{adj}
\end{array}\right] \otimes \left[\begin{array}{ll}
+\text{\boldmath$\alpha${\bf y}} & \text{lemma}:=\text{y}\\
\star+\text{\boldmath$\alpha\varepsilon$} & \text{lemma}:=\text{$\varepsilon$}
\end{array}\right]
\]\end{scriptsize}

\begin{scriptsize}\[
\left[\begin{array}{ll}
\varepsilon- & \text{pref}:=\text{$\varepsilon$}\\
\text{naj}- & \text{pref}:=\text{naj}
\end{array}\right] \otimes \left[\begin{array}{ll}
-\text{{\bf š}y} & \text{flex}:=\text{y}, \text{adj:grad}\\
-\text{{\bf š}yx} & \text{flex}:=\text{ych}, \text{adj:grad}\\
-\text{{\bf š}ym} & \text{flex}:=\text{ym}, \text{adj:grad}\\
-\text{{\bf š}ym$'$i} & \text{flex}:=\text{ymi}, \text{adj:grad}\\
-\text{{\bf š}e} & \text{flex}:=\text{e}, \text{adj:grad}\\
-\text{{\bf š}ego} & \text{flex}:=\text{ego}, \text{adj:grad}\\
-\text{{\bf š}ej} & \text{flex}:=\text{ej}, \text{adj:grad}\\
-\text{{\bf š}emu} & \text{flex}:=\text{emu}, \text{adj:grad}\\
-\text{{\bf š}a} & \text{flex}:=\text{a}, \text{adj:grad}\\
-\text{{\bf š}ą} & \text{flex}:=\text{ą}, \text{adj:grad}\\
-\text{{\bf ši}} & \text{flex}:=\text{i}, \text{adj:grad}\\
D-\text{{\bf š}em} & \text{flex}:=\text{ym}, \text{adj:grad}\\
D-\text{{\bf š}emi} & \text{flex}:=\text{ymi}, \text{adj:grad}\\
D-\text{{\bf š}o} & \text{flex}:=\text{ą}, \text{adj:grad}\\
D-\text{{\bf š}om} & \text{flex}:=\text{ą}, \text{adj:grad}
\end{array}\right] \otimes \left[\begin{array}{ll}
-\text{\boldmath$\kappa'$š} & \text{grad}:=\text{sz}\\
-\text{\boldmath$\lambda'$ejš} & \text{grad}:=\text{iejsz}
\end{array}\right] \otimes \left[\begin{array}{ll}
+\text{\boldmath$\alpha'${\bf y}} & \text{lemma}:=\text{y}\\
\star+\text{\boldmath$\alpha'\varepsilon$} & \text{lemma}:=\text{$\varepsilon$}
\end{array}\right]
\]\end{scriptsize}

\begin{scriptsize}\[
\left[\begin{array}{ll}
\varepsilon- & \text{pref}:=\text{$\varepsilon$}\\
\text{naj}- & \text{pref}:=\text{naj}
\end{array}\right] \otimes \left[\begin{array}{ll}
-\text{{\bf š}y} & \text{flex}:=\text{y}, \text{adj:grad}\\
-\text{{\bf š}yx} & \text{flex}:=\text{ych}, \text{adj:grad}\\
-\text{{\bf š}ym} & \text{flex}:=\text{ym}, \text{adj:grad}\\
-\text{{\bf š}ym$'$i} & \text{flex}:=\text{ymi}, \text{adj:grad}\\
-\text{{\bf š}e} & \text{flex}:=\text{e}, \text{adj:grad}\\
-\text{{\bf š}ego} & \text{flex}:=\text{ego}, \text{adj:grad}\\
-\text{{\bf š}ej} & \text{flex}:=\text{ej}, \text{adj:grad}\\
-\text{{\bf š}emu} & \text{flex}:=\text{emu}, \text{adj:grad}\\
-\text{{\bf š}a} & \text{flex}:=\text{a}, \text{adj:grad}\\
-\text{{\bf š}ą} & \text{flex}:=\text{ą}, \text{adj:grad}\\
-\text{{\bf ši}} & \text{flex}:=\text{i}, \text{adj:grad}\\
D-\text{{\bf š}em} & \text{flex}:=\text{ym}, \text{adj:grad}\\
D-\text{{\bf š}emi} & \text{flex}:=\text{ymi}, \text{adj:grad}\\
D-\text{{\bf š}o} & \text{flex}:=\text{ą}, \text{adj:grad}\\
D-\text{{\bf š}om} & \text{flex}:=\text{ą}, \text{adj:grad}
\end{array}\right] \otimes \left[\begin{array}{ll}
-\text{\boldmath$\kappa$š} & \text{grad}:=\text{sz}\\
-\text{\boldmath$\lambda$ejš} & \text{grad}:=\text{iejsz}
\end{array}\right] \otimes \left[\begin{array}{ll}
+\text{\boldmath$\alpha${\bf y}} & \text{lemma}:=\text{y}\\
\star+\text{\boldmath$\alpha\varepsilon$} & \text{lemma}:=\text{$\varepsilon$}
\end{array}\right]
\]\end{scriptsize}

\begin{scriptsize}\[
\left[\begin{array}{ll}
\varepsilon- & \text{pref}:=\text{$\varepsilon$}\\
\text{naj}- & \text{pref}:=\text{naj}
\end{array}\right] \otimes \left[\begin{array}{ll}
-\text{\boldmath$\alpha'$o} & \text{flex}:=\text{o}, \text{adv}\\
-\text{\boldmath$\xi'$ej} & \text{flex}:=\text{iej}, \text{adv}
\end{array}\right] \otimes \left[\begin{array}{ll}
+\text{\boldmath$\alpha'$o} & \text{lemma}:=\text{o}
\end{array}\right]
\]\end{scriptsize}

\begin{scriptsize}\[
\left[\begin{array}{ll}
\varepsilon- & \text{pref}:=\text{$\varepsilon$}\\
\text{naj}- & \text{pref}:=\text{naj}
\end{array}\right] \otimes \left[\begin{array}{ll}
-\text{\boldmath$\alpha$o} & \text{flex}:=\text{o}, \text{adv}\\
-\text{\boldmath$\alpha${\bf ie}$_1$} & \text{flex}:=\text{ie}, \text{adv}\\
-\text{\boldmath$\alpha${\bf ie}$_2$} & \text{flex}:=\text{ie}, \text{adv}\\
-\text{\boldmath$\xi$ej} & \text{flex}:=\text{iej}, \text{adv}\\
-\text{\boldmath$\alpha${\bf ie}$_1$j} & \text{flex}:=\text{iej}, \text{adv}\\
-\text{\boldmath$\alpha${\bf ie}$_2$j} & \text{flex}:=\text{iej}, \text{adv}
\end{array}\right] \otimes \left[\begin{array}{ll}
+\text{\boldmath$\alpha$o} & \text{lemma}:=\text{o}\\
+\text{\boldmath$\alpha${\bf ie}$_1$} & \text{lemma}:=\text{ie}\\
+\text{\boldmath$\alpha${\bf ie}$_2$} & \text{lemma}:=\text{ie}
\end{array}\right]
\]\end{scriptsize}

\begin{scriptsize}\[
\left[\begin{array}{ll}
-\text{\boldmath$\alpha'${\bf y}} & \text{flex}:=\text{y}, \uparrow, \text{noun}\\
-\text{\boldmath$\alpha'${\bf y}x} & \text{flex}:=\text{ych}, \uparrow, \text{noun}\\
-\text{\boldmath$\alpha'${\bf y}m} & \text{flex}:=\text{ym}, \uparrow, \text{noun}\\
-\text{\boldmath$\alpha'${\bf y}m$'$i} & \text{flex}:=\text{ymi}, \uparrow, \text{noun}\\
-\text{\boldmath$\alpha'$e} & \text{flex}:=\text{e}, \uparrow, \text{noun}\\
-\text{\boldmath$\alpha'$ego} & \text{flex}:=\text{ego}, \uparrow, \text{noun}\\
-\text{\boldmath$\alpha'$ej} & \text{flex}:=\text{ej}, \uparrow, \text{noun}\\
-\text{\boldmath$\alpha'$em} & \text{flex}:=\text{em}, \uparrow, \text{noun}\\
-\text{\boldmath$\alpha'$emu} & \text{flex}:=\text{emu}, \uparrow, \text{noun}\\
-\text{\boldmath$\alpha'$a} & \text{flex}:=\text{a}, \uparrow, \text{noun}\\
-\text{\boldmath$\alpha'$ax} & \text{flex}:=\text{ach}, \uparrow, \text{noun}\\
-\text{\boldmath$\alpha'$am$'$i} & \text{flex}:=\text{ami}, \uparrow, \text{noun}\\
-\text{\boldmath$\alpha'$ą} & \text{flex}:=\text{ą}, \uparrow, \text{noun}\\
-\text{\boldmath$\alpha'$ę} & \text{flex}:=\text{ę}, \uparrow, \text{noun}\\
-\text{\boldmath$\alpha'$o} & \text{flex}:=\text{o}, \uparrow, \text{noun}\\
-\text{\boldmath$\alpha'$om} & \text{flex}:=\text{om}, \uparrow, \text{noun}\\
-\text{\boldmath$\alpha'$ov$'$i} & \text{flex}:=\text{owi}, \uparrow, \text{noun}\\
-\text{\boldmath$\alpha'$ov$'$e} & \text{flex}:=\text{owie}, \uparrow, \text{noun}\\
-\text{\boldmath$\alpha'$óv} & \text{flex}:=\text{ów}, \uparrow, \text{noun}\\
-\text{\boldmath$\alpha'$u} & \text{flex}:=\text{u}, \uparrow, \text{noun}\\
-\text{\boldmath$\alpha'$um} & \text{flex}:=\text{um}, \uparrow, \text{noun}\\
-\text{\boldmath$\alpha'\varepsilon$} & \text{flex}:=\text{$\varepsilon$}, \uparrow, \text{noun}\\
\star-\text{\boldmath$\alpha'\varepsilon$m$'$i} & \text{flex}:=\text{ami}, \uparrow, \text{noun}\\
-\text{{\bf č}e} & \text{flex}:=\text{cze}, \uparrow, \text{noun}\\
D-\text{\boldmath$\alpha'$em} & \text{flex}:=\text{ym}, \uparrow, \text{noun}\\
D-\text{\boldmath$\alpha'$em$'$i} & \text{flex}:=\text{ymi}, \uparrow, \text{noun}\\
D-\text{\boldmath$\alpha'${\bf y}ma} & \text{flex}:=\text{ami}, \uparrow, \text{noun}\\
D-\text{\boldmath$\alpha'$om} & \text{flex}:=\text{ą}, \text{noun}\\
D-\text{\boldmath$\alpha'$o} & \text{flex}:=\text{ą}, \text{noun}\\
D-\text{\boldmath$\alpha'$e} & \text{flex}:=\text{ę}, \text{noun}\\
D-\text{\boldmath$\alpha'$om$'$i} & \text{flex}:=\text{ami}, \uparrow, \text{noun}\\
D-\text{\boldmath$\alpha'$amy} & \text{flex}:=\text{ami}, \uparrow, \text{noun}\\
D-\text{\boldmath$\alpha'$ox} & \text{flex}:=\text{ach}, \uparrow, \text{noun}
\end{array}\right] \otimes \left[\begin{array}{ll}
+\text{\boldmath$\alpha'${\bf y}} & \text{lemma}:=\text{y}\\
+\text{\boldmath$\alpha'$e} & \text{lemma}:=\text{e}\\
+\text{\boldmath$\alpha'$a} & \text{lemma}:=\text{a}\\
+\text{\boldmath$\alpha'$o} & \text{lemma}:=\text{o}\\
+\text{\boldmath$\alpha'$ov$'$e} & \text{lemma}:=\text{owie}\\
+\text{\boldmath$\alpha'$um} & \text{lemma}:=\text{um}\\
+\text{\boldmath$\alpha'\varepsilon$} & \text{lemma}:=\text{$\varepsilon$}
\end{array}\right]
\]\end{scriptsize}

\begin{scriptsize}\[
\left[\begin{array}{ll}
-\text{\boldmath$\alpha${\bf y}} & \text{flex}:=\text{y}, \downarrow, \text{noun}\\
-\text{\boldmath$\alpha${\bf y}x} & \text{flex}:=\text{ych}, \downarrow, \text{noun}\\
-\text{\boldmath$\alpha${\bf y}m} & \text{flex}:=\text{ym}, \downarrow, \text{noun}\\
-\text{\boldmath$\alpha${\bf y}m$'$i} & \text{flex}:=\text{ymi}, \downarrow, \text{noun}\\
-\text{\boldmath$\alpha${\bf e}} & \text{flex}:=\text{e}, \downarrow, \text{noun}\\
-\text{\boldmath$\alpha${\bf e}go} & \text{flex}:=\text{ego}, \downarrow, \text{noun}\\
-\text{\boldmath$\alpha${\bf e}j} & \text{flex}:=\text{ej}, \downarrow, \text{noun}\\
-\text{\boldmath$\alpha${\bf e}m} & \text{flex}:=\text{em}, \downarrow, \text{noun}\\
-\text{\boldmath$\alpha${\bf e}mu} & \text{flex}:=\text{emu}, \downarrow, \text{noun}\\
-\text{\boldmath$\alpha$a} & \text{flex}:=\text{a}, \downarrow, \text{noun}\\
-\text{\boldmath$\alpha$ax} & \text{flex}:=\text{ach}, \downarrow, \text{noun}\\
-\text{\boldmath$\alpha$am$'$i} & \text{flex}:=\text{ami}, \downarrow, \text{noun}\\
-\text{\boldmath$\alpha$ą} & \text{flex}:=\text{ą}, \downarrow, \text{noun}\\
-\text{\boldmath$\alpha$ę} & \text{flex}:=\text{ę}, \downarrow, \text{noun}\\
-\text{\boldmath$\alpha$o} & \text{flex}:=\text{o}, \downarrow, \text{noun}\\
-\text{\boldmath$\alpha$om} & \text{flex}:=\text{om}, \downarrow, \text{noun}\\
-\text{\boldmath$\alpha$ov$'$i} & \text{flex}:=\text{owi}, \downarrow, \text{noun}\\
-\text{\boldmath$\alpha$ov$'$e} & \text{flex}:=\text{owie}, \downarrow, \text{noun}\\
-\text{\boldmath$\alpha$óv} & \text{flex}:=\text{ów}, \downarrow, \text{noun}\\
-\text{\boldmath$\alpha_1$u} & \text{flex}:=\text{u}, \downarrow, \leftarrow, \text{noun}\\
-\text{\boldmath$\alpha_2$u} & \text{flex}:=\text{u}, \downarrow, \rightarrow, \text{noun}\\
-\text{\boldmath$\alpha$um} & \text{flex}:=\text{um}, \downarrow, \text{noun}\\
-\text{\boldmath$\alpha${\bf i}} & \text{flex}:=\text{i}, \downarrow, \text{noun}\\
-\text{\boldmath$\alpha${\bf ie}$_1$} & \text{flex}:=\text{ie}, \downarrow, \leftarrow, \text{noun}\\
-\text{\boldmath$\alpha${\bf ie}$_2$} & \text{flex}:=\text{ie}, \downarrow, \rightarrow, \text{noun}\\
-\text{\boldmath$\alpha\varepsilon$} & \text{flex}:=\text{$\varepsilon$}, \downarrow, \text{noun}\\
\star-\text{\boldmath$\alpha\varepsilon$m$'$i} & \text{flex}:=\text{ami}, \downarrow, \text{noun}\\
D-\text{\boldmath$\alpha${\bf e}m} & \text{flex}:=\text{ym}, \downarrow, \text{noun}\\
D-\text{\boldmath$\alpha${\bf e}m$'$i} & \text{flex}:=\text{ymi}, \downarrow, \text{noun}\\
D-\text{\boldmath$\alpha$om} & \text{flex}:=\text{ą}, \text{noun}\\
D-\text{\boldmath$\alpha$o} & \text{flex}:=\text{ą}, \text{noun}\\
D-\text{\boldmath$\alpha${\bf e}} & \text{flex}:=\text{ę}, \text{noun}\\
D-\text{\boldmath$\alpha$om$'$i} & \text{flex}:=\text{ami}, \downarrow, \text{noun}\\
D-\text{\boldmath$\alpha$amy} & \text{flex}:=\text{ami}, \downarrow, \text{noun}\\
D-\text{\boldmath$\alpha$ox} & \text{flex}:=\text{ach}, \downarrow, \text{noun}\\
D-\text{\boldmath$\alpha${\bf y}ma} & \text{flex}:=\text{ami}, \downarrow, \text{noun}
\end{array}\right] \otimes \left[\begin{array}{ll}
+\text{\boldmath$\alpha${\bf y}} & \text{lemma}:=\text{y}\\
+\text{\boldmath$\alpha${\bf e}} & \text{lemma}:=\text{e}\\
+\text{\boldmath$\alpha$a} & \text{lemma}:=\text{a}\\
+\text{\boldmath$\alpha$o} & \text{lemma}:=\text{o}\\
+\text{\boldmath$\alpha$ov$'$e} & \text{lemma}:=\text{owie}\\
+\text{\boldmath$\alpha$um} & \text{lemma}:=\text{um}\\
\star+\text{\boldmath$\alpha$us} & \text{lemma}:=\text{us}\\
+\text{\boldmath$\alpha${\bf i}} & \text{lemma}:=\text{i}\\
+\text{\boldmath$\alpha\varepsilon$} & \text{lemma}:=\text{$\varepsilon$}
\end{array}\right]
\]\end{scriptsize}

\begin{scriptsize}\[
\left[\begin{array}{ll}
-\text{\boldmath$\alpha'$ę} & \text{flex}:=\text{ę}, \text{noun}\\
-\text{\boldmath$\alpha'$ęt$'$a} & \text{flex}:=\text{ęcia}, \text{noun}\\
-\text{\boldmath$\alpha'$ęt$'$u} & \text{flex}:=\text{ęciu}, \text{noun}\\
-\text{\boldmath$\alpha'$ęt$'$em} & \text{flex}:=\text{ęciem}, \text{noun}\\
-\text{\boldmath$\alpha'$ęta} & \text{flex}:=\text{ęta}, \text{noun}\\
-\text{\boldmath$\alpha'$ąt} & \text{flex}:=\text{ąt}, \text{noun}\\
-\text{\boldmath$\alpha'$ętom} & \text{flex}:=\text{ętom}, \text{noun}\\
-\text{\boldmath$\alpha'$ętam$'$i} & \text{flex}:=\text{ętam$'$i}, \text{noun}\\
-\text{\boldmath$\alpha'$ętax} & \text{flex}:=\text{ętach}, \text{noun}
\end{array}\right] \otimes \left[\begin{array}{ll}
+\text{\boldmath$\alpha'$ę} & \text{lemma}:=\text{ę}
\end{array}\right]
\]\end{scriptsize}

\begin{scriptsize}\[
\left[\begin{array}{ll}
-\text{an$'$e} & \text{flex}:=\text{anie}, \text{noun}\\
-\text{an} & \text{flex}:=\text{an}, \text{noun}\\
-\text{anom} & \text{flex}:=\text{anom}, \text{noun}\\
-\text{anóv} & \text{flex}:=\text{anów}, \text{noun}\\
-\text{anam$'$i} & \text{flex}:=\text{anami}, \text{noun}\\
-\text{anax} & \text{flex}:=\text{anach}, \text{noun}\\
-\text{any} & \text{flex}:=\text{any}, \text{noun}
\end{array}\right] \otimes \left[\begin{array}{ll}
+\text{an$'$in} & \text{lemma}:=\text{anin}
\end{array}\right]
\]\end{scriptsize}

\begin{scriptsize}\[
\left[\begin{array}{ll}
-\text{m$'$ę} & \text{flex}:=\text{mię}, \text{noun}\\
-\text{m$'$en$'$a} & \text{flex}:=\text{mienia}, \text{noun}\\
-\text{m$'$en$'$u} & \text{flex}:=\text{mieniu}, \text{noun}\\
-\text{m$'$en$'$em} & \text{flex}:=\text{mieniem}, \text{noun}\\
-\text{m$'$ona} & \text{flex}:=\text{miona}, \text{noun}\\
-\text{m$'$on} & \text{flex}:=\text{mion}, \text{noun}\\
-\text{m$'$onom} & \text{flex}:=\text{mionom}, \text{noun}\\
-\text{m$'$onam$'$i} & \text{flex}:=\text{mionami}, \text{noun}\\
-\text{m$'$onax} & \text{flex}:=\text{mionach}, \text{noun}
\end{array}\right] \otimes \left[\begin{array}{ll}
+\text{m$'$ę} & \text{lemma}:=\text{mię}
\end{array}\right]
\]\end{scriptsize}

\begin{scriptsize}\[
\left[\begin{array}{ll}
-\text{o} & \text{flex}:=\text{o}, \text{noun}\\
-\text{ona} & \text{flex}:=\text{ona}, \text{noun}\\
-\text{onov$'$i} & \text{flex}:=\text{onowi}, \text{noun}\\
-\text{onem} & \text{flex}:=\text{onem}, \text{noun}\\
-\text{on$'$e} & \text{flex}:=\text{onie}, \text{noun}\\
-\text{onov$'$e} & \text{flex}:=\text{onowie}, \text{noun}\\
-\text{ony} & \text{flex}:=\text{ony}, \text{noun}\\
-\text{onóv} & \text{flex}:=\text{onów}, \text{noun}\\
-\text{onom} & \text{flex}:=\text{onom}, \text{noun}\\
-\text{onam$'$i} & \text{flex}:=\text{onami}, \text{noun}\\
-\text{onax} & \text{flex}:=\text{onach}, \text{noun}
\end{array}\right] \otimes \left[\begin{array}{ll}
+\text{o} & \text{lemma}:=\text{o(n)}
\end{array}\right]
\]\end{scriptsize}

\begin{scriptsize}\[
\left[\begin{array}{ll}
-\text{stvo} & \text{flex}:=\text{stwo}, \text{noun}\\
-\text{stva} & \text{flex}:=\text{stwa}, \text{noun}\\
-\text{stvu} & \text{flex}:=\text{stwu}, \text{noun}\\
-\text{stvo} & \text{flex}:=\text{stwo}, \text{noun}\\
-\text{stvem} & \text{flex}:=\text{stwem}, \text{noun}
\end{array}\right] \otimes \left[\begin{array}{ll}
+\text{stwo} & \text{lemma}:=\text{stwo}
\end{array}\right]
\]\end{scriptsize}

\begin{scriptsize}\[
\left[\begin{array}{ll}
-\text{\boldmath$\iota$n$'$i} & \text{flex}:=\text{ni}, \text{noun}\\
-\text{\boldmath$\iota$n$'$ą} & \text{flex}:=\text{nią}, \text{noun}\\
-\text{\boldmath$\iota$n$'$ę} & \text{flex}:=\text{nię}, \text{noun}\\
-\text{\boldmath$\iota$n$'$e} & \text{flex}:=\text{nie}, \text{noun}\\
-\text{\boldmath$\iota$n$'$} & \text{flex}:=\text{ń}, \text{noun}\\
-\text{\boldmath$\iota$n$'$ax} & \text{flex}:=\text{niach}, \text{noun}\\
-\text{\boldmath$\iota$n$'$om} & \text{flex}:=\text{niom}, \text{noun}\\
-\text{\boldmath$\iota$n$'$am$'$i} & \text{flex}:=\text{niami}, \text{noun}
\end{array}\right] \otimes \left[\begin{array}{ll}
+\text{n$'$i} & \text{lemma}:=\text{ni}
\end{array}\right]
\]\end{scriptsize}

\begin{scriptsize}\[\hspace{-2cm}
\left[\begin{array}{ll}
\varepsilon- & \text{pref}:=\text{$\varepsilon$}\\
\text{n$'$e}- & \text{pref}:=\text{nie}
\end{array}\right] \otimes \left[\begin{array}{ll}
-\varepsilon & \text{flex}:=\text{$\varepsilon$}, \text{flex2}:=\text{$\varepsilon$}\\
-\text{my} & \text{flex}:=\text{my}, \text{flex2}:=\text{$\varepsilon$}\\
-\text{t$'$e} & \text{flex}:=\text{cie}, \text{flex2}:=\text{$\varepsilon$}\\
-\text{š} & \text{flex}:=\text{sz}, \text{flex2}:=\text{$\varepsilon$}\\
D-\text{m} & \text{flex}:=\text{my}, \text{flex2}:=\text{$\varepsilon$}\\
D-\text{s} & \text{flex}:=\text{sz}, \text{flex2}:=\text{$\varepsilon$}\\
-\text{\boldmath$\beta${\bf ł}šy} & \text{flex}:=\text{szy}, \text{flex2}:=\text{ł}\\
-\text{\boldmath$\beta${\bf ł}} & \text{flex}:=\text{$\varepsilon$}, \text{flex2}:=\text{ł}\\
-\text{ł} & \text{flex}:=\text{$\varepsilon$}, \text{flex2}:=\text{ł}\\
-\text{ła} & \text{flex}:=\text{a}, \text{flex2}:=\text{ł}\\
-\text{ło} & \text{flex}:=\text{o}, \text{flex2}:=\text{ł}\\
-\text{ły} & \text{flex}:=\text{y}, \text{flex2}:=\text{ł}\\
-\text{\boldmath$\beta${\bf li}} & \text{flex}:=\text{i}, \text{flex2}:=\text{ł}\\
-\text{tyx} & \text{flex}:=\text{ych}, \text{flex2}:=\text{t}\\
-\text{tym} & \text{flex}:=\text{ym}, \text{flex2}:=\text{t}\\
-\text{tym$'$i} & \text{flex}:=\text{ymi}, \text{flex2}:=\text{t}\\
-\text{te} & \text{flex}:=\text{e}, \text{flex2}:=\text{t}\\
-\text{ty} & \text{flex}:=\text{y}, \text{flex2}:=\text{t}\\
-\text{tą} & \text{flex}:=\text{ą}, \text{flex2}:=\text{t}\\
-\text{ta} & \text{flex}:=\text{a}, \text{flex2}:=\text{t}\\
-\text{to} & \text{flex}:=\text{o}, \text{flex2}:=\text{t}\\
-\text{tego} & \text{flex}:=\text{ego}, \text{flex2}:=\text{t}\\
-\text{temu} & \text{flex}:=\text{emu}, \text{flex2}:=\text{t}\\
-\text{tej} & \text{flex}:=\text{ej}, \text{flex2}:=\text{t}\\
-\text{t$'$i} & \text{flex}:=\text{i}, \text{flex2}:=\text{t}\\
D-\text{to} & \text{flex}:=\text{ą}, \text{flex2}:=\text{t}\\
D-\text{tom} & \text{flex}:=\text{ą}, \text{flex2}:=\text{t}\\
D-\text{tem} & \text{flex}:=\text{ym}, \text{flex2}:=\text{t}\\
D-\text{tem$'$i} & \text{flex}:=\text{ymi}, \text{flex2}:=\text{t}\\
-\text{t$'$om} & \text{flex}:=\text{om}, \text{flex2}:=\text{ć}\\
-\text{t$'$am$'$i} & \text{flex}:=\text{ami}, \text{flex2}:=\text{ć}\\
-\text{t$'$ax} & \text{flex}:=\text{ach}, \text{flex2}:=\text{ć}\\
-\text{t$'$e} & \text{flex}:=\text{e}, \text{flex2}:=\text{ć}\\
-\text{t$'$a} & \text{flex}:=\text{a}, \text{flex2}:=\text{ć}\\
-\text{t$'$u} & \text{flex}:=\text{u}, \text{flex2}:=\text{ć}\\
-\text{t$'$em} & \text{flex}:=\text{em}, \text{flex2}:=\text{ć}\\
-\text{t$'$} & \text{flex}:=\text{$\varepsilon$}, \text{flex2}:=\text{ć}\\
D-\text{t$'$om$'$i} & \text{flex}:=\text{ami}, \text{flex2}:=\text{ć}\\
D-\text{t$'$amy} & \text{flex}:=\text{ami}, \text{flex2}:=\text{ć}\\
D-\text{t$'$ox} & \text{flex}:=\text{ach}, \text{flex2}:=\text{ć}\\
\star-\text{\boldmath$\beta${\bf t}$'$} & \text{flex}:=\text{$\varepsilon$}, \text{flex2}:=\text{ć}\\
-\text{všy} & \text{flex}:=\text{szy}, \text{flex2}:=\text{w}
\end{array}\right] \otimes \left[\begin{array}{ll}
-\text{a} & \text{group}:=\text{a}, \text{verb}\\
-\text{u} & \text{group}:=\text{u}, \text{verb}\\
-\text{\boldmath$\alpha${\bf y}} & \text{group}:=\text{y}, \text{verb}\\
-\text{uje} & \text{group}:=\text{uje}, \text{verb}\\
-\text{eje} & \text{group}:=\text{eje}, \text{verb}\\
-\text{aje} & \text{group}:=\text{aje}, \text{verb}\\
-\text{\boldmath$\alpha${\bf y}je} & \text{group}:=\text{yje}, \text{verb}\\
-\text{uj} & \text{group}:=\text{uj}, \text{verb}\\
-\text{ej} & \text{group}:=\text{ej}, \text{verb}\\
-\text{aj} & \text{group}:=\text{aj}, \text{verb}\\
-\text{\boldmath$\alpha${\bf y}j} & \text{group}:=\text{yj}, \text{verb}\\
D-\text{o} & \text{group}:=\text{a}, \text{verb}\\
D-\text{oje} & \text{group}:=\text{aje}, \text{verb}\\
D-\text{oj} & \text{group}:=\text{aj}, \text{verb}\\
-\text{\boldmath$\gamma\varepsilon$} & \text{group}:=\text{J$\varepsilon$}, \text{verb}\\
-\text{\boldmath$\gamma$e} & \text{group}:=\text{Je}, \text{verb}\\
-\text{\boldmath$\gamma$} & \text{group}:=\text{J}, \text{verb}
\end{array}\right] \otimes \left[\begin{array}{ll}
+\text{ovat$'$} & \text{lemma}:=\text{ować}\\
+\text{yvat$'$} & \text{lemma}:=\text{ywać}\\
+\text{avat$'$} & \text{lemma}:=\text{awać}\\
+\text{at$'$} & \text{lemma}:=\text{ać}\\
+\text{ut$'$} & \text{lemma}:=\text{uć}\\
+\text{yt$'$} & \text{lemma}:=\text{yć}
\end{array}\right]
\]\end{scriptsize}

\begin{scriptsize}\[\hspace{-2cm}
\left[\begin{array}{ll}
\varepsilon- & \text{pref}:=\text{$\varepsilon$}\\
\text{n$'$e}- & \text{pref}:=\text{nie}
\end{array}\right] \otimes \left[\begin{array}{ll}
-\text{{\bf a}m} & \text{flex}:=\text{m}, \text{flex2}:=\text{$\varepsilon$}\\
-\text{ą} & \text{flex}:=\text{ą}, \text{flex2}:=\text{$\varepsilon$}\\
-\text{ę} & \text{flex}:=\text{ę}, \text{flex2}:=\text{$\varepsilon$}\\
D-\text{o} & \text{flex}:=\text{ą}, \text{flex2}:=\text{$\varepsilon$}\\
D-\text{om} & \text{flex}:=\text{ą}, \text{flex2}:=\text{$\varepsilon$}\\
D-\text{e} & \text{flex}:=\text{ę}, \text{flex2}:=\text{$\varepsilon$}\\
-\text{nyx} & \text{flex}:=\text{ych}, \text{flex2}:=\text{n}\\
-\text{nym} & \text{flex}:=\text{ym}, \text{flex2}:=\text{n}\\
-\text{nym$'$i} & \text{flex}:=\text{ymi}, \text{flex2}:=\text{n}\\
-\text{ne} & \text{flex}:=\text{e}, \text{flex2}:=\text{n}\\
-\text{ny} & \text{flex}:=\text{y}, \text{flex2}:=\text{n}\\
-\text{ną} & \text{flex}:=\text{ą}, \text{flex2}:=\text{n}\\
-\text{na} & \text{flex}:=\text{a}, \text{flex2}:=\text{n}\\
-\text{no} & \text{flex}:=\text{o}, \text{flex2}:=\text{n}\\
-\text{nego} & \text{flex}:=\text{ego}, \text{flex2}:=\text{n}\\
-\text{nemu} & \text{flex}:=\text{emu}, \text{flex2}:=\text{n}\\
-\text{nej} & \text{flex}:=\text{ej}, \text{flex2}:=\text{n}\\
-\text{\boldmath$\nu$n$'$i} & \text{flex}:=\text{i}, \text{flex2}:=\text{n}\\
D-\text{no} & \text{flex}:=\text{ą}, \text{flex2}:=\text{n}\\
D-\text{nom} & \text{flex}:=\text{ą}, \text{flex2}:=\text{n}\\
D-\text{nem} & \text{flex}:=\text{ym}, \text{flex2}:=\text{n}\\
D-\text{nem$'$i} & \text{flex}:=\text{ymi}, \text{flex2}:=\text{n}\\
-\text{n$'$om} & \text{flex}:=\text{om}, \text{flex2}:=\text{ń}\\
-\text{n$'$am$'$i} & \text{flex}:=\text{ami}, \text{flex2}:=\text{ń}\\
-\text{n$'$ax} & \text{flex}:=\text{ach}, \text{flex2}:=\text{ń}\\
-\text{n$'$e} & \text{flex}:=\text{e}, \text{flex2}:=\text{ń}\\
-\text{n$'$a} & \text{flex}:=\text{a}, \text{flex2}:=\text{ń}\\
-\text{n$'$u} & \text{flex}:=\text{u}, \text{flex2}:=\text{ń}\\
-\text{n$'$em} & \text{flex}:=\text{em}, \text{flex2}:=\text{ń}\\
-\text{n$'$} & \text{flex}:=\text{$\varepsilon$}, \text{flex2}:=\text{ń}\\
D-\text{n$'$om$'$i} & \text{flex}:=\text{ami}, \text{flex2}:=\text{ń}\\
D-\text{n$'$amy} & \text{flex}:=\text{ami}, \text{flex2}:=\text{ń}\\
D-\text{n$'$ox} & \text{flex}:=\text{ach}, \text{flex2}:=\text{ń}\\
-\text{ącyx} & \text{flex}:=\text{ych}, \text{flex2}:=\text{ąc}\\
-\text{ącym} & \text{flex}:=\text{ym}, \text{flex2}:=\text{ąc}\\
-\text{ącym$'$i} & \text{flex}:=\text{ymi}, \text{flex2}:=\text{ąc}\\
-\text{ące} & \text{flex}:=\text{e}, \text{flex2}:=\text{ąc}\\
-\text{ąco} & \text{flex}:=\text{o}, \text{flex2}:=\text{ąc}\\
-\text{ący} & \text{flex}:=\text{y}, \text{flex2}:=\text{ąc}\\
-\text{ącą} & \text{flex}:=\text{ą}, \text{flex2}:=\text{ąc}\\
-\text{ąca} & \text{flex}:=\text{a}, \text{flex2}:=\text{ąc}\\
-\text{ącego} & \text{flex}:=\text{ego}, \text{flex2}:=\text{ąc}\\
-\text{ącemu} & \text{flex}:=\text{emu}, \text{flex2}:=\text{ąc}\\
-\text{ącej} & \text{flex}:=\text{ej}, \text{flex2}:=\text{ąc}\\
-\text{ąc} & \text{flex}:=\text{$\varepsilon$}, \text{flex2}:=\text{ąc}\\
D-\text{ącom} & \text{flex}:=\text{ą}, \text{flex2}:=\text{ąc}\\
D-\text{ąco} & \text{flex}:=\text{ą}, \text{flex2}:=\text{ąc}\\
D-\text{ącem} & \text{flex}:=\text{ym}, \text{flex2}:=\text{ąc}\\
D-\text{ącem$'$i} & \text{flex}:=\text{ymi}, \text{flex2}:=\text{ąc}
\end{array}\right] \otimes \left[\begin{array}{ll}
-\text{a} & \text{group}:=\text{a}, \text{verb}\\
-\text{u} & \text{group}:=\text{u}, \text{verb}\\
-\text{\boldmath$\alpha${\bf y}} & \text{group}:=\text{y}, \text{verb}\\
-\text{uje} & \text{group}:=\text{uje}, \text{verb}\\
-\text{eje} & \text{group}:=\text{eje}, \text{verb}\\
-\text{aje} & \text{group}:=\text{aje}, \text{verb}\\
-\text{\boldmath$\alpha${\bf y}je} & \text{group}:=\text{yje}, \text{verb}\\
-\text{uj} & \text{group}:=\text{uj}, \text{verb}\\
-\text{ej} & \text{group}:=\text{ej}, \text{verb}\\
-\text{aj} & \text{group}:=\text{aj}, \text{verb}\\
-\text{\boldmath$\alpha${\bf y}j} & \text{group}:=\text{yj}, \text{verb}\\
D-\text{o} & \text{group}:=\text{a}, \text{verb}\\
D-\text{oje} & \text{group}:=\text{aje}, \text{verb}\\
D-\text{oj} & \text{group}:=\text{aj}, \text{verb}\\
-\text{\boldmath$\gamma\varepsilon$} & \text{group}:=\text{J$\varepsilon$}, \text{verb}\\
-\text{\boldmath$\gamma$e} & \text{group}:=\text{Je}, \text{verb}\\
-\text{\boldmath$\gamma$} & \text{group}:=\text{J}, \text{verb}
\end{array}\right] \otimes \left[\begin{array}{ll}
+\text{ovat$'$} & \text{lemma}:=\text{ować}\\
+\text{yvat$'$} & \text{lemma}:=\text{ywać}\\
+\text{avat$'$} & \text{lemma}:=\text{awać}\\
+\text{at$'$} & \text{lemma}:=\text{ać}\\
+\text{ut$'$} & \text{lemma}:=\text{uć}\\
+\text{yt$'$} & \text{lemma}:=\text{yć}
\end{array}\right]
\]\end{scriptsize}

\begin{scriptsize}\[\hspace{-2cm}
\left[\begin{array}{ll}
\varepsilon- & \text{pref}:=\text{$\varepsilon$}\\
\text{n$'$e}- & \text{pref}:=\text{nie}
\end{array}\right] \otimes \left[\begin{array}{ll}
-\varepsilon & \text{flex}:=\text{$\varepsilon$}, \text{flex2}:=\text{$\varepsilon$}\\
-\text{my} & \text{flex}:=\text{my}, \text{flex2}:=\text{$\varepsilon$}\\
-\text{t$'$e} & \text{flex}:=\text{cie}, \text{flex2}:=\text{$\varepsilon$}\\
-\text{š} & \text{flex}:=\text{sz}, \text{flex2}:=\text{$\varepsilon$}\\
D-\text{m} & \text{flex}:=\text{my}, \text{flex2}:=\text{$\varepsilon$}\\
D-\text{s} & \text{flex}:=\text{sz}, \text{flex2}:=\text{$\varepsilon$}\\
-\text{\boldmath$\beta${\bf ł}šy} & \text{flex}:=\text{szy}, \text{flex2}:=\text{ł}\\
-\text{\boldmath$\beta${\bf ł}} & \text{flex}:=\text{$\varepsilon$}, \text{flex2}:=\text{ł}\\
-\text{ł} & \text{flex}:=\text{$\varepsilon$}, \text{flex2}:=\text{ł}\\
-\text{ła} & \text{flex}:=\text{a}, \text{flex2}:=\text{ł}\\
-\text{ło} & \text{flex}:=\text{o}, \text{flex2}:=\text{ł}\\
-\text{ły} & \text{flex}:=\text{y}, \text{flex2}:=\text{ł}\\
-\text{\boldmath$\beta${\bf li}} & \text{flex}:=\text{i}, \text{flex2}:=\text{ł}\\
-\text{tyx} & \text{flex}:=\text{ych}, \text{flex2}:=\text{t}\\
-\text{tym} & \text{flex}:=\text{ym}, \text{flex2}:=\text{t}\\
-\text{tym$'$i} & \text{flex}:=\text{ymi}, \text{flex2}:=\text{t}\\
-\text{te} & \text{flex}:=\text{e}, \text{flex2}:=\text{t}\\
-\text{ty} & \text{flex}:=\text{y}, \text{flex2}:=\text{t}\\
-\text{tą} & \text{flex}:=\text{ą}, \text{flex2}:=\text{t}\\
-\text{ta} & \text{flex}:=\text{a}, \text{flex2}:=\text{t}\\
-\text{to} & \text{flex}:=\text{o}, \text{flex2}:=\text{t}\\
-\text{tego} & \text{flex}:=\text{ego}, \text{flex2}:=\text{t}\\
-\text{temu} & \text{flex}:=\text{emu}, \text{flex2}:=\text{t}\\
-\text{tej} & \text{flex}:=\text{ej}, \text{flex2}:=\text{t}\\
-\text{t$'$i} & \text{flex}:=\text{i}, \text{flex2}:=\text{t}\\
D-\text{to} & \text{flex}:=\text{ą}, \text{flex2}:=\text{t}\\
D-\text{tom} & \text{flex}:=\text{ą}, \text{flex2}:=\text{t}\\
D-\text{tem} & \text{flex}:=\text{ym}, \text{flex2}:=\text{t}\\
D-\text{tem$'$i} & \text{flex}:=\text{ymi}, \text{flex2}:=\text{t}\\
-\text{t$'$om} & \text{flex}:=\text{om}, \text{flex2}:=\text{ć}\\
-\text{t$'$am$'$i} & \text{flex}:=\text{ami}, \text{flex2}:=\text{ć}\\
-\text{t$'$ax} & \text{flex}:=\text{ach}, \text{flex2}:=\text{ć}\\
-\text{t$'$e} & \text{flex}:=\text{e}, \text{flex2}:=\text{ć}\\
-\text{t$'$a} & \text{flex}:=\text{a}, \text{flex2}:=\text{ć}\\
-\text{t$'$u} & \text{flex}:=\text{u}, \text{flex2}:=\text{ć}\\
-\text{t$'$em} & \text{flex}:=\text{em}, \text{flex2}:=\text{ć}\\
-\text{t$'$} & \text{flex}:=\text{$\varepsilon$}, \text{flex2}:=\text{ć}\\
D-\text{t$'$om$'$i} & \text{flex}:=\text{ami}, \text{flex2}:=\text{ć}\\
D-\text{t$'$amy} & \text{flex}:=\text{ami}, \text{flex2}:=\text{ć}\\
D-\text{t$'$ox} & \text{flex}:=\text{ach}, \text{flex2}:=\text{ć}\\
\star-\text{\boldmath$\beta${\bf t}$'$} & \text{flex}:=\text{$\varepsilon$}, \text{flex2}:=\text{ć}\\
-\text{všy} & \text{flex}:=\text{szy}, \text{flex2}:=\text{w}
\end{array}\right] \otimes \left[\begin{array}{ll}
-\text{\boldmath$\alpha'${\bf y}} & \text{group}:=\text{y}, \text{verb}\\
-\text{\boldmath$\alpha'$e} & \text{group}:=\text{e}, \text{verb}\\
-\text{\boldmath$\alpha'$eje} & \text{group}:=\text{eje}, \text{verb}\\
-\text{\boldmath$\alpha'$ej} & \text{group}:=\text{ej}, \text{verb}\\
-\text{\boldmath$\alpha'${\bf y}j} & \text{group}:=\text{yj}, \text{verb}\\
-\text{\boldmath$\alpha'\varepsilon$} & \text{group}:=\text{$\varepsilon$}, \text{verb}\\
-\text{\boldmath$\alpha'$a} & \text{group}:=\text{a}, \text{verb}\\
D-\text{\boldmath$\alpha'$o} & \text{group}:=\text{a}, \text{verb}
\end{array}\right] \otimes \left[\begin{array}{ll}
+\boldmath\alpha'{\bf y}\text{t$'$} & \text{lemma}:=\text{palat-ć}\\
+\boldmath\alpha'\text{et$'$} & \text{lemma}:=\text{palat-eć}
\end{array}\right]
\]\end{scriptsize}

\begin{scriptsize}\[\hspace{-2cm}
\left[\begin{array}{ll}
\varepsilon- & \text{pref}:=\text{$\varepsilon$}\\
\text{n$'$e}- & \text{pref}:=\text{nie}
\end{array}\right] \otimes \left[\begin{array}{ll}
-\text{{\bf a}m} & \text{flex}:=\text{m}, \text{flex2}:=\text{$\varepsilon$}\\
-\text{ą} & \text{flex}:=\text{ą}, \text{flex2}:=\text{$\varepsilon$}\\
-\text{ę} & \text{flex}:=\text{ę}, \text{flex2}:=\text{$\varepsilon$}\\
D-\text{o} & \text{flex}:=\text{ą}, \text{flex2}:=\text{$\varepsilon$}\\
D-\text{om} & \text{flex}:=\text{ą}, \text{flex2}:=\text{$\varepsilon$}\\
D-\text{e} & \text{flex}:=\text{ę}, \text{flex2}:=\text{$\varepsilon$}\\
-\text{nyx} & \text{flex}:=\text{ych}, \text{flex2}:=\text{n}\\
-\text{nym} & \text{flex}:=\text{ym}, \text{flex2}:=\text{n}\\
-\text{nym$'$i} & \text{flex}:=\text{ymi}, \text{flex2}:=\text{n}\\
-\text{ne} & \text{flex}:=\text{e}, \text{flex2}:=\text{n}\\
-\text{ny} & \text{flex}:=\text{y}, \text{flex2}:=\text{n}\\
-\text{ną} & \text{flex}:=\text{ą}, \text{flex2}:=\text{n}\\
-\text{na} & \text{flex}:=\text{a}, \text{flex2}:=\text{n}\\
-\text{no} & \text{flex}:=\text{o}, \text{flex2}:=\text{n}\\
-\text{nego} & \text{flex}:=\text{ego}, \text{flex2}:=\text{n}\\
-\text{nemu} & \text{flex}:=\text{emu}, \text{flex2}:=\text{n}\\
-\text{nej} & \text{flex}:=\text{ej}, \text{flex2}:=\text{n}\\
-\text{\boldmath$\nu$n$'$i} & \text{flex}:=\text{i}, \text{flex2}:=\text{n}\\
D-\text{no} & \text{flex}:=\text{ą}, \text{flex2}:=\text{n}\\
D-\text{nom} & \text{flex}:=\text{ą}, \text{flex2}:=\text{n}\\
D-\text{nem} & \text{flex}:=\text{ym}, \text{flex2}:=\text{n}\\
D-\text{nem$'$i} & \text{flex}:=\text{ymi}, \text{flex2}:=\text{n}\\
-\text{n$'$om} & \text{flex}:=\text{om}, \text{flex2}:=\text{ń}\\
-\text{n$'$am$'$i} & \text{flex}:=\text{ami}, \text{flex2}:=\text{ń}\\
-\text{n$'$ax} & \text{flex}:=\text{ach}, \text{flex2}:=\text{ń}\\
-\text{n$'$e} & \text{flex}:=\text{e}, \text{flex2}:=\text{ń}\\
-\text{n$'$a} & \text{flex}:=\text{a}, \text{flex2}:=\text{ń}\\
-\text{n$'$u} & \text{flex}:=\text{u}, \text{flex2}:=\text{ń}\\
-\text{n$'$em} & \text{flex}:=\text{em}, \text{flex2}:=\text{ń}\\
-\text{n$'$} & \text{flex}:=\text{$\varepsilon$}, \text{flex2}:=\text{ń}\\
D-\text{n$'$om$'$i} & \text{flex}:=\text{ami}, \text{flex2}:=\text{ń}\\
D-\text{n$'$amy} & \text{flex}:=\text{ami}, \text{flex2}:=\text{ń}\\
D-\text{n$'$ox} & \text{flex}:=\text{ach}, \text{flex2}:=\text{ń}\\
-\text{ącyx} & \text{flex}:=\text{ych}, \text{flex2}:=\text{ąc}\\
-\text{ącym} & \text{flex}:=\text{ym}, \text{flex2}:=\text{ąc}\\
-\text{ącym$'$i} & \text{flex}:=\text{ymi}, \text{flex2}:=\text{ąc}\\
-\text{ące} & \text{flex}:=\text{e}, \text{flex2}:=\text{ąc}\\
-\text{ąco} & \text{flex}:=\text{o}, \text{flex2}:=\text{ąc}\\
-\text{ący} & \text{flex}:=\text{y}, \text{flex2}:=\text{ąc}\\
-\text{ącą} & \text{flex}:=\text{ą}, \text{flex2}:=\text{ąc}\\
-\text{ąca} & \text{flex}:=\text{a}, \text{flex2}:=\text{ąc}\\
-\text{ącego} & \text{flex}:=\text{ego}, \text{flex2}:=\text{ąc}\\
-\text{ącemu} & \text{flex}:=\text{emu}, \text{flex2}:=\text{ąc}\\
-\text{ącej} & \text{flex}:=\text{ej}, \text{flex2}:=\text{ąc}\\
-\text{ąc} & \text{flex}:=\text{$\varepsilon$}, \text{flex2}:=\text{ąc}\\
D-\text{ącom} & \text{flex}:=\text{ą}, \text{flex2}:=\text{ąc}\\
D-\text{ąco} & \text{flex}:=\text{ą}, \text{flex2}:=\text{ąc}\\
D-\text{ącem} & \text{flex}:=\text{ym}, \text{flex2}:=\text{ąc}\\
D-\text{ącem$'$i} & \text{flex}:=\text{ymi}, \text{flex2}:=\text{ąc}
\end{array}\right] \otimes \left[\begin{array}{ll}
-\text{\boldmath$\alpha'$} & \text{group}:=\text{J}, \text{verb}\\
-\text{\boldmath$\alpha'$ej} & \text{group}:=\text{J}, \text{verb}\\
-\text{\boldmath$\zeta$} & \text{group}:=\text{J}, \text{verb}\\
-\text{\boldmath$\alpha'$e} & \text{group}:=\text{Je}, \text{verb}\\
-\text{\boldmath$\zeta$e} & \text{group}:=\text{Je}, \text{verb}\\
-\text{\boldmath$\alpha'$a} & \text{group}:=\text{Ja}, \text{verb}\\
-\text{\boldmath$\alpha'$o} & \text{group}:=\text{Jo}, \text{verb}\\
-\text{\boldmath$\zeta$o} & \text{group}:=\text{Jo}, \text{verb}
\end{array}\right] \otimes \left[\begin{array}{ll}
+\text{t$'$} & \text{lemma}:=\text{palat-ć}\\
+\text{et$'$} & \text{lemma}:=\text{palat-eć}
\end{array}\right]
\]\end{scriptsize}

\begin{scriptsize}\[\hspace{-2cm}
\left[\begin{array}{ll}
\varepsilon- & \text{pref}:=\text{$\varepsilon$}\\
\text{n$'$e}- & \text{pref}:=\text{nie}
\end{array}\right] \otimes \left[\begin{array}{ll}
-\varepsilon & \text{flex}:=\text{$\varepsilon$}, \text{flex2}:=\text{$\varepsilon$}\\
-\text{my} & \text{flex}:=\text{my}, \text{flex2}:=\text{$\varepsilon$}\\
-\text{t$'$e} & \text{flex}:=\text{cie}, \text{flex2}:=\text{$\varepsilon$}\\
-\text{š} & \text{flex}:=\text{sz}, \text{flex2}:=\text{$\varepsilon$}\\
D-\text{m} & \text{flex}:=\text{my}, \text{flex2}:=\text{$\varepsilon$}\\
D-\text{s} & \text{flex}:=\text{sz}, \text{flex2}:=\text{$\varepsilon$}\\
-\text{\boldmath$\beta${\bf ł}šy} & \text{flex}:=\text{szy}, \text{flex2}:=\text{ł}\\
-\text{\boldmath$\beta${\bf ł}} & \text{flex}:=\text{$\varepsilon$}, \text{flex2}:=\text{ł}\\
-\text{ł} & \text{flex}:=\text{$\varepsilon$}, \text{flex2}:=\text{ł}\\
-\text{ła} & \text{flex}:=\text{a}, \text{flex2}:=\text{ł}\\
-\text{ło} & \text{flex}:=\text{o}, \text{flex2}:=\text{ł}\\
-\text{ły} & \text{flex}:=\text{y}, \text{flex2}:=\text{ł}\\
-\text{\boldmath$\beta${\bf li}} & \text{flex}:=\text{i}, \text{flex2}:=\text{ł}\\
-\text{tyx} & \text{flex}:=\text{ych}, \text{flex2}:=\text{t}\\
-\text{tym} & \text{flex}:=\text{ym}, \text{flex2}:=\text{t}\\
-\text{tym$'$i} & \text{flex}:=\text{ymi}, \text{flex2}:=\text{t}\\
-\text{te} & \text{flex}:=\text{e}, \text{flex2}:=\text{t}\\
-\text{ty} & \text{flex}:=\text{y}, \text{flex2}:=\text{t}\\
-\text{tą} & \text{flex}:=\text{ą}, \text{flex2}:=\text{t}\\
-\text{ta} & \text{flex}:=\text{a}, \text{flex2}:=\text{t}\\
-\text{to} & \text{flex}:=\text{o}, \text{flex2}:=\text{t}\\
-\text{tego} & \text{flex}:=\text{ego}, \text{flex2}:=\text{t}\\
-\text{temu} & \text{flex}:=\text{emu}, \text{flex2}:=\text{t}\\
-\text{tej} & \text{flex}:=\text{ej}, \text{flex2}:=\text{t}\\
-\text{t$'$i} & \text{flex}:=\text{i}, \text{flex2}:=\text{t}\\
D-\text{to} & \text{flex}:=\text{ą}, \text{flex2}:=\text{t}\\
D-\text{tom} & \text{flex}:=\text{ą}, \text{flex2}:=\text{t}\\
D-\text{tem} & \text{flex}:=\text{ym}, \text{flex2}:=\text{t}\\
D-\text{tem$'$i} & \text{flex}:=\text{ymi}, \text{flex2}:=\text{t}\\
-\text{t$'$om} & \text{flex}:=\text{om}, \text{flex2}:=\text{ć}\\
-\text{t$'$am$'$i} & \text{flex}:=\text{ami}, \text{flex2}:=\text{ć}\\
-\text{t$'$ax} & \text{flex}:=\text{ach}, \text{flex2}:=\text{ć}\\
-\text{t$'$e} & \text{flex}:=\text{e}, \text{flex2}:=\text{ć}\\
-\text{t$'$a} & \text{flex}:=\text{a}, \text{flex2}:=\text{ć}\\
-\text{t$'$u} & \text{flex}:=\text{u}, \text{flex2}:=\text{ć}\\
-\text{t$'$em} & \text{flex}:=\text{em}, \text{flex2}:=\text{ć}\\
-\text{t$'$} & \text{flex}:=\text{$\varepsilon$}, \text{flex2}:=\text{ć}\\
D-\text{t$'$om$'$i} & \text{flex}:=\text{ami}, \text{flex2}:=\text{ć}\\
D-\text{t$'$amy} & \text{flex}:=\text{ami}, \text{flex2}:=\text{ć}\\
D-\text{t$'$ox} & \text{flex}:=\text{ach}, \text{flex2}:=\text{ć}\\
\star-\text{\boldmath$\beta${\bf t}$'$} & \text{flex}:=\text{$\varepsilon$}, \text{flex2}:=\text{ć}\\
-\text{všy} & \text{flex}:=\text{szy}, \text{flex2}:=\text{w}
\end{array}\right] \otimes \left[\begin{array}{ll}
-\text{\boldmath$\eta$n$'$e} & \text{group}:=\text{nie}, \text{verb}\\
-\text{\boldmath$\eta$ną} & \text{group}:=\text{ną}, \text{verb}\\
-\text{\boldmath$\eta$n$'$ę} & \text{group}:=\text{nię}, \text{verb}\\
-\text{\boldmath$\eta$nę} & \text{group}:=\text{nę}, \text{verb}\\
-\text{\boldmath$\eta$n$'$ij} & \text{group}:=\text{nij}, \text{verb}\\
-\text{\boldmath$\eta$n} & \text{group}:=\text{n}, \text{verb}\\
-\text{\boldmath$\eta$} & \text{group}:=\text{$\varepsilon$}, \text{verb}
\end{array}\right] \otimes \left[\begin{array}{ll}
+\text{\boldmath$\eta$nąt$'$} & \text{lemma}:=\text{nąć}\\
\star+\text{\boldmath$\beta${\bf t}$'$} & \text{lemma}:=\text{$\varepsilon$ć}
\end{array}\right]
\]\end{scriptsize}

\begin{scriptsize}\[\hspace{-2cm}
\left[\begin{array}{ll}
\varepsilon- & \text{pref}:=\text{$\varepsilon$}\\
\text{n$'$e}- & \text{pref}:=\text{nie}
\end{array}\right] \otimes \left[\begin{array}{ll}
-\text{{\bf a}m} & \text{flex}:=\text{m}, \text{flex2}:=\text{$\varepsilon$}\\
-\text{ą} & \text{flex}:=\text{ą}, \text{flex2}:=\text{$\varepsilon$}\\
-\text{ę} & \text{flex}:=\text{ę}, \text{flex2}:=\text{$\varepsilon$}\\
D-\text{o} & \text{flex}:=\text{ą}, \text{flex2}:=\text{$\varepsilon$}\\
D-\text{om} & \text{flex}:=\text{ą}, \text{flex2}:=\text{$\varepsilon$}\\
D-\text{e} & \text{flex}:=\text{ę}, \text{flex2}:=\text{$\varepsilon$}\\
-\text{nyx} & \text{flex}:=\text{ych}, \text{flex2}:=\text{n}\\
-\text{nym} & \text{flex}:=\text{ym}, \text{flex2}:=\text{n}\\
-\text{nym$'$i} & \text{flex}:=\text{ymi}, \text{flex2}:=\text{n}\\
-\text{ne} & \text{flex}:=\text{e}, \text{flex2}:=\text{n}\\
-\text{ny} & \text{flex}:=\text{y}, \text{flex2}:=\text{n}\\
-\text{ną} & \text{flex}:=\text{ą}, \text{flex2}:=\text{n}\\
-\text{na} & \text{flex}:=\text{a}, \text{flex2}:=\text{n}\\
-\text{no} & \text{flex}:=\text{o}, \text{flex2}:=\text{n}\\
-\text{nego} & \text{flex}:=\text{ego}, \text{flex2}:=\text{n}\\
-\text{nemu} & \text{flex}:=\text{emu}, \text{flex2}:=\text{n}\\
-\text{nej} & \text{flex}:=\text{ej}, \text{flex2}:=\text{n}\\
-\text{\boldmath$\nu$n$'$i} & \text{flex}:=\text{i}, \text{flex2}:=\text{n}\\
D-\text{no} & \text{flex}:=\text{ą}, \text{flex2}:=\text{n}\\
D-\text{nom} & \text{flex}:=\text{ą}, \text{flex2}:=\text{n}\\
D-\text{nem} & \text{flex}:=\text{ym}, \text{flex2}:=\text{n}\\
D-\text{nem$'$i} & \text{flex}:=\text{ymi}, \text{flex2}:=\text{n}\\
-\text{n$'$om} & \text{flex}:=\text{om}, \text{flex2}:=\text{ń}\\
-\text{n$'$am$'$i} & \text{flex}:=\text{ami}, \text{flex2}:=\text{ń}\\
-\text{n$'$ax} & \text{flex}:=\text{ach}, \text{flex2}:=\text{ń}\\
-\text{n$'$e} & \text{flex}:=\text{e}, \text{flex2}:=\text{ń}\\
-\text{n$'$a} & \text{flex}:=\text{a}, \text{flex2}:=\text{ń}\\
-\text{n$'$u} & \text{flex}:=\text{u}, \text{flex2}:=\text{ń}\\
-\text{n$'$em} & \text{flex}:=\text{em}, \text{flex2}:=\text{ń}\\
-\text{n$'$} & \text{flex}:=\text{$\varepsilon$}, \text{flex2}:=\text{ń}\\
D-\text{n$'$om$'$i} & \text{flex}:=\text{ami}, \text{flex2}:=\text{ń}\\
D-\text{n$'$amy} & \text{flex}:=\text{ami}, \text{flex2}:=\text{ń}\\
D-\text{n$'$ox} & \text{flex}:=\text{ach}, \text{flex2}:=\text{ń}\\
-\text{ącyx} & \text{flex}:=\text{ych}, \text{flex2}:=\text{ąc}\\
-\text{ącym} & \text{flex}:=\text{ym}, \text{flex2}:=\text{ąc}\\
-\text{ącym$'$i} & \text{flex}:=\text{ymi}, \text{flex2}:=\text{ąc}\\
-\text{ące} & \text{flex}:=\text{e}, \text{flex2}:=\text{ąc}\\
-\text{ąco} & \text{flex}:=\text{o}, \text{flex2}:=\text{ąc}\\
-\text{ący} & \text{flex}:=\text{y}, \text{flex2}:=\text{ąc}\\
-\text{ącą} & \text{flex}:=\text{ą}, \text{flex2}:=\text{ąc}\\
-\text{ąca} & \text{flex}:=\text{a}, \text{flex2}:=\text{ąc}\\
-\text{ącego} & \text{flex}:=\text{ego}, \text{flex2}:=\text{ąc}\\
-\text{ącemu} & \text{flex}:=\text{emu}, \text{flex2}:=\text{ąc}\\
-\text{ącej} & \text{flex}:=\text{ej}, \text{flex2}:=\text{ąc}\\
-\text{ąc} & \text{flex}:=\text{$\varepsilon$}, \text{flex2}:=\text{ąc}\\
D-\text{ącom} & \text{flex}:=\text{ą}, \text{flex2}:=\text{ąc}\\
D-\text{ąco} & \text{flex}:=\text{ą}, \text{flex2}:=\text{ąc}\\
D-\text{ącem} & \text{flex}:=\text{ym}, \text{flex2}:=\text{ąc}\\
D-\text{ącem$'$i} & \text{flex}:=\text{ymi}, \text{flex2}:=\text{ąc}
\end{array}\right] \otimes \left[\begin{array}{ll}
-\text{\boldmath$\eta$n$'$e} & \text{group}:=\text{nie}, \text{verb}\\
-\text{\boldmath$\eta$ną} & \text{group}:=\text{ną}, \text{verb}\\
-\text{\boldmath$\eta$n$'$ę} & \text{group}:=\text{nię}, \text{verb}\\
-\text{\boldmath$\eta$nę} & \text{group}:=\text{nę}, \text{verb}\\
-\text{\boldmath$\eta$n$'$ij} & \text{group}:=\text{nij}, \text{verb}\\
-\text{\boldmath$\eta$n} & \text{group}:=\text{n}, \text{verb}\\
-\text{\boldmath$\eta$} & \text{group}:=\text{$\varepsilon$}, \text{verb}
\end{array}\right] \otimes \left[\begin{array}{ll}
+\text{\boldmath$\eta$nąt$'$} & \text{lemma}:=\text{nąć}\\
\star+\text{\boldmath$\beta${\bf t}$'$} & \text{lemma}:=\text{$\varepsilon$ć}
\end{array}\right]
\]\end{scriptsize}

\subsection{Reguły dla rzeczowników z lematami wygłosowymi}

Tytuły kolumn oznaczają tutaj przedostatnią głoskę, zaś wierszy -- ostatnią.

\begin{longtable}{p{4mm}|p{4mm}p{4mm}p{4mm}p{4mm}p{4mm}p{4mm}p{4mm}p{4mm}p{4mm}}
 & a & ą & e & ę & i & o & ó & u & y\\
\hline
b & ab & ąb & eb & ąb & ib & ob & ób & ub & yb\\
 &  &  &  & ęb &  & ób &  &  & \\
\hline
bi & ab & ąb & eb & ąb &  & ób &  & ub & \\
\hline
c & ac & ąc & ec & ęc & ic & oc &  & uc & yc\\
\hline
ch & ach & ąch & ech & ęch & ich & och & óch & uch & ych\\
\hline
ci & ać & ąć & eć & ęć & ić & oć &  & uć & yc\\
 &  &  &  &  &  & óć &  &  & yć\\
\hline
cz & acz & ącz & ecz & ęcz & icz & ocz & ócz & ucz & ycz\\
\hline
dz & adz & ądz & edz & ędz & idz & ódz &  & udz & ydz\\
\hline
dzi & adź &  & edź & ądź & idź & odź & ódź & udź & \\
 &  &  &  & ędź &  & ódź &  &  & \\
\hline
dż & adż &  & edż &  & idż & odż &  & udż & ydż\\
\hline
f & af &  & ef &  & if & of &  & uf & yf\\
\hline
g & ag & ąg & eg & ąg & ig & og &  & ug & yg\\
 &  &  &  & ęg &  & óg &  &  & \\
\hline
gi & ag &  &  & ęg & ig & óg &  & ug & \\
\hline
h & ah &  &  &  &  & oh &  & uh & \\
\hline
j & aj &  & ej &  & ij & oj & ój & uj & yj\\
 &  &  &  &  &  & ój &  &  & \\
\hline
k & ak & ąk & ek & ąk & ik & ok & ók & uk & yk\\
 &  &  &  & ęk &  &  &  &  & \\
\hline
ki &  &  &  & ąk & ik & ok &  & uk & yk\\
 &  &  &  & ęk &  &  &  &  & \\
\hline
l & al &  & el &  & il & ol & ól & ul & yl\\
 &  &  &  &  &  & ól &  &  & \\
\hline
ł & ał &  & eł &  & ił & oł & ół & uł & ył\\
 &  &  &  &  &  & ół &  &  & \\
\hline
m & am &  & em &  & im & om &  & um & ym\\
\hline
mi &  &  & em &  & im & om &  &  & ym\\
\hline
n & an &  & en &  & in & on &  & un & yn\\
\hline
ni & ań &  & eń &  & iń & oń &  & uń & yń\\
\hline
p & ap &  & ep & ęp & ip & op &  & up & yp\\
 &  &  &  &  &  & óp &  &  & \\
\hline
pi & ap & ąp & ep &  & ip & op &  & up & \\
\hline
q & aq &  &  &  &  &  &  &  & \\
\hline
r & ar &  & er &  & ir & or & ór & ur & yr\\
 &  &  &  &  &  & ór &  &  & \\
\hline
rz & arz &  & erz &  &  & orz & órz & urz & yrz\\
 &  &  &  &  &  & órz &  &  & \\
\hline
s & as & ąs & es & ąs & is & os &  & us & ys\\
 &  &  &  & ęs &  &  &  &  & \\
\hline
si & aś & ąś & eś & ęś & iś & oś &  & uś & yś\\
\hline
sz & asz &  & esz &  & isz & osz & ósz & usz & ysz\\
\hline
t & at & ąt & et & ąt & it & ot & ót & ut & yt\\
 &  &  &  & ęt &  & ót &  &  & \\
\hline
u & au &  &  &  &  &  &  &  & \\
\hline
v & av &  &  &  & iv &  &  &  & \\
\hline
w & aw &  & ew &  & iw & ow &  & uw & yw\\
 &  &  &  &  &  & ów &  &  & \\
\hline
wi & aw &  & ew &  &  & ów &  &  & \\
\hline
y & ay &  & ey &  &  & oy &  &  & \\
\hline
z & az & ąz & ez & ąz & iz & oz &  & uz & yz\\
 &  &  &  & ęz &  & óz &  &  & \\
\hline
zi & aź &  & eź & ąź & iź & oź & óź & uź & yź\\
 &  &  &  & ęź &  & óź &  &  & \\
\hline
ż & aż & ąż & eż & ąż & iż & oż & óż & uż & yż\\
 &  &  &  & ęż &  & óż &  &  & \\
\hline
\end{longtable}

\begin{longtable}{p{4mm}|p{4mm}p{4mm}p{4mm}p{4mm}p{4mm}p{4mm}p{4mm}p{4mm}p{4mm}p{4mm}p{4mm}p{4mm}p{4mm}p{4mm}}
 & c & ć & cz & dz & dź & dż & j & l & ń & rz & sz & ś & ź & ż\\
\hline
b &  & ćb & czb &  & dźb &  & jb & lb & ńb & rzb &  & śb & źb & żb\\
\hline
c & cc & ciec & czec &  & dziec &  & jc & lc & ńc & rzec & szec & siec & ziec & żec\\
 &  &  &  &  &  &  & jec & lec & niec &  &  &  &  & \\
\hline
ch &  &  &  &  &  &  &  & lch &  & rzch & szech &  &  & \\
\hline
ci &  &  &  &  &  &  &  & lć & ńć &  &  & ść &  & \\
\hline
cz &  &  &  &  &  &  & jcz & lcz & ńcz &  & szcz &  &  & \\
\hline
dzi &  &  &  &  &  &  &  & ldź & ńdź &  &  &  &  & \\
\hline
dż &  &  &  &  &  & dżdż &  &  &  &  &  &  &  & \\
\hline
f &  &  &  &  &  &  & jf & lf &  &  &  &  &  & \\
\hline
g &  &  &  &  &  &  & jg & lg &  & rzg &  &  &  & \\
\hline
h &  &  &  &  &  &  &  & lh &  &  &  &  &  & \\
\hline
k & ck & ciek & czek & dzk & dziek & dżek & jk & lk & niek & rzek & szek & siek & ziek & żek\\
 & cek &  &  & dzek &  &  & jek & lek &  &  &  &  &  & \\
\hline
ki & cek & ciek & czek &  &  &  & jek & lek & niek &  & szek & siek &  & żek\\
\hline
l & cel &  & czel & dzel &  &  & jl & ll &  &  & szel & śl &  & żel\\
\hline
ł & ceł &  &  &  &  &  & jeł &  &  &  &  &  &  & żeł\\
\hline
m &  & ćm & czm &  & dźm &  & jm & lm &  & rzm & szem & śm &  & żm\\
 &  & ciem & czem &  &  &  & jem &  &  & rzem &  &  &  & żem\\
\hline
n &  &  & czn &  &  &  & jn & ln &  & rzn & szn & śn & zien & żn\\
 &  &  & czen &  &  &  & jen & len &  &  & szen & sien &  & żen\\
\hline
ni &  &  & czeń &  &  &  & jń & lń &  & rzeń & szeń & śń & źń & żeń\\
 &  &  &  &  &  &  & jen &  &  &  &  & sien & zień & \\
 &  &  &  &  &  &  &  &  &  &  &  & sień &  & \\
\hline
p & cp &  &  &  &  &  & jp & lp &  &  &  &  &  & \\
\hline
q & cq &  &  &  &  &  &  &  &  &  &  &  &  & \\
\hline
r & cr &  &  &  &  & dżr & jr &  &  &  &  &  &  & \\
 &  &  &  &  &  &  & jer &  &  &  &  &  &  & \\
\hline
s & cs &  &  &  &  &  & js & ls &  &  &  &  &  & \\
\hline
sz &  &  &  &  &  &  & jsz & lsz &  &  &  &  &  & ższ\\
\hline
t & ct &  & czt &  &  &  & jt & lt &  & rzt & szt &  &  & \\
 & cet &  & czet &  &  &  &  &  &  &  &  &  &  & \\
\hline
v &  &  &  &  &  &  &  & lv &  &  &  &  &  & \\
\hline
w &  & ćw &  &  &  &  & jw & lw &  & rzw & szw &  &  & żw\\
 &  &  &  &  &  &  &  & lew &  &  & szew &  &  & żew\\
\hline
wi &  &  &  &  &  &  &  & lew &  &  &  &  &  & \\
\hline
y &  &  &  &  &  &  &  & ly &  &  &  &  &  & \\
\hline
z &  &  &  &  &  &  & jz & lz &  &  &  &  &  & \\
\hline
ż &  &  &  &  &  &  &  & lż &  &  &  &  &  & \\
\hline
\end{longtable}

\begin{longtable}{p{2mm}|p{2mm}p{4mm}p{2mm}p{1mm}p{3mm}p{2mm}p{3mm}p{2mm}p{4mm}p{3mm}p{3mm}p{2mm}p{2mm}p{2mm}p{1mm}p{1mm}p{1mm}p{4mm}}
 & b & ch & d & f & g & h & k & ł & m & n & p & r & s & t & v & w & x & z\\
\hline
b &  &  &  &  &  &  &  & łb & mb &  &  & rb & sb &  &  & wb &  & zb\\
 &  &  &  &  &  &  &  & łeb &  &  &  & reb &  &  &  &  &  & \\
\hline
bi &  &  &  &  &  &  &  & łb &  &  &  &  &  &  &  &  &  & \\
\hline
c & biec &  & dc & fiec &  &  &  &  & miec & nc & pc & rc & sc & ciec &  & wc &  & ziec\\
 &  &  & dec &  &  &  &  &  &  & niec & pec & rzec & sec &  &  & wiec &  & \\
 &  &  & dziec &  &  &  &  &  &  &  & piec &  & siec &  &  &  &  & \\
\hline
ch &  &  &  &  &  &  & kch &  & mech & nch &  & rch & sch & tch &  &  &  & \\
\hline
ci & bć & chć &  & fć & gieć &  & kć & łć & mć & nć & pć & rć &  &  &  & wć &  & \\
 &  & cheć &  &  &  &  & kieć & łeć &  &  & peć &  &  &  &  &  &  & \\
\hline
cz &  &  & decz &  &  &  &  & łcz &  & ncz &  & rcz &  &  &  &  &  & \\
\hline
dz &  &  &  &  &  &  &  &  &  & ndz &  & rdz &  &  &  &  &  & \\
\hline
dzi &  &  &  &  & gdź &  &  &  & mdź & ndź &  & rdź &  &  &  &  &  & \\
\hline
dż &  &  &  &  &  &  &  &  &  & ndż &  &  &  &  &  &  &  & \\
\hline
f &  &  &  & ff &  &  &  &  & mf & nf & pf & rf &  &  &  &  &  & \\
\hline
g &  &  &  &  & gg &  &  & łg & mg & ng &  & rg &  &  &  &  &  & zg\\
 &  &  &  &  &  &  &  &  &  &  &  &  &  &  &  &  &  & zeg\\
\hline
gi &  &  &  &  &  &  &  &  &  & ng &  &  &  &  &  &  &  & \\
\hline
h &  &  & dh &  & gh &  & kh &  &  & nh & ph & rh & sh & th &  &  &  & \\
\hline
k & bek & chek & dek & fek &  &  & kk & łk & mek & nk & pek & rk & sk & tek & vek & wk &  & zek\\
 &  &  &  &  &  &  &  & łek &  & nek &  & rek & sek &  &  & wek &  & \\
\hline
ki & bek &  & dek &  &  &  &  & łk &  & nek & pek & rek & sek & tek &  & wek &  & \\
 &  &  &  &  &  &  &  & łek &  &  &  &  &  &  &  &  &  & \\
\hline
l & bl & chel & del & fl & gel & hl & kl &  & ml & nel & pl & rl & sel & tl & vel & wl & xel & zel\\
 & bel &  &  & fel & giel & hel & kel &  & mel &  & pel &  &  & tel &  & wel &  & \\
 &  &  &  &  &  &  & kiel &  &  &  &  &  &  &  &  &  &  & \\
\hline
ł & bł & cheł & deł &  & gł &  & kieł & łł & mł &  & peł & reł & sł & cieł &  & weł &  & zł\\
 & beł &  &  &  & gieł &  &  &  &  &  &  & rzeł & seł & teł &  &  &  & zeł\\
 &  &  &  &  &  &  &  &  &  &  &  &  &  &  &  &  &  & zieł\\
\hline
m &  & chm & dm &  & gm & hm &  & łm & mm &  &  & rm & sm & tm &  &  &  & zm\\
 &  &  & dem &  &  &  &  &  &  &  &  &  & sem &  &  &  &  & \\
\hline
n & bn & chn & dn & fn & gn & hn & kn & łn & mn & nn & pn & rn & sn & tn &  & wn &  & zn\\
 & ben & chen & den &  & gien & hen & kien & łen & men & nien & pien & ren & sen & cien &  & wien &  & zen\\
 &  &  &  &  &  &  &  &  & mien &  &  &  & sien &  &  &  &  & \\
\hline
ni &  & chen & deń &  & gien &  & kien & łń & mń &  & pń & rń &  & cień &  & wien &  & \\
 &  & cheń & dzien &  & gień &  &  &  &  &  & pień & reń &  & teń &  & wień &  & \\
 &  &  & dzień &  &  &  &  &  &  &  &  &  &  &  &  &  &  & \\
\hline
p & bp &  &  &  &  &  & kiep & łp & mp &  & pp & rp & sp &  &  &  &  & \\
 &  &  &  &  &  &  &  &  &  &  &  &  & sep &  &  &  &  & \\
\hline
pi &  &  &  &  &  &  &  & łp & mp &  &  & rp &  &  &  &  &  & \\
\hline
r & br & chr & dr & fr & gr & hr & kr &  & mr & nr & pr & rr &  & tr &  & wr &  & \\
 & ber & cher & der & fer & ger &  & kier &  & mer &  & per &  &  & ter &  & wer &  & \\
 &  &  &  &  & gier &  &  &  &  &  &  &  &  &  &  &  &  & \\
\hline
rz & brz & chrz &  &  & gierz &  & kierz &  &  &  & prz &  &  & trz &  &  &  & \\
 & berz &  &  &  &  &  &  &  &  &  &  &  &  &  &  &  &  & \\
\hline
s & bs &  & ds &  &  &  & ks &  & ms & ns & ps & rs & ss & ts &  & ws &  & zs\\
 &  &  &  &  &  &  & x &  &  &  & pies &  &  &  &  & wies &  & \\
\hline
si &  &  &  &  &  &  &  &  &  &  &  & rś &  &  &  & wieś &  & \\
\hline
sz &  &  &  &  &  &  & ksz & łsz & msz & nsz &  & rsz &  &  &  & wesz &  & \\
\hline
t &  & cht & dt & ft & gt & ht & kt & łt &  & nt & pt & rt & st & tt &  &  &  & \\
 &  &  &  &  &  &  &  &  &  &  &  &  & set &  &  &  &  & \\
\hline
w &  & chw & dw &  & gw &  & kw & łw &  & nw &  & rw &  & tw &  &  &  & zw\\
 &  & chew & dew &  &  &  & kiew &  &  &  &  & rew &  & tew &  &  &  & zew\\
\hline
wi &  & chew &  &  & giew &  & kiew & łw &  & new &  & rw &  & tew &  &  &  & zew\\
 &  &  &  &  &  &  &  &  &  &  &  & rew &  &  &  &  &  & \\
\hline
y &  &  &  &  & gy &  &  &  &  &  &  &  &  &  &  &  &  & \\
\hline
z & bz &  &  &  & gz &  &  & łz & mz & nz &  &  &  & tz &  &  &  & zz\\
 & bez &  &  &  & giez &  &  & łez &  &  &  &  &  &  &  &  &  & \\
\hline
ż &  &  &  &  & gż &  &  & łż & mż & nż &  & rż &  &  &  &  &  & \\
 &  &  &  &  &  &  &  & łeż & meż &  &  & reż &  &  &  &  &  & \\
\hline
\end{longtable}

\subsection{Reguły warstwy interpretacji}

\begin{longtable}{p{7cm}|l|l|l}
interpretacja & cat & flex & lemma\\
\hline
adj:sg:nom.acc.voc:n:pos adj:pl:nom.acc.voc:m2.m3.f.n:pos & adj & e & y\\
adj:sg:nom.voc:m1.m2.m3:pos adj:sg:acc:m3:pos adj:pl:nom.voc:m1:pos & adj & y & y\\
adj:sg:nom.voc:m1.m2.m3:pos adj:sg:acc:m3:pos & adj & y & y\\
adj:sg:nom.voc:f:pos & adj & a & y\\
adj:sg:gen.dat.loc:f:pos & adj & ej & y\\
adj:sg:gen:m1.m2.m3.n:pos adj:sg:acc:m1.m2:pos & adj & ego & y\\
adj:sg:dat:m1.m2.m3.n:pos & adj & emu & y\\
adj:sg:acc.inst:f:pos & adj & ą & y\\
adj:sg:inst.loc:m1.m2.m3.n:pos adj:pl:dat:m1.m2.m3.f.n:pos & adj & ym & y\\
adj:pl:nom.voc:m1:pos & adj & i & y\\
adj:pl:gen.loc:m1.m2.m3.f.n:pos adj:pl:acc:m1:pos & adj & ych & y\\
adj:pl:inst:m1.m2.m3.f.n:pos & adj & ymi & y\\
adjp & adj & u & y\\
adja & adj & o & y\\
$\star$adj:sg:nom:m1.m2.m3:pos adj:sg:acc:m3:pos & adj & $\varepsilon$ & y\\
$\star$adjc & adj & $\varepsilon$ & y\\
$\star$adj:sg:nom.acc.voc:n:pos adja & adj & o & $\varepsilon$\\
$\star$adj:sg:nom.acc.voc:n:pos & adj & o & $\varepsilon$\\
$\star$adj:sg:nom.acc.voc:n:pos adj:pl:nom.acc.voc:m2.m3.f.n:pos & adj & e & $\varepsilon$\\
$\star$adj:sg:nom.voc:m1.m2.m3:pos adj:sg:acc:m3:pos & adj & $\varepsilon$ & $\varepsilon$\\
$\star$adj:sg:nom.voc:m1.m2.m3:pos adj:sg:acc:m3:pos & adj & y & $\varepsilon$\\
$\star$adj:sg:nom.voc:f:pos & adj & a & $\varepsilon$\\
$\star$adj:sg:nom:m1.m2.m3:pos adj:sg:acc:m3:pos & adj & $\varepsilon$ & $\varepsilon$\\
$\star$adj:sg:nom:m1.m2.m3:pos & adj & $\varepsilon$ & $\varepsilon$\\
$\star$adj:sg:nom:n:pos adj:pl:nom:m2.m3.f.n:pos & adj & e & $\varepsilon$\\
$\star$adj:sg:nom:f:pos & adj & a & $\varepsilon$\\
$\star$adj:sg:gen.dat.loc:f:pos & adj & ej & $\varepsilon$\\
$\star$adj:sg:gen:m1.m2.m3.n:pos adj:sg:acc:m1.m2:pos & adj & ego & $\varepsilon$\\
$\star$adj:sg:dat:m1.m2.m3.n:pos & adj & emu & $\varepsilon$\\
$\star$adj:sg:acc.inst:f:pos & adj & ą & $\varepsilon$\\
$\star$adj:sg:inst.loc:m1.m2.m3.n:pos adj:pl:dat:m1.m2.m3.f.n:pos & adj & ym & $\varepsilon$\\
$\star$adj:pl:nom.acc.voc:m2.m3.f.n:pos & adj & e & $\varepsilon$\\
$\star$adj:pl:nom.voc:m1:pos & adj & y & $\varepsilon$\\
$\star$adj:pl:nom.voc:m1:pos & adj & i & $\varepsilon$\\
$\star$adj:pl:nom:m1:pos & adj & i & $\varepsilon$\\
$\star$adj:pl:gen.loc:m1.m2.m3.f.n:pos adj:pl:acc:m1:pos & adj & ych & $\varepsilon$\\
$\star$adj:pl:inst:m1.m2.m3.f.n:pos & adj & ymi & $\varepsilon$\\
$\star$adj:sg.pl:nom.gen.dat.acc.inst.loc.voc & adj & $\varepsilon$ & $\varepsilon$ \\
	\quad \quad :m1.m2.m3.f.n:pos adja & & & \\
$\star$adj:sg.pl:nom.gen.dat.acc.inst.loc.voc & adj & $\varepsilon$ & $\varepsilon$\\
	\quad \quad :m1.m2.m3.f.n:pos & & & \\
\end{longtable}
\begin{longtable}{p{7cm}|l|l|l|l}
interpretacja & cat & pref & flex & lemma\\
\hline
adj:sg:acc.inst:f:com & adj:grad & $\varepsilon$ & ą & y\\
adj:pl:inst:m1.m2.m3.f.n:com & adj:grad & $\varepsilon$ & ymi & y\\
adj:sg:inst.loc:m1.m2.m3.n:com adj:pl:dat:m1.m2.m3.f.n:com & adj:grad & $\varepsilon$ & ym & y\\
adj:pl:gen.loc:m1.m2.m3.f.n:com adj:pl:acc:m1:com & adj:grad & $\varepsilon$ & ych & y\\
adj:sg:nom.voc:m1.m2.m3:com adj:sg:acc:m3:com & adj:grad & $\varepsilon$ & y & y\\
adj:pl:nom.voc:m1:com & adj:grad & $\varepsilon$ & i & y\\
adj:sg:dat:m1.m2.m3.n:com & adj:grad & $\varepsilon$ & emu & y\\
adj:sg:gen.dat.loc:f:com & adj:grad & $\varepsilon$ & ej & y\\
adj:sg:gen:m1.m2.m3.n:com adj:sg:acc:m1.m2:com & adj:grad & $\varepsilon$ & ego & y\\
adj:sg:nom.acc.voc:n:com adj:pl:nom.acc.voc:m2.m3.f.n:com & adj:grad & $\varepsilon$ & e & y\\
adj:sg:nom.voc:f:com & adj:grad & $\varepsilon$ & a & y\\
adj:sg:acc.inst:f:sup & adj:grad & naj & ą & y\\
adj:pl:inst:m1.m2.m3.f.n:sup & adj:grad & naj & ymi & y\\
adj:sg:inst.loc:m1.m2.m3.n:sup adj:pl:dat:m1.m2.m3.f.n:sup & adj:grad & naj & ym & y\\
adj:pl:gen.loc:m1.m2.m3.f.n:sup adj:pl:acc:m1:sup & adj:grad & naj & ych & y\\
adj:sg:nom.voc:m1.m2.m3:sup adj:sg:acc:m3:sup & adj:grad & naj & y & y\\
adj:pl:nom.voc:m1:sup & adj:grad & naj & i & y\\
adj:sg:dat:m1.m2.m3.n:sup & adj:grad & naj & emu & y\\
adj:sg:gen.dat.loc:f:sup & adj:grad & naj & ej & y\\
adj:sg:gen:m1.m2.m3.n:sup adj:sg:acc:m1.m2:sup & adj:grad & naj & ego & y\\
adj:sg:nom.acc.voc:n:sup adj:pl:nom.acc.voc:m2.m3.f.n:sup & adj:grad & naj & e & y\\
adj:sg:nom.voc:f:sup & adj:grad & naj & a & y\\
\end{longtable}

\begin{longtable}{p{7cm}|l|l|l}
interpretacja & cat & flex & lemma\\
\hline
adv:pos & adv & o & o\\
adv:com & adv & iej & o\\
adv:com & adv & iej & ie\\
adv:pos & adv & ie & ie\\
adv:sup & adv & iej & o\\
adv:sup & adv & iej & ie\\
\end{longtable}
\begin{longtable}{p{7cm}|l}
interpretacja & cat\\
\hline
subst:sg.pl:nom.gen.dat.acc.inst.loc.voc:n:ncol & ndm\\
subst:sg.pl:nom.gen.dat.acc.inst.loc.voc:m3 & ndm\\
subst:sg.pl:nom.gen.dat.acc.inst.loc.voc:m2 & ndm\\
$\star$subst:sg.pl:nom.gen.dat.acc.inst.loc.voc:m1 & ndm\\
subst:sg.pl:nom.gen.dat.acc.inst.loc.voc:m1 & ndm\\
	\quad \quad |depr:pl:nom.acc.voc:m2 & \\
subst:sg.pl:nom.gen.dat.acc.inst.loc.voc:f & ndm\\
$\star$subst:pl:nom.gen.dat.acc.inst.loc.voc:n:pt & ndm\\
$\star$adj:sg.pl:nom.gen.dat.acc.inst.loc.voc & ndm\\
	\quad \quad :m1.m2.m3.f.n:pos adja & \\
$\star$adj:sg.pl:nom.gen.dat.acc.inst.loc.voc & ndm\\
	\quad \quad :m1.m2.m3.f.n:pos & \\
\end{longtable}

\begin{longtable}{p{7cm}|l|l|l|l}
interpretacja & cat & pref & flex & flex2\\
\hline
$\star$fin:sg:pri & verb & $\varepsilon$ & ę & $\varepsilon$\\
$\star$fin:sg:sec & verb & $\varepsilon$ & sz & $\varepsilon$\\
$\star$fin:sg:ter & verb & $\varepsilon$ & $\varepsilon$ & $\varepsilon$\\
$\star$fin:pl:pri & verb & $\varepsilon$ & my & $\varepsilon$\\
$\star$fin:pl:sec & verb & $\varepsilon$ & cie & $\varepsilon$\\
$\star$fin:pl:ter & verb & $\varepsilon$ & ą & $\varepsilon$\\
$\star$fin:sg:pri & verb & $\varepsilon$ & ę & n\\
$\star$fin:pl:ter & verb & $\varepsilon$ & ą & n\\
$\star$fin:sg:pri & verb & $\varepsilon$ & m & $\varepsilon$\\
$\star$impt:sg:sec & verb & $\varepsilon$ & $\varepsilon$ & $\varepsilon$\\
$\star$impt:pl:pri & verb & $\varepsilon$ & my & $\varepsilon$\\
$\star$impt:pl:sec & verb & $\varepsilon$ & cie & $\varepsilon$\\
$\star$pcon:imperf & verb & $\varepsilon$ & $\varepsilon$ & ąc\\
$\star$pacta & verb & $\varepsilon$ & o & ąc\\
$\star$pact:sg:nom.acc.voc:n:imperf:aff pact:pl:nom.acc.voc:m2.m3.f.n:imperf:aff & verb & $\varepsilon$ & e & ąc\\
$\star$pact:sg:nom.acc.voc:n:imperf:neg pact:pl:nom.acc.voc:m2.m3.f.n:imperf:neg & verb & nie & e & ąc\\
$\star$pact:sg:nom.voc:m1.m2.m3:imperf:aff pact:sg:acc:m3:imperf:aff pact:pl:nom.voc:m1:imperf:aff & verb & $\varepsilon$ & y & ąc\\
$\star$pact:sg:nom.voc:f:imperf:aff & verb & $\varepsilon$ & a & ąc\\
$\star$pact:sg:nom.voc:m1.m2.m3:imperf:neg pact:sg:acc:m3:imperf:neg pact:pl:nom.voc:m1:imperf:neg & verb & nie & y & ąc\\
$\star$pact:sg:nom.voc:f:imperf:neg & verb & nie & a & ąc\\
$\star$pact:sg:gen.dat.loc:f:imperf:aff & verb & $\varepsilon$ & ej & ąc\\
$\star$pact:sg:gen.dat.loc:f:imperf:neg & verb & nie & ej & ąc\\
$\star$pact:sg:gen:m1.m2.m3.n:imperf:aff pact:sg:acc:m1.m2:imperf:aff & verb & $\varepsilon$ & ego & ąc\\
$\star$pact:sg:gen:m1.m2.m3.n:imperf:neg pact:sg:acc:m1.m2:imperf:neg & verb & nie & ego & ąc\\
$\star$pact:sg:dat:m1.m2.m3.n:imperf:aff & verb & $\varepsilon$ & emu & ąc\\
$\star$pact:sg:dat:m1.m2.m3.n:imperf:neg & verb & nie & emu & ąc\\
$\star$pact:sg:acc.inst:f:imperf:aff & verb & $\varepsilon$ & ą & ąc\\
$\star$pact:sg:acc.inst:f:imperf:neg & verb & nie & ą & ąc\\
$\star$pact:sg:inst.loc:m1.m2.m3.n:imperf:aff pact:pl:dat:m1.m2.m3.f.n:imperf:aff & verb & $\varepsilon$ & ym & ąc\\
$\star$pact:sg:inst.loc:m1.m2.m3.n:imperf:neg pact:pl:dat:m1.m2.m3.f.n:imperf:neg & verb & nie & ym & ąc\\
$\star$pact:pl:gen.loc:m1.m2.m3.f.n:imperf:aff pact:pl:acc:m1:imperf:aff & verb & $\varepsilon$ & ych & ąc\\
$\star$pact:pl:gen.loc:m1.m2.m3.f.n:imperf:neg pact:pl:acc:m1:imperf:neg & verb & nie & ych & ąc\\
$\star$pact:pl:inst:m1.m2.m3.f.n:imperf:aff & verb & $\varepsilon$ & ymi & ąc\\
$\star$pact:pl:inst:m1.m2.m3.f.n:imperf:neg & verb & nie & ymi & ąc\\
$\star$inf & verb & $\varepsilon$ & $\varepsilon$ & ć\\
$\star$pant & verb & $\varepsilon$ & szy & w\\
$\star$pant & verb & $\varepsilon$ & szy & ł\\
$\star$imps & verb & $\varepsilon$ & o & t\\
$\star$imps & verb & $\varepsilon$ & o & n\\
$\star$praet:sg:m1.m2.m3:nagl & verb & $\varepsilon$ & $\varepsilon$ & ł\\
$\star$praet:sg:m1.m2.m3:agl & verb & $\varepsilon$ & $\varepsilon$ & ł\\
$\star$praet:sg:m1.m2.m3 & verb & $\varepsilon$ & $\varepsilon$ & ł\\
$\star$praet:pl:m2.m3.f.n & verb & $\varepsilon$ & y & ł\\
$\star$praet:sg:n & verb & $\varepsilon$ & o & ł\\
$\star$praet:pl:m1 & verb & $\varepsilon$ & i & ł\\
$\star$praet:sg:f & verb & $\varepsilon$ & a & ł\\
$\star$ger:pl:gen:n:aff & verb & $\varepsilon$ & $\varepsilon$ & ć\\
$\star$ger:sg:dat.loc:n:aff & verb & $\varepsilon$ & u & ć\\
$\star$ger:pl:dat:n:aff & verb & $\varepsilon$ & om & ć\\
$\star$ger:sg:inst:n:aff & verb & $\varepsilon$ & em & ć\\
$\star$ger:sg:nom.acc:n:aff & verb & $\varepsilon$ & e & ć\\
$\star$ger:pl:inst:n:aff & verb & $\varepsilon$ & ami & ć\\
$\star$ger:pl:loc:n:aff & verb & $\varepsilon$ & ach & ć\\
$\star$ger:sg:gen:n:aff ger:pl:nom.acc:n:aff & verb & $\varepsilon$ & a & ć\\
$\star$ger:pl:gen:n:neg & verb & nie & $\varepsilon$ & ć\\
$\star$ger:sg:dat.loc:n:neg & verb & nie & u & ć\\
$\star$ger:pl:dat:n:neg & verb & nie & om & ć\\
$\star$ger:sg:inst:n:neg & verb & nie & em & ć\\
$\star$ger:sg:nom.acc:n:neg & verb & nie & e & ć\\
$\star$ger:pl:inst:n:neg & verb & nie & ami & ć\\
$\star$ger:pl:loc:n:neg & verb & nie & ach & ć\\
$\star$ger:sg:gen:n:neg ger:pl:nom.acc:n:neg & verb & nie & a & ć\\
$\star$ger:pl:gen:n:aff & verb & $\varepsilon$ & $\varepsilon$ & ń\\
$\star$ger:sg:dat.loc:n:aff & verb & $\varepsilon$ & u & ń\\
$\star$ger:pl:dat:n:aff & verb & $\varepsilon$ & om & ń\\
$\star$ger:sg:inst:n:aff & verb & $\varepsilon$ & em & ń\\
$\star$ger:sg:nom.acc:n:aff & verb & $\varepsilon$ & e & ń\\
$\star$ger:pl:inst:n:aff & verb & $\varepsilon$ & ami & ń\\
$\star$ger:pl:loc:n:aff & verb & $\varepsilon$ & ach & ń\\
$\star$ger:sg:gen:n:aff ger:pl:nom.acc:n:aff & verb & $\varepsilon$ & a & ń\\
$\star$ger:pl:gen:n:neg & verb & nie & $\varepsilon$ & ń\\
$\star$ger:sg:dat.loc:n:neg & verb & nie & u & ń\\
$\star$ger:pl:dat:n:neg & verb & nie & om & ń\\
$\star$ger:sg:inst:n:neg & verb & nie & em & ń\\
$\star$ger:sg:nom.acc:n:neg & verb & nie & e & ń\\
$\star$ger:pl:inst:n:neg & verb & nie & ami & ń\\
$\star$ger:pl:loc:n:neg & verb & nie & ach & ń\\
$\star$ger:sg:gen:n:neg ger:pl:nom.acc:n:neg & verb & nie & a & ń\\
$\star$ppas:sg:nom.acc.voc:n:aff ppas:pl:nom.acc.voc:m2.m3.f.n:aff & verb & $\varepsilon$ & e & t\\
$\star$ppas:sg:nom.acc.voc:n:neg ppas:pl:nom.acc.voc:m2.m3.f.n:neg & verb & nie & e & t\\
$\star$ppas:sg:nom.voc:m1.m2.m3:aff ppas:sg:acc:m3:aff & verb & $\varepsilon$ & y & t\\
$\star$ppas:sg:nom.voc:f:aff & verb & $\varepsilon$ & a & t\\
$\star$ppas:sg:nom.voc:m1.m2.m3:neg ppas:sg:acc:m3:neg & verb & nie & y & t\\
$\star$ppas:sg:nom.voc:f:neg & verb & nie & a & t\\
$\star$ppas:sg:gen.dat.loc:f:aff & verb & $\varepsilon$ & ej & t\\
$\star$ppas:sg:gen.dat.loc:f:neg & verb & nie & ej & t\\
$\star$ppas:sg:gen:m1.m2.m3.n:aff ppas:sg:acc:m1.m2:aff & verb & $\varepsilon$ & ego & t\\
$\star$ppas:sg:gen:m1.m2.m3.n:neg ppas:sg:acc:m1.m2:neg & verb & nie & ego & t\\
$\star$ppas:sg:dat:m1.m2.m3.n:aff & verb & $\varepsilon$ & emu & t\\
$\star$ppas:sg:dat:m1.m2.m3.n:neg & verb & nie & emu & t\\
$\star$ppas:sg:acc.inst:f:aff & verb & $\varepsilon$ & ą & t\\
$\star$ppas:sg:acc.inst:f:neg & verb & nie & ą & t\\
$\star$ppas:sg:inst.loc:m1.m2.m3.n:aff ppas:pl:dat:m1.m2.m3.f.n:aff & verb & $\varepsilon$ & ym & t\\
$\star$ppas:sg:inst.loc:m1.m2.m3.n:neg ppas:pl:dat:m1.m2.m3.f.n:neg & verb & nie & ym & t\\
$\star$ppas:pl:nom.voc:m1:aff & verb & $\varepsilon$ & i & t\\
$\star$ppas:pl:nom.voc:m1:neg & verb & nie & i & t\\
$\star$ppas:pl:gen.loc:m1.m2.m3.f.n:aff ppas:pl:acc:m1:aff & verb & $\varepsilon$ & ych & t\\
$\star$ppas:pl:gen.loc:m1.m2.m3.f.n:neg ppas:pl:acc:m1:neg & verb & nie & ych & t\\
$\star$ppas:pl:inst:m1.m2.m3.f.n:aff & verb & $\varepsilon$ & ymi & t\\
$\star$ppas:pl:inst:m1.m2.m3.f.n:neg & verb & nie & ymi & t\\
$\star$ppas:sg:nom.acc.voc:n:aff ppas:pl:nom.acc.voc:m2.m3.f.n:aff & verb & $\varepsilon$ & e & n\\
$\star$ppas:sg:nom.acc.voc:n:neg ppas:pl:nom.acc.voc:m2.m3.f.n:neg & verb & nie & e & n\\
$\star$ppas:sg:nom.voc:m1.m2.m3:aff ppas:sg:acc:m3:aff & verb & $\varepsilon$ & y & n\\
$\star$ppas:sg:nom.voc:f:aff & verb & $\varepsilon$ & a & n\\
$\star$ppas:sg:nom.voc:m1.m2.m3:neg ppas:sg:acc:m3:neg & verb & nie & y & n\\
$\star$ppas:sg:nom.voc:f:neg & verb & nie & a & n\\
$\star$ppas:sg:gen.dat.loc:f:aff & verb & $\varepsilon$ & ej & n\\
$\star$ppas:sg:gen.dat.loc:f:neg & verb & nie & ej & n\\
$\star$ppas:sg:gen:m1.m2.m3.n:aff ppas:sg:acc:m1.m2:aff & verb & $\varepsilon$ & ego & n\\
$\star$ppas:sg:gen:m1.m2.m3.n:neg ppas:sg:acc:m1.m2:neg & verb & nie & ego & n\\
$\star$ppas:sg:dat:m1.m2.m3.n:aff & verb & $\varepsilon$ & emu & n\\
$\star$ppas:sg:dat:m1.m2.m3.n:neg & verb & nie & emu & n\\
$\star$ppas:sg:acc.inst:f:aff & verb & $\varepsilon$ & ą & n\\
$\star$ppas:sg:acc.inst:f:neg & verb & nie & ą & n\\
$\star$ppas:sg:inst.loc:m1.m2.m3.n:aff ppas:pl:dat:m1.m2.m3.f.n:aff & verb & $\varepsilon$ & ym & n\\
$\star$ppas:sg:inst.loc:m1.m2.m3.n:neg ppas:pl:dat:m1.m2.m3.f.n:neg & verb & nie & ym & n\\
$\star$ppas:pl:nom.voc:m1:aff & verb & $\varepsilon$ & i & n\\
$\star$ppas:pl:nom.voc:m1:neg & verb & nie & i & n\\
$\star$ppas:pl:gen.loc:m1.m2.m3.f.n:aff ppas:pl:acc:m1:aff & verb & $\varepsilon$ & ych & n\\
$\star$ppas:pl:gen.loc:m1.m2.m3.f.n:neg ppas:pl:acc:m1:neg & verb & nie & ych & n\\
$\star$ppas:pl:inst:m1.m2.m3.f.n:aff & verb & $\varepsilon$ & ymi & n\\
$\star$ppas:pl:inst:m1.m2.m3.f.n:neg & verb & nie & ymi & n\\
\end{longtable}
\begin{longtable}{p{7cm}|l|l|l|l|l|l}
interpretacja & cat & pref & group & flex & flex2 & lemma\\
\hline
fin:sg:pri & verb & $\varepsilon$ & a & m & $\varepsilon$ & ać\\
fin:sg:sec & verb & $\varepsilon$ & a & sz & $\varepsilon$ & ać\\
fin:sg:ter & verb & $\varepsilon$ & a & $\varepsilon$ & $\varepsilon$ & ać\\
fin:pl:pri & verb & $\varepsilon$ & a & my & $\varepsilon$ & ać\\
fin:pl:sec & verb & $\varepsilon$ & a & cie & $\varepsilon$ & ać\\
fin:pl:ter & verb & $\varepsilon$ & aj & ą & $\varepsilon$ & ać\\
impt:sg:sec & verb & $\varepsilon$ & aj & $\varepsilon$ & $\varepsilon$ & ać\\
impt:pl:pri & verb & $\varepsilon$ & aj & my & $\varepsilon$ & ać\\
impt:pl:sec & verb & $\varepsilon$ & aj & cie & $\varepsilon$ & ać\\
pcon:imperf & verb & $\varepsilon$ & aj & $\varepsilon$ & ąc & ać\\
pacta & verb & $\varepsilon$ & aj & o & ąc & ać\\
pact:sg:nom.acc.voc:n:imperf:aff pact:pl:nom.acc.voc:m2.m3.f.n:imperf:aff & verb & $\varepsilon$ & aj & e & ąc & ać\\
pact:sg:nom.acc.voc:n:imperf:neg pact:pl:nom.acc.voc:m2.m3.f.n:imperf:neg & verb & nie & aj & e & ąc & ać\\
pact:sg:nom.voc:m1.m2.m3:imperf:aff pact:sg:acc:m3:imperf:aff pact:pl:nom.voc:m1:imperf:aff & verb & $\varepsilon$ & aj & y & ąc & ać\\
pact:sg:nom.voc:f:imperf:aff & verb & $\varepsilon$ & aj & a & ąc & ać\\
pact:sg:nom.voc:m1.m2.m3:imperf:neg pact:sg:acc:m3:imperf:neg pact:pl:nom.voc:m1:imperf:neg & verb & nie & aj & y & ąc & ać\\
pact:sg:nom.voc:f:imperf:neg & verb & nie & aj & a & ąc & ać\\
pact:sg:gen.dat.loc:f:imperf:aff & verb & $\varepsilon$ & aj & ej & ąc & ać\\
pact:sg:gen.dat.loc:f:imperf:neg & verb & nie & aj & ej & ąc & ać\\
pact:sg:gen:m1.m2.m3.n:imperf:aff pact:sg:acc:m1.m2:imperf:aff & verb & $\varepsilon$ & aj & ego & ąc & ać\\
pact:sg:gen:m1.m2.m3.n:imperf:neg pact:sg:acc:m1.m2:imperf:neg & verb & nie & aj & ego & ąc & ać\\
pact:sg:dat:m1.m2.m3.n:imperf:aff & verb & $\varepsilon$ & aj & emu & ąc & ać\\
pact:sg:dat:m1.m2.m3.n:imperf:neg & verb & nie & aj & emu & ąc & ać\\
pact:sg:acc.inst:f:imperf:aff & verb & $\varepsilon$ & aj & ą & ąc & ać\\
pact:sg:acc.inst:f:imperf:neg & verb & nie & aj & ą & ąc & ać\\
pact:sg:inst.loc:m1.m2.m3.n:imperf:aff pact:pl:dat:m1.m2.m3.f.n:imperf:aff & verb & $\varepsilon$ & aj & ym & ąc & ać\\
pact:sg:inst.loc:m1.m2.m3.n:imperf:neg pact:pl:dat:m1.m2.m3.f.n:imperf:neg & verb & nie & aj & ym & ąc & ać\\
pact:pl:gen.loc:m1.m2.m3.f.n:imperf:aff pact:pl:acc:m1:imperf:aff & verb & $\varepsilon$ & aj & ych & ąc & ać\\
pact:pl:gen.loc:m1.m2.m3.f.n:imperf:neg pact:pl:acc:m1:imperf:neg & verb & nie & aj & ych & ąc & ać\\
pact:pl:inst:m1.m2.m3.f.n:imperf:aff & verb & $\varepsilon$ & aj & ymi & ąc & ać\\
pact:pl:inst:m1.m2.m3.f.n:imperf:neg & verb & nie & aj & ymi & ąc & ać\\
inf & verb & $\varepsilon$ & a & $\varepsilon$ & ć & ać\\
pant & verb & $\varepsilon$ & a & szy & w & ać\\
imps & verb & $\varepsilon$ & a & o & n & ać\\
praet:sg:m1.m2.m3 & verb & $\varepsilon$ & a & $\varepsilon$ & ł & ać\\
praet:pl:m2.m3.f.n & verb & $\varepsilon$ & a & y & ł & ać\\
praet:sg:n & verb & $\varepsilon$ & a & o & ł & ać\\
praet:pl:m1 & verb & $\varepsilon$ & a & i & ł & ać\\
praet:sg:f & verb & $\varepsilon$ & a & a & ł & ać\\
ppas:sg:nom.acc.voc:n:aff ppas:pl:nom.acc.voc:m2.m3.f.n:aff & verb & $\varepsilon$ & a & e & n & ać\\
ppas:sg:nom.acc.voc:n:neg ppas:pl:nom.acc.voc:m2.m3.f.n:neg & verb & nie & a & e & n & ać\\
ppas:sg:nom.voc:m1.m2.m3:aff ppas:sg:acc:m3:aff & verb & $\varepsilon$ & a & y & n & ać\\
ppas:sg:nom.voc:f:aff & verb & $\varepsilon$ & a & a & n & ać\\
ppas:sg:nom.voc:m1.m2.m3:neg ppas:sg:acc:m3:neg & verb & nie & a & y & n & ać\\
ppas:sg:nom.voc:f:neg & verb & nie & a & a & n & ać\\
ppas:sg:gen.dat.loc:f:aff & verb & $\varepsilon$ & a & ej & n & ać\\
ppas:sg:gen.dat.loc:f:neg & verb & nie & a & ej & n & ać\\
ppas:sg:gen:m1.m2.m3.n:aff ppas:sg:acc:m1.m2:aff & verb & $\varepsilon$ & a & ego & n & ać\\
ppas:sg:gen:m1.m2.m3.n:neg ppas:sg:acc:m1.m2:neg & verb & nie & a & ego & n & ać\\
ppas:sg:dat:m1.m2.m3.n:aff & verb & $\varepsilon$ & a & emu & n & ać\\
ppas:sg:dat:m1.m2.m3.n:neg & verb & nie & a & emu & n & ać\\
ppas:sg:acc.inst:f:aff & verb & $\varepsilon$ & a & ą & n & ać\\
ppas:sg:acc.inst:f:neg & verb & nie & a & ą & n & ać\\
ppas:sg:inst.loc:m1.m2.m3.n:aff ppas:pl:dat:m1.m2.m3.f.n:aff & verb & $\varepsilon$ & a & ym & n & ać\\
ppas:sg:inst.loc:m1.m2.m3.n:neg ppas:pl:dat:m1.m2.m3.f.n:neg & verb & nie & a & ym & n & ać\\
ppas:pl:nom.voc:m1:aff & verb & $\varepsilon$ & a & i & n & ać\\
ppas:pl:nom.voc:m1:neg & verb & nie & a & i & n & ać\\
ppas:pl:gen.loc:m1.m2.m3.f.n:aff ppas:pl:acc:m1:aff & verb & $\varepsilon$ & a & ych & n & ać\\
ppas:pl:gen.loc:m1.m2.m3.f.n:neg ppas:pl:acc:m1:neg & verb & nie & a & ych & n & ać\\
ppas:pl:inst:m1.m2.m3.f.n:aff & verb & $\varepsilon$ & a & ymi & n & ać\\
ppas:pl:inst:m1.m2.m3.f.n:neg & verb & nie & a & ymi & n & ać\\
ger:pl:gen:n:aff & verb & $\varepsilon$ & a & $\varepsilon$ & ń & ać\\
ger:sg:dat.loc:n:aff & verb & $\varepsilon$ & a & u & ń & ać\\
ger:pl:dat:n:aff & verb & $\varepsilon$ & a & om & ń & ać\\
ger:sg:inst:n:aff & verb & $\varepsilon$ & a & em & ń & ać\\
ger:sg:nom.acc:n:aff & verb & $\varepsilon$ & a & e & ń & ać\\
ger:pl:inst:n:aff & verb & $\varepsilon$ & a & ami & ń & ać\\
ger:pl:loc:n:aff & verb & $\varepsilon$ & a & ach & ń & ać\\
ger:sg:gen:n:aff ger:pl:nom.acc:n:aff & verb & $\varepsilon$ & a & a & ń & ać\\
ger:pl:gen:n:neg & verb & nie & a & $\varepsilon$ & ń & ać\\
ger:sg:dat.loc:n:neg & verb & nie & a & u & ń & ać\\
ger:pl:dat:n:neg & verb & nie & a & om & ń & ać\\
ger:sg:inst:n:neg & verb & nie & a & em & ń & ać\\
ger:sg:nom.acc:n:neg & verb & nie & a & e & ń & ać\\
ger:pl:inst:n:neg & verb & nie & a & ami & ń & ać\\
ger:pl:loc:n:neg & verb & nie & a & ach & ń & ać\\
ger:sg:gen:n:neg ger:pl:nom.acc:n:neg & verb & nie & a & a & ń & ać\\
\end{longtable}
\begin{longtable}{p{7cm}|l|l|l|l|l|l}
interpretacja & cat & pref & group & flex & flex2 & lemma\\
\hline
fin:sg:sec & verb & $\varepsilon$ & nie & sz & $\varepsilon$ & nąć\\
fin:sg:ter & verb & $\varepsilon$ & nie & $\varepsilon$ & $\varepsilon$ & nąć\\
fin:pl:pri & verb & $\varepsilon$ & nie & my & $\varepsilon$ & nąć\\
fin:pl:sec & verb & $\varepsilon$ & nie & cie & $\varepsilon$ & nąć\\
fin:sg:pri & verb & $\varepsilon$ & n & ę & $\varepsilon$ & nąć\\
fin:pl:ter & verb & $\varepsilon$ & n & ą & $\varepsilon$ & nąć\\
impt:sg:sec & verb & $\varepsilon$ & nij & $\varepsilon$ & $\varepsilon$ & nąć\\
impt:pl:pri & verb & $\varepsilon$ & nij & my & $\varepsilon$ & nąć\\
impt:pl:sec & verb & $\varepsilon$ & nij & cie & $\varepsilon$ & nąć\\
pcon:imperf & verb & $\varepsilon$ & n & $\varepsilon$ & ąc & nąć\\
pacta & verb & $\varepsilon$ & n & o & ąc & nąć\\
pact:sg:nom.acc.voc:n:imperf:aff pact:pl:nom.acc.voc:m2.m3.f.n:imperf:aff & verb & $\varepsilon$ & n & e & ąc & nąć\\
pact:sg:nom.acc.voc:n:imperf:neg pact:pl:nom.acc.voc:m2.m3.f.n:imperf:neg & verb & nie & n & e & ąc & nąć\\
pact:sg:nom.voc:m1.m2.m3:imperf:aff pact:sg:acc:m3:imperf:aff pact:pl:nom.voc:m1:imperf:aff & verb & $\varepsilon$ & n & y & ąc & nąć\\
pact:sg:nom.voc:f:imperf:aff & verb & $\varepsilon$ & n & a & ąc & nąć\\
pact:sg:nom.voc:m1.m2.m3:imperf:neg pact:sg:acc:m3:imperf:neg pact:pl:nom.voc:m1:imperf:neg & verb & nie & n & y & ąc & nąć\\
pact:sg:nom.voc:f:imperf:neg & verb & nie & n & a & ąc & nąć\\
pact:sg:gen.dat.loc:f:imperf:aff & verb & $\varepsilon$ & n & ej & ąc & nąć\\
pact:sg:gen.dat.loc:f:imperf:neg & verb & nie & n & ej & ąc & nąć\\
pact:sg:gen:m1.m2.m3.n:imperf:aff pact:sg:acc:m1.m2:imperf:aff & verb & $\varepsilon$ & n & ego & ąc & nąć\\
pact:sg:gen:m1.m2.m3.n:imperf:neg pact:sg:acc:m1.m2:imperf:neg & verb & nie & n & ego & ąc & nąć\\
pact:sg:dat:m1.m2.m3.n:imperf:aff & verb & $\varepsilon$ & n & emu & ąc & nąć\\
pact:sg:dat:m1.m2.m3.n:imperf:neg & verb & nie & n & emu & ąc & nąć\\
pact:sg:acc.inst:f:imperf:aff & verb & $\varepsilon$ & n & ą & ąc & nąć\\
pact:sg:acc.inst:f:imperf:neg & verb & nie & n & ą & ąc & nąć\\
pact:sg:inst.loc:m1.m2.m3.n:imperf:aff pact:pl:dat:m1.m2.m3.f.n:imperf:aff & verb & $\varepsilon$ & n & ym & ąc & nąć\\
pact:sg:inst.loc:m1.m2.m3.n:imperf:neg pact:pl:dat:m1.m2.m3.f.n:imperf:neg & verb & nie & n & ym & ąc & nąć\\
pact:pl:gen.loc:m1.m2.m3.f.n:imperf:aff pact:pl:acc:m1:imperf:aff & verb & $\varepsilon$ & n & ych & ąc & nąć\\
pact:pl:gen.loc:m1.m2.m3.f.n:imperf:neg pact:pl:acc:m1:imperf:neg & verb & nie & n & ych & ąc & nąć\\
pact:pl:inst:m1.m2.m3.f.n:imperf:aff & verb & $\varepsilon$ & n & ymi & ąc & nąć\\
pact:pl:inst:m1.m2.m3.f.n:imperf:neg & verb & nie & n & ymi & ąc & nąć\\
inf & verb & $\varepsilon$ & ną & $\varepsilon$ & ć & nąć\\
pant & verb & $\varepsilon$ & ną & szy & w & nąć\\
imps & verb & $\varepsilon$ & nię & o & t & nąć\\
praet:sg:m1.m2.m3 & verb & $\varepsilon$ & ną & $\varepsilon$ & ł & nąć\\
praet:pl:m2.m3.f.n & verb & $\varepsilon$ & nę & y & ł & nąć\\
praet:sg:n & verb & $\varepsilon$ & nę & o & ł & nąć\\
praet:pl:m1 & verb & $\varepsilon$ & nę & i & ł & nąć\\
praet:sg:f & verb & $\varepsilon$ & nę & a & ł & nąć\\
ppas:sg:nom.acc.voc:n:aff ppas:pl:nom.acc.voc:m2.m3.f.n:aff & verb & $\varepsilon$ & nię & e & t & nąć\\
ppas:sg:nom.acc.voc:n:neg ppas:pl:nom.acc.voc:m2.m3.f.n:neg & verb & nie & nię & e & t & nąć\\
ppas:sg:nom.voc:m1.m2.m3:aff ppas:sg:acc:m3:aff & verb & $\varepsilon$ & nię & y & t & nąć\\
ppas:sg:nom.voc:f:aff & verb & $\varepsilon$ & nię & a & t & nąć\\
ppas:sg:nom.voc:m1.m2.m3:neg ppas:sg:acc:m3:neg & verb & nie & nię & y & t & nąć\\
ppas:sg:nom.voc:f:neg & verb & nie & nię & a & t & nąć\\
ppas:sg:gen.dat.loc:f:aff & verb & $\varepsilon$ & nię & ej & t & nąć\\
ppas:sg:gen.dat.loc:f:neg & verb & nie & nię & ej & t & nąć\\
ppas:sg:gen:m1.m2.m3.n:aff ppas:sg:acc:m1.m2:aff & verb & $\varepsilon$ & nię & ego & t & nąć\\
ppas:sg:gen:m1.m2.m3.n:neg ppas:sg:acc:m1.m2:neg & verb & nie & nię & ego & t & nąć\\
ppas:sg:dat:m1.m2.m3.n:aff & verb & $\varepsilon$ & nię & emu & t & nąć\\
ppas:sg:dat:m1.m2.m3.n:neg & verb & nie & nię & emu & t & nąć\\
ppas:sg:acc.inst:f:aff & verb & $\varepsilon$ & nię & ą & t & nąć\\
ppas:sg:acc.inst:f:neg & verb & nie & nię & ą & t & nąć\\
ppas:sg:inst.loc:m1.m2.m3.n:aff ppas:pl:dat:m1.m2.m3.f.n:aff & verb & $\varepsilon$ & nię & ym & t & nąć\\
ppas:sg:inst.loc:m1.m2.m3.n:neg ppas:pl:dat:m1.m2.m3.f.n:neg & verb & nie & nię & ym & t & nąć\\
ppas:pl:nom.voc:m1:aff & verb & $\varepsilon$ & nię & i & t & nąć\\
ppas:pl:nom.voc:m1:neg & verb & nie & nię & i & t & nąć\\
ppas:pl:gen.loc:m1.m2.m3.f.n:aff ppas:pl:acc:m1:aff & verb & $\varepsilon$ & nię & ych & t & nąć\\
ppas:pl:gen.loc:m1.m2.m3.f.n:neg ppas:pl:acc:m1:neg & verb & nie & nię & ych & t & nąć\\
ppas:pl:inst:m1.m2.m3.f.n:aff & verb & $\varepsilon$ & nię & ymi & t & nąć\\
ppas:pl:inst:m1.m2.m3.f.n:neg & verb & nie & nię & ymi & t & nąć\\
ger:pl:gen:n:aff & verb & $\varepsilon$ & nię & $\varepsilon$ & ć & nąć\\
ger:sg:dat.loc:n:aff & verb & $\varepsilon$ & nię & u & ć & nąć\\
ger:pl:dat:n:aff & verb & $\varepsilon$ & nię & om & ć & nąć\\
ger:sg:inst:n:aff & verb & $\varepsilon$ & nię & em & ć & nąć\\
ger:sg:nom.acc:n:aff & verb & $\varepsilon$ & nię & e & ć & nąć\\
ger:pl:inst:n:aff & verb & $\varepsilon$ & nię & ami & ć & nąć\\
ger:pl:loc:n:aff & verb & $\varepsilon$ & nię & ach & ć & nąć\\
ger:sg:gen:n:aff ger:pl:nom.acc:n:aff & verb & $\varepsilon$ & nię & a & ć & nąć\\
ger:pl:gen:n:neg & verb & nie & nię & $\varepsilon$ & ć & nąć\\
ger:sg:dat.loc:n:neg & verb & nie & nię & u & ć & nąć\\
ger:pl:dat:n:neg & verb & nie & nię & om & ć & nąć\\
ger:sg:inst:n:neg & verb & nie & nię & em & ć & nąć\\
ger:sg:nom.acc:n:neg & verb & nie & nię & e & ć & nąć\\
ger:pl:inst:n:neg & verb & nie & nię & ami & ć & nąć\\
ger:pl:loc:n:neg & verb & nie & nię & ach & ć & nąć\\
ger:sg:gen:n:neg ger:pl:nom.acc:n:neg & verb & nie & nię & a & ć & nąć\\
$\star$inf & verb & $\varepsilon$ & $\varepsilon$ & $\varepsilon$ & ć & nąć\\
$\star$pant & verb & $\varepsilon$ & $\varepsilon$ & szy & ł & nąć\\
$\star$praet:sg:m1.m2.m3:nagl & verb & $\varepsilon$ & $\varepsilon$ & $\varepsilon$ & ł & nąć\\
$\star$praet:sg:m1.m2.m3:agl & verb & $\varepsilon$ & $\varepsilon$ & $\varepsilon$ & ł & nąć\\
praet:sg:m1.m2.m3 & verb & $\varepsilon$ & $\varepsilon$ & $\varepsilon$ & ł & nąć\\
praet:pl:m2.m3.f.n & verb & $\varepsilon$ & $\varepsilon$ & y & ł & nąć\\
praet:sg:n & verb & $\varepsilon$ & $\varepsilon$ & o & ł & nąć\\
praet:pl:m1 & verb & $\varepsilon$ & $\varepsilon$ & i & ł & nąć\\
praet:sg:f & verb & $\varepsilon$ & $\varepsilon$ & a & ł & nąć\\
\end{longtable}
\begin{longtable}{p{7cm}|l|l|l|l|l|l}
interpretacja & cat & pref & group & flex & flex2 & lemma\\
\hline
fin:sg:sec & verb & $\varepsilon$ & uje & sz & $\varepsilon$ & ować\\
fin:sg:ter & verb & $\varepsilon$ & uje & $\varepsilon$ & $\varepsilon$ & ować\\
fin:pl:pri & verb & $\varepsilon$ & uje & my & $\varepsilon$ & ować\\
fin:pl:sec & verb & $\varepsilon$ & uje & cie & $\varepsilon$ & ować\\
fin:sg:pri & verb & $\varepsilon$ & uj & ę & $\varepsilon$ & ować\\
fin:pl:ter & verb & $\varepsilon$ & uj & ą & $\varepsilon$ & ować\\
impt:sg:sec & verb & $\varepsilon$ & uj & $\varepsilon$ & $\varepsilon$ & ować\\
impt:pl:pri & verb & $\varepsilon$ & uj & my & $\varepsilon$ & ować\\
impt:pl:sec & verb & $\varepsilon$ & uj & cie & $\varepsilon$ & ować\\
pcon:imperf & verb & $\varepsilon$ & uj & $\varepsilon$ & ąc & ować\\
pacta & verb & $\varepsilon$ & uj & o & ąc & ować\\
pact:sg:nom.acc.voc:n:imperf:aff pact:pl:nom.acc.voc:m2.m3.f.n:imperf:aff & verb & $\varepsilon$ & uj & e & ąc & ować\\
pact:sg:nom.acc.voc:n:imperf:neg pact:pl:nom.acc.voc:m2.m3.f.n:imperf:neg & verb & nie & uj & e & ąc & ować\\
pact:sg:nom.voc:m1.m2.m3:imperf:aff pact:sg:acc:m3:imperf:aff pact:pl:nom.voc:m1:imperf:aff & verb & $\varepsilon$ & uj & y & ąc & ować\\
pact:sg:nom.voc:f:imperf:aff & verb & $\varepsilon$ & uj & a & ąc & ować\\
pact:sg:nom.voc:m1.m2.m3:imperf:neg pact:sg:acc:m3:imperf:neg pact:pl:nom.voc:m1:imperf:neg & verb & nie & uj & y & ąc & ować\\
pact:sg:nom.voc:f:imperf:neg & verb & nie & uj & a & ąc & ować\\
pact:sg:gen.dat.loc:f:imperf:aff & verb & $\varepsilon$ & uj & ej & ąc & ować\\
pact:sg:gen.dat.loc:f:imperf:neg & verb & nie & uj & ej & ąc & ować\\
pact:sg:gen:m1.m2.m3.n:imperf:aff pact:sg:acc:m1.m2:imperf:aff & verb & $\varepsilon$ & uj & ego & ąc & ować\\
pact:sg:gen:m1.m2.m3.n:imperf:neg pact:sg:acc:m1.m2:imperf:neg & verb & nie & uj & ego & ąc & ować\\
pact:sg:dat:m1.m2.m3.n:imperf:aff & verb & $\varepsilon$ & uj & emu & ąc & ować\\
pact:sg:dat:m1.m2.m3.n:imperf:neg & verb & nie & uj & emu & ąc & ować\\
pact:sg:acc.inst:f:imperf:aff & verb & $\varepsilon$ & uj & ą & ąc & ować\\
pact:sg:acc.inst:f:imperf:neg & verb & nie & uj & ą & ąc & ować\\
pact:sg:inst.loc:m1.m2.m3.n:imperf:aff pact:pl:dat:m1.m2.m3.f.n:imperf:aff & verb & $\varepsilon$ & uj & ym & ąc & ować\\
pact:sg:inst.loc:m1.m2.m3.n:imperf:neg pact:pl:dat:m1.m2.m3.f.n:imperf:neg & verb & nie & uj & ym & ąc & ować\\
pact:pl:gen.loc:m1.m2.m3.f.n:imperf:aff pact:pl:acc:m1:imperf:aff & verb & $\varepsilon$ & uj & ych & ąc & ować\\
pact:pl:gen.loc:m1.m2.m3.f.n:imperf:neg pact:pl:acc:m1:imperf:neg & verb & nie & uj & ych & ąc & ować\\
pact:pl:inst:m1.m2.m3.f.n:imperf:aff & verb & $\varepsilon$ & uj & ymi & ąc & ować\\
pact:pl:inst:m1.m2.m3.f.n:imperf:neg & verb & nie & uj & ymi & ąc & ować\\
\end{longtable}
\begin{longtable}{p{7cm}|l|l|l|l|l|l}
interpretacja & cat & pref & group & flex & flex2 & lemma\\
\hline
fin:sg:sec & verb & $\varepsilon$ & eje & sz & $\varepsilon$ & palat-eć\\
fin:sg:ter & verb & $\varepsilon$ & eje & $\varepsilon$ & $\varepsilon$ & palat-eć\\
fin:pl:pri & verb & $\varepsilon$ & eje & my & $\varepsilon$ & palat-eć\\
fin:pl:sec & verb & $\varepsilon$ & eje & cie & $\varepsilon$ & palat-eć\\
$\star$fin:sg:sec & verb & $\varepsilon$ & y & sz & $\varepsilon$ & palat-eć\\
$\star$fin:sg:ter & verb & $\varepsilon$ & y & $\varepsilon$ & $\varepsilon$ & palat-eć\\
$\star$fin:pl:pri & verb & $\varepsilon$ & y & my & $\varepsilon$ & palat-eć\\
$\star$fin:pl:sec & verb & $\varepsilon$ & y & cie & $\varepsilon$ & palat-eć\\
fin:sg:pri & verb & $\varepsilon$ & J & ę & $\varepsilon$ & palat-eć\\
fin:pl:ter & verb & $\varepsilon$ & J & ą & $\varepsilon$ & palat-eć\\
impt:sg:sec & verb & $\varepsilon$ & ej & $\varepsilon$ & $\varepsilon$ & palat-eć\\
impt:pl:pri & verb & $\varepsilon$ & ej & my & $\varepsilon$ & palat-eć\\
impt:pl:sec & verb & $\varepsilon$ & ej & cie & $\varepsilon$ & palat-eć\\
$\star$impt:sg:sec & verb & $\varepsilon$ & yj & $\varepsilon$ & $\varepsilon$ & palat-eć\\
$\star$impt:pl:pri & verb & $\varepsilon$ & yj & my & $\varepsilon$ & palat-eć\\
$\star$impt:pl:sec & verb & $\varepsilon$ & yj & cie & $\varepsilon$ & palat-eć\\
$\star$impt:sg:sec & verb & $\varepsilon$ & $\varepsilon$ & $\varepsilon$ & $\varepsilon$ & palat-eć\\
$\star$impt:pl:pri & verb & $\varepsilon$ & $\varepsilon$ & my & $\varepsilon$ & palat-eć\\
$\star$impt:pl:sec & verb & $\varepsilon$ & $\varepsilon$ & cie & $\varepsilon$ & palat-eć\\
pcon:imperf & verb & $\varepsilon$ & J & $\varepsilon$ & ąc & palat-eć\\
pacta & verb & $\varepsilon$ & J & o & ąc & palat-eć\\
pact:sg:nom.acc.voc:n:imperf:aff pact:pl:nom.acc.voc:m2.m3.f.n:imperf:aff & verb & $\varepsilon$ & J & e & ąc & palat-eć\\
pact:sg:nom.acc.voc:n:imperf:neg pact:pl:nom.acc.voc:m2.m3.f.n:imperf:neg & verb & nie & J & e & ąc & palat-eć\\
pact:sg:nom.voc:m1.m2.m3:imperf:aff pact:sg:acc:m3:imperf:aff pact:pl:nom.voc:m1:imperf:aff & verb & $\varepsilon$ & J & y & ąc & palat-eć\\
pact:sg:nom.voc:f:imperf:aff & verb & $\varepsilon$ & J & a & ąc & palat-eć\\
pact:sg:nom.voc:m1.m2.m3:imperf:neg pact:sg:acc:m3:imperf:neg pact:pl:nom.voc:m1:imperf:neg & verb & nie & J & y & ąc & palat-eć\\
pact:sg:nom.voc:f:imperf:neg & verb & nie & J & a & ąc & palat-eć\\
pact:sg:gen.dat.loc:f:imperf:aff & verb & $\varepsilon$ & J & ej & ąc & palat-eć\\
pact:sg:gen.dat.loc:f:imperf:neg & verb & nie & J & ej & ąc & palat-eć\\
pact:sg:gen:m1.m2.m3.n:imperf:aff pact:sg:acc:m1.m2:imperf:aff & verb & $\varepsilon$ & J & ego & ąc & palat-eć\\
pact:sg:gen:m1.m2.m3.n:imperf:neg pact:sg:acc:m1.m2:imperf:neg & verb & nie & J & ego & ąc & palat-eć\\
pact:sg:dat:m1.m2.m3.n:imperf:aff & verb & $\varepsilon$ & J & emu & ąc & palat-eć\\
pact:sg:dat:m1.m2.m3.n:imperf:neg & verb & nie & J & emu & ąc & palat-eć\\
pact:sg:acc.inst:f:imperf:aff & verb & $\varepsilon$ & J & ą & ąc & palat-eć\\
pact:sg:acc.inst:f:imperf:neg & verb & nie & J & ą & ąc & palat-eć\\
pact:sg:inst.loc:m1.m2.m3.n:imperf:aff pact:pl:dat:m1.m2.m3.f.n:imperf:aff & verb & $\varepsilon$ & J & ym & ąc & palat-eć\\
pact:sg:inst.loc:m1.m2.m3.n:imperf:neg pact:pl:dat:m1.m2.m3.f.n:imperf:neg & verb & nie & J & ym & ąc & palat-eć\\
pact:pl:gen.loc:m1.m2.m3.f.n:imperf:aff pact:pl:acc:m1:imperf:aff & verb & $\varepsilon$ & J & ych & ąc & palat-eć\\
pact:pl:gen.loc:m1.m2.m3.f.n:imperf:neg pact:pl:acc:m1:imperf:neg & verb & nie & J & ych & ąc & palat-eć\\
pact:pl:inst:m1.m2.m3.f.n:imperf:aff & verb & $\varepsilon$ & J & ymi & ąc & palat-eć\\
pact:pl:inst:m1.m2.m3.f.n:imperf:neg & verb & nie & J & ymi & ąc & palat-eć\\
inf & verb & $\varepsilon$ & e & $\varepsilon$ & ć & palat-eć\\
pant & verb & $\varepsilon$ & a & szy & w & palat-eć\\
imps & verb & $\varepsilon$ & Ja & o & n & palat-eć\\
praet:sg:m1.m2.m3 & verb & $\varepsilon$ & a & $\varepsilon$ & ł & palat-eć\\
praet:pl:m2.m3.f.n & verb & $\varepsilon$ & a & y & ł & palat-eć\\
praet:sg:n & verb & $\varepsilon$ & a & o & ł & palat-eć\\
praet:pl:m1 & verb & $\varepsilon$ & a & i & ł & palat-eć\\
praet:sg:f & verb & $\varepsilon$ & a & a & ł & palat-eć\\
ger:pl:gen:n:aff & verb & $\varepsilon$ & Je & $\varepsilon$ & ń & palat-eć\\
ger:sg:dat.loc:n:aff & verb & $\varepsilon$ & Je & u & ń & palat-eć\\
ger:pl:dat:n:aff & verb & $\varepsilon$ & Je & om & ń & palat-eć\\
ger:sg:inst:n:aff & verb & $\varepsilon$ & Je & em & ń & palat-eć\\
ger:sg:nom.acc:n:aff & verb & $\varepsilon$ & Je & e & ń & palat-eć\\
ger:pl:inst:n:aff & verb & $\varepsilon$ & Je & ami & ń & palat-eć\\
ger:pl:loc:n:aff & verb & $\varepsilon$ & Je & ach & ń & palat-eć\\
ger:sg:gen:n:aff ger:pl:nom.acc:n:aff & verb & $\varepsilon$ & Je & a & ń & palat-eć\\
ger:pl:gen:n:neg & verb & nie & Je & $\varepsilon$ & ń & palat-eć\\
ger:sg:dat.loc:n:neg & verb & nie & Je & u & ń & palat-eć\\
ger:pl:dat:n:neg & verb & nie & Je & om & ń & palat-eć\\
ger:sg:inst:n:neg & verb & nie & Je & em & ń & palat-eć\\
ger:sg:nom.acc:n:neg & verb & nie & Je & e & ń & palat-eć\\
ger:pl:inst:n:neg & verb & nie & Je & ami & ń & palat-eć\\
ger:pl:loc:n:neg & verb & nie & Je & ach & ń & palat-eć\\
ger:sg:gen:n:neg ger:pl:nom.acc:n:neg & verb & nie & Je & a & ń & palat-eć\\
$\star$ppas:sg:nom.acc.voc:n:aff ppas:pl:nom.acc.voc:m2.m3.f.n:aff & verb & $\varepsilon$ & Ja & e & n & palat-eć\\
$\star$ppas:sg:nom.acc.voc:n:neg ppas:pl:nom.acc.voc:m2.m3.f.n:neg & verb & nie & Ja & e & n & palat-eć\\
$\star$ppas:sg:nom.voc:m1.m2.m3:aff ppas:sg:acc:m3:aff & verb & $\varepsilon$ & Ja & y & n & palat-eć\\
$\star$ppas:sg:nom.voc:f:aff & verb & $\varepsilon$ & Ja & a & n & palat-eć\\
$\star$ppas:sg:nom.voc:m1.m2.m3:neg ppas:sg:acc:m3:neg & verb & nie & Ja & y & n & palat-eć\\
$\star$ppas:sg:nom.voc:f:neg & verb & nie & Ja & a & n & palat-eć\\
$\star$ppas:sg:gen.dat.loc:f:aff & verb & $\varepsilon$ & Ja & ej & n & palat-eć\\
$\star$ppas:sg:gen.dat.loc:f:neg & verb & nie & Ja & ej & n & palat-eć\\
$\star$ppas:sg:gen:m1.m2.m3.n:aff ppas:sg:acc:m1.m2:aff & verb & $\varepsilon$ & Ja & ego & n & palat-eć\\
$\star$ppas:sg:gen:m1.m2.m3.n:neg ppas:sg:acc:m1.m2:neg & verb & nie & Ja & ego & n & palat-eć\\
$\star$ppas:sg:dat:m1.m2.m3.n:aff & verb & $\varepsilon$ & Ja & emu & n & palat-eć\\
$\star$ppas:sg:dat:m1.m2.m3.n:neg & verb & nie & Ja & emu & n & palat-eć\\
$\star$ppas:sg:acc.inst:f:aff & verb & $\varepsilon$ & Ja & ą & n & palat-eć\\
$\star$ppas:sg:acc.inst:f:neg & verb & nie & Ja & ą & n & palat-eć\\
$\star$ppas:sg:inst.loc:m1.m2.m3.n:aff ppas:pl:dat:m1.m2.m3.f.n:aff & verb & $\varepsilon$ & Ja & ym & n & palat-eć\\
$\star$ppas:sg:inst.loc:m1.m2.m3.n:neg ppas:pl:dat:m1.m2.m3.f.n:neg & verb & nie & Ja & ym & n & palat-eć\\
$\star$ppas:pl:nom.voc:m1:aff & verb & $\varepsilon$ & Ja & i & n & palat-eć\\
$\star$ppas:pl:nom.voc:m1:neg & verb & nie & Ja & i & n & palat-eć\\
$\star$ppas:pl:gen.loc:m1.m2.m3.f.n:aff ppas:pl:acc:m1:aff & verb & $\varepsilon$ & Ja & ych & n & palat-eć\\
$\star$ppas:pl:gen.loc:m1.m2.m3.f.n:neg ppas:pl:acc:m1:neg & verb & nie & Ja & ych & n & palat-eć\\
$\star$ppas:pl:inst:m1.m2.m3.f.n:aff & verb & $\varepsilon$ & Ja & ymi & n & palat-eć\\
$\star$ppas:pl:inst:m1.m2.m3.f.n:neg & verb & nie & Ja & ymi & n & palat-eć\\
\end{longtable}
\begin{longtable}{p{7cm}|l|l|l|l|l|l}
interpretacja & cat & pref & group & flex & flex2 & lemma\\
\hline
fin:sg:sec & verb & $\varepsilon$ & y & sz & $\varepsilon$ & palat-ć\\
fin:sg:ter & verb & $\varepsilon$ & y & $\varepsilon$ & $\varepsilon$ & palat-ć\\
fin:pl:pri & verb & $\varepsilon$ & y & my & $\varepsilon$ & palat-ć\\
fin:pl:sec & verb & $\varepsilon$ & y & cie & $\varepsilon$ & palat-ć\\
fin:sg:pri & verb & $\varepsilon$ & J & ę & $\varepsilon$ & palat-ć\\
fin:pl:ter & verb & $\varepsilon$ & J & ą & $\varepsilon$ & palat-ć\\
impt:sg:sec & verb & $\varepsilon$ & yj & $\varepsilon$ & $\varepsilon$ & palat-ć\\
impt:pl:pri & verb & $\varepsilon$ & yj & my & $\varepsilon$ & palat-ć\\
impt:pl:sec & verb & $\varepsilon$ & yj & cie & $\varepsilon$ & palat-ć\\
impt:sg:sec & verb & $\varepsilon$ & $\varepsilon$ & $\varepsilon$ & $\varepsilon$ & palat-ć\\
impt:pl:pri & verb & $\varepsilon$ & $\varepsilon$ & my & $\varepsilon$ & palat-ć\\
impt:pl:sec & verb & $\varepsilon$ & $\varepsilon$ & cie & $\varepsilon$ & palat-ć\\
pcon:imperf & verb & $\varepsilon$ & J & $\varepsilon$ & ąc & palat-ć\\
pacta & verb & $\varepsilon$ & J & o & ąc & palat-ć\\
pact:sg:nom.acc.voc:n:imperf:aff pact:pl:nom.acc.voc:m2.m3.f.n:imperf:aff & verb & $\varepsilon$ & J & e & ąc & palat-ć\\
pact:sg:nom.acc.voc:n:imperf:neg pact:pl:nom.acc.voc:m2.m3.f.n:imperf:neg & verb & nie & J & e & ąc & palat-ć\\
pact:sg:nom.voc:m1.m2.m3:imperf:aff pact:sg:acc:m3:imperf:aff pact:pl:nom.voc:m1:imperf:aff & verb & $\varepsilon$ & J & y & ąc & palat-ć\\
pact:sg:nom.voc:f:imperf:aff & verb & $\varepsilon$ & J & a & ąc & palat-ć\\
pact:sg:nom.voc:m1.m2.m3:imperf:neg pact:sg:acc:m3:imperf:neg pact:pl:nom.voc:m1:imperf:neg & verb & nie & J & y & ąc & palat-ć\\
pact:sg:nom.voc:f:imperf:neg & verb & nie & J & a & ąc & palat-ć\\
pact:sg:gen.dat.loc:f:imperf:aff & verb & $\varepsilon$ & J & ej & ąc & palat-ć\\
pact:sg:gen.dat.loc:f:imperf:neg & verb & nie & J & ej & ąc & palat-ć\\
pact:sg:gen:m1.m2.m3.n:imperf:aff pact:sg:acc:m1.m2:imperf:aff & verb & $\varepsilon$ & J & ego & ąc & palat-ć\\
pact:sg:gen:m1.m2.m3.n:imperf:neg pact:sg:acc:m1.m2:imperf:neg & verb & nie & J & ego & ąc & palat-ć\\
pact:sg:dat:m1.m2.m3.n:imperf:aff & verb & $\varepsilon$ & J & emu & ąc & palat-ć\\
pact:sg:dat:m1.m2.m3.n:imperf:neg & verb & nie & J & emu & ąc & palat-ć\\
pact:sg:acc.inst:f:imperf:aff & verb & $\varepsilon$ & J & ą & ąc & palat-ć\\
pact:sg:acc.inst:f:imperf:neg & verb & nie & J & ą & ąc & palat-ć\\
pact:sg:inst.loc:m1.m2.m3.n:imperf:aff pact:pl:dat:m1.m2.m3.f.n:imperf:aff & verb & $\varepsilon$ & J & ym & ąc & palat-ć\\
pact:sg:inst.loc:m1.m2.m3.n:imperf:neg pact:pl:dat:m1.m2.m3.f.n:imperf:neg & verb & nie & J & ym & ąc & palat-ć\\
pact:pl:gen.loc:m1.m2.m3.f.n:imperf:aff pact:pl:acc:m1:imperf:aff & verb & $\varepsilon$ & J & ych & ąc & palat-ć\\
pact:pl:gen.loc:m1.m2.m3.f.n:imperf:neg pact:pl:acc:m1:imperf:neg & verb & nie & J & ych & ąc & palat-ć\\
pact:pl:inst:m1.m2.m3.f.n:imperf:aff & verb & $\varepsilon$ & J & ymi & ąc & palat-ć\\
pact:pl:inst:m1.m2.m3.f.n:imperf:neg & verb & nie & J & ymi & ąc & palat-ć\\
inf & verb & $\varepsilon$ & y & $\varepsilon$ & ć & palat-ć\\
pant & verb & $\varepsilon$ & y & szy & w & palat-ć\\
imps & verb & $\varepsilon$ & Jo & o & n & palat-ć\\
praet:sg:m1.m2.m3 & verb & $\varepsilon$ & y & $\varepsilon$ & ł & palat-ć\\
praet:pl:m2.m3.f.n & verb & $\varepsilon$ & y & y & ł & palat-ć\\
praet:sg:n & verb & $\varepsilon$ & y & o & ł & palat-ć\\
praet:pl:m1 & verb & $\varepsilon$ & y & i & ł & palat-ć\\
praet:sg:f & verb & $\varepsilon$ & y & a & ł & palat-ć\\
ger:pl:gen:n:aff & verb & $\varepsilon$ & Je & $\varepsilon$ & ń & palat-ć\\
ger:sg:dat.loc:n:aff & verb & $\varepsilon$ & Je & u & ń & palat-ć\\
ger:pl:dat:n:aff & verb & $\varepsilon$ & Je & om & ń & palat-ć\\
ger:sg:inst:n:aff & verb & $\varepsilon$ & Je & em & ń & palat-ć\\
ger:sg:nom.acc:n:aff & verb & $\varepsilon$ & Je & e & ń & palat-ć\\
ger:pl:inst:n:aff & verb & $\varepsilon$ & Je & ami & ń & palat-ć\\
ger:pl:loc:n:aff & verb & $\varepsilon$ & Je & ach & ń & palat-ć\\
ger:sg:gen:n:aff ger:pl:nom.acc:n:aff & verb & $\varepsilon$ & Je & a & ń & palat-ć\\
ger:pl:gen:n:neg & verb & nie & Je & $\varepsilon$ & ń & palat-ć\\
ger:sg:dat.loc:n:neg & verb & nie & Je & u & ń & palat-ć\\
ger:pl:dat:n:neg & verb & nie & Je & om & ń & palat-ć\\
ger:sg:inst:n:neg & verb & nie & Je & em & ń & palat-ć\\
ger:sg:nom.acc:n:neg & verb & nie & Je & e & ń & palat-ć\\
ger:pl:inst:n:neg & verb & nie & Je & ami & ń & palat-ć\\
ger:pl:loc:n:neg & verb & nie & Je & ach & ń & palat-ć\\
ger:sg:gen:n:neg ger:pl:nom.acc:n:neg & verb & nie & Je & a & ń & palat-ć\\
ppas:sg:nom.acc.voc:n:aff ppas:pl:nom.acc.voc:m2.m3.f.n:aff & verb & $\varepsilon$ & Ja & e & n & palat-ć\\
ppas:sg:nom.acc.voc:n:neg ppas:pl:nom.acc.voc:m2.m3.f.n:neg & verb & nie & Ja & e & n & palat-ć\\
ppas:sg:nom.voc:m1.m2.m3:aff ppas:sg:acc:m3:aff & verb & $\varepsilon$ & Ja & y & n & palat-ć\\
ppas:sg:nom.voc:f:aff & verb & $\varepsilon$ & Ja & a & n & palat-ć\\
ppas:sg:nom.voc:m1.m2.m3:neg ppas:sg:acc:m3:neg & verb & nie & Ja & y & n & palat-ć\\
ppas:sg:nom.voc:f:neg & verb & nie & Ja & a & n & palat-ć\\
ppas:sg:gen.dat.loc:f:aff & verb & $\varepsilon$ & Ja & ej & n & palat-ć\\
ppas:sg:gen.dat.loc:f:neg & verb & nie & Ja & ej & n & palat-ć\\
ppas:sg:gen:m1.m2.m3.n:aff ppas:sg:acc:m1.m2:aff & verb & $\varepsilon$ & Ja & ego & n & palat-ć\\
ppas:sg:gen:m1.m2.m3.n:neg ppas:sg:acc:m1.m2:neg & verb & nie & Ja & ego & n & palat-ć\\
ppas:sg:dat:m1.m2.m3.n:aff & verb & $\varepsilon$ & Ja & emu & n & palat-ć\\
ppas:sg:dat:m1.m2.m3.n:neg & verb & nie & Ja & emu & n & palat-ć\\
ppas:sg:acc.inst:f:aff & verb & $\varepsilon$ & Ja & ą & n & palat-ć\\
ppas:sg:acc.inst:f:neg & verb & nie & Ja & ą & n & palat-ć\\
ppas:sg:inst.loc:m1.m2.m3.n:aff ppas:pl:dat:m1.m2.m3.f.n:aff & verb & $\varepsilon$ & Ja & ym & n & palat-ć\\
ppas:sg:inst.loc:m1.m2.m3.n:neg ppas:pl:dat:m1.m2.m3.f.n:neg & verb & nie & Ja & ym & n & palat-ć\\
ppas:pl:nom.voc:m1:aff & verb & $\varepsilon$ & Ja & i & n & palat-ć\\
ppas:pl:nom.voc:m1:neg & verb & nie & Ja & i & n & palat-ć\\
ppas:pl:gen.loc:m1.m2.m3.f.n:aff ppas:pl:acc:m1:aff & verb & $\varepsilon$ & Ja & ych & n & palat-ć\\
ppas:pl:gen.loc:m1.m2.m3.f.n:neg ppas:pl:acc:m1:neg & verb & nie & Ja & ych & n & palat-ć\\
ppas:pl:inst:m1.m2.m3.f.n:aff & verb & $\varepsilon$ & Ja & ymi & n & palat-ć\\
ppas:pl:inst:m1.m2.m3.f.n:neg & verb & nie & Ja & ymi & n & palat-ć\\
ppas:sg:nom.acc.voc:n:aff ppas:pl:nom.acc.voc:m2.m3.f.n:aff & verb & $\varepsilon$ & Jo & e & n & palat-ć\\
ppas:sg:nom.acc.voc:n:neg ppas:pl:nom.acc.voc:m2.m3.f.n:neg & verb & nie & Jo & e & n & palat-ć\\
ppas:sg:nom.voc:m1.m2.m3:aff ppas:sg:acc:m3:aff & verb & $\varepsilon$ & Jo & y & n & palat-ć\\
ppas:sg:nom.voc:f:aff & verb & $\varepsilon$ & Jo & a & n & palat-ć\\
ppas:sg:nom.voc:m1.m2.m3:neg ppas:sg:acc:m3:neg & verb & nie & Jo & y & n & palat-ć\\
ppas:sg:nom.voc:f:neg & verb & nie & Jo & a & n & palat-ć\\
ppas:sg:gen.dat.loc:f:aff & verb & $\varepsilon$ & Jo & ej & n & palat-ć\\
ppas:sg:gen.dat.loc:f:neg & verb & nie & Jo & ej & n & palat-ć\\
ppas:sg:gen:m1.m2.m3.n:aff ppas:sg:acc:m1.m2:aff & verb & $\varepsilon$ & Jo & ego & n & palat-ć\\
ppas:sg:gen:m1.m2.m3.n:neg ppas:sg:acc:m1.m2:neg & verb & nie & Jo & ego & n & palat-ć\\
ppas:sg:dat:m1.m2.m3.n:aff & verb & $\varepsilon$ & Jo & emu & n & palat-ć\\
ppas:sg:dat:m1.m2.m3.n:neg & verb & nie & Jo & emu & n & palat-ć\\
ppas:sg:acc.inst:f:aff & verb & $\varepsilon$ & Jo & ą & n & palat-ć\\
ppas:sg:acc.inst:f:neg & verb & nie & Jo & ą & n & palat-ć\\
ppas:sg:inst.loc:m1.m2.m3.n:aff ppas:pl:dat:m1.m2.m3.f.n:aff & verb & $\varepsilon$ & Jo & ym & n & palat-ć\\
ppas:sg:inst.loc:m1.m2.m3.n:neg ppas:pl:dat:m1.m2.m3.f.n:neg & verb & nie & Jo & ym & n & palat-ć\\
ppas:pl:nom.voc:m1:aff & verb & $\varepsilon$ & Jo & i & n & palat-ć\\
ppas:pl:nom.voc:m1:neg & verb & nie & Jo & i & n & palat-ć\\
ppas:pl:gen.loc:m1.m2.m3.f.n:aff ppas:pl:acc:m1:aff & verb & $\varepsilon$ & Jo & ych & n & palat-ć\\
ppas:pl:gen.loc:m1.m2.m3.f.n:neg ppas:pl:acc:m1:neg & verb & nie & Jo & ych & n & palat-ć\\
ppas:pl:inst:m1.m2.m3.f.n:aff & verb & $\varepsilon$ & Jo & ymi & n & palat-ć\\
ppas:pl:inst:m1.m2.m3.f.n:neg & verb & nie & Jo & ymi & n & palat-ć\\
\end{longtable}
\begin{longtable}{p{7cm}|l|l|l|l|l|l}
interpretacja & cat & pref & group & flex & flex2 & lemma\\
\hline
fin:sg:sec & verb & $\varepsilon$ & uje & sz & $\varepsilon$ & ywać\\
fin:sg:ter & verb & $\varepsilon$ & uje & $\varepsilon$ & $\varepsilon$ & ywać\\
fin:pl:pri & verb & $\varepsilon$ & uje & my & $\varepsilon$ & ywać\\
fin:pl:sec & verb & $\varepsilon$ & uje & cie & $\varepsilon$ & ywać\\
fin:sg:pri & verb & $\varepsilon$ & uj & ę & $\varepsilon$ & ywać\\
fin:pl:ter & verb & $\varepsilon$ & uj & ą & $\varepsilon$ & ywać\\
impt:sg:sec & verb & $\varepsilon$ & uj & $\varepsilon$ & $\varepsilon$ & ywać\\
impt:pl:pri & verb & $\varepsilon$ & uj & my & $\varepsilon$ & ywać\\
impt:pl:sec & verb & $\varepsilon$ & uj & cie & $\varepsilon$ & ywać\\
pcon:imperf & verb & $\varepsilon$ & uj & $\varepsilon$ & ąc & ywać\\
pacta & verb & $\varepsilon$ & uj & o & ąc & ywać\\
pact:sg:nom.acc.voc:n:imperf:aff pact:pl:nom.acc.voc:m2.m3.f.n:imperf:aff & verb & $\varepsilon$ & uj & e & ąc & ywać\\
pact:sg:nom.acc.voc:n:imperf:neg pact:pl:nom.acc.voc:m2.m3.f.n:imperf:neg & verb & nie & uj & e & ąc & ywać\\
pact:sg:nom.voc:m1.m2.m3:imperf:aff pact:sg:acc:m3:imperf:aff pact:pl:nom.voc:m1:imperf:aff & verb & $\varepsilon$ & uj & y & ąc & ywać\\
pact:sg:nom.voc:f:imperf:aff & verb & $\varepsilon$ & uj & a & ąc & ywać\\
pact:sg:nom.voc:m1.m2.m3:imperf:neg pact:sg:acc:m3:imperf:neg pact:pl:nom.voc:m1:imperf:neg & verb & nie & uj & y & ąc & ywać\\
pact:sg:nom.voc:f:imperf:neg & verb & nie & uj & a & ąc & ywać\\
pact:sg:gen.dat.loc:f:imperf:aff & verb & $\varepsilon$ & uj & ej & ąc & ywać\\
pact:sg:gen.dat.loc:f:imperf:neg & verb & nie & uj & ej & ąc & ywać\\
pact:sg:gen:m1.m2.m3.n:imperf:aff pact:sg:acc:m1.m2:imperf:aff & verb & $\varepsilon$ & uj & ego & ąc & ywać\\
pact:sg:gen:m1.m2.m3.n:imperf:neg pact:sg:acc:m1.m2:imperf:neg & verb & nie & uj & ego & ąc & ywać\\
pact:sg:dat:m1.m2.m3.n:imperf:aff & verb & $\varepsilon$ & uj & emu & ąc & ywać\\
pact:sg:dat:m1.m2.m3.n:imperf:neg & verb & nie & uj & emu & ąc & ywać\\
pact:sg:acc.inst:f:imperf:aff & verb & $\varepsilon$ & uj & ą & ąc & ywać\\
pact:sg:acc.inst:f:imperf:neg & verb & nie & uj & ą & ąc & ywać\\
pact:sg:inst.loc:m1.m2.m3.n:imperf:aff pact:pl:dat:m1.m2.m3.f.n:imperf:aff & verb & $\varepsilon$ & uj & ym & ąc & ywać\\
pact:sg:inst.loc:m1.m2.m3.n:imperf:neg pact:pl:dat:m1.m2.m3.f.n:imperf:neg & verb & nie & uj & ym & ąc & ywać\\
pact:pl:gen.loc:m1.m2.m3.f.n:imperf:aff pact:pl:acc:m1:imperf:aff & verb & $\varepsilon$ & uj & ych & ąc & ywać\\
pact:pl:gen.loc:m1.m2.m3.f.n:imperf:neg pact:pl:acc:m1:imperf:neg & verb & nie & uj & ych & ąc & ywać\\
pact:pl:inst:m1.m2.m3.f.n:imperf:aff & verb & $\varepsilon$ & uj & ymi & ąc & ywać\\
pact:pl:inst:m1.m2.m3.f.n:imperf:neg & verb & nie & uj & ymi & ąc & ywać\\
\end{longtable}

Poniższe tablice w lewej kolumnie zawierają interpretację, w prawej wartości atrybutu \texttt{flex}, \texttt{palat} i \texttt{velar},
a w nagłówku opis pozostałych atrybutów. Brak określenia wartości dla \texttt{palat} lub \texttt{velar} oznacza, że wartość może być dowolna.

\begin{multicols}{2}
cat=noun lemma=a gender:=f\\
\begin{tabular}{l|l}
sg:nom & a\\
sg:gen & y $\star\varepsilon$\\
sg:gen.dat.loc & ej\\
sg:dat.loc & y$\uparrow$ ie$\downarrow$\\
sg:acc & ę ą\\
sg:inst & ą\\
sg:voc & u$\uparrow$ o a\\
pl:nom.acc.voc & y$\downarrow$ e\\
pl:gen & $\varepsilon$ y$\uparrow$\\
pl:gen.loc & ych\\
pl:dat & ym om\\
pl:inst & ymi ami\\
pl:loc & ach\\
\end{tabular}\\

cat=noun lemma=a gender:=m1\\
\begin{tabular}{l|l}
sg:nom & a\\
sg:gen & y $\star$ego\\
sg:gen.acc & $\star$ego\\
sg:dat & $\star$emu\\
sg:dat.loc & y$\uparrow$ ie$\downarrow$\\
sg:acc & ę\\
sg:inst & ą $\star$ym\\
sg:loc & $\star$ym\\
sg:voc & o $\star$u\\
pl:nom.voc & y$\uparrow$ i$\downarrow$ owie $\star$e\\
pl:gen.acc & ów $\star\varepsilon$\\
pl:dat & om\\
pl:inst & ami\\
pl:loc & ach\\
depr & y e\\
\end{tabular}\\

cat=noun lemma=a gender:=m2\\
\begin{tabular}{l|l}
sg:nom & a\\
sg:gen & y\\
sg:dat.loc & y$\uparrow$ ie$\downarrow$\\
sg:acc & ę\\
sg:inst & ą\\
sg:voc & o\\
pl:nom.acc.voc & y$\downarrow$ e$\uparrow$\\
pl:gen & ów y\\
pl:dat & om\\
pl:inst & ami\\
pl:loc & ach\\
\end{tabular}\\

cat=noun lemma=a gender:=n:ncol\\
\begin{tabular}{l|l}
sg:nom.acc.voc & $\star$a\\
sg:gen & $\star$a\\
sg:dat & $\star$a\\
sg:inst & $\star$a\\
sg:loc & $\star$a\\
\end{tabular}\\

cat=noun lemma=a gender:=n:pt\\
\begin{tabular}{l|l}
pl:nom.acc.voc & a\\
pl:gen & $\varepsilon$ ów y$\uparrow$\\
pl:dat & om\\
pl:inst & ami\\
pl:loc & ach\\
\end{tabular}\\

cat=noun lemma=anin gender:=m1\\
\begin{tabular}{l|l}
pl:nom.voc & anie\\
pl:gen.acc & anów an\\
pl:dat & anom\\
pl:inst & anami\\
pl:loc & anach\\
depr & any\\
\end{tabular}\\

cat=noun lemma=e gender:=m1\\
\begin{tabular}{l|l}
sg:nom.voc|depr & e\\
sg:gen.acc & ego\\
sg:dat & emu\\
sg:inst.loc & em\\
pl:nom.voc & owie\\
pl:gen.acc & ów\\
pl:dat & om\\
pl:inst & ami\\
pl:loc & ach\\
\end{tabular}\\

cat=noun lemma=e gender:=n:ncol\\
\begin{tabular}{l|l}
sg:nom.acc.voc & e\\
sg:gen & ego\\
sg:gen|pl:nom.acc.voc & a$\uparrow$\\
sg:dat & emu\\
sg:dat.loc & u$\uparrow$\\
sg:inst & em$\uparrow$\\
sg:inst.loc & em$\downarrow$ ym\\
pl:nom.acc.voc & e\\
pl:gen & $\varepsilon\uparrow$ $\star$ów y$\uparrow$\\
pl:gen.loc & ych\\
pl:dat & ym om$\uparrow$\\
pl:inst & ymi ami$\uparrow$\\
pl:loc & ach$\uparrow$\\
\end{tabular}\\

cat=noun lemma=e gender:=n:pt\\
\begin{tabular}{l|l}
pl:nom.acc.voc & e\\
pl:gen & $\varepsilon\uparrow$ ów$\uparrow$ ych$\downarrow$ y$\uparrow$\\
pl:dat & ym om\\
pl:inst & ymi ami\\
pl:loc & ych ach\\
\end{tabular}\\

cat=noun lemma=mię gender:=n:ncol\\
\begin{tabular}{l|l}
sg:nom.acc.voc & mię\\
sg:gen & mienia\\
sg:dat.loc & mieniu\\
sg:inst & mieniem\\
pl:nom.acc.voc & miona\\
pl:gen & mion\\
pl:dat & mionom\\
pl:inst & mionami\\
pl:loc & mionach\\
\end{tabular}\\

cat=noun lemma=ni gender:=f\\
\begin{tabular}{l|l}
sg:nom.gen.dat.loc.voc & ni\\
sg:acc & nię $\star$nią\\
sg:inst & nią\\
pl:nom.acc.voc & nie\\
pl:gen & ń\\
pl:dat & niom\\
pl:inst & niami\\
pl:loc & niach\\
\end{tabular}\\

cat=noun lemma=o gender:=m1\\
\begin{tabular}{l|l}
sg:nom & o\\
sg:gen & y$\downarrow$ $\star$a\\
sg:gen.acc & a\\
sg:dat & u$\downarrow$ owi ie$\downarrow$\\
sg:acc & ę$\downarrow$\\
sg:inst & ą$\downarrow$ em\\
sg:loc & u$\uparrow$ u$\downarrow\rightarrow$ ie$\downarrow$\\
sg:voc & u$\uparrow$ u$\downarrow\rightarrow$ o\\
pl:nom.voc & owie a$\downarrow$\\
pl:gen.acc & ów\\
pl:dat & om\\
pl:inst & ami\\
pl:loc & ach\\
depr & y$\downarrow$ o e$\uparrow$ a$\downarrow$\\
\end{tabular}\\

cat=noun lemma=o gender:=m1:pt\\
\begin{tabular}{l|l}
pl:nom.voc & o\\
pl:gen.acc & a\\
pl:dat.loc & u\\
pl:inst & em\\
\end{tabular}\\

cat=noun lemma=o gender:=m2\\
\begin{tabular}{l|l}
sg:nom & o\\
sg:gen.acc & a\\
sg:dat & owi\\
sg:inst & em\\
sg:loc & u$\uparrow$ u$\downarrow\rightarrow$ ie$\downarrow\leftarrow$\\
sg:voc & u o\\
pl:nom.acc.voc & y$\downarrow$ e$\uparrow$ a$\downarrow$\\
pl:gen & ów\\
pl:dat & om\\
pl:inst & ami\\
pl:loc & ach\\
\end{tabular}\\

cat=noun lemma=o gender:=m3\\
\begin{tabular}{l|l}
sg:nom.acc & $\star$o\\
sg:gen & $\star$a\\
sg:dat & $\star$owi\\
sg:inst & $\star$em\\
sg:loc & $\star$u\\
sg:voc & $\star$u\\
pl:nom.acc.voc & $\star$e\\
pl:gen & $\star$ów\\
pl:dat & $\star$om\\
pl:inst & $\star$ami\\
pl:loc & $\star$ach\\
\end{tabular}\\

cat=noun lemma=o gender:=n:ncol\\
\begin{tabular}{l|l}
sg:nom.acc.voc & o\\
sg:gen & a\\
sg:dat & u\\
sg:inst & em\\
sg:loc & u$\uparrow$ u$\downarrow\rightarrow$ ie$\downarrow\leftarrow$\\
pl:nom.acc.voc & a\\
pl:gen & $\varepsilon$ ów $\star$y\\
pl:dat & om\\
pl:inst & ami\\
pl:loc & ach\\
\end{tabular}\\

cat=noun lemma=o(n) gender:=m1\\
\begin{tabular}{l|l}
sg:gen.acc & ona\\
sg:dat & onowi\\
sg:inst & onem\\
sg:loc.voc & onie\\
pl:nom.voc & onowie\\
pl:gen.acc & onów\\
pl:dat & onom\\
pl:inst & onami\\
pl:loc & onach\\
depr & ony\\
\end{tabular}\\

cat=noun lemma=stwo gender:=m1:pt\\
\begin{tabular}{l|l}
pl:nom.voc & stwo\\
pl:gen.acc & stwa\\
pl:dat.loc & stwu\\
pl:inst & stwem\\
\end{tabular}\\

cat=noun lemma=um gender:=n:ncol\\
\begin{tabular}{l|l}
sg:nom.gen.dat.acc.inst.loc.voc & um\\
pl:nom.acc.voc & a\\
pl:gen & ów\\
pl:dat & om\\
pl:inst & ami\\
pl:loc & ach\\
\end{tabular}\\

cat=noun lemma=us gender:=m3\\
\begin{tabular}{l|l}
sg:gen & u\\
sg:dat & owi\\
sg:inst & em\\
sg:loc.voc & ie\\
pl:nom.acc.voc & y\\
pl:gen & ów\\
pl:dat & om\\
pl:inst & ami\\
pl:loc & ach\\
\end{tabular}\\

cat=noun lemma=y gender:=m1\\
\begin{tabular}{l|l}
sg:nom.voc & y\\
sg:gen.acc & ego\\
sg:dat & emu\\
sg:inst.loc|pl:dat & ym\\
pl:nom.voc & y$\uparrow$ owie i$\downarrow$ $\star$y\\
pl:gen.acc.loc & ych\\
pl:inst & ymi\\
depr & e\\
\end{tabular}\\

cat=noun lemma=y gender:=m2\\
\begin{tabular}{l|l}
sg:nom.voc & y\\
sg:gen.acc & ego\\
sg:dat & emu\\
sg:inst.loc|pl:dat & ym\\
pl:nom.acc.voc & e\\
pl:gen.acc.loc & ych\\
pl:inst & ymi\\
\end{tabular}\\

cat=noun lemma=y gender:=m3\\
\begin{tabular}{l|l}
sg:nom.acc.voc & y\\
sg:gen & ego\\
sg:dat & emu\\
sg:inst.loc|pl:dat & ym\\
pl:nom.acc.voc & e\\
pl:gen.loc & ych\\
pl:inst & ymi\\
\end{tabular}\\

cat=noun lemma=y gender:=n:pt\\
\begin{tabular}{l|l}
pl:nom.acc.voc & y\\
pl:gen & $\varepsilon\downarrow$ ów y$\uparrow$\\
pl:dat & om\\
pl:inst & ami\\
pl:loc & ach\\
\end{tabular}\\

cat=noun lemma=ę gender:=n:col\\
\begin{tabular}{l|l}
sg:nom.acc.voc & ę\\
sg:gen & ęcia\\
sg:dat.loc & ęciu\\
sg:inst & ęciem\\
pl:nom.acc.voc & ęta\\
pl:gen & ąt\\
pl:dat & ętom\\
pl:inst & ętami\\
pl:loc & ętach\\
\end{tabular}\\

cat=noun lemma=$\varepsilon$ gender:=f\\
\begin{tabular}{l|l}
sg:nom.acc & $\varepsilon\uparrow$\\
sg:gen.dat.loc.voc|pl:gen & y$\uparrow$\\
sg:inst & ą$\uparrow$\\
pl:nom.acc.voc & y$\uparrow$ e$\uparrow$\\
pl:dat & om$\uparrow$\\
pl:inst & ami$\uparrow$\\
pl:loc & ach$\uparrow$\\
\end{tabular}\\
cat=noun lemma=$\varepsilon$ gender:=m1\\
\begin{tabular}{l|l}
sg:nom & $\varepsilon$\\
sg:gen & $\star$y\\
sg:gen.acc & a\\
sg:dat & owi $\star$u\\
sg:dat.loc & $\star$y\\
sg:acc & $\star$y\\
sg:inst & em $\star$ą\\
sg:loc & $\star$u $\star$ie\\
sg:loc.voc & u$\uparrow$ u$\downarrow\rightarrow$ ie$\downarrow\leftarrow$\\
sg:voc & cze$\uparrow$ $\star$y $\star$ie\\
pl:nom.voc & y$\uparrow$ i$\downarrow$ e$\uparrow$ owie $\star$ie\\
pl:gen.acc & ów y$\uparrow$\\
pl:dat & om\\
pl:inst & ami\\
pl:loc & ach\\
depr & y$\downarrow$ e$\uparrow$\\
\end{tabular}\\
cat=noun lemma=$\varepsilon$ gender:=m2\\
\begin{tabular}{l|l}
sg:nom & $\varepsilon$\\
sg:gen.acc & a\\
sg:dat & owi $\star$u\\
sg:inst & em\\
sg:loc.voc & u$\uparrow$ u$\downarrow\rightarrow$ ie$\downarrow\leftarrow$\\
pl:nom.acc.voc & y$\downarrow$ e$\uparrow$ $\star$e\\
pl:gen & ów y$\uparrow$\\
pl:dat & om\\
pl:inst & ami\\
pl:loc & ach\\
\end{tabular}\\
cat=noun lemma=$\varepsilon$ gender:=m3\\
\begin{tabular}{l|l}
sg:nom.acc & $\varepsilon$\\
sg:gen & u a\\
sg:dat & $\star$u$\downarrow$ owi\\
sg:inst & em\\
sg:loc & $\star$ie\\
sg:loc.voc & u$\uparrow$ u$\downarrow\rightarrow$ ie$\downarrow\leftarrow$\\
sg:voc & $\star$ie\\
pl:nom.acc.voc & y$\downarrow$ e$\uparrow$ $\star$e $\star$a\\
pl:gen & ów y$\uparrow$\\
pl:dat & om\\
pl:inst & ami\\
pl:loc & ach\\
\end{tabular}\\
\end{multicols}
\end{document}
