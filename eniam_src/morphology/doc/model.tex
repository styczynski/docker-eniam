\documentclass{article}
\usepackage{amsmath}
\usepackage{amssymb}
\usepackage[T1]{fontenc}
\usepackage[utf8]{inputenc}
\usepackage[polish]{babel}
% \usepackage{tikz}
% \usetikzlibrary{conceptgraph}
\usepackage{amsthm}

\parindent 0pt
\parskip 4pt

\newcommand{\form}{{\it form}}
\newcommand{\lemma}{{\it lemma}}
\newcommand{\cat}{{\it cat}}
\newcommand{\interp}{{\it interp}}
\newcommand{\fsuf}{{\it fsuf}}
\newcommand{\lsuf}{{\it lsuf}}

\newtheorem{task}{Zadanie}
\newtheorem{answer}{Odpowiedź}

\title{Model probabilistyczny fleksji języka polskiego}
\author{Wojciech Jaworski}
%\date{}

\begin{document}
\maketitle

Zakładamy, że język jest rozkładem probabilistycznym na czwórkach (\form,\lemma,\cat,\interp),
czyli, że wystąpienia kolejnych słów w tekście są od siebie niezależne.
Interpretacja \interp{} jest zbiorem tagów zgodnym a tagsetem SGJP.
Kategoria $\cat \in \{ {\rm noun}, {\rm adj}, {\rm adv}, {\rm verb}, {\rm other} \}$
Zakładamy też, że język jest poprawny, tzn. nie ma literówek, ani błędów gramatycznych.

Dysponujemy następującymi danymi: 
\begin{itemize}
\item słownikiem gramatycznym S, czyli zbiorem czwórek, o których wiemy, że należą do języka;
\item zbiorem reguł, czyli zbiorem czwórek (\fsuf,\lsuf,\cat,\interp)
\item zbiorem wyjątków, czyli zbiorem czwórek, o których wiemy, że należą do języka, które nie są opisywane przez reguły
\item otagowaną listą frekwencyjną.
\end{itemize}
Reguła przyłożona do formy ucina \fsuf{} i przykleja \lsuf.

Lista frekwencyjna wytworzona jest na podstawie NKJP1M. Usunięte zostały z niej symbole 
(formy do których odczytania nie wystarczy znajomość reguł wymowy takie, jak liczby zapisane cyframi, oznaczenia godzin i lat,
znaki interpunkcyjne, skróty, emotikony). Usunięte zostały również formy odmienialne z użyciem myślnika i apostrofu 
(np. odmienione akronimy i nazwiska obce, formy takie jak ,,12-latek``). 
Interpretacje na liście frekwencyjnej zostały skonwertowane do postaci takiej jaka występuje w SGJP, 
łączącej interpretacje form identycznych. Na przykład interpretacje adj:pl:nom:m1:pos, adj:pl:voc:m1:pos, adj:pl:nom:p1:pos i adj:pl:voc:p1:pos 
zostały złączone w adj:pl:nom.voc:m1.p1:pos, a frekwencje form zsumowane.

Celem jest aproksymacja wartości P(\lemma,\cat,\interp|\form).

%Jakość aproksymacji mierzymy licząc jak często wśród $k$ najbardziej prawdopodobnych trójek $\lemma,\cat,\interp$ 
%wskazanych przez model dla zadanej formy znajduje się trójka poprawna. Wyniki dla poszczególnych form agregujemy 
%za pomocą średniej ważonej po ich częstościach.

%Pytanie 0: Ile wynosi powyższa miara liczona z użyciem p-stw wziętych z listy frekwencyjnej? (To jest ograniczenie górne dla modelu)

%Pytanie 0': Ile wynosi powyższa miara liczona z użyciem częstości wziętych ze zbioru reguł? (To jest ograniczenie dolne dla modelu)

Pierwszym kryterium jest przynależność formy do słownika S. 
Jeśli forma należy do S zakładamy, że jedno z haseł S zawierające tę formę
poprawnie opisuje jej lemat, kategorię i interpretację.

\begin{task}
Jakie jest prawdopodobieństwo trafienia na formę, której lemat, kategoria i interpretacja należy do słownika, czyli
\[P((\form,\lemma,\cat,\interp) \in S)\]
Jakie jest prawdopodobieństwo trafienia na formę, która należy do słownika, ale jej lemat, kategoria lub interpretacja należy do słownika, czyli
\[P((\form,\lemma,\cat,\interp) \not\in S \wedge \form \in S)\]
\end{task}

\begin{answer}
Prawdopodobieństwo natrafienia na formę należącą do słownika wynosi 95,67\%, zaś natrafienia na formę należącą do SGJP bez odpowiedniej
interpretacji -- 3,92\% (lista tych form znajduje się w pliku traps.txt).
\end{answer}

W przypadku form należących do słownika różnorodność interpretacji będzie niewielka, 
natomiast istotne będzie prawdopodobieństwo wystąpienia danego lematu.
Zaś w przypadku form nie należących do słownika prawdopodobieństwo wystąpienia lematu
będzie zawsze małe.

Dzielimy teraz listę frekwencyjną na część należącą do S i nie należącą do S. 
Od tej pory budujemy model osobno dla każdej z części.

W przypadku cześci należącej do S zauważamy, że \[P(\lemma,\cat,\interp|\form)=P(\form|\lemma,\cat,\interp)\frac{P(\lemma,\cat,\interp)}{P(\form)}\]

Zakładamy, że \interp{} jest niezależne od \lemma, pod warunkiem określonego \cat
\[P(\lemma,\cat,\interp)=P(\lemma,\cat)P(\interp|\lemma,\cat)=P(\lemma,\cat)P(\interp|\cat)\]

$P(\form)$, $P(\lemma,\cat)$ i $P(\interp|\cat)$ szacujemy na podstawie listy frekwencyjnej,
w przypadku pierwszych dwu stosując wygładzanie. Wyliczenie $P(\form)$ zawiera uogólniona lista frekwencyjna
(ścieżka {\tt resources/NKJP1M/NKJP1M-generalized-frequency.tab} w repozytorium ENIAM), $P(\lemma,\cat)$ -- plik
 {\tt prob\_lemmacat.txt}, zaś $P(\interp|\cat)$ -- {\tt prob\_itp\_givencat.txt} (oba zawarte w katalogu {\tt morphology/doc}).

$P(\form|\lemma,\cat,\interp)$ wynosi 0, gdy w S nie ma krotki postaci (\form,\lemma,\cat,\interp);
1, gdy jest dokładnie jedna krotka z (\lemma,\cat,\interp). Gdy jest ich więcej oznacza to, że
lemat ma przynajmniej dwa warianty odmiany. Są to przypadki rzadkie. Przypisujemy każdej z możliwości
prawdopodobieństwo 1.

\begin{task}
Przejrzeć SGJP i znaleźć wszystkie przykłady, w których dla ustalonego lematu, kategorii i interpretacji
jest więcej niż jedna forma. Znaleźć wystąpienia tych krotek na liście frekwencyjnej.
\end{task}

\begin{answer}
Lista takich form znajduje się w pliku multi\_forms.txt.
\end{answer}

Teraz zanalizujemy drugą część listy frekwencyjnej. 
Załóżmy, że reguły mają postać taką, że sufiks żadnej reguły nie jest podciągiem sufixu innej z nich.
Sufiksy reguł tworzą drzewo, które w każdym węźle ma dowiązania do sufixów o jeden znak dłuższych oraz kategorię pozostałe traktową łącznie.
Przyjmujemy następujące założenie modelowe:
\[P(\lemma,\cat,\interp|\form)\approx P(rule|\form)=P(rule|\fsuf)\]
Wynika ono z tego, że mając nieznaną formę musimy oprzeć się na ogólnych regułach 
odmiany i nie możemy korzystać z tego że ma ona jakieś konkretne brzmienie.
Korzystamy tutaj tylko z reguł oznaczonych jako produktywne.

Problem tu jest taki, że lista frekwencyjna jest zbyt mała by precyzyjnie określić p-stwo ok. 40000 reguł.
Dlatego znowu stosujemy zabieg z prawdopodobieństwem warunkowym.

\[P(rule|\fsuf)=P(\lsuf,\cat,\interp|\fsuf)=P(\fsuf|\lsuf,\cat,\interp)\frac{P(\lsuf,\cat,\interp)}{P(\fsuf)}\]

$P(\fsuf)$ jest prawdopodobieństwem tego, że do języka należy słowo o zadanym sufixie. 
Można je oszacować za pomocą listy frekwencyjnej.

Zakładamy, że \interp{} jest niezależne od \lsuf, pod warunkiem określonego \cat
\[P(\lsuf,\cat,\interp)=P(\lsuf,\cat)P(\interp|\lsuf,\cat)=P(\lsuf,\cat)P(\interp|\cat)\]

$P(\lsuf,\cat)$ i $P(\interp|\cat)$ można oszacować na podstawie listy frekwencyjnej.

\begin{task}
Oszacować $P(\fsuf)$ i $P(\lsuf,\cat)$ na podstawie listy frekwencyjnej.
Sprawdzić dla jakich sufiksów próbka jest mała albo nie ma jej wcale. 
\end{task}

% w razie gdyby był problem można próbować dzielić sufiksy na części i założyć niezależność tych części

$P(\fsuf|\lsuf,\cat,\interp)$ wynosi 0, gdy nie ma reguły postaci (\fsuf,\lsuf,\cat,\interp);
1, gdy jest dokładnie jedna reguła z (\fsuf,\lsuf,\cat,\interp). Ustawiamy produktywność reguł tak 
by nie pojawiało się więcej pasujących reguł. 

\begin{task}
Określić produktywność reguł i sprawdzić, czy nie ma niejednoznacznych dopasowań.
\end{task}

\begin{task}
Określić jakość modelu.
\end{task}

\begin{answer}
Wyliczona jakość modelu (stopień pokrycia listy frekwencyjnej przez co najmniej 95\% najbardziej prawdopodobnych interpretacji wg modelu) wyniosła 79,90\%.
\end{answer}

%czasowniki produktywne to te z lematem ać ować ywać, ić, yć, (nąć)

Pytanie 4: Czy powyższe przybliżenie jest poprawne, jak często jest więcej niż jedna reguła i ile wynoszą wówczas p-stwa?

Zadania poboczne: wytworzenie otagowanej listy frekwencyjnej, wytworzenie (uzupełnienie) zbioru reguł na podstawie SGJP i listy frekwencyjnej, wskazanie, które reguły opisują sytuacje wyjątkowe.

Zadanie na przyszłość: reguły słowotwórstwa i ich interpretacja semantyczna.

Do powyższego modelu trzeba jeszcze dodać prefixy nie i naj.

\end{document}
